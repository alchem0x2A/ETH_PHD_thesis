  % \chapter{Introduction}
\chapter{Field-Dependent Electrostatic Screening in van der Waals Heterostructures}
\label{ch:asym}
\renewcommand*\imgdir{img/asym/}

% \dictum[]{%
  % }%

\vspace{1em}

\chapterabstract{Part of this chapter appears in the following journal
  article: Li, L. H., Tian, T., Cai, Q., Shih, C.-J. \& Santos,
  E. J. G. Asymmetric electric field screening in van der Waals
  heterostructures. Nat Commun. 9, 1–11 (2018).  }

\newcommand*{\E}{\mathscr{E}}


\section{Introduction}
\label{sec:asym-introduction}


Van der Waals heterostructures (vdWHs) composed of two-dimensional
(2D) crystals and precisely assembled opens a new avenue for promising
electronic and optoelectronic applications with promising
characteristics~\autocite{Geim_2013_2D_vdw_Het,Jariwala_2016_mixed_vdw_het,Novoselov_2016_vdW,Liu_2016_rev,Withers_2015_LED_vde_Het,Britnell_2012_FET}.
%
With the properties of individual 2D material layers well-studied
since the discovery of graphene and other 2D
materials~\autocite{Novoselov_2005_2D_crystal},
%
the current challenge is how achieve unprecedented physical and chemical
phenomena different from the isolated 2D sheets via controlled
combination of different layer~\autocite{Geim_2013_2D_vdw_Het}.
%
The continuous development of experimental methods which allows
fabrication of 2D vdWHs with atomistic precision via bottom-up
approaches~\autocite{Jariwala_2016_mixed_vdw_het,Novoselov_2016_vdW,Liu_2016_rev},
is benefited from the
atomic flatness and absence of dangling bonds at the surface of 2D
layered materials, as discussed in \autoref{sec:inter-forc-at}.
%
A large variety of 2D vdWHs are fabricated using building
blocks like graphene, hBN and
TMDCs~\autocite{Dean_2010_BN_gr_highquality,Xue_2011_STM,Withers_2015_LED_vde_Het,Britnell_2012_FET,Cui_2015_multiterm_mos2}.
%
It is significant to remark that a large majority of these studies
employs the electrostatic field effect (see
\autoref{sec:qc-introduction}) to modulate the charge states in the
vdWHs to achieve tunable carrier
transport~\autocite{Britnell_2012_FET,Dean_2013_butterfly} as well
carrier-photon
interaction~\autocite{Britnell_2013_vdWE,Withers_2015_LED_vde_Het}.
%
%
The relative simple setup of electrostatic gating enables insight into the charge transfer, and on the
electric field screening through the 2D vdWHs.
%
The penetration of electrostatic field through a 2D vdWH is similar to the idea of field effect transparency through mono\-layer 2D material introduced in
\autoref{ch:qc}, while with a more complex manner.

In this chapter, we propose a self-consistent multilayer quantum
capacitor model (MQCM) to simulate the penetration of field effect
through vdWHs. By taking fundamental electronic properties (\eg
bandgap, DOS and Fermi level) of isolated layers, MQCM is capable of
solving the charge density and potential distribution in a certain 2D
vdWH.
%
Using the MQCM, we further extend the concept of electrostatic field
effect transparency to vdWHs, and quantifies the penetration depth of
an external field in multilayer graphene stacks.
%
More interestingly, taking graphene/2H-MoS\textsubscript{2} interfaces
as an example, our model predicts that the field effect within a 2D
vdWH may be asymmetric due to the varied DOS and mismatch of
Fermi level between individual layers.
%

Our predictions are backed up by experimental examination and first
principles simulations from our collaborators.
%
Using electric force microscopy (EFM), the
graphene/MoS\textsubscript{2} heterostructures exhibit distinct
electric response to external fields which depends on the stacking
sequence.
%
A large depolarization field is recorded at the
MoS\textsubscript{2} side, which is shown to be dependent majority
on the number of MoS\textsubscript{2} layers into the
heterostructure.
%
At the atomistic level, quantum mechanical \textit{ab initio}
simulations based on vdW-corrected density functional theory (DFT) are
employed to resolve the electronic properties of the vdWHs, including
polarization, charge transfer and band structures,
%
which indicate that
the anisotropic formation of interfacial dipole moments preferable at
the MoS\textsubscript{2} facet accounts for the asymmetric
response.
%
The simplicity of the model developed here allows exploration of
further 2D material combination with appreciable computational speed,
which may greatly benefit the design of 2D vdWH-based functional
devices.

\section{Results and Discussion}
\label{sec:asym-results}


\subsection{Self-Consistent Multilayer Quantum Capacitor Model (MQCM)}
\label{sec:asym-model}

The idea of the self-consistent multilayer quantum capacitor model
(MQCM) is to treat individual 2D material as discrete quantum
capacitors with vanishing thickness, such that the charge--potential
relation can be described by the quantum capacitance $C_{\mathrm{Q}}$.
%
Note here only limit the discussion to randomly-stacked 2D layers,
that the interlayer coupling~\autocite{Fang_2014_intercoupl_vdW} effect is
ignored.

\subsubsection{Field-Effect Transparency of Multilayer Graphene}
\label{sec:asym-field-pene-ml-gr}

To demonstrate the idea of MQCM, we first consider the situation of an
electric displacement field in the multilayer metal-oxide-graphene-semiconductor
quantum capacitor (ML-MOGS QC).
%
The setup resembles the monolayer MOGS QC in \autoref{fig:qc-Scheme},
with monolayer graphene replaced by an $N$-layer graphene stack.
SiO\textsubscript{2} and Si are chosen as the oxide and semiconductor
layers, respectively.
%
The structure of ML-MOGS QC is schematically illustrated in
\autoref{fig:asym-scheme_ML_MOGS}
\begin{figure}[htbp]
  \centering
  \import{\imgdir}{ML_graphene.pdf_tex}
  % \includegraphics[width=0.95\textwidth]{img/SI_ML_graphene.eps}
  \caption{Schematic for the multilayer graphene MOGS QC system}
  \label{fig:asym-scheme_ML_MOGS}
\end{figure}
%

We assume that each graphene layer is separated by a distance $d_0$
($\sim{}$3.3 \AA{}\autocite{Shearer_2016}) with interlayer permittivity
$\varepsilon_{\mathrm{G}} \approx 2.4$ taken from the value of
graphite~\autocite{Lui_2011_tunable,Regan_2012_ScreeningEngineered_PV}.
%
The index of layer, $i$, starts from the first graphene sheet adjacent to
the oxide surface.
%
For simplicity, we neglect the space between the first ($i=1$)
graphene layer and the oxide layer, as well as the interface states
(see \autoref{sec:qc-field-effect-transp}) between the last ($i=N$)
graphene layer and the semiconductor layer.
%
The electric fields in
the oxide layer and at the semiconductor surface
are $\E_{\mathrm{ox}}$ and $\E_{\mathrm{S}}$, respectively.
%
We label the electric field between
the $i$ and $i+1$ graphene layers as $\E_{\mathrm{G}, i}$.
%
The charge density of layer $i$ is labeled as $\sigma_{\mathrm{G}, i}$

Applying the Gauss's law on the $i$-th graphene
with sheet charge density, we get a set of conservation equations:
\begin{enumerate}
  \item $i=1$\\
    \begin{equation*}
      \label{eq:asym-gauss-1}
      \E_{\mathrm{ox}}\varepsilon_{0}\varepsilon_{\mathrm{ox}} + \sigma_{\mathrm{G},1} - \E_{\mathrm{G}, 1}\varepsilon_{0}\varepsilon_{\mathrm{G}} = 0
    \end{equation*} 
  \item $i=2 \cdots N-1$\\
    \begin{equation*}
      \label{eq:asym-gauss-2}
      \E_{\mathrm{G}, i-1}\varepsilon_{0}\varepsilon_{\mathrm{G}} + \sigma_{\mathrm{G},i} - \E_{\mathrm{G},i+1}\varepsilon_{0}\varepsilon_{\mathrm{G}} = 0
    \end{equation*}  
  \item $i=N$\\
    \begin{equation*}
      \E_{\mathrm{G},N-1}\varepsilon_{0}\varepsilon_{\mathrm{G}} + \sigma_{\mathrm{G},N} - \E_{\mathrm{S}}\varepsilon_{0}\varepsilon_{\mathrm{S}} = 0
    \end{equation*}
\end{enumerate}                 %
where $\varepsilon_{\mathrm{ox}}$ and $\varepsilon_{\mathrm{S}}$ are
the permittivities of the oxide and semiconductor layers,
respectively.
%
Note that the difference between the vacuum energy level
$\Delta E_{\mathrm{vac}}(i, i+1)$ between $i$ and $i+1$ graphene layers
($i=1\cdots N-1$) is identical to the difference between work
functions of the two layers ($\phi_{\mathrm{G}, i}$ and $\phi_{\mathrm{G}, i+1}$):
\begin{equation}
  \label{eq:asym-diff-E}
  \Delta E_{\mathrm{vac}}(i, i+1) = e\E_{\mathrm{G}, i}d_0 = e\phi_{\mathrm{G},i+1} - e\phi_{\mathrm{G},i}
\end{equation}
%
The above series of equations uniquely define the electrostatics in
the ML-MOGS QC, and can therefore be solved self-consistently by
giving an initial guess of $\E_{\mathrm{S}}$.
\begin{figure}[!htbp]
  \centering{}
  \import{\imgdir}{ML-eta.pdf_tex}
  \caption{\label{fig:asym-ml-eta} Field-effect transparency in
    multilayer graphene. \textbf{a}. Calculated $σ_{\mathrm{S}}$ as a
    function of number of graphene layers in a MOGS QC
    ($n_{0} = 10^{18}$ cm$^{−3}$). \textbf{b}. Calculated transparency
    index $\eta^{\mathrm{FE}}$ as a function of layer number $N$.
    \textbf{c}. The work function of 100-layer graphene as a function
    of distance from the oxide-graphene interface $x$ (in number of
    layers).  }
\end{figure}


To see the effect of the MQCM, we calculate the induced charge density
of the semiconductor layer $\sigma_{\mathrm{S}}$ as a function of
total graphene layer number $N$, and the charge on the metal gate
$\sigma_{\mathrm{M}}$, as shown in \autoref{fig:asym-ml-eta}\lc{a}.
%
To see a higher degree of field penetration to the semiconductor
layer, we choose the intrinsic carrier density $n_{0}$ to be $10^{18}$
cm$^{-3}$ (\ie a higher $C_{\mathrm{S}}$).
%
We observe that $\sigma_{\mathrm{S}}$ becomes independent of
$\sigma_{\mathrm{M}}$ when more than 10 layers of graphene are used in
the ML-MOGS QC.
%
This is as expected: intuitively speaking, the total DOS of multilayer
graphene increases with the layer number and results in a larger
$C_{\mathrm{Q}}$ value.
%
In the case of graphite ($N \to \infty$), the electric displacement
field is fully screened and $\sigma_{\mathrm{S}}$ becomes independent
of $\sigma_{\mathrm{M}}$.
%
Note in the MQCM, the field effect transparency arises as a collective
effect of the charge density regulation between layers. This avoids
cumbersome derivation the total DOS of multilayer graphene by the
tight-bonding model \autocite{Nilsson_2008_bitri_gr_electron} or
\textit{ab initio} calculations, while provide appreciable insights
and flexibility.
%
We further extend the concept of field-effect transparency
$\eta^{\mathrm{FE}}$ to ML-MOGS QC system, with the same definition
$\eta^{\mathrm{FE}}(N) = -\left(\dfrac{\partial
    \sigma_{\mathrm{S}}}{\partial \sigma_{\mathrm{M}}}\right)$ while
dependent on the total layer number $N$.
%
\autoref{fig:asym-ml-eta}\lc{b} shows $\eta^{\mathrm{FE}}$ as function of both $N$ and $\sigma_{\mathrm{M}}$.
%
Similar to the observation in \autoref{fig:asym-ml-eta}\lc{a}, the
effective penetration length of field effect appears as $\sim{}$10
layers, while field-effect transparency shows an asymmetric response
to different polarity of the gate electrode.
%

The MQCM approach proposed here is excellently scalable with respect
to the layer number to be used for questions such as electrode doping
of graphene
layers~\autocite{Pi_2009_metal_doping_gr,Giovannetti_2008_doping}.
%
As a demonstration, we calculate the work function profiles in the
perpendicular direction of a 100-layer MOGS QC, as shown in
\autoref{fig:asym-ml-eta}\lc{c}.
%
Two boundary layers appear both the oxide/graphene and
semiconductor\allowbreak{}/graphene interfaces, analogous to the
electrical double layer in electrolyte
solutions~\autocite{Bard_1980_electrochem_book}.
%
At the center of multilayer sheet ($i \sim{} 50$), the work function
of graphene is almost identical to the CNP, indicating the electric
field is fully screened inside the multilayer graphene.
%
Considering the small interlayer permittivity, such nano\-scale
electrostatic screening is caused by the charge transfer between the
individual graphene layers.

%
Taking the concept of Debye screening length (see
\autoref{eq:intro-debye}), we estimate the
effective sheet charge density $\sigma_{\mathrm{eff}}$ of the graphene layers to be:
\begin{equation}
\label{eq:asym-debye}
\sigma_{\mathrm{eff}} = \dfrac{\varepsilon_{0} \varepsilon_{\mathrm{G}} k_{\mathrm{B}} T}{e^{2} \lambda_{\mathrm{FE}}^{2}} d_{0}
\end{equation}
where $\lambda_{\mathrm{FE}}$ is the penetration length of field
effect ($\sim{}10\cdot{}d_{0}$). At $T=300$ K, $\sigma_{\mathrm{eff}}$
is estimated to $\sim{}9\times10^{9}\ e\cdot$cm$^{-2}$, which is
close to the value of intrinsic thermal carrier density of graphene
reported~\autocite{Fang_2007_carrier_graphene}.


\subsubsection{Electrostatic Screening in Graphene/MoS\textsubscript{2} vdWHs}
\label{sec:asym-classic}

The approach presented in \autoref{sec:asym-field-pene-ml-gr} can also be applied to 2D vdWHs composed of distinct 2D material layers.
%
As an example, we study the case of graphene (G) /
2H-MoS\textsubscript{2} heterostructures with varied layer numbers
corresponding to the experimental results from our collaborator (will
be shown later).
%
For simplicity, we refer to the 2H-MoS\textsubscript{2} as
MoS\textsubscript{2} throughout this chapter.
%
\autoref{fig:asym-mqcm-scheme} schematically shows the band diagram of
the G/MoS\textsubscript{2} vdWH used in our model.
%
\begin{figure}[!htbp]
  \centering{}
  \import{\imgdir}{mqcm_scheme.pdf_tex}
  \caption{\label{fig:asym-mqcm-scheme} Application of MQCM to a 2D
    vdW composed of graphene and 2H-MoS\textsubscript{2}. The band
    alignment and individual layer charge densities are schematically
    shown.  }
\end{figure}
%
Here we consider an isolated vdWH placed under an external electric
$\E_{\mathrm{ext}}$, and does not have carrier transfer with the
exterior (similar to the experimental conditions in the electric force
microscopy (EFM)~\autocite{Li_2014_screen}). The sign of
$\E_{\mathrm{ext}}$ is positive if the external field points from the
graphene layers to the MoS\textsubscript{2} layers.
%
Similar to the naming convention in \autoref{fig:asym-scheme_ML_MOGS},
the layer index $i$ starts from the outer-most graphene layer.
%
When the system is at equilibrium, we assume that the Fermi level
$E_{\mathrm{F}}$ aligns throughout the G/MoS\textsubscript{2}
vdWH.
%
In this model, we further assume that; (i) the intrinsic
electronic properties of individual layers, including density of
states (DOS), band gap $E_{\mathrm{g}}$, and work function $\phi$ are
invariable with the stacking order and the external electric field and
(ii) the interlayer distance ($d_{\mathrm{i, i+1}}$) between layers
$i$ and $i+1$ is not affected by the external electric field.
%
Note although the layer-dependent transition of band
structure~\autocite{Bhimanapati_2015_2D_rev} is ignored in assumption (i),
it has been shown that such classical treatment using Coulombic
coupling has high consistency with full-scale  \textit{ab initio}
simulations~\autocite{Andersen_2015_dielec_vdWH}.
%
The charge and potential
distribution in a $N$-layer vdWH is solved by several conservation equations
in a self-consistent approach similar to that in \autoref{sec:asym-field-pene-ml-gr}, including:
%

\begin{enumerate}
\item The charge neutrality of vdWH \\
  \begin{equation}
    \label{eq:asym-neutral-vdwh}
    \sum_{i=1}^{\mathrm{N}} \sigma_{i} = 0
  \end{equation}
  where $\sigma_{\mathrm{i}}$ is the sheet charge density of layer $i$.

  
\item Charge balance of the $i$-th layer by Gauss's law \\
  \begin{equation}
    \label{eq:asym-gauss-vdwh}
    \E_{i-1, i}\varepsilon_{0}\varepsilon_{i-1, i} + \sigma_{i} -  \E_{i, i+1} \varepsilon_{0} \varepsilon_{i, i+1} = 0
  \end{equation}
  where $\E_{i-1, i}$ and $\varepsilon_{i-1, i}$ are the electric field and permittivity between layers $i - 1$ and $i$, respectively.
  %
  Note $\E_{0, 1}$ and $\E_{N, N+1}$ are essentially the external electric field $\E_{\mathrm{ext}}$.
  %
  The permittivity $\varepsilon_{i-1, i}$ is treated as the average bulk permittivities of layers $i-1$ and $i$

  
\item The charge--work function relation
  \begin{equation}
    \label{eq:asym-charge-wf}
    \sigma_{i}(\phi_{i}) = {\displaystyle \int_{-\infty}^{\infty}} \mathrm{DOS}(E')
    \left[f(E', -e\phi_{i}) - f(E', -e\phi_{i0})\right] \mathrm{d}E'
  \end{equation}
  where $\phi_{i}$ is the work function (under external field) and
  $\phi_{i0}$ is the intrinsic work function (in isolated form) of
  layer $i$, respectively. $f$ is the Fermi-Dirac distribution
  function.

  
\item The potential drop between layers $i$ and $i+1$
  \begin{equation}
    \label{eq:5}
    \Delta E_{\mathrm{vac}}(i, i+1) = e \phi_{i+1} - e \phi_{i} = e\E_{i, i+1} d_{i, i+1}
  \end{equation}
\end{enumerate}

The set of equations
(\autoref{eq:asym-neutral-vdwh}--\autoref{eq:asym-charge-wf}) allows
describing the vdWH regardless of the current
material involved, as long as the previous assumptions hold.
%
By solving the conservation
equations of electric field and charge density of the
G/MoS\textsubscript{2} vdWH, we can extract information about
interfacial and screening properties.
%
For the simplest case of monolayer(1L) G/ 1L MoS\textsubscript{2},
\autoref{fig:asym-mqcm-result}\lc{a} shows the work functions of both
materials as a result of $\E_{\mathrm{ext}}$ ranging from -8 to +8
V·nm$^{-1}$.
%
\begin{figure}[!htbp]
  \centering{}
  \import{\imgdir}{mqcm_results.pdf_tex}
  \caption{\label{fig:asym-mqcm-result} Asymmetric response of
    G/MoS\textsubscript{2} vdWHs to external field revealed by
    MQCM. \textbf{a}. Work functions of graphene and MoS2 as functions
    of external electric field $\mathcal{E}_{\mathrm{ext}}$
    (V·nm$^{-1}$). The regimes of the n-doped and p-doped MoS$_{2}$
    are highlighted in green and faint red, respectively.  \textbf{b}
    Electric dipole moment $\mu(m, n)$ as a function of
    $\mathcal{E}_{\mathrm{ext}}$ for different number of graphene
    ($m$) and MoS$_{2}$ ($n$) layers. \textbf{c}.  Charge density in
    the graphene layers $\sigma_{\mathrm{G}}$ as a function of
    $\mathcal{E}_{\mathrm{ext}}$ for different G/MoS\textsubscript{2}
    vdWHs.  To illustrate the origin of the asymmetric response to
    external field, band alignment for 1L G/ 3L MoS\textsubscript{2}
    system under −2.0 V·nm$^{-1}$ and 2.0 V·nm$^{-1}$ are shown in \textbf{d} and \textbf{e}
    respectively.  Only band structures near
    the Fermi level are shown.
  }
\end{figure}

When highly n-doped and p-doped
MoS\textsubscript{2} can be found when $\E_{\mathrm{ext}}<-2.3$ V·nm$^{-1}$ or $\E_{\mathrm{ext}}> 5.5$ V·nm$^{-1}$, respectively.
%
The Fermi level of MoS\textsubscript{2} shifts beyond its conduction
band (CB) or valence band (VB) in both case respectively, which is
accounted for the charge accumulation in the vdWH. Note that a
noticeable charge accumulation ($>10^{12}\ e\cdot$cm$^{-2}$) occurs even
when the Fermi level of MoS\textsubscript{2} is $\sim{}$0.1 eV away
from the band edges, when the Fermi level of graphene shifts
significantly from its CNP. The charge separation between the layers
polarizes the 2D vdWH.
%
On the other hand, when $\E_{\mathrm{ext}}$is between -2.3 V·nm$^{-1}$
and 5.5 V·nm$^{-1}$, the Fermi level of MoS\textsubscript{2} lies far away
from the band edges, the vdWH is merely not
polarized.
%
We find that the polarization of the G/MoS\textsubscript{2} vdWH is
more enhanced under negative external electric field.%
In other words, the 1L G/ 1L MoS\textsubscript{2} vdWH asymmetrically
screens the external electric field $\E_{\mathrm{ext}}$.
%

We ascribe such asymmetry to the difference between the electronic
structures of graphene and MoS\textsubscript{2}: MoS\textsubscript{2}
is considered an n-type semiconductor with its intrinsic Fermi level
(-4.5 eV) closer to its CB (-4.0 eV) than its VB (-5.8
eV)~\autocite{Ochedowski_2014_contami_mos2,Das_2012_high_perform,Lu_2014_midgap_mos2},
while graphene has symmetric linear band structure around its CNP
(-4.6 eV)~\autocite{Das_Sarma_2011_electron_gr}.
%
Here all energy levels are
compared with the vacuum level which is set at 0 eV.
%
Due to the resemblance of Fermi level values between graphene and
MoS\textsubscript{2}, little charge transfer occurs under weak
electric field, and the degree of charge transfer is mainly determined
by the position of the Fermi level with respect to the CB or VB of
MoS\textsubscript{2}, when the Fermi level of MoS\textsubscript{2}
lies far from its band edges. %
%
Following the same procedure, we extend the discussions to multilayer
G/Mo\textsubscript{2} vdWHs. For a $m$-layer graphene / $n$-layer
MoS\textsubscript{2} vdWH ($m$L G/ $n$L MoS\textsubscript{2}), we
calculate the sheet dipole moment $\mu(m, n)$, defined
as
$\mu(m, n) = \sum_{i=1}^{m+n} \sigma_{i} (\sum_{i=0}^{i-1} d_{i, i+1})$, as shown in \autoref{fig:asym-mqcm-result}\lc{b}.
%
Clearly, the vdWH is more polarized under negative $\E_{\mathrm{ext}}$.
%
By increasing the number of layer numbers, the magnitude of
$\E_{\mathrm{ext}}$ required to induce significant polarization in the
vdWH also becomes smaller, as compared with the 1L G/ 1L MoS\textsubscript{2} system.
%
The effect is enhanced when more
MoS\textsubscript{2} layers are included, with the largest increment
noticed in 3L G/ 3L MoS\textsubscript{2}/3L system.
%
On the other hand, the number of graphene layers has minor
contributions to the asymmetric behavior effect.
%
The similar effect is also seen in the charge density profile.
%
\autoref{fig:asym-mqcm-result}\lc{c} shows the total charge density
$\sigma_{\mathrm{G}}$ in the graphene layers
($\sigma_{\mathrm{G}} = \sum_{i=1}^{m} \sigma_{i}$) for different
combinations of $m$ and $n$ of the G/MoS\textsubscript{2} interface.
%
Similarly, the charge transfer at the G/MoS\textsubscript{2} interface
is more pronounced at negative $\E_{\mathrm{ext}}$ regime, since
MoS\textsubscript{2} is more easily to be
n-doped~\autocite{Amani_2015_mos2_QY1}.
%
The charge redistribution is more pronounced with increased layer
numbers of graphene and MoS\textsubscript{2}, and the
MoS\textsubscript{2} layers contributes more to such effect than
graphene.
%
Note comparing \autoref{fig:asym-mqcm-result}\lc{b} and
\autoref{fig:asym-mqcm-result}\lc{c}, for thicker graphene layers (e.g. 3L
G/$n$L MoS\textsubscript{2}), a considerable amount of total dipole
moment can still be observed under weak electric field
($|\E_{\mathrm{ext}}|<5$ V·nm$^{-1}$), when the charge transfer between
graphene and MoS\textsubscript{2} is negligible ($<10^{11}$
$e \cdot$cm$^{-2}$).
%
This indicates that the electric field is well screened by
multilayer graphene under such conditions due to its semi\-metallic nature.
%
The screening in the
MoS\textsubscript{2} becomes more appreciable only when the Fermi level
reach the band edges, when the DOS increase drastically.
%
To verify such statement, we reconstructed the band diagram of 1L G/
3L MoS\textsubscript{2} system under electric fields of
$\E_{\mathrm{ext}}=-2$ V·nm$^{-1}$ (~\autoref{fig:asym-mqcm-result}\lc{d})
and $\E_{\mathrm{ext}}=+2$ V·nm$^{-1}$
(~\autoref{fig:asym-mqcm-result}\lc{e}, respectively.
%
Under -2 V·nm$^{-1}$ electric field, the Fermi level reaches
the CB of the outermost ($i=4$) MoS\textsubscript{2} layer, while under +2
V·nm$^{-1}$ electric field, the Fermi level remains within the band gap of
MoS\textsubscript{2} resulting in very small charge transfer,
% in good accordance with the ab initio calculations showed in
% \worktodo{Figure} 5(d)-5(f).
In other words, charge accumulation occurs mostly on graphene and the
outer-most MoS\textsubscript{2} layer, a similar effect as seen in
\autoref{fig:asym-ml-eta}\lc{c}.


\subsection{Experimental Observation of Asymmetric Screening in G/MoS\textsubscript{2} vdWHs}
\label{sec:asym-exp}

\begin{figure}[!htbp]
\centering{}
\import{\imgdir}{EFM_scheme.pdf_tex}
\caption{\label{fig:asym-EFM}%
  Experimental characterization of electrostatic screening in
  vdWHs. \textbf{a}. Optical images of graphene (left), MoS$_{2}$
  (middle) layers exfoliated and the constructed vdWH
  (right). Scale bar: 10 μm. \textbf{b}. Scheme of the EFM measurement. \textbf{c}
  Typical EFM responses of different vdWHs and bulk substrates. In the
  zoomed figure (right), the phase change becomes asymmetric in vdWHs
  with increased MoS\textsubscript{2} layer numbers.  }
\end{figure}

The asymmetric electrostatic screening proposed by the MQCM in
\autoref{sec:asym-classic} is demonstrated by experimental
characterization using electric force microscope (EFM).
%
The measurements are performed in the group of Dr. Lu Hua Li of Deakin University, Australia.
%
Here the principles and results of the experimental measurements are
briefly discussed.

Heterostructures of graphene and MoS\textsubscript{2} of different
thicknesses on gold (Au)-coated Si wafer were achieved by
PMMA-assisted transfer~\autocite{Minemawari_2011_inkjet_PMMA}:
%
As shown in \autoref{fig:asym-EFM}\lc{a}, 1-4L thick graphene nanosheets and 1-3L
thick MoS\textsubscript{2} nanosheetsare mechanically exfoliated onto
separate SiO\textsubscript{2}/Si substrates, with thickness confirmed
by the Raman
spectroscopy~\autocite{Ferrari_2006_raman,Lee_2010_anomal_raman_mos2,Chakraborty_2012_mos2_layer_raman}.
%
To fabricate MoS\textsubscript{2}/graphene heterostructures, the
MoS\textsubscript{2} was first transferred onto the graphene with the
help of PMMA, and then the G/MoS\textsubscript{2} hetero\-structure
was relocated onto a 100 nm-thick Au-coated SiO\textsubscript{2}/Si
substrate.
%

EFM was used to measure the electric field screening properties of the
G/MoS\textsubscript{2} heterostructures, as previously used in other
2D vdWH
systems~\autocite{Datta_2009_ML_Screening,Castellanos_Gomez_2012_interlayer,Li_2014_screen}.
%
A DC voltage sweeping ranging from +9 V to −9 V was applied to the Au
substrate, which generated external electric fields of different
intensities. The tip of the conductive cantilever which was grounded,
and acted as a sensor monitoring the electric field passing through
the heterostructures (~\autoref{fig:asym-EFM}\lc{b}).
%
Subtle changes of electric field could be detected by EFM phase shift
($\Delta \varphi$), following~\autocite{Li_2014_screen}:
\begin{equation}
  \label{eq:asym-phase-change}
  \Delta \varphi = \frac{\partial F}{\partial z} \frac{Q_{\mathrm{cant}}}{k} 
\end{equation}
where $\partial F/ \partial z$ is the local force gradient felt by the
cantilever tip, $k$ is the spring constant of the cantilever, and $Q_{\mathrm{cant}}$
is the Q-factor of the cantilever.
%
For simplicity, the interaction between the cantilever tip and the
sample in EFM is viewed as an ideal capacitor with capacitance
$C_{\mathrm{loc}}$, which is related to $\partial F/ \partial z$ via~\autocite{Li_2014_screen}:
\begin{equation}
  \label{eq:asym-capaci-cant}
  \frac{\partial F}{\partial z} = \frac{1}{2} \frac{\partial^{2} C_{\mathrm{loc}}}{\partial z^{2}} (V_{\mathrm{s}} + V_{\mathrm{CPD}})^{2}
\end{equation}
where $z$ is the distance between the tip
and the sample, $V_{\mathrm{s}}$ is the tip voltage, and $V_{\mathrm{CPD}}$ is the contact potential difference (CPD) due to the mismatch
of the work functions between the tip and the sample.
%
\autoref{eq:asym-phase-change} and \autoref{eq:asym-capaci-cant}
indicate the phase change of EFM is quadratic to the tip voltage in
ideal situations ($\partial^{2} C / \partial z^{2}$ being constant),
while a change in the capacitance results in shift of curve from perfect parabola.
%
As seen from the experimental data, the EFM spectroscopy from the Au
substrate, 4L graphene, bulk MoS\textsubscript{2}, and three
heterostructures, namely 1L G/ 1L MoS\textsubscript{2}, 4L G/2L
MoS\textsubscript{2}, and 4L G/3L MoS\textsubscript{2} are shown in
\autoref{fig:asym-EFM}\lc{b}.  The EFM phase data of all the samples formed
opening-up parabolas as indicated by \autoref{eq:asym-capaci-cant}.
%
However, different
samples give rise to slightly varied shapes of the parabolas.
%
Intriguingly, after normalizing the phase data for different
materials, we found that the parabolas of the EFM phase from some
heterostructures (such as 4L G/ 3L MoS$_{2}$) lacked the
mirror-symmetry as that of a perfect parabola.
%
The EFM data from the Au substrate could be well fitted by the
second-degree polynomial, but the same fitting process was not able to
reproduce the left part (negative bias) of the EFM phase curve from 4L
G/ 3L MoS\textsubscript{2} (see zoomed inset of
\autoref{fig:asym-EFM}\lc{b}).
%
A clear distinction can be observed between different types of 2D
vdWHs: similar to the Au substrate, the 4L graphene, bulk
MoS\textsubscript{2}, and 1L G / 1L MoS\textsubscript{2} systems yield
symmetric parabolas of the EFM phase; on the contrary, the
heterostructures of few-layer MoS\textsubscript{2} and graphene,
i.e. 4L G/2L MoS\textsubscript{2} and 4L G/3L
MoS\textsubscript{2}, exhibit deviated-parabolic EFM phase
values under more negative substrate voltages.
%
%
As discussed previously, most of the parameters determining EFM phase,
except $V_{\mathrm{s}}$, should be constant during each
measurement. The asymmetric parabolic EFM curves suggest that the
electric field screening properties of 4L G/2L MoS\textsubscript{2} and 4L G/3L MoS\textsubscript{2} systems are not
constant under different external electric fields. In turn, the local capacitance $C_{\mathrm{loc}}$ should change slightly
accordingly. This phenomenon was not shown in Au film, 4L graphene,
bulk MoS\textsubscript{2}, or 1L MoS\textsubscript{2}/1L graphene, but
became prominent in the heterostructures with increased thickness of
graphene and MoS\textsubscript{2}.
%
Such experimental results correspond well with our predictions in
\autoref{sec:asym-classic}, for both the asymmetric response and
layer-dependent enhancing of response.

\subsection{First Principles Simulations}
\label{sec:mult-theor-model}

Quantum mechanical \textit{ab initio} simulations based on
density functional theory (DFT) are also performed to elucidate the interfacial properties of G/MoS\textsubscript
heterostructures.
The first principle calculations are performed in the group of Dr. Elton Santos in Queen's University of Belfast, UK.
%
The degree of electric susceptibility $\chi_{\mathrm{vdWH}}$ (equivalent to
$\varepsilon_{\mathrm{r}} - 1$ where $\varepsilon_{\mathrm{r}}$ is the
effective permittivity) in the vdW heterostructures of graphene and
MoS\textsubscript{2} with different number of layers in response to
the applied electric fields are calculated.
%
The quantum mechanical models presented in
\autocite{Santos_2013_tunable_eps_gr,Santos_2013_ACSnano_kaxi} are
employed to extract information about the dielectric response under
finite-electric fields in large supercells.
%


\autoref{fig:asym-first-principles1}\lc{a-c} show the variation of
$\chi_{\mathrm{vdWH}}$ with $\E_{\mathrm{ext}}$ at different
G/MoS\textsubscript{2} layer combinations.
%
\begin{figure}[!htbp]
  \centering{}
  \import{\imgdir}{first_principles1.pdf_tex}
  \caption{\label{fig:asym-first-principles1} First-principles
    calculations for various vdWHs. \textbf{a}–\textbf{c} Electric
    susceptibility $\chi_{\mathrm{vdWH}}$ as a function of external
    field $\E_{\mathrm{ext}}$ in vdWHs with 1L, 2L and 3L MoS$_{2}$,
    respectively. $\chi_{\mathrm{vdWH}} - \E_{\mathrm{ext}}$ data with
    different layers of graphene are shown in each panel.
    % 
    \textbf{d} Summarized results of $\chi$ as a function of the number of graphene
    layers at $\E_{\mathrm{ext}} = 1.0$ V$\cdot$nm$^{-1}$.
  }
\end{figure}
Strikingly, only negative $\E_{\mathrm{ext}}$
affected despite the number of graphene and MoS\textsubscript{2}
sheets present in the heterostructures. This is in remarkable
agreement with our predictions in \autoref{sec:asym-classic} as well as the experimental results in \autoref{sec:asym-exp}, where an asymmetrical response
was observed only from graphene and few-layer MoS\textsubscript{2}
heterostructures (~\autoref{fig:asym-EFM}\lc{c}).
%
The effect is enhanced, as more MoS\textsubscript{2} layers are
included into each graphene system, for instance, in 3L G/3L
MoS\textsubscript{2} system, a four-fold enhanced magnitude of
$\chi_{\mathrm{vdWH}}$ relative to zero field was observed
(~\autoref{fig:asym-first-principles1}\lc{c}.
%
At a fix value of the electric field (-1.0 V·nm$^{-1}$), larger asymmetric
screening was displayed as the number of MoS\textsubscript{2} layers
increased: the screening in the heterostructures containing 3L
MoS\textsubscript{2} was almost doubled that of 1L and 2L
MoS\textsubscript{2} heterostructures.
%
The slope of $\chi_{\mathrm{vdWH}}$ versus the number of graphene
layers also increased with the thickness of MoS\textsubscript{2},
which indicates that thicker MoS\textsubscript{2} tends to contribute
more to the slab polarization.
%
% This follows the behavior observed from EFM measurements,
% which heterostructures involving thicker MoS\textsubscript{2} sheets
% in contact with graphene gave rise to a more asymmetric EFM phase
% parabola (Fig. 2(f)). Based on these results it becomes clear that the
% transition metal dichalcogenide layers play a key role on this
% screening effect. We will analyze in the following the modifications
% of the electronic structure of the heterostructures at finite electric
% fields, and elucidate the origin of this asymmetric susceptibility
% dependence on the external bias.  \worktodo{Figure} 4 shows that the
% behavior of with the electric bias results from the asymmetrical
% polarization associated to which side of the heterostructure the field
% interacts first. At negative bias, a larger amount of induced charge
% was displaced towards the surface-layers of the MoS\textsubscript{2}
% in the heterostructure, which consequently generates a polarization
% that provided a better screening to the external electric fields
% relative to positive bias (Fig. 4(a)-(b)). As the number of
% MoS\textsubscript{2} layer is small, little differences are noticed
% under the reversed electric field, as the dipole moment formed at the
% interface roughly compensated each other (Fig. 4(a)).  This effect is
% enlarged, as thicker MoS\textsubscript{2} sheets are included. This
% was due to the amount of interfacial charge redistribution, which
% generated electric dipole moments preferentially aligned along one
% direction (Fig. 4(b)). Electric fields pointing towards graphene were
% not well screened as those towards MoS\textsubscript{2} layers,
% because the induced polarization was not so efficient to generate
% response fields that would shield the heterostructures
% completely. This means that higher magnitudes of electric field were
% observed inside thinner heterostructures, rather than thicker ones
% (Fig. 4(c)-(d)). We also observed that the induced electric potential
% shows a smooth variation over the interface, and it is almost
% independent of the number of layers composing the junction.  displays
% high magnitudes over fields towards graphene layers, with a change in
% polarity at the MoS\textsubscript{2} layer near the interface with for
% thicker vdWHs (see Fig. 4(d)). This indicates that the
% interfacial-charge balance in both systems that generates is sensitive
% to the amount of polarization charge from the MoS\textsubscript{2}
% layers. That is, the thicker the MoS\textsubscript{2} sheets, the
% larger the polarization. A consequence of this electric field
% direction-dependent polarization is noted in the different magnitudes
% of observed in the vacuum region outside of 3L MoS\textsubscript{2}/3L
% graphene system for positive and negative fields (Fig. 4(d)).  In
% electrostatic boundary conditions, where the normal component of the
% displacement field has to be preserved into the
% system~\autocite{Meyer_2001_dipole_slab}, it gives: where corresponds to
% the field in the sheets and to the induced polarization.  It is worth
% noting that differs to because the latter is calculated directly from
% the average induced charge using the Poisson equation and the former
% directly from the boundary conditions and the input field in the
% simulations (see Methods for details). For negative fields towards
% MoS\textsubscript{2} layers, the second term on the right-hand side in
% Eq. (3) involving the polarization is appreciably large, which
% generates a depolarization or response field that would overcome the
% applied external bias. This resulted in smaller electric fields inside
% the heterojunction (Fig. 4(d)). A similar effect is observed to fields
% directed to graphene layers, but higher in magnitudes inside the sheet
% due to smaller induced polarization. The polarization at the
% G/MoS\textsubscript{2} heterostructure is therefore a contributing
% factor in the special screening field effect we measured by EFM.
%
To analyze observed asymmetric behavior at quantum mechanical level,
further analysis of the electronic structure in the vdWHs are
made.
The effect of the electric field can also be seen by the tuning of the
Fermi level $\Delta E_{\mathrm{F}}$ relative to the charge neutral
point (CNP) of graphene, as shown by the band structure calculated at
different magnitudes of $\E_{\mathrm{ext}}$ for 1 L G/ 3L
MoS\textsubscript{2} system in \autoref{fig:asym-first-principles2}.
\begin{figure}[!htbp]
  \centering{}
  \import{\imgdir}{first_principles2.pdf_tex}
  \caption{\label{fig:asym-first-principles2} %
    Electronic band structures of the 3L MoS2/1L Graphene vdWHs under
    different $\E_{\mathrm{ext}}$ (\textbf{a} -0.7 V·nm$^{-1}$,
    \textbf{b} 0 V·nm$^{-1}$ and \textbf{c} 0.7 V·nm$^{-1}$). The
    states of graphene and MoS$_{2}$ are highlighted in blue and pink,
    relatively.  An asymmetrical shift of states with respect to the
    Fermi level under external electric field is observed. }
\end{figure}
%
% This charge rearrangement is smaller for positive $\E_{\mathrm{ext}}$
% which is ascribed to the difference between semiconducting nature of
% the MoS\textsubscript{2} layer, and the semi-metallic character of

% %
% This results in less polarizable field-dependent facet,
% smaller charge-transfer from MoS\textsubscript{2} to graphene, and
% consequently better screening.
% The effect of the electric field can also be seen by the tuning of the
% Fermi level $\Delta E_{\mathrm{F}}$ relative to the charge neutral
% point (CNP) of graphene, as shown by the band structure calculated at
% different magnitudes of $\E_{\mathrm{ext}}$ for 1 L G/ 3L
% MoS\textsubscript{2} system in \autoref{fig:asym-first-principles2}.
% The 1L G/1L MoS\textsubscript{2}) tend to tune
% their Fermi level almost linearly with the electric field, similar to
% the case of monolayer
% graphene~\autocite{Yu_2009_Tuning,Wang_2010_carrier_gr,Li_2013_WF_mos2}.
% As the number of MoS\textsubscript{2} sheets increase, displayed
% variations at positive fields as large as 0.34 eV (for 3L G/
% 3L MoS\textsubscript{2} system), but almost negligible ar the negative
% $\E_{\mathrm{ext}}$ can be observed.
% Such asymmetric behavior has been observed when
% MoS\textsubscript{2} layers are used in metal-insulator-semiconductor
% junctions~\autocite{Chu_2017_eh_tunneling}.
% Carrier doping induced by the
% electric field was responsible for the variation of the Fermi level or
% the work function of MoS\textsubscript{2} mainly along one direction,
% which is directly related to the unbalance of charge density between
% both sides of the semimetal and the semiconductor interface. This
% indicates that the intrinsic character of the electronic structure of
% each system in vdW heterostructures contributes to the formation of
% the anisotropic screening observed. This effect has several main
% implications on the fundamental electronic structure of the
% G/MoS\textsubscript{2} interfaces as can be appreciated
%
Without external field, the Fermi level crossed the Dirac
point of graphene, as no charge-imbalance was present between both
systems. Several MoS\textsubscript{2} states at the conduction band
were observed at $\sim{}$50 meV relative to the Fermi level, which also
corresponds to the Schottky barrier presents at the
interface~\autocite{Yu_2014_gr_mos2}.
%
At finite fields, those states are observed to shift up (down) with
positive (negative) external field, which alters the band occupation.
It is observed that the Fermi level changes more significantly at
negative $\E_{\mathrm{ext}}$, leading to a larger degree of charge
transfer, which is in good agreement with our model predictions in
\autoref{sec:asym-classic}.



\subsection{Unifying Classical and Quantum Approaches}
\label{sec:asym-unify}

The semi-classical MQCM model shows sound agreement with the
\textit{ab initio} calculations in predicting the asymmetric screening
behavior of the G/MoS\textsubscript{2} vdWH, as a result of the
different energy levels and DOS of both materials.
%
As indicated by the analysis of EFM experiments, the asymmetric
response is likely to be induced by the change of local capacitance.
%
Inspired by this, we further propose that such asymmetry can be
described by the quantum capacitance $C_{\mathrm{Q}}$ of vdWHs within
both theoretical frameworks.
%
Similar to the case of monolayer 2D material, $C_{\mathrm{Q}}$ of the
vdWH is related to the total DOS of the system, which can be
accurately obtained using quantum mechanical approaches.
%
On the other hand, the apparent quantum capacitance in the MQCM
approach can be calculated by the differentiating the dipolar charge
density $\sigma_{\mathrm{dip}} = \mu / d_{\mathrm{tot}}$, where $d_{\mathrm{tot}}$ is the total thick of the vdWH,  by the potential drop across the vacuum level
$\Delta E_{\mathrm{vac}}^{\mathrm{tot}}$ (equivalent to the
$\sum_{i=1}^{N-1} \Delta E_{\mathrm{vac}}(i, i+1)$), such that $C_{\mathrm{Q}} = \partial \sigma_{\mathrm{dip}} / \partial \Delta E_{\mathrm{vac}}^{\mathrm{tot}}$
%
We compare the quantum capacitances calculated by the MQCM approach
($C_{\mathrm{Q}}^{\mathrm{MQCM}}$) and DFT calculations
($C_{\mathrm{Q}}^{\mathrm{MQCM}}$) as functions of $\E_{\mathrm{ext}}$ for various
G/MoS\textsubscript{2} systems in \autoref{fig:asym-qc}\lc{a} and \lc{b},
respectively.
%
Interesting, the quantum capacitances calculated by both methods show
similar behavior under external electric field, with the maximum
quantum capacitance reaching $\sim{}$30-33 $\mathrm{\mu}$F·cm$^{-2}$
in 3L G/3L MoS\textsubscript{2} system, under an electric field of -1
V·nm$^{-1}$.

\begin{figure}[!htbp]
  \import{\imgdir}{qc-compare.pdf_tex}
  \caption{\label{fig:asym-qc} %
    Quantum capacitance $C_{\mathrm{Q}}^{\mathrm{MQCM}}$ (\textbf{a}) and $C_{\mathrm{Q}}^{\mathrm{DFT}}$ for different MoS2/Graphene
    vdWHs calculated by the MQCM and DFT approaches, respectively.
  }
\end{figure}
%
We find the layer dependency of quantum capacitance is very
similar to that of the dipole moment and charge transfer: the layer
number of MoS\textsubscript{2} is dominating the magnitude of the
total quantum capacitance under strong electric field. This is
reasonable due to the higher quantum capacitance of
MoS\textsubscript{2} than graphene when Fermi level shifts to the band
edges (see \autoref{sec:qc-charge-distr-sg}).
%
Note that the DFT calculations predict a non-zero quantum capacitance
of vdWHs with 2L and 3L graphene even without external electric field,
as a result of the interlayer coupling, which is not included in the
classical model. Despite the minute difference between both
theoretical frameworks, it is clear that the quantum capacitance of
the G/MoS\textsubscript{2} vdWH, as a combination of the energy levels
and DOS, can describe the asymmetric screening behavior with good
precision. Our multiscale theoretical framework is thus readily
applicable for a variety of vdWHs beyond
graphene/MoS\textsubscript{2}, by utilizing the electronic “genome”,
in particular the quantum capacitances of individual 2D layers. A full
picture of electric screening of 2D vdWHs can be built benefited from
the framework proposed in this work, shedding light on researches
concerning the electrostatic nature of two-layer 2D vdWHs revealed by
recent
studies~\autocite{Chu_2017_eh_tunneling,Lee_2014_pn_vdw_het,Furchi_2014_PV_vdwH}.

\section{Conclusions}
\label{sec:asym-conclusions}


In this chapter, we extend the electrostatic model in \autoref{ch:qc}
to multilayer systems. The proposed multilayer quantum capacitor model
(MQCM) shows its potential in calculating the field-effect
transparency through multilayer graphene sheets as well as
graphene/MoS\textsubscript{2} van der Waals heterostructures(vdWHs). 

The proposed model reveals fundamental knowledge about the
electrostatic screening in vdWHs.
In particular, the MQCM model predicts an
asymmetric electric field response in
graphene/MoS\textsubscript{2} heterostructures. Such phenomenon is endorsed by
both high-resolution EFM spectroscopy
and density functional theory simulations.
%
The asymmetric response to external field is attributed to the charge transfer between the G/MoS\textsubscript{2} interface, which clearly distinguished from bulk materials.
%
Our theoretical framework paves the way to
understand and engineer the electronic and dielectric properties of a
broad class of 2D materials assembled in heterojunctions for different
technological applications.

\section{Methods}
\label{sec:asym-methods}

\subsection*{Experimental Procedures}
\label{sec:asym-exper-proc}

The graphene and MoS\textsubscript{2} nanosheets were mechanically
exfoliated by Scotch tape. Highly oriented pyrolytic graphite (HOPG)
(Momentive, US) and synthetic MoS\textsubscript{2} crystals (2D
Semiconductors, US) were used as received. Si wafers with 90 and 270
nm thick thermal SiO\textsubscript{2} were used for graphene and
MoS\textsubscript{2}, respectively. An Olympus optical microscopy
(BX51) equipped with a DP71 camera was used to search atomically thin
nanosheets, and a Cypher AFM (Asylum Research, US) was employed for
topography measurements. The Raman spectra were collected by a
Renishaw Raman microscope using 514.5 nm (for MoS\textsubscript{2})
and 633 nm (for graphene) lasers and an objective lens of 100× (a
numerical aperture of 0.9). For the fabrication of the
heterostructure, the identified MoS\textsubscript{2} nanosheets were
firstly coated by thin layer of PMMA, then peeled off from the
SiO\textsubscript{2}/Si via etching by NaOH, stacked on graphene on
SiO\textsubscript{2}/Si under the optical microscope. The
MoS\textsubscript{2}/graphene structure was transferred to Au (100 nm)
coated SiO\textsubscript{2}/Si substrate following a similar
procedure. The Au coating was produced by a Leica ACE600 sputter with
a crystal balance monitoring the coating thickness in real time. The
EFM measurements were conducted on the Cypher AFM and  EFM phase data
were collected by a Pt/Ti coated cantilever with a spring constant of
~2 N/m (ElectricLever, Asylum Research, US) dwelling above the
heterostructure at a sampling rate of 2 kHz, while a voltage sweeping
linearly from +9 to −9 V was applied on the Au substrate over a period
of 20 s.

\subsection*{\textit{Ab initio} Simulations}
\label{sec:ab-init-simul}
\textit{Ab initio} quantum calculations were based on ab initio
density functional theory using the
\texttt{SIESTA}\autocite{Soler_02_siesta} and the \texttt{VASP}
codes~\autocite{Kresse_1996_1,Kresse_1996_2}.  Projected augmented wave
method (PAW)~\autocite{Blochl_1994_PW} for the latter, and norm-conserving
(NC) Troullier-Martins pseudo\-potentials~\autocite{troullier91} for the
former, have been used in the description of the bonding environment
for Mo, S and C. The shape of the numerical atomic orbitals (NAOs) was
automatically determined by the algorithms described in
Ref. \cite{Soler_02_siesta}. The generalized gradient
approximation\autocite{Perdew_1996_GGA} along with the DRSLL
functional\autocite{Dion_2004_vdw} was used in both methods, together with
a double-zeta polarized basis set in Siesta, and a well-converged
plane-wave cutoff of 500 eV in VASP. The cutoff radii of the different
orbitals in SIESTA were obtained using an energy shift of 50 meV,
which proved to be sufficiently accurate to describe the geometries
and the energetics. Atoms were allowed to relax under the
conjugate-gradient algorithm until the forces acting on the atoms were
less than 1$\times$10$^{-8}$ eV·Å$^{-1}$.  To model the system studied in the
experiments, we created large supercells containing up to 394 atoms to
simulate the interface between different number of graphene and
MoS\textsubscript{2} layers. We have optimized the supercell for the
G/MoS\textsubscript{2} interface using a 5$\times{}$5 graphene cell on
a 4$\times{}$4 MoS\textsubscript{2} cell, where the mismatch between
different lattice constants is smaller than ~2.0\% (i.e. the systems
are commensurate). We have kept the lattice constant of the
MoS\textsubscript{2} at equilibrium, and stretched the one for
graphene by that amount. Negligible variations of the graphene
electronic properties are observed with the preservation of the Dirac
cone for all systems. To avoid any interactions between supercells in
the non-periodic direction, a 20 Å vacuum space was used in all
calculations. In addition to this, a cutoff energy of 120 Ry was used
to resolve the real-space grid used to calculate the Hartree and
exchange correlation contribution to the total energy in SIESTA. The
Brillouin zone was sampled with a 9×9×1 grid to perform relaxations
with and without van der Waals interactions. Electronic
band structures were calculated using a converged 44×44×1 k-sampling
for the unit cell of Graphene/MoS\textsubscript{2}. In addition to
this, a Fermi-Dirac distribution with an electronic temperature
of 20 meV was used to resolve the electronic structure.  The electric
field across the vdW heterostructures is simulated using a spatially
periodic sawtooth-like potential perpendicular to the
G/MoS\textsubscript{2} heterostructures. Such potential is convenient
to analyze the response of finite systems (e.g. slabs) to electric
fields\autocite{Santos_2013_tunable_eps_gr,Santos_2013_ACSnano_kaxi}.
%
The magnitudes of the spatially varying electrostatic potential and
charge density across the supercell are determined via a convolution
with a filter function to eliminate undesired oscillations and
conserve the main features important in the analysis.

\subsection*{Parameters of MQCM Approach}
\label{sec:param-mqcm-appr}
We simplify the quantum capacitance of graphene using a linear model:
$C_{\mathrm{G}, i} = 26.1$
μF·cm$^{-2}$·V$^{-1}\cdotΔ\phi_{\mathrm{G}, i}$, while the quantum
capacitance of MoS2 is a step function, such that no density of states
exist within the band gap, while $C_{\mathrm{MoS2}}^{\mathrm{n}} = 48$
μF$\cdot$cm$^{-2}$ and $C_{\mathrm{MoS2}}^{\mathrm{p}} = 180$
μF·cm$^{-2}$ for VB and CB,
respectively~\autocite{Lu_2014_midgap_mos2}. The intrinsic work function
of graphene and MoS$_{2}$ are set at
4.6~\autocite{Das_Sarma_2011_electron_gr} and 4.5 V~\autocite{Mak_2010_mos2},
respectively. The energy levels of CB and VB of MoS$_{2}$ are taken as
−4.0 and −5.8 eV, respectively.





\section{Author Contributions}
\label{sec:asym-author-contributions}

This project is a joint collaboration with the groups of Dr. Lu Hua Li
of Deakin University, Australia (experiment) and Dr. Elton
J. G. Santos in Queen's University of Belfast, UK. E.J.G.S. and
L.H.L. designed the research. L.H.L. fabricated the samples, perform
characterization, and EFM measurements. Q.C. performed Raman
characterization. T.T. and C.J.S. developed the capacitor
model. E.J.G.S. developed the first-principles calculations and
performed analysis.  All authors contributed to the discussions.




%%% Local Variables:
%%% mode: latex
%%% TeX-master: "../thesis"
%%% End:
