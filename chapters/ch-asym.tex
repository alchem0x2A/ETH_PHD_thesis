% \chapter{Introduction}
\chapter{Field-Dependent Electrostatic Screening in van der Waals Heterostructures}
\label{ch:asym}
\renewcommand*\imgdir{img/ch-asym/}

% \dictum[Wolfgang Pauli]{%
  % God made the bulk;\\Surfaces were invented by the devil
  % }%
\worktodo{find the quote or quit}

\vspace{1em}

\chapterabstract{Part of this chapter appears in the following journal
  article: Li, L. H., Tian, T., Cai, Q., Shih, C.-J. \& Santos,
  E. J. G. Asymmetric electric field screening in van der Waals
  heterostructures. Nat Commun. 9, 1–11 (2018).  }



Introduction Van der Waals heterostructures (vdWHs) composed of
two-dimensional (2D) crystals and precisely assembled in a
deterministic order constitutes a new paradigm for promising
electronic and optoelectronic applications with enhanced features and
performance1-6. With the properties of the individual layers being
characterized since the discovery of graphene and other 2D materials7,
the great challenge is how to combine them in order to obtain new
physical and chemical phenomena not observed on the original
sheets1. The continuous development of experimental methods that allow
atomically thin materials to be fabricated on-demand and to be placed
on desired locations with an unprecedented control and accuracy have
opened new avenues to fabricate complex device architectures using
regular bottom-up approaches2-4. The atomic flatness and lack of
dangling bonds at the surface of 2D layered materials, such as
graphene, boron nitride (BN) and transition metal dichalcogenides
(TMDCs), allow them to form non-covalent interactions with a wide
range of materials without the condition of lattice matching that
normally heterojunctions would have. One of the first realizations of
such vdWHs was the fabrication of BN/graphene interfaces and the
probing of their electronic properties using gate bias8. Such system
remarkably showed that graphene could develop superior electrical
properties, achieving levels of performance comparable to those
observed with freestanding layers. In this assembly BN layers work as
an underlying substrate screening out graphene from any dangling
bonds, corrugations and charge inhomogeneities that are inherent to
standards SiO2 surfaces to play a role on its electronic
properties8,9. A direct extension of this system was the complete
encapsulation of graphene, TMDCs and the combination of them between
BN layers5,6,10,11. This rapidly showed that such device framework
displayed improved transport properties with high-mobilities10. It is
important to remark that in all these systems electrostatic gating
schemes have been one of the main driving force to probe the chemical
and physical properties of either isolated 2D materials before
assembling, or their vdWHs afterwards. Electric gate bias measurements
have become a feasible way to control, induce and manipulate
electronic properties of almost any vdW crystals since graphene early
days12,13,14. This relative simple setup allows deep insight into the
charge density reorganization between different stacking sequences,
and on the electric-field screening, which determines most of the
exquisite device properties observed in vdWHs. However, no direct
measurements of the electrostatic screening features of different
combinations of 2D materials have been reported yet, which ramp up
further understanding and developments of vdWHs into any technological
platform.  Here we show that the dielectric properties of an
archetypal vdWH, involving graphene and MoS2, displays an asymmetrical
behavior relative to the polarization of the applied gate bias into
the surface allowing an anisotropic screening. Using electric force
microscopy (EFM), we observe that graphene/MoS2 heterostructures
exhibit distinct electric response to external fields, respect to the
stacking sequence. A large depolarization field is recorded at the
MoS2 side, which is shown to be dependent majority on the number of
MoS2 layers into the heterostructure. A multiscale theoretical
framework is developed for elucidating such stacking-dependent
asymmetric electric screening phenomenon in graphene/MoS2 vdWHs. At
the atomistic level, we employed quantum mechanical ab initio
simulations based on vdW-corrected density functional theory (DFT) to
resolve various electronic properties of the vdWHs, including
polarization, charge transfer and band structures. Our results
indicate that the anisotropic formation of interfacial dipole moments
preferable at the MoS2 facet accounts for the asymmetric response. We
further show that such asymmetric behavior can be explained using a
classical quantum capacitor model, described by a set of
self-consistent electrostatic conservation equations and treating the
2D layers according to their individual electronic genomes
(i.e. energy levels, band gap and quantum capacitance). Our
theoretical framework has good consistency with the experimental
results at both ab initio and classical levels, showing that the
combinations of 2D materials with distinct electronic structures can
result in vdWHs with rich screening features. Furthermore, our
theoretical framework is readily applicable for other vdWHs beyond
graphene/MoS2 to explore a wide range of 2D material combinations with
programmable electronic screening properties, which may greatly
benefit the design of 2D vdWH-based functional devices.

Results and Discussion
Characterization and Measurements
Heterostructures of graphene and MoS2 of different thicknesses on Au coated Si wafer were achieved by two rounds of PMMA transfer.15 Figure 1a shows the optical image of 1-4L thick graphene nanosheets mechanically exfoliated on a SiO2 (90 nm)/Si substrate. Their thickness was confirmed by the Raman spectra, as shown in Figure 1b. The intensity of the 2D band of the 1L graphene was much stronger than that of the G band; while the intensity of the 2D and G bands of the 2L graphene is comparable. The Raman results are consistent with previous reports.16 The absence of Raman D band suggests the high quality of the graphene. Atomically thin MoS2 nanosheets were exfoliated on a SiO2 (270 nm)/Si substrate (Figure 1c). The portions with different purplish optical contrast gave $E_2g^1$ and $A_1g$ Raman bands centered at 385.3 and 404.7 cm−1, 384.3 and 406.3 cm−1, and 383.6 and 407.2 cm−1, respectively (Figure 1d). Therefore, they corresponded to 1-3L MoS2.17,18 The AFM image of the MoS2 sample can be found in Supporting Information, Figure S1. To fabricate MoS2/graphene heterostructures, the MoS2 was first transferred onto the graphene with the help of PMMA, and then the MoS2/graphene structure was relocated onto a 100 nm-thick Au coated SiO2/Si substrate, as shown by the optical microscopy photo in Figure 1(e). The corresponding AFM image of the heterostructure on Au is displayed in Figure 1(f).
We used EFM to measure the electric field screening properties of the MoS2/graphene heterostructures. EFM has been used to investigate the electric field screening in graphene, BN, and MoS2 nanosheets.19-21 However, the experimental setup in the current study was slightly different from those in these previous reports. We applied a DC voltage sweeping from +9 to −9 V to the Au substrate, which generated external electric fields of different intensities. The tip of the conductive cantilever which was grounded but oscillating at its first resonant frequency stayed at a few nanometers above the heterostructures, and acted as a sensor monitoring the electric field passing through the heterostructures (Figure 2a). Subtle electric field changes could be detected by EFM phase shift (Δϕ), which can be described as:21,22
% Δϕ=  (∂F⁄∂z)/k  ∙Q_cant      (1)
where ∂F/∂z is the local force gradient, representing the derivative of the electric force felt by the cantilever tip; k is the spring constant of the cantilever; and Qcant is the Q factor of the cantilever. For simplicity, the interaction between the cantilever tip and the sample in EFM is often viewed as an ideal capacitance. Therefore, the local force gradient due to the capacitive interaction becomes:21,22
% ∂F⁄∂z=  1/2   (∂^2 C)/(∂z^2 )  (V_s+V_CPD )^2      (2)
where C and z are the local capacitance and distance between the tip and the sample, respectively; Vs is the unscreened electric field which originated from the substrate voltage and passed through the samples, or in other words, the unscreened surface charge of the samples; VCPD is the contact potential difference due to the mismatch of the work functions between the tip and the sample. The raw data of the EFM spectroscopy from the Au substrate, 4L graphene, bulk MoS2, and three heterostructures, namely 1L MoS2/1L graphene, 2L MoS2/4L graphene, and 3L MoS2/4L graphene are shown in Figure 2b. The EFM phase of all the samples formed opening-up parabolas with the axis symmetry parallel with the y-axis as a function of the DC voltage. Considering the quadratic function in Eq. 2, we were not surprised that the parabolas were recorded. As VCPD was fixed for each sample, the formation of the parabolic curves was due to the sweeping Vs. In other words, Vs was a dominant parameter in our EFM measurements. The opening-up means attractive capacitive interactions under both positive and negative voltages.21 However, the different samples gave rise to slightly different shapes of the parabolas. This was caused by the other parameters, especially C and z. The distance z was inevitably slightly different from sample to sample during the EFM spectroscopy measurements. Although the local capacitance C is very difficult to define as it depends on many factors, including the shape, size, conductivity, and dielectric property of a sample, C was different from sample to sample. Therefore, it is understandable that the different samples resulted in the slightly different parabolic shapes in EFM spectroscopy. 
Intriguingly, we found that the parabolas of the EFM phase from some heterostructures lacked the mirror-symmetry as that of a perfect parabola. We give typical examples on symmetric and asymmetric EFM phase in Figure 2c-d, respectively. The EFM data from the Au substrate could be well fitted by the second-degree polynomial (grey vs. brown in Figure 2c), but the same fitting process was not able to reproduce the left part (negative bias) of the EFM phase curve from 3L MoS2/4L graphene (grey vs. blue in Figure 2d). To compare this phenomenon from the different samples, we normalized all the EFM data, as shown in Figure 2e. The enlarged view from the dashed area in Figure 2e is displayed in Figure 2f. Similar to the Au substrate, the 4L graphene, bulk MoS2, and 1L MoS2/1L graphene gave rise to symmetric parabolas of the EFM phase; in contrast, the heterostructures of few-layer MoS2 and graphene, i.e. 2L MoS2/4L graphene and 3L MoS2/4L graphene, showed deviated-parabolic EFM phase values under more negative substrate voltages. These EFM results were peculiar. As discussed previously, most of the parameters determining EFM phase, except Vs, should be constant during each measurement. The asymmetric parabolic EFM curves suggest that the electric field screening properties of 2L MoS2/4L graphene and 3L MoS2/4L graphene were probably not constant under different substrate voltages, i.e. external electric fields. In turn, the local capacitance C should change slightly accordingly. This phenomenon was not shown in Au film, 4L graphene, bulk MoS2, or 1L MoS2/1L graphene, but became prominent in the heterostructures with increased thickness of graphene and MoS2.

Multiscale Theoretical modeling
Quantum mechanical first-principle simulations
To better understand this intriguing phenomenon, we performed two levels of theoretical analysis using quantum mechanical ab initio calculations based on density functional theory (DFT); and at a classical level electrostatic approach using a capacitance model based on charge conservation equation solved variationally (see Methods for details). We first address the quantum mechanical part of the G/MoS2 heterostructures. We calculated the degree of polarization in the vdW heterostructures of graphene and MoS2 of different number of layers in response to the applied electric fields in terms of the electric susceptibility  . We used the quantum mechanical model presented in Refs.23,24 where a fully based ab initio approach was employed to extract information about the dielectric response at finite-electric fields and large supercells. No external parameters apart from the magnitude of the external electric fields were utilized in a self-consistent calculation. The simulations also took into account vdW dispersion forces, electrostatic interactions, and exchange-correlation potential within DFT at the same footing. 
  	Figure 3(a)-(c) show the variation of   with the electric bias at 1L, 2L and 3L MoS2, respectively, but with a distinct number of graphene layers. Strikingly, only negative bias affected   despite of the number of graphene and MoS2 sheets present in the heterostructures. This is in remarkable agreement with the experimental results where an asymmetrical response was recorded only from graphene and few-layer MoS2 heterostructures (Fig. 2(f)). The effect is enhanced, as more MoS2 layers are included into each graphene system. The largest increment is noticed on 3L MoS2/3L graphene, where a four-fold enhanced magnitude of   relative to zero field was calculated (Fig. 3(c)). We also observed that graphene layers have minor contributions to the effect, as  slightly varied as more graphene was putted together on top of MoS2 (Fig. 3(d)). At a fixed value of the electric field (-1.0 V/nm), larger asymmetric screening was displayed as the number of MoS2 layers increased: the screening in the heterostructures containing 3L MoS2 was almost doubled that of 1L and 2L MoS2 heterostructures. The slope of   versus the number of graphene layers also increased with the thickness of MoS2 (1L (0.008), 2L (0.03) and 3L (0.05)), which indicates that thicker MoS2 tends to be more correlated with variations in the number of graphene layers. This follows the behavior observed from EFM measurements, which heterostructures involving thicker MoS2 sheets in contact with graphene gave rise to a more asymmetric EFM phase parabola (Fig. 2(f)). Based on these results it becomes clear that the transition metal dichalcogenide layers play a key role on this screening effect. We will analyze in the following the modifications of the electronic structure of the heterostructures at finite electric fields, and elucidate the origin of this asymmetric susceptibility dependence on the external bias. 
Figure 4 shows that the behavior of   with the electric bias results from the asymmetrical polarization   associated to which side of the heterostructure the field interacts first. At negative bias, a larger amount of induced charge   was displaced towards the surface-layers of the MoS2 in the heterostructure, which consequently generates a polarization   that provided a better screening to the external electric fields relative to positive bias (Fig. 4(a)-(b)). As the number of MoS2 layer is small, little differences are noticed under the reversed electric field, as the dipole moment formed at the interface roughly compensated each other (Fig. 4(a)).  This effect is enlarged, as thicker MoS2 sheets are included. This was due to the amount of interfacial charge redistribution, which generated electric dipole moments preferentially aligned along one direction (Fig. 4(b)). Electric fields pointing towards graphene were not well screened as those towards MoS2 layers, because the induced polarization was not so efficient to generate response fields   that would shield the heterostructures completely. This means that higher magnitudes of electric field were observed inside thinner heterostructures, rather than thicker ones (Fig. 4(c)-(d)). We also observed that the induced electric potential   shows a smooth variation over the interface, and it is almost independent of the number of layers composing the junction.    displays high magnitudes over fields towards graphene layers, with a change in polarity at the MoS2 layer near the interface with   for thicker vdWHs (see Fig. 4(d)). This indicates that the interfacial-charge balance in both systems that generates   is sensitive to the amount of polarization charge from the MoS2 layers. That is, the thicker the MoS2 sheets, the larger the polarization. A consequence of this electric field direction-dependent polarization is noted in the different magnitudes of  observed in the vacuum region outside of 3L MoS2/3L graphene system for positive and negative fields (Fig. 4(d)).  In electrostatic boundary conditions, where the normal component of the displacement field   has to be preserved into the system25, it gives: 
where  corresponds to the field in the sheets and   to the induced polarization.  It is worth noting that   differs to   because the latter is calculated directly from the average induced charge using the Poisson equation and the former directly from the boundary conditions and the input field in the simulations (see Methods for details). For negative fields towards MoS2 layers, the second term on the right-hand side in Eq. (3) involving the polarization is appreciably large, which generates a depolarization or response field  that would overcome the applied external bias. This resulted in smaller electric fields inside the heterojunction (Fig. 4(d)). A similar effect is observed to fields directed to graphene layers, but higher in magnitudes inside the sheet due to smaller induced polarization. The polarization at the G/MoS2 heterostructure is therefore a contributing factor in the special screening field effect we measured by EFM. 
Based on the previous analysis, several implications on the electronic structure of the heterojunction can be foreseen. Figure 5 shows an asymmetric electronic response with the external electric field for several quantities. At negative bias, the induced dipole moments associated to the S-Mo-S bonds displace the charge towards the surface of the MoS2 sheet (Fig. 5(a)), which generated charge transfer from graphene to MoS2 at the interface (Fig. 5(b)). This charge rearrangement is smaller for positive fields because of the semiconducting nature of the MoS2 layer with less charge-carriers on its surface, and the semi-metallic character of graphene. This results in less polarizable field-dependent facet, smaller charge-transfer from MoS2 to graphene, and consequently better screening. The effect of the electric field can also be noted on the tuning of the Fermi level   relative to the charge-neutrality point when no doping concentration was considered on either graphene or MoS2 (Fig. 5(c)). Thin systems (e.g. 1L MoS2/1L G) tend to tune their Fermi level almost linearly with the electric field, which is a property intrinsically present at the pristine layers.26,27,28 As the number of MoS2 sheets increased,   displayed variations at positive fields as large as 0.34 eV for 3L MoS2/3L G but almost negligible for the negative electric field with the formation of a plateau at -0.2 V/nm and beyond. Such asymmetric behavior has been observed when MoS2 layers are used in metal-insulator-semiconductor junctions.29 Carrier doping induced by the electric field was responsible for the variation of the Fermi level or the work function of MoS2 mainly along one direction, which is directly related to the unbalance of charge density between both sides of the semimetal and the semiconductor interface. This indicates that the intrinsic character of the electronic structure of each system in vdW heterostructures contributes to the formation of the anisotropic screening observed. This effect has several main implications on the fundamental electronic structure of the G/MoS2 interfaces as can be appreciated on the band structure calculated at different magnitudes of gate bias for a sample system (e.g. 3L MoS2/1L G) in Figure 5(d)-(f). Similar trends are observed for different thicknesses of graphene and MoS2. At 0.0 V/nm, the Fermi level crossed the Dirac point of graphene, as no charge-imbalance was present between both systems. Several MoS2 states at the conduction band were observed at 247.86 meV relative to the Fermi level, which also corresponds to the Schottky barrier presents at the interface.30 At finite fields, those states are observed to shift up (down) with positive (negative) electric bias, with a consequent split as large as ~0.20 eV. This modifies their occupation, as some graphene states can become occupied (positive bias) or unoccupied (negative bias) according to the field polarization. The insets in Fig. 5(c) summarize the main effect of the bias on graphene and MoS2 states near the Fermi level.  This indicates that for electric fields toward the dichalcogenide layer, the states mainly composed of the conduction band of MoS2 with minor contribution from graphene were responsible for the charge-screening effect, and vice versa. This suggests the important role of the interface on the electrical properties of the vdW heterostructures, as the polarity of the electric field can select which states can screen the system against external bias. 
ClassicalClassical electrostatic approach approach: capacitor model
Apart from the mighty ab initio approach that gives the full picture of electronic states in the vdWHs, it is more desirable that such assymetric behavior of the vdWHs can be modeled inexpensively using several key electronic properties from the individual 2D material layers. we can extract several key properties that govern the asymmetric field response. Here we show that the asymmetric electronic screening of G/MoS¬2 vdW heterostructures under an external electric field can be well described using a classical electrostatic model, taking the quantum capacitance into consideration.31 As the 2D vdWHs are stacked via non-covalent interactions, it is found that the individual properties of 2D materials can still be largely preserved in their stacked layers, which are coupled by the Coulombic interactions.32 The idea behind our classical electrostatic model is that each individual layer has its own electronic “genome” (i.e. energy level, band gap, quantum capacitance) extracted from ab initio calculations, which can be used as building blocks in solving the electrostatic conservation equations of the whole vdWH, under the non-coupling assumption. Figure 6(a) schematically shows the band diagram of the G/MoS2 vdWH used in the classical model. Due to the fact that there is no electron drift in the vdWH at equilibrium (which is consistent with the EFM experimental setup and ab initio configurations), We assume that: (i) the Fermi level $E_Fermi$ aligns throughout the G/MoS2 vdWH. For simplicity of the model, we further assume that; (i)  (ii) the density of states (DOS) of individual layer is invariable with the stacking order and the external electric field and ; (iiiii) the interlayer distances ( ) are not affected by the external electric field. Note that although the transition of band structure is ignored in assumption (i), it has been shown that such classical treatment using Coulombic coupling has relatively high consistency with the ab initio simulations32. The charge and potential distribution in the vdWH is solved by several conservation equations in a self-consistent approach37: (i) the charge of individual 2D layer   follows the charge conservation of the vdWH, (ii)   is determined by the electric displacement field adjacent to the 2D layer, (iii)  determines the work function of layer i,   and (iv) the work function difference between two adjacent layers is determined by the electric displacement field. More details about the mean-field model can be found in Methods. By solving the conservation equations of electric field and charge density of the G/MoS2 vdWH, we can extract information about interfacial and screening properties (see details in Methods). For the simplest case of 1L G/ 1L MoS2, we plot in Figure 6(b) the work functions φ\_i of both materials, as a function of the external electric field strength (E\_ext) ranging from -85 V/nm to +85 V/nm. When Highly n-doped and p-doped MoS2 regimes can be found when   < -2.3 V/nm or   > 5.5 V/nm, respectively. The fermi level of MoS2 shifts close to its conduction band (CB) or valence band (VB) in both regimes respectively, which is accounted for the charge accumulation in the vdWH. Note that a noticeable charge accumulation (> 1012 e/cm2) occurs even when the Fermi level of MoS2 is still ca 0.1 eV away from the band edges, due to the fact that the low quantum capacitance of graphene near its intrinsic Fermi level. On the other hand, when    is between -2.3 V/nm and 5.5 V/nm, the Fermi level of MoS2 lies far away from the band edges, resulting that tand  , the Fermi level of MoS2 lies within the band gap, the vdWH is merely not polarized and the work function of graphene  shows little change with  . On the other hand, charge redistribution is greatly enhanced when the Fermi level shifts up to either the valence band (VB)  ( ), or the conduction band (CB) ( ) of MoS2. We find that the polarization of the G/MoS2 vdWH is more enhanced under negative external electric field, which corresponds with the findings in Figure 3(a). We ascribe such asymmetry to the difference between the electronic structures of graphene and MoS2: MoS2 is considered as a n-type semiconductor with its intrinsic Fermi level (-4.5 eV) closer to its CB (-4.0 eV) than its VB (-5.8 eV),33-35 while graphene has symmetric linear band structure around its Dirac point (-4.6 eV),36 all energy levels are compared with the vacuum level which is set at 0 eV.  Due to their close Fermi level values, little charge transfer occurs between graphene and MoS2 under weak electric field, and the degree of charge transfer is mainly determined by the position of the Fermi level with respect to the CB or VB of MoS2. Following the same procedure, we calculated the dipole moment  (Figure 6(c)), and the charge density of the graphene layers   (Figure 6(d)) for a different number of layers at the G/MoS2 interface. The results from the classical model show good consistency compared with the quantum mechanical ab initio calculations of dipole moment (Figure 5(a)) and charge transfer (Figure 5(b)): the charge redistribution is more pronounced with increased layer numbers of graphene and MoS2, and the MoS2 layers contributes more to such effect than graphene. Note that for thicker graphene layers (e.g. 3L G/nL MoS2), a considerable amount of total dipole moment can still be observed under weak electric field ( ), when the charge transfer between graphene and MoS2 is negligible ( ). This indicates that the electric field is well screened by multilayer graphene under such conditions, since the DOS of graphene is finite around the intrinsic Fermi level. The screening in the MoS2 becomes more important only when the Fermi level reach the band edges, that is, when the DOS increases greatly. To verify such statement, we reconstructed the band diagram of 1L G/ 3L MoS2 system under electric fields of -2 V/nm (Figure 6(e)) and +2 V/nm (Figure 6(f)). Under -2 V/nm electric field, the Fermi level reaches the CB out the outermost MoS2 layer, while under +2 V/nm electric field, the Fermi level remains within the band gap of MoS2 and has very little shift from the Dirac point of graphene, in good accordance with the ab initio calculations showed in Figure 5(d)-5(f). Charge accumulation occurs mostly on graphene and the outmost MoS2 layer, due to the larger DOS of both layers.
Unifying classical and quantum approaches
The classical electrostatic model shows sound agreement with the ab initio calculations in predicting the asymmetric screening behavior of the G/MoS2 vdWH, as a result of the different energy levels and DOS of both materials. Here we further propose that such asymmetry can be described by the quantum capacitance, within both theoretical frameworks. The vdWH can be considered as a capacitor, characterized by the quantum capacitance ( ), which is a function of the DOS at the Fermi level of the whole vdWH:    . The total DOS of the vdWH can be obtained quantum mechanically at high accuracy. On the other hand, the apparent quantum capacitance in the classical model can be calculated by the differentiating the charge density by the potential drop across the vacuum level   (equivalent to the total difference of work functions  ):  . We compare the quantum capacitances calculated by the classical model ( ) and DFT calculations ( ) as functions of   inof various G/MoS2 systems in Figures 7(a) and 7(b), respectively. Interesting, the quantum capacitances calculated by both methods show very similar behavior under external electric field, with the maximum quantum capacitance reaching ~30-33  in 3L G/ 3L MoS2 configuration, under an electric field of -1 V/nm. We find the layer dependency of quantum capacitance is very similar to that of the dipole moment and charge transfer: the layer number of MoS2 is dominating the magnitude of the total quantum capacitance under strong electric field. This is reasonable due to the higher quantum capacitance of MoS2 than graphene when Fermi level shifts to the band edges.37 Note that the DFT calculations predict a non-zero quantum capacitance of vdWHs with 2L and 3L graphene even without external electric field, as a result of the interlayer coupling, which is not included in the classical model. Despite the minute difference between both theoretical frameworks, it is clear that the quantum capacitance of the G/MoS2 vdWH, as a combination of the energy levels and DOS, can describe the asymmetric screening behavior with good precision. Our multiscale theoretical framework is thus readily applicable for a variety of vdWHs beyond graphene/MoS2, by utilizing the electronic “genome”, in particular the quantum capacitances of individual 2D layers. A full picture of electric screening of 2D vdWHs can be built benefited from the framework proposed in this work, extending the “tip of the iceberg” of the electrostatic nature of two-layer 2D vdWHs revealed by recent studies.29,38-40

Conclusions 
In summary, our findings reveal fundamental knowledge of the screening properties of van der Waals heterostructures using widely used two-dimensional materials, such as graphene and MoS2. Graphene/MoS2 constitutes an archetypal of vdW heterostructure with exciting possibilities for electronic devices based on atomically thin films.  We have shown an asymmetric electric field response on the screening properties of G/MoS2 stackings via high-resolution EFM spectroscopy and a multiscale theoretical analysis that involve quantum mechanical ab initio density functional theory including vdW dispersion forces, and a classical capacitor modelelectrostatic approach considering the quantum capacitance.  Our ab initio calculations are further unified in a quantum capacitance-based model, showing that the difference between the energy levels and band structures between graphene and MoS2 is account for the asymmetric screening behavior. After the transfer of MoS2 on graphene, the screening of either isolated graphene or MoS2 changes accordingly to the sign of the electric bias utilized becoming polarity-dependent. The EFM phase spectrum shows an asymmetry with the substrate voltage as the number of MoS2 layers increases relative to that of the graphene. Electric fields towards MoS2 tend to be better screened than those directed to graphene as an asymmetrical polarization associated to charge transfer at the graphene/MoS2 interface generate response fields that opposed to external bias. Such charge rearrangement also polarized the interface inducing the appearance of dipole moments and consequently giving a directional character to the underlying electronic structure. In particular, external fields in such vdW heterostructures can select which electronic states can be used to screen the gate bias, which clearly give an external control on the screening properties according to the stacking order and thickness. Our computational-experimental framework paves the way to understand and engineer the electronic and dielectric properties of a broad class of 2D materials assembled in heterojunctions for different technological applications, such as optoelectronics and plasmonics. 


 
Figure 1. Sample fabrication and optical characterization. (a) Optical microscopy photo of the mechanically exfoliated graphene of different thicknesses on 90 nm SiO2/Si; (b) the corresponding Raman spectra ; (c) optical microscopy photo of the as-exfoliated MoS2 on 270 nm SiO2/Si; (d) the corresponding Raman spectra; (e) optical microscopy photo of the MoS2 (top)/graphene (bottom) heterostructures transferred onto 100 nm-thick Au coated SiO2/Si by PMMA method; (f) the corresponding AFM image of the heterostructure.
 
Figure 2. EFM response of vdW heterostructures at different electric bias. (a) Drawing of the EFM spectroscopy setup with the Au substrate underneath graphene and MoS2 layers. Positive (negative) bias at the Au substrate generates electric fields towards graphene (MoS2), which are detected at the negative (positive) charged tip. (b) The raw EFM data from the six samples under a sweeping substrate DC voltage (+9 to −9 V); (c) The excellent fitting of the EFM phase spectrum from the Au substrate (brown) using second-degree polynomials (grey); (d) The asymmetry in the EFM phase spectrum from the 3L MoS2/4L G heterostructure demonstrated using the same fitting process; (e) Normalized EFM phase spectra from the six samples to show the asymmetric parabolas obtained from some of the heterostructures; (f) Enlarged view of the dashed area in (e).
 
Figure 3. Ab initio vdW first-principles calculations for G/MoS2 heterostructures. (a)-(c) Electric susceptibility   as a function of external electric fields (V/nm) at 1L, 2L and 3L MoS2, respectively (see insets). The number of graphene layers systematically increases in each panel at a fixed number of MoS2 sheets following the labeling shown in (a). The polarization of the field follows the orientation in Fig. 2a, where positive (negative) fields go towards graphene (MoS2) firstly.  (d)   as a function of the number of graphene layers at a static field of -1.0 V/nm. Different curves correspond to different number of MoS2 layers on the each vdW-heterostructures. Straight lines are fitting curves using a linear equation ( ) where the angular coefficients   are: 1L (0.008), 2L (0.035) and 3L (0.047).  
 
Figure 4. Asymmetric dipolar contributions at the G/MoS2 interface. (a)-(b) Induced charge density   (left y-axis) and electric polarization   (right y-axis) for 1L MoS2/1L Graphene and 3L MoS2/3L Graphene, respectively. The applied electric field is 1.0 V/nm. Blue (green) curves correspond to positive (negative) fields. Positive (negative) fields go towards graphene (MoS2), and vice-versa. The G/MoS2 interface is highlighted to show the unbalanced formation of electric dipole moments between graphene and MoS2 accordingly with the number of layer layers used to form the heterostructures. (c)-(d) Difference in electrostatic potential  in the slabs with and without the external electric field of 1.0 V/nm, and their corresponding response field  for 1L MoS2/1L Graphene and 3L MoS2/3L Graphene, respectively. The absolute values of   and   are taking in (c) and (d) for comparison at the same side of the plot. Geometries for all systems are highlighted at the background of each panel in opacity tone.  
 
Figure 5: Asymmetric electronic response under external electric fields. (a)-(c) Electric dipole moment (Debye), charge transfer QGraphene(electrons/cell), between graphene and MoS2, and Fermi level variation, EFermi(eV), as a function of electric field (V/nm), respectively. The different curves and colors correspond to the different number of graphene and MoS2 layers. Blue (1L MoS2), orange (2L MoS2) and green (3L MoS2). The same labeling in (a) for the number of graphene layers and colors for the MoS2 are used throughout the different plots. Positive (negative) fields point towards graphene (MoS2) layers as shown in the inset in (a). The calculated charge transfer QGraphene follows the trend displayed in the insets in (b).  That is, positive bias induces charge transfer from MoS2 to graphene, and vice-versa. The insets in (c) summarize the effect of the electric field on EFermi(eV) and the resulting electronic structure: positive (negative) bias shifts downward (upward) in energy the Fermi level. Also notice the relative shifts of graphene and MoS2 states with the electric bias. (d)-(f) Electronic band structures of the 3L MoS2/1L G heterostructure at different gate bias. Graphene states are highlighted in blue and MoS2 bands in faint pink. Fermi level is shown by the dashed-line in each panel. An asymmetrical dependence of the electronic properties with the electric field is noted in all calculated quantities. 

 
Figure 6: Classical model electrostatic approach for charge redistribution of G/MoS2 vdW heterostructures. (a) Schematic band diagram of the multilayer G/MoS2heterostructures. Vacuum level (Evac), work function ( i), surface charge density ( ), interlayer distance ( ),  and relative permittivity dielectric constant (ii) are shown for each stacking i. (b) Work functions of graphene and MoS2 as functions of external electric field Eext (V/nm). The regimes of the n-doped and p-doped MoS2 where the Fermi level reaches the conduction (CB) and valence (VB) bands of MoS2 are highlighted in green and faint red, respectively. The Fermi level reaches the conduction band (CB) or valence band (VB) of MoS2 when large negative or positive Eext is applied, relatively (as shown by the arrows). (c) Electric dipole moment  (Debye) as a function of Eext (V/nm) for different number of graphene and MoS2 layers on the heterostructures. (d) Surface charge density QGraphene (left axis) and the amount of charge transfer between graphene and MoS2 layersGr (right axis) of graphene as a function of Eext (V/nm) for different G/MoS2 systems. (e)-(f) Band alignments for an illustrative case, e.g. 1 L G/3L MoS2 under -2.0 V/nm and +2.0 V/nm electric fields, respectively. Positive charges are shown in faint red and negative charges are shown in blue, respectively. Only band structures near the Fermi level are shown for illustration. 
 
Figure 7: Quantum capacitance CQ for different G/MoS2 heterostructures. (a) Quantum capacitance calculated from the classical model    (F/cm2)  and (b) Quantum capacitance calculated from the DOS profile of ab initio  calculations   as a function of the external field Eext(V/nm)for various G/MoS2 vdWHs, respectively. The quantum capacitance increase more under negative electric field than positive field. Both classical and DFT calculations show similar trend of relationship between layer numbers and quantum capacitance: the layer numbers of MoS2 contributes more to the total quantum capacitance than graphene under strong electric field, consistent with the findings of layer-dependent dipole moment and charge transfer. 

Methods:
Experimental
The graphene and MoS2 nanosheets were mechanically exfoliated by Scotch tape. Highly oriented pyrolytic graphite (HOPG) (Momentive, US) and synthetic MoS2 crystals (2D Semiconductors, US) were used as received. Si wafers with 90 and 270 nm thick thermal SiO2 were used for graphene and MoS2, respectively. An Olympus optical microscopy (BX51) equipped with a DP71 camera was used to search atomically thin nanosheets, and a Cypher AFM (Asylum Research, US) was employed for topography measurements. The Raman spectra were collected by a Renishaw Raman microscope using 514.5 nm (for MoS2) and 633 nm (for graphene) lasers and an objective lens of 100× (a numerical aperture of 0.9). For the fabrication of the heterostructure, the identified MoS2 nanosheets were firstly coated by thin layer of PMMA, then peeled off from the SiO2/Si via etching by NaOH, stacked on graphene on SiO2/Si under the optical microscope. The MoS2/graphene structure was transferred to Au (100 nm) coated SiO2/Si substrate following a similar procedure. The Au coating was produced by a Leica ACE600 sputter with a crystal balance monitoring the coating thickness in real time. The EFM measurements were conducted on the Cypher AFM. The EFM phase data were collected by a Pt/Ti coated cantilever with a spring constant of ~2 N/m (ElectricLever, Asylum Research, US) dwelling above the heterostructure at a sampling rate of 2 kHz, while a voltage sweeping linearly from +9 to −9 V was applied on the Au substrate over a period of 20 s.
Ab initio quantum calculations
Calculations were based on ab initio density functional theory using the SIESTA41 and the VASP codes.42,43 Projected augmented wave method (PAW)44,45 for the latter, and norm-conserving (NC) Troullier-Martins pseudopotentials46 for the former, have been used in the description of the bonding environment for Mo, S and C. The shape of the numerical atomic orbitals (NAOs) was automatically determined by the algorithms described in41. The generalized gradient approximation47 along with the DRSLL48 functional was used in both methods, together with a double-zeta polarized basis set in Siesta, and a well-converged plane-wave cutoff of 500 eV in VASP. The cutoff radii of the different orbitals in SIESTA were obtained using an energy shift of 50 meV, which proved to be sufficiently accurate to describe the geometries and the energetics. Atoms were allowed to relax under the conjugate-gradient algorithm until the forces acting on the atoms were less than 1x10-8 eV/Å. The self-consistent field (SCF) convergence was also set to 1.0x10-8 eV. To model the system studied in the experiments, we created large supercells containing up to 394 atoms to simulate the interface between different number of graphene and MoS2 layers. We have optimized the supercell for the G/MoS2 interface using a 55 graphene cell on a 44 MoS2 cell, where the mismatch between different lattice constants is smaller than ~2.0% (i.e. the systems are commensurate). We have kept the lattice constant of the MoS2 at equilibrium, and stretched the one for graphene by that amount. Negligible variations of the graphene electronic properties are observed with the preservation of the Dirac cone for all systems. To avoid any interactions between supercells in the non-periodic direction, a 20 Å vacuum space was used in all calculations. In addition to this, a cutoff energy of 120 Ry was used to resolve the real-space grid used to calculate the Hartree and exchange correlation contribution to the total energy in SIESTA. The Brillouin zone was sampled with a 9×9×1 grid under the Monkhorst-Pack scheme49 to perform relaxations with and without van der Waals interactions. Energetics and electronic band structure were calculated using a converged 44×44×1 k-sampling for the unit cell of Graphene/MoS2. In addition to this we used a Fermi-Dirac distribution with an electronic temperature of kBT = 20 meV to resolve the electronic structure. 
The electric field   across the vdW heterostructures is simulated using a spatially periodic sawtooh-like potential   perpendicular to the G/MoS2 heterostructures. Such potential is convenient to analyze the response of finite systems (e.g. slabs) to electric fields22,50-52, while problematic for extended systems (e.g. bulk).  The magnitudes of the spatially varying electrostatic potential and charge density   across the supercell are determined via a convolution with a filter function to eliminate undesired oscillations and conserve the main features important in the analysis. The variations of both quantities,  and  are determined relative to zero field. The polarization   is calculated by the integration of   through  .   is defined through   where   , with   the thickness of the vdW heterostructure, and    the height of the supercell23,24.  
ClassicalMean-field Model
In the classicalmean-field model, the graphene and MoS2 layers are treated as individual layers, with the the band structures (band gap, DOS and intrinsic work functions) considered as invariable to external electric field (i.e. the effect interlayer coupling on band structure is neglected). We further consider that the interlayer distance  between the i-1 and i-th layers is fixed, and taken as the interlayer distance in DFT calculations under zero field. The interlayer dielectric constant  between the i-1 and i-th layer is considered as uniform. For simplicity, we consider that  is independent of the external field, while the current model can be easily adapted for field-dependent dielectric constant in multilayer 2D materials using ab initio calculation results23,24. We consider the interlayer electric field   to be uniform. The charge density   and work function   of each layer can thus be solved through the following conservation equations in a self-consistent way:
	The charge neutrality of the vdWH:
	Charge balance of the i-th layer by Gauss law:
	 as a function of the work function  of i-th layer:
Where   is the Fermi-Dirac distribution function, and   is the intrinsic work function of the i-th layer. 
	The potential drop between 2 adjacent layers:

Note that for individual layers,   . We simplify the quantum capacitance of graphene using a linear model:  , while the quantum capacitance of MoS2 is a step function where no density of states exist within the band gap, while  and   for VB and CB, respectively. The intrinsic work function of graphene and MoS2 are set at 4.6 V and 4.5 V, respectively. The energy levels of CB and VB of MoS2 are taken as -4.0 eV and -5.8 eV, respectively.






%%% Local Variables:
%%% mode: latex
%%% TeX-master: "../thesis"
%%% End:
