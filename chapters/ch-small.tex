% \chapter{Introduction}
\chapter[Elastic Interfacial Effect Field Transistor]{An Elastic Interfacial Effect Field Transistor Based on Multiscale Interfacial Phenomena}
\label{ch:small}
\renewcommand*\imgdir{img/small/}

% \dictum[]{%
  % }%
% \worktodo{find the quote or quit}

\vspace{1em}

\chapterabstract{Part of this chapter appears in the following journal
  article: Tian, T., Sharma, C., Navanshu, A., Varga, M., Lee, Y.-T.,
  Chiu, Y.-C. \& Shih, C.-J. An Elastic Interfacial Transistor Enabled
  by Superhydrophobicity. Small 14, 1804006 (2018).  }


\section{Introduction}
\label{sec:small-introduction}

In this chapter, we demonstrate an application combining our
understandings of 2D material interfaces in previous chapters.  The
proposed 2D material-based platform, which we refer to as the
interfacial field effect transistor (IFET), makes use of various
multiscale physical phenomena discussed previous, including the
molecular epitaxy, field-effect transparency, wetting phenomena and
carrier transport.

Having been the central component of the semiconductor industry for
more than 70 years, the field-effect transistor (FET) still attracta
substantial interest and development.
%
Recent advances in semiconductor materials and physics have driven the
development of field-effect transistors (FETs) for a variety of
applications \cite{Torsi_2013_rev,Ben_Sasson_2014_fet_rev}.
%
One
long-existing challenge is to incorporate mechanical responsiveness
and durability into FETs, which potentially offers new technological
opportunities beyond the reach of existing platforms
\cite{Someya_2004,Oh_2016_stretch_polym_FET,Shin_2017_air_FET}. 
%
It is
desirable to intrinsically incorporate the responsiveness in
semiconductor by rational materials design
\cite{Oh_2016_stretch_polym_FET} but often compromises the device
performance and processability \cite{Lee_2018}. For example, the
elasticity, the ability of a soft material to return to its original
shape when the deforming stress is released, requires an entropic
force acting within the material, fundamentally impeding carrier
transport \cite{OConnor_2011_strain_P3HT}.
%
As a result, there are only few reports that have successfully
demonstrated stress-sensing FETs using intrinsically flexible
semiconductors \cite{Oh_2016_stretch_polym_FET}. On the other hand,
incorporating transistor components into a mechanically flexible
matrix also offers opportunities for stress-sensing FETs, including
strategies using conductive elastic polymer as the pressure-sensitive
unit
\cite{Someya_2004,Sekitani_2009,Kaltenbrunner_2013_elastic_device,Takei_2010_NW_skin}.
However sensitivity of such approach is limited by the properties of
the polymer-based components, which requires more micro- to nano-scale
engineering to enhances the mechanical durability
\cite{Jang_2015_soft_network}, the flexibility and responsiveness of
flexible rubber-based FETs
\cite{Mannsfeld_2010_pressure,Schwartz_2013_polymer_transistor}.
Recently, highly sensitive FETs using air as the compressible
dielectric layer has as also been proposed
\cite{Zang_2015_suspend_gate_FET,Shin_2017_air_FET}. The elasticity of
the polymer-based FETs essentially depends on the elastic modulus of
the flexible material, which is typically in the range of
\(10^{1}\sim{}10^{3}\) kPa \cite{Amjadi_2016_stretchable_sensor}.
%
While further reducing the elastic modulus to kPa level is possible
via soft materials like hydrogels, these materials suffer from
 poor mechanical stability which hinders their applications.
 %

 In this chapter, we introduce an alternative force-sensing concept by
 controlling the wettability of a semiconductor surface, referring to
 the interfacial field-effect transistor (IFET).  We design an IFET
 made by superhydrophobic semiconductor nanowires (NWs) sandwiched
 between a layer of two-dimensional electron gas (2DEG) and a
 conductive Cassie-Baxter (CB) sessile droplet.  The mechanical
 responsiveness of IFET is harnessed from a the deformation of the
 conductive droplet on superhydrophobic semiconducting nanowires. The
 extremely low elastic modulus of the droplet is calculated to be
 $\sim{}8\times{}10^{2}$ Pa, much lower than any existing rubbers,
 ensures an excellent elastic sensiting limit down to \textless{}10 Pa
 and a stress sensitivity of 36 kPa$^{-1}$.
 %
 The partial penetration
 of electric field through the 2DEG modulates the carrier profile at
 the NW/2DEG interface and a thermal-tunable current on/off ratio
 exceeding $3\times{}10^{4}$.
 %
  This study demonstrates a
  versatile platform that bridges multiple macroscopic interfacial
  phenomena with nanoelectronic responses.


\section{Results and Discussions}
\label{sec:small-results-discussions}


\subsection{The Interfacial Field Effect Transistor (IFET)}
\label{sec:small-interf-field-effect}

 \autoref{fig:small-main-1} presents the device architecture of the proposed
IFET. We consider a quantum capacitor device \cite{Luryi_1988_Quantum}, in which
a semiconductor layer is sandwiched between a conductive drain (D) and
a 2DEG source (S) electrodes.
%
\begin{figure}[htbp]
  \centering
  \import{\imgdir}{scheme_IFET.pdf_tex}
% \includegraphics[width=0.95\linewidth]{img/scheme-1.pdf}
  \caption{\label{fig:small-main-1} Design concept of the proposed
    elastic interfacial transistor. \textbf{a}, Schematic showing the device
    architecture at the micrometer (top) and nanometer (bottom)
    scales. Superhydrophobic semiconductor nanowires (NWs) are
    sandwiched between a conductive droplet at the CB state and a
    layer of 2DEG that allows partial penetration of an electrostatic
    field. \textbf{b}, Upon applying an compressive stress \(\sigma\) to the
    droplet, a vertical displacement \(\Delta H\) advances the contact
    line by \(\Delta x\), with an apparent advancing contact angle
    \(\theta_{\mathrm{adv}}^{*}\). When the stress is released, the
    contact line moves back towards its equilibrium shape, with an
    apparent receding contact angel of
    \(\theta_{\mathrm{rec}}^{*}\). At the nanometer scale, may also
    result in an increase of penetration length \(\Delta l\) that
    effectively reduces the transistor channel length.}
\end{figure}
%
%
As repeatedly mentioned in previous chapters, the ultrathin and
low-quantum-capacitance (low \(C_{\mathrm{Q}}\)) nature of 2DEG allows
partial penetration of an electrostatic field exerted by the
underlying gate (G) electrode~\cite{Shih_2015_PartiallyScreened},
modulating the carrier density in semiconductor that controls
drain-to-source current (\(I_{\mathrm{DS}}\)) (
\autoref{fig:small-main-1}a).
%
In addition, the fact that \(I_{\mathrm{DS}}\) passes through two
semiconductor-electrode interfaces maximizes the response to the
interfacial properties as compared to a FET device with horizontal
architecture~\cite{Ben_Sasson_2014_fet_rev}. The IFET, as its name
indicates, is a collective product of multiscale physical phenomena
that are schematically shown in \autoref{fig:small-toc}.
\begin{figure}[!htbp]
  \centering
  \import{\imgdir}{TOC.pdf_tex}
% \includegraphics[width=0.95\linewidth]{img/scheme-1.pdf}
  \caption{\label{fig:small-toc} Multiscale phenomena in the
    interfacial transistor. From left to right: macroscopic wetting at
    the NW/liquid interface; nanoscale molecule orientaion at the
    molecule/2DEG interface; atomistic scale energy level alignment at
    the semiconductor/2DEG interface.  }
\end{figure}


The operational principles of the IFET requires the top surface to
have a low surface energy, in order to allow free mechanical
deformation of a sessile droplet.
%
This can be achieved by high-aspect-ratio structures such as nanowires (NWs)~\cite{Yang_2010_rev_NW} directly on the 2DEG,
forming a superhydrophobic surface at the Cassie-Baxter (CB) state
\cite{Cassie_1944_wet}.
%
A conductive sessile droplet is then placed on top
of the NWs, acting as the D electrode, with the radius of contact,
\(R\).
%
Accordingly, applying a compressive stress \(\sigma\) on top
of the sessile leads to a horizontal displacement \(\Delta x\) that
advances the contact line, with an apparent advancing contact angle
\(\theta_{\mathrm{adv}}^{*}\) (\autoref{fig:small-main-1}b left).
%
When the
stress is released, the contact line moves back towards its
equilibrium shape, with an apparent receding contact angel of
\(\theta_{\mathrm{rec}}^{*}\) ( \autoref{fig:small-main-1}b right), enabling the
mechanical reversibility.
%
We hypothesize that the electrical response during the
process results from: (i) an increase of the droplet contact area
\(\Delta A \approx 2 \pi R \Delta x\) at the macroscopic level and (ii)
an increase of penetration length \(\Delta l \approx \sigma
r^{2}/2\gamma_{\mathrm{L}}\) that effectively reduces the semiconductor
channel length in the NWs (see  \autoref{fig:small-main-1}b bottom), where
\(\gamma_{\mathrm{L}}\) is the surface tension of liquid and \(r\) is the
average radius between NWs.
%
In order to realize the concept of a conductive 2DEG surface as well
as growth template of semiconducting nanowires, monolayer graphene (Gr) is used.
%
The choice is based on the following considerations: (i) The template
effect (see \autoref{sec:intro-mol-2D}) of graphene is know to
facilitate of several organic and inorganic
NWs~\cite{Huang_2016_laury_nanowire_gr,Wang_2015_vertical_nanowire_gr,Fu_2012_gr_ZnONW},
and (ii) As a 2DEG material, graphene has the smallest quantum
capacitance $C_{\mathrm{Q}}$ near its Dirac point, allowing effective
modulation of interfacial charge states due to the field-effect
transparency (see \autoref{sec:qc-gener-behav-2deg}).
%
As for the semiconductor material, it is well-known
that the covalently bonded semiconductors generally possess a high
surface energy, owing to the high surface polarity
\cite{Azimi_2013_wetting_RO} and dangling bonds
\cite{Zhang_2004_dangling}, presumably leading to the Wenzel state.
%
To
this end, the organic semiconductor solids containing the
\(\pi\)-conjugated molecules, with the molecular polarizability
considerably reduced by perfluoration, were considered.
%
We have managed to grow NWs of Copper(II) hexadecafluorophthalocyanine
(F\(_{\text{16}}\)CuPc, structure see \autoref{fig:intro-formula}), a
n-type organic semiconductor with high air stability
\cite{Bao_1998_FCuPC}, on a sheet of monolayer graphene transferred
onto a SiO\(_{\text{2}}\)/Si substrate (details see
\autoref{sec:small-method}).

\begin{figure}[htbp]
\centering
\import{\imgdir}{cupc-orien.pdf_tex}
\caption{\label{fig:small-cupc-orien} Molecular orientation-induced
  superhydrophobicity in F$_{16}$CuPc NWs.  \textbf{a}. Top-view
  (left) and cross-sectional (right) SEM micrographs showing the
  morphologies of F\(_{\text{16}}\)CuPc/SiO\(_{\text{2}}\) (green) and
  F\(_{\text{16}}\)CuPc NWs (cyan). Scale bar: 200 nm. \textbf{b},
  GIXD patterns and molecular orientation schematics of
  F\(_{\text{16}}\)CuPc/SiO\(_{\text{2}}\) (left) and
  F\(_{\text{16}}\)CuPc NWs (right), showing dominant edge-on (while
  circles) and face-on (black circles) orientations, respectively. The
  four representative planes (\(\alpha\) to \(\delta\)) are labeled
  accordingly. \textbf{c}. (left) Orientation-induced Raman responses
  changes between F$_{16}$CuPc/SiO$_{2}$ and F$_{16}$CuPc NWs samples.
  SEM micrograph (top right) and Raman mapping image (bottom right) of
  F\(_{\text{16}}\)CuPc molecules deposited onto a graphene sheet with
  a crack, clearly identifying the NW region. Scale bar: 2 \(\mu\)m. }
\end{figure}

The F\(_{\text{16}}\)CuPc NWs have high aspect ratios, with diameters
of 30--60 nm and while height of several hundred nm (
\autoref{fig:small-cupc-orien}a).
%
We notice that the NW morphology is similar to the
F\(_{\text{16}}\)CuPc films templated by PTCDA
\cite{Yang_2009_F16_PTCDA} or gold nanoparticles
\cite{Mbenkum_2006_F16_1D}, presumably sharing the same growth
mechanism.
%
Indeed, the strong \(\pi-\pi\) interactions between
graphene and F\(_{\text{16}}\)CuPc basal planes lower the critical
free energy of heterogeneous nucleation; in combination with the weak
van der Waals (vdW) interactions between the perfluorinated molecular
edges that limit the lateral growth, it explains why the NW structure
is thermo\-dynamically favored.
%
The structural driving force resulting from the subtle asymmetry
between in-plane and out-of-plane interactions, however, disappears
when the molecule-substrate interactions are not sufficiently strong
(see \autoref{sec:intro-mol-subst}), as observed in the thin-film
structure grown on bare SiO\(_{\text{2}}\) (
\autoref{fig:small-cupc-orien}a).
%
Using frequency change  the quartz crystal
micro\-balance (QCM), we are able to compare the amount of
F$_{\mathrm{16}}$CuPc molecules deposited onto both
films. Interesting, we observe the nominal film thickness of the
F\(_{\text{16}}\)CuPc NW is roughly 4.2 times compared with that of
the F\(_{\text{16}}\)CuPc/SiO\(_{\text{2}}\) regardless of the amount
of molecules deposited. This indicates the growth of NWs is consistent
for both thin and thick film.

The origin of the distinct morphology observed between the two
samples, is further analyzed by grazing incidence X-ray diffraction
(GIXD), as shown in \autoref{fig:small-cupc-orien}b.
% The patterns for
% the F\(_{\text{16}}\)CuPc NW and
% F\(_{\text{16}}\)CuPc/SiO\(_{\text{2}}\) samples.
The F\(_{\text{16}}\)CuPc/SiO\(_{\text{2}}\) sample has an intense
peak along the \(q_{\mathrm{z}}\) axis, labeled as \(\alpha\)
(\emph{d}-spacing = 1.470 nm). It corresponds to the (002) plane in
the F\(_{\text{16}}\)CuPc single crystal \cite{Yang_2009_F16_PTCDA}
that is perpendicular to the substrate normal vector, suggesting a
dominant ``edge-on'' orientation ( \autoref{fig:small-cupc-orien}b
bottom left). It is also endorsed by the position of the
(\(\bar{1}\)22) plane peak near the \(q_{\mathrm{xy}}\) axis
(\(\beta\)) \cite{Yoon_2010_crystal_F16cuPc}, with a \emph{d}-spacing
of 0.305 nm. On the other hand, for the F\(_{\text{16}}\)CuPc NW
sample, we observe an additional peak corresponding to the (002) plane
(labeled as \(\gamma\)) ( \autoref{fig:small-cupc-orien}b right), with
its wave vector rotated by \(\sim 76^{\circ}\) with respect to the
\(q_{\mathrm{z}}\) axis and a \emph{d}-spacing of 1.440 nm, slightly
smaller than that in the edge-on orientation. Together with the
(\(\bar{1}\)22) peak located at the \(q_{\mathrm{z}}\) axis, we point
out the emergence of the ``face-on'' orientation (
\autoref{fig:small-cupc-orien}b bottom right) in the NWs.


The substrate-induced orientation transition also results in the
angle-dependent Raman scattering of the Davydov multiplets that alters
the Raman characteristics \cite{Cerdeira_2013_RamanF16}, as shown in
different heights of Raman peaks at 1380 cm\(^{\text{-1}}\) to that at
1315 cm\(^{\text{-1}}\) in \autoref{fig:small-cupc-orien}c (left
panel).  The right panels \autoref{fig:small-cupc-orien}c present the scanning
electron microscopy (SEM) micrograph and Raman map giving the
intensity ratio of the peak at 1380 cm\(^{\text{-1}}\) to that at 1315
cm\(^{\text{-1}}\), I(1380)/I(1315), on the F\(_{\text{16}}\)CuPc film
deposited on a sheet of SiO\(_{\text{2}}\)-supported graphene with a
break. The NW region, with a higher I(1380)/I(1315) ratio, is clearly
identified.


\subsection{Superhydrophobicity on Semiconducting Nanowires}
\label{sec:small-superhydr-nw}


With the successful establishment of semiconducting NWs on graphene
surface, we examined the wettability of the two F\(_{\text{16}}\)CuPc
samples, as shown in \autoref{fig:small-water-wet}).
%
\begin{figure}[htbp]
  \centering
  \import{\imgdir}{NW-water-wet.pdf_tex}
% \includegraphics[width=0.9\linewidth]{img/scheme-2.pdf}
  \caption{\label{fig:small-water-wet} Molecular
    orientation-induced superhydrophobicity in NWs, with mages of
    water sessile droplets sitting on
    F\(_{\text{16}}\)CuPc/SiO\(_{\text{2}}\) (\textbf{a}) and
    F\(_{\text{16}}\)CuPc NWs (\textbf{c}) taken from the ESEM and CAG
    (inset) techniques, respectively.
     The CAG-determined water
    \(\theta_{\mathrm{adv}}^{*}\), \(\theta_{\mathrm{rec}}^{*}\), and
    \(\theta_{\mathrm{s}}^{*}\) values (dots) are also shown with respect to the film
    thickness (NW height) for the
    F\(_{\text{16}}\)CuPc/SiO\(_{\text{2}}\) (\textbf{b}) and
    F\(_{\text{16}}\)CuPc NW (\textbf{d}) samples, which are nicely
    described by the Wenzel and CB models (solid curves),
    respectively.}
\end{figure}
%
The apparent
static contact angles (\(\theta_{\mathrm{s}}^{*}\)) of water were
independently determined by (i) a contact angle goniometer (CAG) and
(ii) condensation of water vapor in an environmental scanning electron
microscope (ESEM) setup. For example, \autoref{fig:small-water-wet}a and c
present the CAG/ESEM-measured \(\theta_{\mathrm{s}}^{*}\) of a 50
nm-thick F\(_{\text{16}}\)CuPc/SiO\(_{\text{2}}\) and a 300 nm-thick
F\(_{\text{16}}\)CuPc NW samples, showing 103.0\(\pm\)5.4\(^{\circ}\)
/ 107.0\(\pm\)3.9\(^{\circ}\) and 152.8\(\pm\)3.9\(^{\circ}\) /
149.8\(\pm\)3.4\(^{\circ}\), respectively. To gain more insights into
the superhydrophobic states of the two surfaces, we conducted a series
of static and dynamic contact angle measurements which allow us to
determine \(\theta_{\mathrm{adv}}^{*}\), \(\theta_{\mathrm{rec}}^{*}\)
and \(\theta_{\mathrm{s}}^{*}\), with respect to the film thickness (
\autoref{fig:small-water-wet}b and d). The apparent contact angle values
\(\theta_{\mathrm{i}}^{*}\), where i=adv, rec, and s, were numerically
fitted by the Wenzel (W) \cite{Wenzel_1936_wetting} and CB
\cite{Cassie_1944_wet} models, given by:

\begin{eqnarray}
\label{eq:small-2}
&\cos \theta^{*}_{\mathrm{i,W}} =& r_{\mathrm{W}} \cos \theta_{\mathrm{i}} \\
&\cos \theta^{*}_{\mathrm{i,CB}} =& r_{\mathrm{CB}} f \cos \theta_{\mathrm{i}} + f - 1
\end{eqnarray}
where \(r_{\mathrm{W}}\) and \(r_{\mathrm{CB}}\) are the roughness
ratio of the wet surface area in the two (W and CB) states,
\(\theta_{\mathrm{i}}\) is the respective contact angles (advancing,
receding, or static) on an ideal surface and \(f\) is the fraction of
solid surface area wet by water in the CB
model~\cite{Yeh_2008_CBW_hys,McHale_2004,Joanny_1984,Patankar_2010_CBW_hys}.
%
By assuming \(r_{\mathrm{W}} \approx r_{\mathrm{CB}}\) that linearly
increases with film thickness owing to the intrinsically identical
chemical nature of the two surfaces~\cite{Yeh_2008_CBW_hys}, we
find that the water wettability on
F\(_{\text{16}}\)CuPc/SiO\(_{\text{2}}\) and F\(_{\text{16}}\)CuPc NW
samples can be nicely described by the Wenzel and CB models,
respectively (solid lines in \autoref{fig:small-water-wet}b and d), allowing us
to determine \(\theta_{\mathrm{adv}}\), \(\theta_{\mathrm{rec}}\),
\(\theta_{\mathrm{s}}\), and \emph{f}, as shown
in \autoref{tbl:small-fitted}.
%
\begin{table}[!htbp]
\caption{\label{tbl:small-fitted}
Fitted parameters for the dynamic contact angle model described}
\centering
\begin{tabular}{llllr}
\hline
Quantity & \(\theta_{\mathrm{adv}}\) & \(\theta_{\mathrm{rec}}\) & \(\theta_{\mathrm{s}}\) & \(f_{\infty}\) \\
\hline
Fitted value & 107\(^{\circ}\) & 78\(^{\circ}\) & 94\(^{\circ}\) & 0.093 \\
\hline
\end{tabular}
\end{table}
%
Although some discrepancy
can be seen between the experimental and model data, the simple CB
and Wenzel models still nicely capture the distinct wetting behaviors
on the two F\(_{\text{16}}\)CuPc morphologies. We point out that the
graphene-induced molecular orientation effects have eventually led to
the transition between Wenzel and CB states, which to our knowledge
has never been demonstrated in molecular solids.


The superhydrophobicity is further applicable to conductive sessile droplets,
specifically the room-temperature liquid metals\cite{Dickey_2008_EGAIN} (LMs) on the
F\(_{\text{16}}\)CuPc samples.
%
It is noteworthy that the key challenge hindering
the development of LM-based devices is the undesirable adhesion of LM
to the device surface, in particular with the eutectic gallium indium
(EGaIn), in which the formation of gallium oxide layer effectively
reduces the interfacial tension
\cite{Dickey_2008_EGAIN,Doudrick_2014_oxide}.
%
\begin{figure}[htbp]
  \centering
  \import{\imgdir}{LM-wetting.pdf_tex}
% \includegraphics[width=0.9\linewidth]{img/scheme-2.pdf}
  \caption{\label{fig:small-lm-wet} \worktodo{more} Wetting of liquid
    metal (LM) droplets on superhydrophobicity NWs.
    \textbf{a}. Mercury sessile droplets sitting on
    F\(_{\text{16}}\)CuPc/SiO\(_{\text{2}}\) (top) and
    F\(_{\text{16}}\)CuPc NWs (bottom), showing
    \(\theta_{\mathrm{s}}^{*}\) values of 147.7\(\pm\)2.5\(^{\circ}\)
    and 157.1\(\pm\)3.8\(^{\circ}\), respectively. Scale bar: 500
    \(\mu\)m. \textbf{b}. Touching (left) - detaching (right) of a
    suspended EGaIn droplet from
    F\(_{\text{16}}\)CuPc/SiO\(_{\text{2}}\) (top) and
    F\(_{\text{16}}\)CuPc NWs (bottom) surfaces, showing non-stick
    characteristics of EGaIn to the NWs.}
\end{figure}
  %
Mercury sessile droplets sitting on the
F\(_{\text{16}}\)CuPc/SiO\(_{\text{2}}\) and F\(_{\text{16}}\)CuPc NW
surfaces show \(\theta_{\mathrm{s}}^{*}\) values of
147.7\(\pm\)2.5\(^{\circ}\) and 157.1\(\pm\)3.8\(^{\circ}\),
respectively( \autoref{fig:small-lm-wet}a).  Regarding the EGaIn
sessile droplets, although the \(\theta_{\mathrm{s}}^{*}\) values of
fresh droplets are close on both surfaces, interestingly, the adhesive
property is significantly different, as revealed in
\autoref{fig:small-lm-wet}b. We suspended an EGaIn droplet using a
microcapillary, followed by repeatedly touching/removing the droplet
from the two surfaces. We found that the EGaIn droplet can be easily
detached from the F\(_{\text{16}}\)CuPc NW surface (
\autoref{fig:small-lm-wet}b right), compared to that stuck to the
F\(_{\text{16}}\)CuPc/SiO\(_{\text{2}}\) surface
(\autoref{fig:small-lm-wet}b left). The observation was further
corroborated by their apparent sliding angles
\(\theta_{\mathrm{sl}}^{*}\), equivalent to the degree of contact
angle hysteresis, of 56.1\(\pm\)13.2\(^{\circ}\) and
14.6\(\pm\)2.6\(^{\circ}\) on F\(_{\text{16}}\)CuPc/SiO\(_{\text{2}}\)
and F\(_{\text{16}}\)CuPc NW samples, respectively. Accordingly, the
excellent CB characteristics of LM sessile droplets on the
semiconducting NWs form a solid basis for the proposed elastic IFET.


\subsection{Performance of Field Effect Transistor }
\label{sec:small-field-effect-trans}

Next, we build a prototype IFET benefited from the superhydrophobic
F$_{16}$CuPc NWs formed on graphene.
%
\begin{figure}[!htbp]
  \centering
  \scalebox{0.85}{\import{\imgdir}{VFET.pdf_tex}}
% \includegraphics[width=0.95\linewidth]{img/scheme-3.pdf}
  \caption{\label{fig:small-main-3} Transport characteristics of the
    fabricated IFETs at zero strain. \textbf{a}. Schematic of the
    device architecture, together with an optical micrograph taken by
    CAG. Scale bar: 500 \(\mu\)m. \textbf{b}. A representative
    transfer curve at \(V_{\mathrm{D}}\) = 1 V, showing an on/off
    current ratio of 3.9\(\times{}\)10\(^{\text{4}}\). The inset shows
    the histogram of the on/off current ratio extracted from
    \textgreater{}100 IFET devices. \textbf{c}. Representative
    transfer curves at different drain voltages. (d). The schematic
    (left), photographs (middle), and output current
    \(I_{\mathrm{tot}}\) (right) of an IFET-based circuit that
    controls the light intensity of a commercial LED with
    \(V_{\mathrm{G}}\).}
\end{figure}
\autoref{fig:small-main-3}a presents the
schematic of the proposed elastic IFET.
%
A sheet of monolayer graphene grown by chemical vapor deposition (CVD)
was transferred onto a 300 nm SiO\(_{\text{2}}\)/Si substrate
functionalized by octadecyltrichrolosilane self-assembled monolayer
(OTS SAM) \cite{Yan_2011}, in order to minimize the substrate-induced
traps \cite{Wang_2011_quanti_doping_gr}.  The lateral graphene FET has
an on/off current ratio of \(\sim\)10 and a field-effect mobility
(\(\mu_{\mathrm{FE}}\)) of \(\sim\)2000
cm\(^{\text{2}}\)V\(^{\text{-1}}\)s\(^{\text{-1}}\) at room
temperature, with the charge neutrality point (CNP) at the gate
voltage of $<10$ V. Subsequently, a layer of \(\sim\)300 nm-high
F\(_{\text{16}}\)CuPc NWs was deposited on graphene, followed by
placing a cantilever-attached LM droplet on top as the D electrode
(details see \autoref{sec:small-method}).
%
The current density \(J_{\mathrm{DS}}\) from the LM (D) to graphene
(S) in the IFET was then quantified by normalizing the drain current
by the contact area of the sessile droplet with the NW layer, such
that
\(J_{\mathrm{DS}} = {\displaystyle \frac{I_{\mathrm{DS}}}{\pi
    R^{2}}}\), where \(R\) is determined from the CAG optical
micrograph of the LM droplet ( \autoref{fig:small-main-3}a inset), as
a function of drain and gate voltages (\(V_{\mathrm{D}}\) and
\(V_{\mathrm{G}}\)). Note that since the NWs are only partially
wetted, the actual current density passing through individual NWs is
estimated to be \(J_{\mathrm{DS}}/f\). The carrier density in the
F\(_{\text{16}}\)CuPc molecules adjacent to the NW/Gr interface is
modulated by the partially penetrated field effect through monolayer
graphene (see explanations in ~\autoref{ch:qc}), which in turn alters
the Schottky barrier height at the interface
\cite{Yang_2012_Barristor}.
%
The face-on orientation of the F\(_{\text{16}}\)CuPc molecules also
facilitates interfacial carrier transport in two ways. The overlap
between \(\pi\) orbitals of F\(_{\text{16}}\)CuPc molecules along the
face-on stacking direction ensures high carrier mobility in the NWs
\cite{Bao_1998_FCuPC}.
%
Moreover, as discussed in \autoref{sec:org2cdd8f0}, the molecular
orientation also affects the interfacial carrier transport: the work
function of face-on F\(_{\text{16}}\)CuPc molecules aligns with
graphene, compared with large work function mismatch between edge-on
F\(_{\text{16}}\)CuPc and graphene
\cite{Mao_2010_F16_level_orien,Ren_2011_DFT_CuPc_epi_gr}.
%
At zero strain of the LM droplet,  corresponding to the droplet height at \(H_{0}\),
the transfer curve for a representative IFET at \(V_{\mathrm{D}}\)= 1
V demonstrates an on/off current ratio of
3.9\(\times\)10\(^{\text{4}}\) (\autoref{fig:small-main-3}b), in line
with the state-of-the-art vertical field-effect transistor (VFET)
technology
\cite{Yang_2012_Barristor,Shih_2015_PartiallyScreened,Sun_2017_COF_VFET,Ben_Sasson_2014_fet_rev}.
%
Moreover,
the fact that the present IFET is well-functional at a low
\(V_{\mathrm{D}}\) reflects it potential towards the low-power
electronics design, which remains challenging for most resistive
force-sensing components \cite{Pan_2014_pressure,Pang_2012_gauge}. A
number of IFET samples were characterized, showing an average on/off
current ratio of 5\(\times\)10\(^{\text{3}}\) at \(V_{\mathrm{D}}\)=1
V ( \autoref{fig:small-main-3}b inset). The transfer curves at
different \(V_{\mathrm{D}}\) are shown in \autoref{fig:small-main-3}c,
with the on/off current ratio gradually decreases with
\(V_{\mathrm{D}}\), owing to a mechanism analogous to the
drain-induced barrier lowering (DIBL) effect in short-channel FETs
\cite{Lundstrom_2003_moore}. We note the slight difference each device
may be caused by the air-borne doping of the IFET, and further efforts
will be made to improve the repeatability of such devices.
% More
% discussions about the transport mechanisms, together with the band
% diagrams, are stated in Section S4, s S12, S16-S20 in the Supporting
% Information.
Three-terminal operation of the IFET offers integrated
device functionalities sharing with typical FETs. We demonstrate a
circuit composed by an IFET, a green light-emitting diode (LED), and a
bipolar junction transistor (BJT) amplifier (circuit schematic see
\autoref{fig:small-main-3}d left), with the contact angle monitored by
a CAG ( \autoref{fig:small-main-3}d middle). Through continuous
forward-reverse scan of \(V_{\mathrm{G}}\), the total current
\(I_{\mathrm{tot}}\) that passes through the circuit, is modulated by
\(\sim\)250 folds and switches the LED on and off, showing a high
current stability and reproducibility ( \autoref{fig:small-main-3}d
right). The circuit design maximized the LED response to compressive
stress as well, as will be
discussed in following sections.

\subsection{Pressure Sensing on IFET}
\label{sec:press-sens-ifet}

The reversible mechanical response of a CB droplet on the
superhydrophobic NW surface, is the key principle behind the IFET.
%
To model the elastic response for a CB droplet, we consider a droplet
sandwiched between two flat plates, with two apparent contact angles
\(\theta_{\mathrm{t}}^{*}\) and \(\theta_{\mathrm{b}}^{*}\),
corresponding to the top and bottom liquid-solid interfaces,
respectively (see \autoref{sec:small-method}). Under the assumptions
of (i) the Bond number Bo\(\ll\)1 and (ii) the contact angles remain
constant independent of \(\sigma\), the cross-sectional boundary of
the droplet can be described as a segment of a perfect circle
\cite{berthier_2012_microdroplet}. Accordingly, the Laplace pressure
\(p\) of the droplet is given by:
\(p = \gamma_{\mathrm{L}} (R_{1}^{-1} + R_{2}^{-1})\), where \(R_{1}\)
and \(R_{2}\) are the principal radii of the droplet, as schematically
shown in \autoref{fig:small-main-4}a. Upon applying a compressive
stress \(\sigma\) between the plates, the droplet experiences an
uniaxial strain \(\varepsilon = (H_{0} - H) / H_{0}\), where \(H_{0}\)
and \(H\) are the droplet heights before and after stress,
respectively. The compressive stress varies with height, following
\(\sigma = p(H) - p(H_{0})\). Note that here the liquid phase itself
is nearly incompressible, and the ``elasticity'' is originated from a
thermodynamic driving force counteracting the increase of interfacial
tension upon mechanical stress, conceptually different from the
deformation of a bulk material.
\begin{figure}[!htbp]
  \centering
  \scalebox{0.95}{\import{\imgdir}{pressure-1.pdf_tex}}
% \includegraphics[width=0.75\linewidth]{img/scheme-4.pdf}
  \caption{\label{fig:small-main-4} Elastic response of the LM droplet
    in the proposed IFET. \textbf{a}, Schematics of a droplet
    sandwiched between two plates before (top) and after (bottom) of a
    compressive stress, resulting a change of the principal radii,
    \(R_{1}\) and \(R_{2}\). \textbf{b}, Compressive stress \(\sigma\)
    as a function of strain \(\varepsilon\) for a 0.1 \(\mu\)L mercury
    CB droplet obtained from experiments (dots) and our hydrostatic
    model (curve). Inset: model-predicted elastic modulus \(E\) as a
    function of the droplet volume \(V_{\mathrm{drop}}\). \textbf{c},
    Transfer curves under various compressive stress values. Insets:
    CAG images of the LM droplet under various external stress levels
    (scale bars: 500 \(\mu\)m.)}
\end{figure}

We formulate the principal radii of the compressed droplet as a
function of droplet height \(H\) for \(H < H_{0}\), namely
\(R_{1}(H)\) and \(R_{2}(H)\)  (detailed
derivations see \autoref{sec:small-method}). First, the maximum height
\(H_{0}\) corresponding to \(\sigma=0\) is given by:

  \begin{equation}
  \label{eq:small-H0}
  \begin{aligned}
    H_{0} &= \sqrt[3]{\frac{3 V_{\mathrm{drop}}}{4 \pi}} \sqrt[3]{\frac{1}{ 
   g(\theta_{\mathrm{t}}^{*}) + g(\theta_{\mathrm{b}}^{*}) -1 }}  \left(\cos \theta_{\mathrm{t}}^{*} + \cos \theta_{\mathrm{b}}^{*
}\right) \\
    g(\theta) &= \left(\frac{1 + \cos \theta}{2} \right)^{2} \left(2 - \cos \theta \right)
  \end{aligned}
  \end{equation}
where \(V_{\mathrm{drop}}\) is the the droplet volume following
\(V_{\mathrm{drop}} = w(R_{1}, H, \theta_{\mathrm{t}}^{*},
  \theta_{\mathrm{b}}^{*})\), in which \(w\) is an implicit function of
\(R_{1}\) that can be solved
numerically for a given \(H\). On the other hand, the second principal
radius is geometrically given by:
\begin{equation}
\label{eq:small-1}
R_{2} = -\frac{H}{\cos \theta_{\mathrm{t}}^{*} + \cos \theta_{\mathrm{b}}^{*}}
\end{equation}
By using the above equations, the compressive stress \(\sigma\) as a
function of \(\varepsilon\), as well as the effective elastic modulus
\(\mathscr{M} = \left({\displaystyle \frac{\mathrm{d} \sigma}{\mathrm{d}
  \varepsilon}}\right)_{H_{0}}\), can be calculated numerically. We
note that \(\theta^{*}_{\mathrm{b}}\) depend on the factor \(f\), rather
than the average radius of the NWs.  To validate our model, a
mercury droplet having an air-stable surface tension
\(\gamma_{\mathrm{L}}\)=0.487 J\(\cdot\)m\(^{\text{-2}}\) is used. Note that
mercury often forms alloys with commonly-used metals
\cite{Kieffer_1959}, so the top contact angle
\(\theta_{\mathrm{t}}^{*}\) may vary depending on the sample
preparation process. For each droplet height, we determined the
experimental \(\sigma\) by extracting the principal radii from the CAG
images, with the height controlled by a stage
micromanipulator.  \autoref{fig:small-main-4}b compares the experimental
and calculated elastic stress of a 0.1 \(\mathrm{\mu}\)L droplet as
a function of strain, showing excellent agreement. We notice that
within the strain range considered here (up to 13.5\%), the \(\sigma -
  \varepsilon\) profile is nearly linear, following the Hooke's law. We
determine the effective elastic modulus to be 820 Pa, which is more
than 100 times softer than the widely used Ecoflex rubbers
(\(\mathscr{M}\sim{}\)0.1 MPa) \cite{Mosadegh_2014_soft_robot}, and to our knowledge, softer than any solid
materials including the state-of-the-art ultrasoft elastomers
\cite{Miriyev_2017_soft_mater,Jang_2015_soft_network}.
%
Using our model, we further calculate \(\mathscr{M}\) versus
\(V_{\mathrm{drop}}\) (\autoref{fig:small-main-4}b inset), predicting
an adjustable elastic modulus by simply controlling the droplet
volume. Unsurprisingly, a smaller droplet tends to be stiffer due to
an intrinsically large Laplace pressure. Another degree of freedom for
tunning the elastic modulus is the bottom contact angle
$\theta_{\mathrm{b}}^{*}$: in other words, through controlling the
wettability of a semiconductor surface, one may design an IFET with a
desirable mechanical responsiveness. In general, superhydrophobic
surface with $\theta_{\mathrm{b}}^{*} \to 180°$ is favored for larger
response of the droplet.

We next examine the current response of the IFET under mechanical
stress.  \autoref{fig:small-main-4}c presents the transfer curves at
different \(\sigma\) values. The current from drain to source
increases with \(\sigma\), with the on/off current ratio unaffected by
the elastic stress. The elastic response at \(V_{\mathrm{G}}\)= 0 V
shows a sensitivity, \(\eta = (I/I_{0} - 1) / \sigma\) , where
\(I_{0}\) and \(I\) are \(I_{\mathrm{DS}}\) before and after stress,
of 36 kPa\(^{-1}\), together
with the detection limit of down to \textless{}10 Pa, comparable to
the most sensitive resistive pressure sensor
\cite{Mannsfeld_2010_pressure,Pang_2012_gauge,Pan_2014_pressure,Zang_2015_suspend_gate_FET},
with a considerably lower driving voltage that favors
low-power-consumption designs.
%
Following the design concept stated earlier, we point out that the
major mechanism responsible for the current response is an increased
contact area \(\Delta A\) upon stress \(\sigma\), and the change of
penetration length \(\Delta l\) (see \autoref{fig:small-main-1}b) is
negligible within the stress range considered.  The above analysis is
further endorsed by the finite element method (FEM) simulations, as
shown in \autoref{fig:small-FEM}a).  We further demonstrate the
reversibility of current response by repeatedly applying and releasing
a compressive stress of 204.2\(\pm\)12.3 Pa to our IFET device
(\autoref{fig:small-FEM}b). It is also noteworthy that in our tests,
the EGaIn sessile droplets function equally well on the NWs, in spite
of their sticky oxide surface.
\begin{figure}[!htbp]
  \centering
  \scalebox{0.95}{\import{\imgdir}{pressure-2.pdf_tex}}
% \includegraphics[width=0.75\linewidth]{img/scheme-4.pdf}
  \caption{\label{fig:small-FEM} Pressure sensing mechanism of IFET.
    \textbf{d}.  Comparison of CAG-captured (top) and finite-element
    simulated (bottom) droplet shape and stress values under different
    degrees of compression. Scale bar: 200
    \(\mu\)m. \textbf{b}. Real-time monitoring of \(I_{\mathrm{DS}}\)
    in an IFET by repeatedly applying and releasing a compressive
    stress.}
\end{figure}

\subsection{Carrier Transport at Nanowire / Graphene Interface}
\label{sec:small-carrier-transport}

Finally, the transport properties at the NW/Gr interface is
discussed. As illustrated earlier, since the carrier density at the
interface is modulated by a partially-penetrated electrostatic field
(see \autoref{ch:qc}), it has been suggested that the current density
can be described by the thermionic emission model
\cite{Sze_2006_Mosfets}.
%
The current density $J_{\mathrm{DS}}$ flowing
through the NW/2DEG interface is be described as:
\begin{equation}
\label{eq:small-therm}
J_{\mathrm{DS}} = A^{**} T^{2} \exp(- \frac{e \varphi_{\mathrm{SB}}}{k_{\mathrm{B}}T}) 
                \left[ \exp(\frac{e V_{\mathrm{D}}}{k_{\mathrm{B}}T}) - 1\right]
\end{equation}
where $\varphi_{\mathrm{SB}}$ is the Schottky barrier height, \(T\) is the temperature, \(A^{**}\) is the reduced effective
Richardson constant, \(e\) is the elementary charge and \(k_{\mathrm{B}}\) is the Boltzmann
constant. 
%
\autoref{eq:small-therm} allows to quantify the Schottky barrier
height, \(\varphi_{\mathrm{SB}}\), from the temperature-dependent
measurements. We find that our IFET transfer current substantially
increases with temperature (\autoref{fig:small-main-5}a).
%
\begin{figure}[!htbp]
\centering
\scalebox{0.9}{\import{\imgdir}{temperature.pdf_tex}}
\caption{\label{fig:small-main-5} Temperature-dependent electron
  transport characteristics at the NW/Gr interface. \textbf{a}.
  \(J_{\mathrm{DS}}\) as a function of \(V_{\mathrm{G}}\) at various
  temperatures, showing the thermoionic emission effect. \textbf{b}.
  Experimentally-obtained (dots) and calculated (dashed curves)
  current gains, \(G(T)\) , as a function of temperature \(T\),
  suggesting that the thermoionic emission model can describe the
  interfacial current well. \textbf{c}. Extracted Schottky barrier
  height \(\varphi_{\mathrm{SB}}\) (blue dots) as a function
  \(V_{\mathrm{G}}\). The dashed curve corresponds to our theoretical
  prediction using the elementary electronic properties of graphene,
  suggesting a degree of Fermi level pinning at the NW/Gr interface.}
\end{figure}
%
The possibility that such change of current is caused by
temperature-induced increase of contact area $A$ or penetration length
$\Delta l$ is excluded, since the thermal expansion of LM is
small~\cite{Dickey_2008_EGAIN} (\textless{}1\%) within the temperature
range considered here.
%
\autoref{fig:small-main-5}b presents the experimentally-obtained
current gain as a function of temperature,
\(G(T)=J_{\mathrm{DS}}(T)/J_{\mathrm{DS}}(T=20\ ^{\circ} \mathrm{C})\)
at different \(V_{\mathrm{G}}\) levels together with the least-square
fitting curves using the thermionic emission model. Accordingly, at a
more negative \(V_{\mathrm{G}}\), the temperature dependence appears
to be stronger, suggesting a higher \(\varphi_{\mathrm{SB}}\) blocking
thermally-induced transport of electrons. The extracted
\(\varphi_{\mathrm{SB}}\) values as a function of \(V_{\mathrm{G}}\)
is shown in \autoref{fig:small-main-5}c, ranging from 0.46 V at
\(V_{\mathrm{G}}\)=-100 V to 0.16 V at \(V_{\mathrm{G}}\)=100 V.  The
range of gate-tunable \(\varphi_{\mathrm{SB}}\) is comparable to that
of the Si/graphene heterojunction \cite{Yang_2012_Barristor}, but
considerably lower than our theoretical prediction using the
elementary electronic properties of graphene (
\autoref{fig:small-main-5}c and Supporting Information Section S4). A
degree of Fermi level pinning due to the surface-bound traps
\cite{Meric_2008_saturation_gr_FET} may explain the observation.

\section{Conclusions}
\label{sec:small-conclusions}

In this chapter, we establish a new concept to reliably incorporate
mechanical durability and responsiveness in a transistor by
engineering the wettability of a semiconductor surface. Using the
superhydrophobic semiconductor NWs in an IFET, we systematically
analyze the origin of its ultrasoft elasticity driven by the
minimization of interfacial tension upon a compressive stress, as well
as the current response that can be modulated by gating through the
atomically-thin 2DEG.
%
This study is an example of how our fundamental understandings of the
2D material interfaces and engineering of multiscale physical
phenomena would be utilized for a wide range of ultrasensitive and
stimuli-responsive nano\-electronics.




\section{Methods}
\label{sec:small-method}

\subsection*{Materials}
\label{sec:small-org2cee247}
Copper foil (Cu, 20 \(\mu \mathrm{m}\) thickness, 99.999\%, Alfa Aesar),
F\(_{\text{16}}\)CuPc (99\%, sublimed, TCI Chemical), Hydrochloric acid (37\%,
Sigma-Aldrich), Nitric acid (65\%, Sigma-Aldrich), Methane (99.9\%,
PanGas Schweiz), Hydrogen (99.9\%, PanGas Schweiz), SiO\(_{\text{2}}\)/Si\(^{\text{++}}\)
wafer (300 nm thermal oxide, Si-Mat -Silicon Materials e.K.), Aluminum
pellets (99.99\%, Sigma Aldrich), Poly methylmethacrylate (PMMA, 950
kDa, 4\% solution in anisole, MicroChem), Iron chloride (FeCl\(_{\text{3}}\), 99\%,
Sigma-Aldrich), Isopropanol (IPA, 99\%, Thommen-Furler AG), Acetone
(99\%, Thommen-Furler AG), Trichloro(octadecyl)silane (OTS, 90\%,
Sigma-Aldrich), Chlorobenzene (99\%, Sigma-Aldrich) .

\subsection*{Characterizations}
\label{sec:small-org048d24a}
SEM images were taken by Zeiss ULTRA plus with 2 kV beam voltage
and 20 \(\mathrm{\mu}\)m aperture. Confocal Raman spectroscopy was
measured by Renishaw inVia\(^{\textrm{TM}}\) confocal microscope using
532 nm laser and 1800 mm\(^{\text{-1}}\) grating. AFM mapping were measured by
a Asylum Cyber ES AFM. Sessile droplet contact angles were measured
by a MCA-3 Microscopic Contact Angle Goniometer (Kyowa Interface
Science, co. ltd.). ESEM measurements were carried out in a FEI
Quanta 600 ESEM equipped with a gaseous back scattered electron
detector. Electric properties of the interfacial transistor were
measured by an Agilent B1500A Semiconductor Device Parameter
Analyzer. Stress and thermal characterizations see the following
subsections. Grazing incidence X-ray diffraction (GIXD) patterns
were measured on beamline BL13A at the National Synchrotron
Radiation Research Center (NSRRC), Taiwan. A monochromatic beam of
\(\lambda\) = 1.0205 \AA{} was used, and the incident angle was
0.12\(^{\circ}\). X-ray Diffraction (XRD) spectra were measured by
a Bruker D8 Power Diffractometer with a Cu K\(_{\alpha}\) beam of
1.5406 \AA{} wavelength.

\subsection*{Graphene Synthesis and Transfer}
\label{sec:small-org7dad228}

The Cu foil was cut into pieces of 2.5 cm \(\times\) 3.5 cm, cleaned in 1M
HCl solution to remove native oxide, rinsed in DI water and dried in
argon flow. The Cu foil was placed into a quartz boat which was
cleaned by oxygen plasma in advance. The quartz boat was then loaded
into a quartz tube with diameter of 2.5 inch. The quart tube was
mounted onto a home-made CVD system. The quartz boat was placed in the
center of the heating zone and kept at 30\(\sim\)35 cm from the position of
the pump valve. The tube was purged and refilled with H\(_{\text{2}}\) 3 times
and subsequently heated up to 1000 \(^\circ \mathrm{C}\) with a ramp
rate of 20 \(^{\circ}\mathrm{C} \cdot \mathrm{min}^{-1}\). The copper
foil was annealed at 1000 \(^\circ \mathrm{C}\) for 30 min under 15 sccm
H\(_{\text{2}}\) flow, with the pressure kept at 150 mTorr. 40 sccm CH\(_{\text{4}}\) was
then added into the CVD system and the pressure was kept at 5 Torr for
30 min to grow graphene on the Cu foil. After the graphene growth
was completed, the CH\(_{\text{4}}\) flow and heating were stopped, and the quartz
boat was moved out of the heating zone for a fast cooling at a rate of
200 \(^{\circ}\mathrm{C} \cdot \mathrm{min}^{-1}\). When the temperature
of the CVD system cooled under 100 \(^{\circ}\mathrm{C}\), the quartz tube
was refilled with argon and the Cu foil was taken out.

The graphene grown on the backside (near the quartz boat) was used for
all processes involved. 4\% PMMA solution in anisole was spin-coated
onto the desired side of Cu foil at 2000 rpm for 40 s. The PMMA-coated
Cu foil was then annealed on a hot plate at 200 \(^{\circ}\mathrm{C}\)
for 1 min. The graphene on the non-coated side was removed by 400 W
O\(_{\text{2}}\) plasma for 1 min. The Cu foil was then cut into desired size and
placed onto the surface of FeCl\(_{\text{3}}\) solution in a petri dish for 1.5 h
until all the Cu was etched. The floating PMMA/graphene membranes were
subsequently transferred onto DI water (3X), HCl (30 min) and DI water
(5X) for further cleaning, and transferred onto the desired
substrates. For substrates that were highly hydrophobic
(e.g. OTS-treated SiO\(_{\text{2}}\)/Si substrates), the PMMA/graphene membrane
was transferred onto a 5:1 mixture of water-isoprapanol at the last
transfer step. The substrates were kept in ambient for 1 h, and then heated at 150 \(^{\circ}\mathrm{C}\) for
2 min to remove residual water. The substrates with transferred
PMMA-graphene were cooled to room temperature and dipped into a bath of
acetone for 30 s to 2 min to remove the PMMA. The substrate was rinsed
repeatedly in acetone and IPA, and dried in argon flow.

The transfer method can be used on surfaces substrates like metal, Si,
SiO\(_{\text{2}}\) (including chemically-modified SiO\(_{\text{2}}\)), glass and even
textured materials like wood. 

\subsection*{Deposition of F\(_{\text{16}}\)CuPc on Graphene}
\label{sec:small-org80e1aed}

The desired substrate was mounted onto a glass of 3 cm \(\times\) 3 cm
and loaded into a mBraun high vacuum thermal evaporation system. The
surface to be deposited was placed facing the crucible. 3\(\sim\)10 mg
of F\(_{\text{16}}\)CuPc powder was added into a quartz crucible and placed into
the heating zone. The evaporation system was then purged under 10\(^{\text{-6}}\)
mBar. F\(_{\text{16}}\)CuPc was deposited at 300\(\sim\)360 \(^{\circ}\mathrm{C}\)
with a deposition rate of 0.2\(\sim\)0.8 \AA{}/s indicated by the quartz
crystal microabalance (QCM). Since the thickness of F\(_{\text{16}}\)CuPc film on
different substrates differs a lot, the change of QCM oscillation
frequency \(\Delta \nu_{\mathrm{QCM}}\) during the deposition process
was used as the measurement of amount of F\(_{\text{16}}\)CuPc deposited. \(\Delta
\nu_{\mathrm{QCM}}\) is related to the change of mass \(\Delta m\) by the
Sauerbrey equation \cite{Sauerbrey_1959}:


\begin{equation}
\label{eq:small-1}
\Delta \nu_{\mathrm{QCM}} = -\frac{2 \nu_{0}^{2}}{A \sqrt{\rho_{\mathrm{Q}} \mu_{\mathrm{Q}}}} \Delta m
\end{equation}
where the parameters \(A\), \(\rho_{\mathrm{Q}}\) and \(\mu_{\mathrm{Q}}\)
are quartz-related properties. Since \(\Delta \nu_{\mathrm{QCM}}\) is very
small compared with the resonance frequency \(\nu_{0}\), we can assume
\(\nu_{0}\) as constant. Therefore \(\Delta \nu_{\mathrm{QCM}}\) is
proportional to the mass change, which is directly related to the
amount of F\(_{\text{16}}\)CuPc deposited.

\subsection{Fabrication of F\(_{\text{16}}\)CuPc NW Interfacial Transistor}
\label{sec:small-orgdb060bd}

A 4-inch SiO\(_{\text{2}}\)/Si wafer was cut to 2 cm \(\times\) 2 cm pieces and subsequently
cleaned by sonication in DI water, acetone, IPA. The wafer slices were dried
and further cleaned in 400 W O\(_{\text{2}}\) plasma for 10 min. The wafers were
immersed in 20 mL solution of 0.1 mM OTS in dry chlorobenzene for 2
h. The wafers were then rinsed by dry chlorobenzene and heated in
nitrogen atmosphere at 200 \(^{\circ}\mathrm{C}\) for 5 min. Graphene
(1.0 cm \(\times\) 1.0 cm) was transferred onto the OTS-treated wafers
using the steps described in \autoref{sec:small-gr-transfer}.

Source electrodes (100 nm Al) were deposited onto graphene using
shadow masks at deposition rate of 1\(\sim\)3 \AA{}/s. F\(_{\text{16}}\)CuPc were further
deposited by changing the mask onto graphene. The thickness of
F\(_{\text{16}}\)CuPc were controlled by \(\Delta \nu_{\mathrm{QCM}}\) and checked by
SEM. After the deposition, all F\(_{\text{16}}\)CuPc deposited outside the
graphene region were removed by scratching using the cone of an
Eppendorf tip. A contact to the doped Si at backside was made for the
gate electrode. A liquid metal droplet (mercury or EGaIn) were used as
the drain electrode.

\subsection*{Measurements Involving the IFET}
\label{sec:small-orgec772cc}

The F\(_{\text{16}}\)CuPc NW interfacial transistors were mounted onto an
insulating probing block of a modified MCA-3 surface tensiometer. The
gate and source terminals were connected using micropositioners
(Linkam UK). A custom-made Ag-coated Cu cantilever with a plate
(fabricated by mechanical stamping) at the head were used as the drain
electrode. The plate was adjusted parallel to the SiO\(_{\text{2}}\) wafer. A LM
droplet with volume \(\sim\)0.1 \(\mu \mathrm{L}\) was dispensed and
attached to the bottom side of the plate. The droplet was monitored
using a Mitutoyo 2X long working distance lens, and carefully
contacted with the surface of the surface of F\(_{\text{16}}\)CuPc. Electronic
measurements were performed using an Agilent 1500B semiconductor
analyzer.

We observe that the sudden change of electric field when applying the
potential across F\(_{\text{16}}\)CuPc-LM interface may cause undesired breakdown
of the F\(_{\text{16}}\)CuPc NWs (\worktodo{Fig.} \autoref{fig:electric-breakdown}). Therefore,
in tests involving gate modulation, the potentials were first added
onto the terminals without the LM droplet contacting F\(_{\text{16}}\)CuPc NWs.
\(V_{\mathrm{D}}\) was maintained until the droplet was in contact with
F\(_{\text{16}}\)CuPc NWs and a valid current between the drain-source was
established.

For stress tests, the LM droplet was pressed against the F\(_{\text{16}}\)CuPc
NWs surface using a manual pneumatic valve. The moving distance of the
droplet was controlled within 200 \(\mu \mathrm{m}\) to avoid mechanical
deformation of F\(_{\text{16}}\)CuPc NWs

For thermal response tests, the probing block was replaced by a LT-600
heating unit (Linkam UK). The temperature was switched between 20
\(^{\circ} \mathrm{C}\) and 100 \(^{\circ} \mathrm{C}\) at 30 \(^{\circ}
\mathrm{C} \cdot \mathrm{min}^{-1}\). The cooling is controlled by a
LN95 liquid nitrogen flowmeter (Linkam UK).

\subsection*{FEM Analysis of Pressure Distribution}
\label{sec:small-orge90523c}
The FEM analysis for the stress of LM droplets under strain were
carried out using COMSOL Multiphysics\(^{\text{@}}\) 5.3a. The geometry for
each droplet was constructed using the real droplet shape in the
optical images taken by the CAG. The spatial distribution of
Laplace pressure was calculated using built-in curvature
components. The value of FEM stress was calculated by the
difference between average Laplace pressure on the droplet surface
\(\bar{p}\) before and after applying the stress. The average Laplace
pressure is calculated by:

\begin{equation}
\label{eq:small-7}
\bar{p} = {\displaystyle \frac{\int_{\mathrm{\Omega}} p \mathrm{d} \Omega}{\int_{\mathrm{\Omega}} \mathrm{d} \Omega}}
\end{equation}

where \(\Omega\) is the surface of the droplet (excluding the contact
surface with the top and bottom planes).

\subsection*{Analytical model for stress-strain relation}
\label{sec:small-orgc12d6dd}

As stated in the main text, the capillary pressure at the boundary of a droplet which is sit
between two parallel plates can be modeled by the Young-Laplace
equation:
\begin{equation}
\label{eq:small-3}
p = \gamma (R_{1}^{-1} + R_{2}^{-1})
\end{equation}

where \(R_{1}\) and \(R_{2}\) are the two principal radii of the
droplet. Since the size of LM droplets used in the interfacial FET are
within the low Bond number regime (\(Bo = \Delta p g R^{2} / \gamma <
1\)), the effect of gravity can be ignored. Under such conditions,
the cross-sectional boundary of the droplet between two parallel
plates (top and bottom) can be regarded as part of a sphere
\cite{berthier_2012_microdroplet}, and thus possible to be modeled by an
analytical model.

\begin{figure}[htbp]
\centering
\scalebox{1.05}{\import{\imgdir}{model_2D.pdf_tex}}
\caption{\label{fig-SI-drop-model}%
  Analytical model of droplet geometry between two horizontal parallel
  plates, the droplets have a convex shape in both cases. \textbf{a},
  Symmetrical case, the volume of droplet is described by a function
  \(v(R_{1}, \delta, \theta)\) as a function of principal radii
  \(R_{1}\) and \(R_{2}\), the half height of the droplet \(\delta\)
  and contact angle \(\theta\). \textbf{b}, Asymmetrical case where
  top and bottom contact angles \(\theta_{\mathrm{t}}\) and
  \(\theta_{\mathrm{b}}\) are different. The volume of droplet is
  given by:
  \(V_{\mathrm{drop}} = [v(R_{1}, \delta_{\mathrm{t}},
  \theta_{\mathrm{t}}) + v(R_{1}, \delta_{\mathrm{b}},
  \theta_{\mathrm{b}})]/2\). \textbf{c}, The limiting case when
  \(H=H_{0}\). The geometry of the droplet is a sphere segment which
  the volume is given by \autoref{eq:small-9}.  }
\end{figure}

\subsubsection*{Symmetric case}
\label{sec:small-org5934fd4}

First consider the simplest case where the contact angle
between a convex droplet and both plates are the same, the
characteristic geometric parameters of the droplet are the principle
radii \(R_{1}\) and \(R_{2}\), the half height \(\delta=H/2\) and contact
angle \(\theta\) of the droplet. As
shown in \autoref{fig-SI-drop-model}a, the droplet is axial
symmetric and \(R_{1}\) is the maximum radius of horizontal cross
sections, while \(R_{2}\) is the radius of the smaller arc of the
vertical cross section, when gravity an be ignored.

The volume of the droplet \(V_{\mathrm{drop}}\) is then expressed as:

\begin{equation}
\label{eq:small-sym-1}
\begin{aligned}
V_{\mathrm{drop}} &= 2 \pi \int_{0}^{\delta} \left[ (R_{1} - R_{2}) + \sqrt{R_{2}^{2} - z^{2}}\right]^{2} \mathrm{d}z \\
  &= 2\pi \left\{ \left[(R_{1} - R_{2})^{2} + R_{2}^{2} \right] z 
- \frac{z^{3}}{3} \right\} \Bigg|_{0}^{\delta}
 + 2 \pi \left\{(R_{1} - R_{2}) R_{2}^{2} (\theta' + \sin \theta' \cos \theta')
\right\} \Bigg |_{0}^{\theta - \pi/2} \\
&= 2 \pi \left \{ [(R_{1} - R_{2})^{2} + R_{2}^{2}]\delta - \frac{\delta^{3}}{3} + (R_{1} - R_{2}) R_{2}^{2} (\theta - \pi/2 -\sin \theta \cos \theta)\right\} \\
&= u(R_{1}, R_{2}, \delta, \theta)
\end{aligned}
\end{equation}

Note that \(R_{1}\) and \(R_{2}\) are
related with \(\delta\) and the contact radius \(r\):

\begin{eqnarray}
\label{eq:small-R1}
&R_{1} &= {\displaystyle r + \delta \frac{\sin \theta - 1}{ \cos \theta}}  \\
\label{eq:small-R2}
&R_{2} &= -{\displaystyle \frac{\delta}{\cos \theta}}
\end{eqnarray}
plug  \autoref{eq:small-R2} into \autoref{eq:small-sym-1}, we can express
\(V_{\mathrm{drop}}\) alternatively as:

\begin{equation}
\label{eq:small-sym-2}
\begin{aligned}
V_{\mathrm{drop}} &= u(R_{1}, -\frac{\delta}{\cos \theta}, \theta)\\
                  &= v(R_{1}, \delta, \theta)
\end{aligned}
\end{equation}

\(\theta\) and \(\delta\) values are normally determined from the
experimental data, thus we can get the value of \(R_{1}\) by the inverse
function of \(u\) as \(R_{1} = v^{-1}(V_{\mathrm{drop}, \delta,
 \theta})\). The values of \(r\) and \(R_{2}\) are further converted
via \autoref{eq:small-R1} and \autoref{eq:small-R2}.

\subsubsection*{Asymmetric case}
\label{sec:small-orgba3824f}
The symmetric case does not represent the real LM droplet in the
interfacial transistor, since the contact angles on the top plane
(\(\theta_{\mathrm{t}}\)) and bottom plane (\(\theta_{\mathrm{b}}\))
can be quite different. Therefore we need to derive the relation
between \(V_{\mathrm{drop}}\) and \(R_{1}\) \(R_{2}\) of an asymmetric
droplet between two parallel plates. We use a similar approach:
divide a droplet with asymmetric contact angles into two parts with
heights \(\delta_{\mathrm{t}}\) and \(\delta_{\mathrm{b}}\)
\autoref{fig-SI-drop-model}b. Each of the two individual parts
corresponds to half of a symmetric droplet between plates with the
same \(R_{1}\) and \(R_{2}\). \(\delta_{\mathrm{t}}\) and
\(\delta_{\mathrm{b}}\) are determined by:

\begin{eqnarray}
\label{eq:small-deltas-1}
\delta_{\mathrm{t}} &= {\displaystyle \frac{H \cos \theta_{\mathrm{t}}}{\cos \theta_{\mathrm{t}} 
                  + \cos \theta_{\mathrm{b}}}} \\
\label{eq:small-deltas-2}
\delta_{\mathrm{b}} &= {\displaystyle \frac{H \cos \theta_{\mathrm{b}}}{\cos \theta_{\mathrm{t}} 
                  + \cos \theta_{\mathrm{b}}}}
\end{eqnarray}
where \(H\) is the height of the droplet.
From the calculation of droplet volume in the symmetric case, we know
the volume of the asymmetric droplet can be written as:
\begin{equation}
\label{eq:small-V-assym-1}
\begin{aligned}
V_{\mathrm{drop}} &= \frac{u(R_{1}, R_{2}, \delta_{\mathrm{t}}, \theta_{\mathrm{t}}) + u(R_{1}, R_{2}, \delta_{\mathrm{b}}, \theta_{\mathrm{b}})}{2}\\
                  &= \frac{v(R_{1}, \delta_{\mathrm{t}}, \theta_{\mathrm{t}}) +
                           v(R_{2}, \delta_{\mathrm{b}}, \theta_{\mathrm{b}})}{2} \\
                  &= w(R_{1}, H, \theta_{\mathrm{t}}, \theta_{\mathrm{b}})
\end{aligned}
\end{equation}

As can be seen, when the values \(\theta_{\mathrm{t}}\),
\(\theta_{\mathrm{b}}\) and \(V_{\mathrm{drop}}\) are known, we can also
calculate \(R_{1}\) via: \(R_{1} = w^{-1}(V_{\mathrm{drop}}, H,
 \theta_{\mathrm{t}}, \theta_{\mathrm{b}})\). The value of \(R_{2}\) can
be calculated by:

\begin{equation}
\label{eq:small-asym-R2}
R_{2} = -{\displaystyle \frac{H}{\cos \theta_{\mathrm{t}} + \cos \theta_{\mathrm{b}}}}
\end{equation}

The top and bottom
contact radii \(r_{\mathrm{t}}\) and \(r_{\mathrm{b}}\) follow:

\begin{eqnarray}
\label{eq:small-t-1}
r_{\mathrm{t}} &= R_{1} + {\displaystyle \frac{2\delta \cos \theta_{\mathrm{t}}}{\cos \theta_{\mathrm{t}} 
                                + \cos \theta_{\mathrm{b}}}
                          \frac{\cos \theta_{\mathrm{t}} - 1}{\sin \theta_{\mathrm{t}}}}\\
\label{eq:small-t-2}
r_{\mathrm{b}} &= R_{1} + {\displaystyle \frac{2\delta \cos \theta_{\mathrm{b}}}{\cos \theta_{\mathrm{t}} 
                                + \cos \theta_{\mathrm{b}}}
                          \frac{\cos \theta_{\mathrm{b}} - 1}{\sin \theta_{\mathrm{b}}}}
\end{eqnarray}
And thus all the components needed for the asymmetric case are calculated.

\subsubsection{Determination of pressure reference}
\label{sec:small-orgdade735}
Since the stress \(p\) calculated by the Young-Laplace equation is
the stress between the LM and air across the LM boundary, we need
to determine the pressure reference (\(p_{0} = p(H=H_{0})\)) for
calculating the change of stress \(\Delta p=p(H) - p(H=H_{0})\). The
physical meaning of \(H_{0}\) is the maximal height of the droplet
between the two plates when no external stress is applied. In this
case the whole droplet has a shape of a sphere segment 
\autoref{fig-SI-drop-model}c. Since two principal radii coincide in this
case, \(R_{1}=R_{2}=R\), we have:

\begin{equation}
\label{eq:small-5}
V_{\mathrm{drop}} = \frac{4 \pi}{3} R^{3} - \frac{\pi}{3} R^{3} (1 + \cos \theta_{\mathrm{t}})^{2}(2 - \cos \theta_{\mathrm{t}})
                                          - \frac{\pi}{3} R^{3} (1 + \cos \theta_{\mathrm{b}})^{2}(2 - \cos \theta_{\mathrm{b}})
\end{equation}
Further convert it back to \(R\), we get:

\begin{equation}
\label{eq:small-8}
R = \sqrt[3]{\frac{3 V_{\mathrm{drop}}}{4 \pi}} \sqrt[3]{\left[ 
1 - \left(\frac{1 + \cos \theta_{\mathrm{t}}}{2} \right)^{2} \left(2 - \cos \theta_{\mathrm{t}}\right)
- \left(\frac{1 + \cos \theta_{\mathrm{b}}}{2} \right)^{2} \left(2 - \cos \theta_{\mathrm{b}}\right)
\right]^{-1}}
\end{equation}
and

 \begin{equation}
 \label{eq:small-9}
 \begin{aligned}
 H_{0} &= -R(\cos \theta_{\mathrm{t}} + \cos \theta_{\mathrm{b}})  \\
 &= \sqrt[3]{\frac{3 V_{\mathrm{drop}}}{4 \pi}} \sqrt[3]{\left[ 
\left(\frac{1 + \cos \theta_{\mathrm{t}}}{2} \right)^{2} \left(2 - \cos \theta_{\mathrm{t}}\right) +
 \left(\frac{1 + \cos \theta_{\mathrm{b}}}{2} \right)^{2} \left(2 - \cos \theta_{\mathrm{b}}\right) -1 
\right]^{-1}}\\
&\left(\cos \theta_{\mathrm{t}} + \cos \theta_{\mathrm{b}}\right)
 \end{aligned}
\end{equation}



\section{Author contributions}
\label{sec:small-contrib}
T.T. and C.J.S. conceived the concept and designed the
experiments. T.T. and C.S.S. carried out the ESEM
measurements. T.T., N.A. and R.S. fabricated and characterized the
morphology and dynamic wetting properties of the F\(_{\text{16}}\)CuPc
samples. Y.T. L. and Y.C. C. measured the GIXD spectroscopy of
F\(_{\text{16}}\)CuPc samples. T.T. fabricated and tested the interfacial
transistors. T.T. and M.V. designed the experiments using
EGaIn. T.T. and C.J.S. developed the model for the stress-strain
relation of droplets.  All
authors contributed to the discussion of the results and to the
revision of the manuscript.



% \input{paper.bbl.bak}
























%%% Local Variables:
%%% mode: latex
%%% TeX-master: "../thesis"
%%% End:
