  % \chapter{Introduction}
\chapter{Conclusions and Outlook}
\label{ch:outlook}


% \dictum[]{%
  % }%

\vspace{1em}

In summary, the focus of this thesis is to develop insights into the
properties at 2D material interfaces, to facilitate the understanding
and engineering of 2D-material-based applications.
%
There are two main keywords describing the major properties of 2D
material interfaces , that they are geometrically \textit{thin} layers
with thickness down to nano\-meter scale, which gives rise to the
\textit{confined} electronic structures. Many interfacial phenomena
actually involve the interplay between these two characteristics of 2D
materials, making them distinct from their bulk counterparts. The main
conclusions and achievements of this thesis concerning different
projects are summarized as follows:
\begin{enumerate}
\item Importance of quantum capacitance (~\autoref{ch:introduction})\\
  
  Theoretical frameworks concerning the electrostatic interactions at
  2D material interfaces are developed, highlighting the significance
  of the quantum capacitance $C_{\mathrm{Q}}$ that comes from the
  quantum confinement. The field-effect transparency
  $\eta^{\mathrm{FE}}$ of a 2D material is thus quantified, which
  shows to be a collective effect between the 2D quantum capacitance
  and the semiconductor capacitor.

  
\item Layer order matters in 2D van der Waals heterostructures (~\autoref{ch:asym})\\
  Using a multilayer quantum capacitor model, we are able to calculate
  the penetration of electrostatic field through multilayer graphene
  as well as graphene/MoS\textsubscript{2} van der Waals
  heterostructures (vdWHs), using only a few material parameters while
  obtaining similar accuracy as compared to full-scale \textit{ab
    initio} calculations. The model reveals the importance of layer
  ordering in determining the apparent electrostatic screening
  behavior of a vdWH. Unlike a bulk material, the electrostatic
  screening is asymmetric with respect to the direction of the
  external field, which brings unprecedented physical phenomena. The
  proposed asymmetric response is further validated by \textit{ab
    initio} simulations as well as experimental demonstrations.
  
\item The critical role of electronic polarizability of 2D materials
  (~\autoref{ch:diel})\\
  Dielectric responses of 2D materials and their heterostructures are
  studied, with a clear evidence showing the importance of the 2D
  electronic polarizability $\alpha_{\mathrm{2D}}$ over the
  conventionally used macroscopic dielectric constant of 2D
  materials. Using material database screening, we identify two
  universal scaling relations that connect $\alpha_{\mathrm{2D}}$ to
  their bandgap and thickness. For the first time, we propose the
  out-of-plane polarizability $\alpha^{\perp}$ as the descriptor for
  the intrinsic thickness of a 2D material.

  
\item Existence of many-body van der Waals effect at 2D material
  interfaces (~\autoref{ch:diel})\\
  Combining the quantum confined dielectric properties and ultra\-thin
  feature of 2D materials, a theoretical framework that simulates the
  transmission of many-body van der Waals (vdW) interactions through
  2D layers, is developed. Based on a modified Lifshitz-vdW formalism,
  the vdW transparency of 2D material-containing medium
  $\eta^{\mathrm{vdW}}$ is derived and quantified.  The
  dielectric anisotropy of a 2D material selectively screens the vdW
  interactions at low frequency regime. Moreover, by proper
  engineering dielectric properties of 2D and bulk materials,
  repulsive vdW interactions are both predicted by the model and validated
  by  experimental molecular epitaxy on substrate-supported graphene surfaces.

  
\item Multiscale wetting phenomena at 2D material interfaces (~\autoref{ch:wet})\\
  A theoretical framework to model the wettability of the doped 2D
  material is proposed, which bridges multiscale physical phenomena
  including quantum capacitance, molecular reorientation, electrical
  double layer effect. Using the proposed model, the change of
  liquid-2D material interfacial energy upon doping is
  investigated. Our results suggest the dominance of molecular
  reorientation on the doping-driven wettability reported in
  experimental studies, which shed light on further applications using
  2D materials as coatings with controllable wetting behavior.

  
\item Understanding gate-tunable ion transport through nanoporous graphene (~\autoref{ch:np})\\
  Using a self-consistent continuum model combining transport
  phenomena and fundamental electronic structures of 2D materials, we
  systematically investigate the gate-induced salt rejection through
  nanoporous graphene as seen in experimental studies.  Effects of
  salt concentration, species, and pore size are considered. The
  interplay between graphene quantum capacitance and the electrical
  double layer is found to selectively lead the anionic and cationic
  transport pathways by creating voltage-dependent potential barriers
  when the pore size is comparable to the Debye length. Fundamental
  studies on the transport phenomena using porous 2D materials may be
  benefited from this study.

\item Interfacial field effect transistor enabled by multiscale phenomena (~\autoref{ch:small})\\
  
  An novel electronic platform known as the interfacial field effect
  transistor (IFET) is proposed by bridging multiscale physical
  phenomena at 2D material interfaces. Following the hydrostatic
  deformation of a liquid metal droplet upon mechanical stress, an
  extremely small elastic modulus of 820 pascals, enabling an
  excellent stress detection limit down below 10 Pa is enabled in the
  IFET. The concept of IFET demonstrates a versatile platform that
  bridges multiple macroscopic interfacial phenomena for novel
  2D-material-based applications.
\end{enumerate}




%%% Local Variables:
%%% mode: latex
%%% TeX-master: "../thesis"
%%% End:

%%% Local Variables:
%%% mode: latex
%%% TeX-master: "../thesis"
%%% End:
