% \chapter{Introduction}
\chapter{Many-Body Transmission of vdW Interactions Through 2D Materials}
\label{ch:vdw}
\renewcommand*\imgdir{img/ch-vdw/}

% \dictum[Wolfgang Pauli]{%
  % God made the bulk;\\Surfaces were invented by the devil
  % }%
\worktodo{find the quote or quit}

\vspace{1em}

\chapterabstract{Part of this chapter appears in the following work:
  Tian, T., Naef, F., Lee, Y.-T., Zhang, S.-W., Chiu, Y.-C. \& Shih,
  C.-J. Two-Dimensional Materials as High-Pass Filter of van der Waals
  Interactions, in preparation.
}

\section{Introduction}
\label{sec:vdw-introduction}
% 20191014-1539
The van der Waals (vdW) interactions are inter\-atomic /
inter\-molecular forces that stem from the electromagnetic (EM)
fluctuations~\cite{Israelachvili_2011_book,Woods_2016_rev_vdw,parsegian_van_2010_book}.
%
Unlike trivial electrostatic (Coulombic) forces that depends on the
charge stat, the vdW interaction has a complex electro\-dynamic
origin, and acts ubiquitously between all atoms and molecules,
regardless of their charge
state~\cite{Israelachvili_2011_book,Hermann_2017_vdW_rev}.
%
As briefly discussed in \autoref{sec:inter-forc-at}, although being a
relatively weak force, the vdW interactions can be dominating at
molecule-2D interface due to the absence of dangling bonds.
%
The understanding and engineering of the vdW interactions at the 2D
material interface would greatly benefit molecular
self-assembly~\cite{Kumar_2017_rev_assemb_2D},
electronic~\cite{Wang_2015_phys_chem_tuning_TMDC,Lazar_2013} as well
as optical properties~\cite{Geim_2013_2D_vdw_Het,Novoselov_2016_vdW}
of 2D material-based hetero\-structures.
%
\worktodo{try to say more about the dielectric properties}

As introduced in \autoref{sec:van-der-waals}, one particular question
of interest is the transmission of vdW interactions through the
atomically-thin 2D materials.
%
The effective distance of vdW forces, depending on the medium, can
vary from sub-nm 10$^{1}$ nm~\cite{Israelachvili_2011_book}, and are
relatively longer compared with the sub-nm thickness of 2D materials.
%
Therefore, the vdW forces from an macroscopic substrate may
(partially) penetrate through the 2D layer and be felt by an molecule
on the 2D material interface~\cite{shih_2013_wetting_natmat}.
%
This feature
brings potentially unprecedented surface properties by simply coating
an existing bulk surface with 2D
material(s)~\cite{Prasai_2012_coating,rafiee_2012_transparency,Tsoi_2014_vdW_screening_2D}.
%
Based on classic pair-wise model~\cite{Hamaker_1937_vdW} of vdW
forces, the interaction free energy per area $\Phi_{\mathrm{vdW}}$,
between two semi-infinite bulk bodies (A and B) separated by distance
$d$, is:
\begin{equation}
  \label{eq:vdw-hamaker-res}
  \Phi^{\mathrm{vdW}} = - \frac{A_{\mathrm{AB}}}{12 \pi d^{2}}
\end{equation}
where $A_{\mathrm{AB}}$ is known as the Hamaker constant of vdW
interactions between materials A and B.
%
A quantitative statement can be drawn from such simple model: by inserting a 2D material (\eg graphene
with thickness $\sim{}$3.3 \AA{}~\cite{Shearer_2016}) between two bulk
bodies (3$\sim{}$5 \AA{} \worktodo{cite 1 here}), 20--40\% of the vdW
interactions can still penetrate through the 2D layer.
%
Several unprecedented phenomena concerning the interfacial vdW forces
on 2D materials take credit from the analysis above, including wetting
transparency / translucency
\cite{Shih_2012_prl,rafiee_2012_transparency,Gurarslan_2016_screen_MoS2} and remote
epitaxy \cite{Kim_2017_remote_epi_Gr,Kong_2018_pol}.

Despite the seemingly accordance with experimental observations in
these studies, the prerequisites for the classical model are the
perfect additivity of vdW interactions, and a constant
$A_{\mathrm{AB}}$. 
%
In other words, the classic model assumes that the existence of the
2DEG has no influence on the EM waves that arise from charge
fluctuations.
%
Such picture is however challenged by recent experimental
\cite{Tsoi_2014_vdW_screening_2D} and first-principles
\cite{Ambrosetti_2018_carbon,Liu_2018_gr,Li_2018_screen} studies,
revealing the 2D material-dependent Hamaker coefficient.
%
To correctly capture the many-body picture of vdW interactions at 2D
material interfaces, electro\-dynamic effects beyond the classic model
is needed.
%
In practice, this is usually a challenging task, since state-of-art
density functional theory (DFT) used for first principle simulations
is limited ground state electron density, intrinsically lacking charge
fluctuation~\cite{Woods_2016_rev_vdw}.
%
On the other hand, full-scale interaction energy calculations based on
dynamic non-local dielectric responses are usually time-consuming and
only limited to small
systems.~\cite{Hermann_2017_vdW_rev,Zhou_2017_lifshitz}
%
Therefore, novel theoretical frameworks with both appreciable accuracy
while maintaining simplicity, is of high demand to shed light on the
vdW interactions at 2D material interfaces.

In this chapter, we present
the a generalized model to study the many-body vdW transmission
through 2D materials, based on the modified Lifshitz-vdW theory
\cite{Dzyaloshinskii_1961_lifshitz,parsegian_van_2010_book}.
%
Based on the model, the vdW transmission through a 2D material is
selective, such that a 2D material acts as a high-pass filter of vdW
interaction, filtering the low-frequency interactions.
%
Such phenomenon is related with the anisotropic dielectric nature (\worktodo{ref to chapter
  5}).
%
We further show that such effect is concerted by the bandgap of both the
2D and bulk materials through both large-scale material database
screening.
%
More interestingly, repulsive vdW interactions are possible with careful dielectric engineering of the 2D material interface.
%
As a preliminary experimental demonstration, we show that the such
repulsive vdW forces can be indirectly probed using molecular epitaxy
on gold-supported graphene surfaced, as backed by the model presented
here.

\section{Modified Lifshitz-vdW Model For 2D Material Interfaces}
\label{sec:vdw-model-lifshitz}
% 20191014-1529
\worktodo{Add scheme here}
We start the discussion in a system with simplified geometry: consider
two infinitely-large bulk materials (A and B) separated by a medium m
at distance \(d\), the total many-body interaction potential between A
and B over m, \(\Phi\), consists of the contributions from vdW
interactions \(\Phi^{\mathrm{vdW}}\) and mono\-polar surface
interactions \(\Phi^{\pm}\)
\cite{van_Oss_1987_monopolar,Van_Oss_1988}, such that:
\begin{equation}
\label{eq:vdw-phi-oss}
\Phi = \Phi^{\mathrm{vdW}} + \Phi^{\pm}
\end{equation}
where \(\Phi^{\mathrm{vdW}}\) includes the Keesom (dipole -- dipole),
Debye (dipole -- induced dipole) and London dispersion (induced dipole
-- induced dipole) interactions~\cite{Israelachvili_2011_book}, while
\(\Phi^{\pm}\) originates from interfacial monopoles (Coulomb
interactions) \cite{van_Oss_1987_monopolar}.
%
Note that $\Phi^{\mathrm{vdW}}$ and $\Phi^{\pm}$ are all relative
energy scales compared with the infinitely-separated systems.
%
From the Lifshitz-vdW theory \cite{Dzyaloshinskii_1961_lifshitz},
\(\Phi^{\mathrm{vdW}}\) stems from the fluctuations of electromagnetic
(EM) field, and can be regarded as the sum of oscillatory surface EM
mode energies, when the length scale of interest is larger than
atomistic details.
%
When the vdW contribution dominates (\ie chemically inert surface),
and retardation effect (due to limited speed of light, usually seen at
distance $>$20 nm~\cite{parsegian_van_2010_book}) can be ignored,
$\Phi^{\mathrm{vdW}}$ is total energy summed from all allowed EM modes
\cite{Li_2005_diele}:
\begin{equation}
\label{eq:vdw-EM-energy}
\Phi^{\mathrm{vdW}} = k_{\mathrm{B}} T \sum_{j} \ln \left[2 \sinh\left(\frac{\hbar \omega_{j}(\mathbf{k})}{2 k_{\mathrm{B}} T}\right)\right] 
\end{equation}
where \(\omega_{j}\) is EM frequencies of allowed mode $j$ that
corresponds to the (i) geometry and dielectric profile of the system
and (ii) the in-plane wave vector \(\mathbf{k}\).
%
The system-specific information of the EM modes can be generalized by
writing the dispersion relation ($\mathcal{D}$)of system, such that
\(\mathcal{D}(\omega_{j}(\mathbf{k})) \equiv
0\)~\cite{Guttinger_1966_dispersion,Mahanty_1976_dispersion_book}.
%
Analyzing the mode frequency \(\omega_{j}\) is usually a non-trivial
task, while on the other hand, using the Cauchy integral theorem, the
integral in \autoref{eq:vdw-EM-energy} is equivalent to the summation
over the imaginary Matsubara frequencies $\xi$
\cite{Mahanty_1976_dispersion_book} as:
\begin{equation}
\label{eq:vdw-lifshitz-general-2}
\Phi^{\mathrm{vdW}} = \frac{1}{2} k_{\mathrm{B}} T \sum_{n=-\infty}^{\infty} \sum_{\mathbf{k}}\ln \mathcal{D}(i \xi_{n}, \mathbf{k})
= k_{\mathrm{B}} T \sideset{}{'}\sum_{n=0}^{\infty} \sum_{\mathbf{k}} \ln \mathcal{D}(i \xi_{n}, \mathbf{k})
\end{equation}
where \(k_{\mathrm{B}}\) is the Boltzmann constant, \(T\) is
temperature, \(\xi_{n} = 2 \pi n k_{\mathrm{B}} T / \hbar\) is the n-th
Matsubara frequency and the prime in the summation refers to the
prefactor of 1/2 when \(n=0\). Here we assume the dispersion relation
\(\mathcal{D}\) has time-reverse symmetry, i.e.
\(\mathcal{D}(i\xi, \mathbf{k}) = \mathcal{D}(-i\xi, \mathbf{k})\).
%
The summation over \(\mathbf{k}\) can be further carried out in
continuous integral if the system is infinitely large
%
For the 3-layer system considered here (refer to AmB in the remaining
text) that is infinite in \textit{xy}-direction while non-periodic in
\textit{z}-direction, the summation over \(\mathbf{k}\) in
\autoref{eq:vdw-lifshitz-general-2} can be written using in-plane wave
vector \(\mathbf{k} = (k_{x}, k_{y})\):
\begin{equation}
\label{eq:vdw-lifshitz-integral}
\Phi^{\mathrm{vdW}} = \frac{k_{\mathrm{B}} T}{(2 \pi)^{2}} \sideset{}{'} \sum_{n=0}^{\infty} \int_{0}^{\infty} \ln \mathcal{D}(i\xi_{n}, \mathbf{k}) \mathrm{d}^{2} \mathbf{k}
\end{equation}
%
While along the \textit{z}-direction, the electrostatic potential
\(\psi\) follows the Poisson-Laplace equation, such that:
\begin{equation}
\label{eq:vdw-poisson-laplace}
\nabla \cdot [ \varepsilon_{0} \varepsilon_{\mathrm{r}}(\mathbf{r}, z) \cdot \nabla \psi]
= -\rho(\mathbf{r}, z) = 0
\end{equation}
where \(\mathbf{r}=(x, y)\) is the in-plane coordinates,
$\varepsilon_{\mathrm{r}}$ is the dielectric tensor of each material,
and \(\rho\) is the free charge density in the system, and for van der
Waals interactions alone, \(\rho \equiv 0\).
%
The periodicity of the \textit{xy}-plane leads to the form of
plane-wave form of $\psi$~\cite{parsegian_van_2010_book}\worktodo{if
  cite is correct?}:
\begin{equation}
\label{eq:vdw-deriv-pot-pw}
\psi(\mathbf{r}, z) =  
f(z) e^{i \mathbf{k} \cdot \mathbf{r}} 
\end{equation}
where \(f(z)\) is the variation in the \emph{z}-dierction.
%
Here we use a mean-field assumption that $\varepsilon_{\mathrm{r}}$ is
uniform inside each material \worktodo{cite if this is proper}.
%
Due to the dielectric anisotropy discussed in \worktodo{ref to
  previous section}, we consider a general case, that in material $i$,
the dielectric tensor $\varepsilon_{\mathrm{i}}$ consists of different
diagonal components \(\varepsilon^{xx}\), \(\varepsilon^{yy}\) and
\(\varepsilon^{zz}\). \autoref{eq:vdw-poisson-laplace} for individual
layer now writes:
\begin{equation}
\label{eq:vdw-laplace-2}
\varepsilon_{i}^{zz} \frac{\partial^{2} f_{i}(z)}{\partial z^{2}}
- (k_{x}^{2} \varepsilon_{i}^{xx} + k_{y}^{2} \varepsilon_{i}^{yy}) f_{i}(z)
\end{equation}
which is further simplified to:
\begin{equation}
\label{eq:vdw-laplace-3}
\frac{\partial^{2} f_{i}(z)}{\partial z^{2}}
- g_{i}^{2}(\vartheta) \kappa^{2} f_{i}(z) =0
\end{equation}
where $\kappa$ and $\vartheta$ are the polar coordinates of
$\mathbf{k}$ such that
\(\mathbf{k} = (\kappa \cos\mathcal{\vartheta}, \kappa\sin
\mathcal{\vartheta})\), and
\(g_{i}^{2} = {\displaystyle
  [\frac{\varepsilon_{i}^{xx}}{\varepsilon_{i}^{zz}} \cos^{2}
  \vartheta + \frac{\varepsilon_{i}^{yy}}{\varepsilon_{i}^{zz}}
  \sin^{2} \vartheta]}\).
%
\(g_{i}^{2}\) characterizes the dielectric anisotropy of medium $i$
and upscales the wavevector modulus \(\kappa\). \worktodo{modify this
  sentence}\worktodo{need to unify the anisotropy with last chapter}

% 
Following the similar procedure of dispersion relations in isotropic
systems \cite{parsegian_van_2010_book}, the dispersion relation
\worktodo{add something in the brackets} of an AmB layered system with
dielectric anisotropy is written as:
\begin{equation}
\label{eq:vdw-disper-D}
\begin{aligned}
\mathcal{D}
&=
1 - 
\underbrace{\left[
\frac{\hat{\varepsilon}_{\mathrm{A}} - \varepsilon_{\mathrm{m}}^{zz} g_{m}(\vartheta) }{\hat{\varepsilon}_{\mathrm{A}} + \varepsilon_{\mathrm{m}}^{zz} g_{\mathrm{m}}(\vartheta)}
\right]}_{\Delta_{\mathrm{Am}}}
\underbrace{\left[
\frac{\hat{\varepsilon}_{\mathrm{B}} - \varepsilon_{\mathrm{m}}^{zz} g_{\mathrm{m}}(\vartheta) }{\hat{\varepsilon}_{\mathrm{B}} + \varepsilon_{\mathrm{m}}^{zz} g_{\mathrm{m}}(\vartheta)}
\right]}_{\Delta_{\mathrm{Bm}}}
e^{-2 g_{\mathrm{m}}(\vartheta) \kappa d} \\
&= 1 - \Delta_{\mathrm{Am}}(\vartheta) \Delta_{\mathrm{Bm}}(\vartheta) e^{-2 g_{\mathrm{m}}(\vartheta) \kappa d}
\end{aligned}
\end{equation}
where \(\hat{\varepsilon}_{i}\) is the geometric averaged dielectric
function of material $i$ ($i$=A, m or B). Combining with~\autoref{eq:vdw-lifshitz-integral}, we obtain:
\begin{equation}
\label{eq:vdw-lifshitz-aniso-final}
\begin{aligned}
\Phi^{\mathrm{vdW}}
&= \frac{k_{\mathrm{B}} T}{(2 \pi)^{2}} \sideset{}{'}\sum_{n=0}^{\infty}
\int_{0}^{2 \pi} \mathrm{d}\vartheta
\int_{0}^{\infty} \kappa \mathrm{d}\kappa 
\ln[1 - \Delta_{\mathrm{Am}}(\vartheta) 
\Delta_{\mathrm{Bm}}(\vartheta) e^{-2 g_{\mathrm{m}}(\vartheta) \kappa d}] \\
&= \frac{k_{\mathrm{B}} T}{16 \pi^{2} d^{2}}
\sideset{}{'}\sum_{n=0}^{\infty} \int_{0}^{2 \pi} 
\frac{1}{g^{2}_{\mathrm{m}}(\vartheta)}\mathrm{d}\vartheta
\int_{0}^{\infty} x \mathrm{d}x
\ln[1 - \Delta_{\mathrm{Am}}(\vartheta) \Delta_{\mathrm{Bm}}(\vartheta) e^{-x}] \\
\end{aligned}
\end{equation}
where the last step is obtained by auxiliary variable \(x = 2
g_{\mathrm{m}}(\mathcal{\theta}) \kappa d\).
%
\worktodo{add figure about convergence}

% As can be seen, only the dielectric anisotropy of the medium
% ($1/g_{\mathrm{m}}^{2}$) affects the total interaction energy.
% Following the discussions in \worktodo{ref to dielectric chapter, anisotropy}, 

% The integrant with \(x\) is
% dominated
% by the Due to the fact that \(g_{\mathrm{m}}\) is usually much
% larger than unity for a 2D material medium, the integrant is governed
% by the range \(x \leq 5\) (see Figure \ref{fig:integral}), and
% corresponding to the regime of \(\kappa \leq 2.5 (g_{\mathrm{m}}
% d)^{-1}\). Since in this letter, the distance \(d\) is generally at the
% order of 2 nm, with \(g_{\mathrm{m}}\) = 2.5 (typical value for
% dielectric anisotropy), we see that the majority of interaction comes
% from oscillations with \(\kappa<\) 0.0625 \(\mathrm{\AA{}}^{-1}\),
% which is close to the optical limit, and therefore
% \(\Delta_{\mathrm{Am}}\) and \(\Delta_{\mathrm{Bm}}\) can be regarded as
% constant within the regime of integration.



% To be modified later!
\begin{equation}
\label{eq:vdw-vdw-vdw-Phi-EM}
\Delta \Phi \approx \Delta \Phi^{\mathrm{vdW}}
= \sum_{n} G(i \xi_{n})
= \frac{k_{\mathrm{B}}T}{4 \pi^{2}}   
 \sideset{}{'}\sum_{n=0}^{\infty} \int_{0}^{\infty} \mathcal{D}(i \xi_{n}, \mathbf{k}) 
\mathrm{d}^{2} \mathbf{k}
\end{equation}

\(\xi_{n} = 2 \pi n k_{\mathrm{B}} T / \hbar\) is the n-th Matsubara
frequency, \(\hbar\) is the reduced Planck constant, \(\mathbf{k}=(k_{x},
k_{y})\) is the transverse wavevector, and the prime over summation
denotes a prefactor of 1/2 when \(n=0\). \(\mathcal{D}\) is the dispersion
relation of the particular system and geometry that for any allowed
complex frequency \(\omega\) and wavevector \(\mathbf{k}\),
\(\mathcal{D}(\omega, \mathbf{k}) \equiv 0\). The dispersion relation
constructed by solving the Poisson-Laplace equation \(\nabla \cdot
[\varepsilon_{0} \varepsilon_{\mathrm{r}}(\mathbf{r}) \cdot \nabla
\psi(\mathbf{r})] = 0\), where \(\varepsilon_{0}\) is the vacuum
permittivity, \(\varepsilon_{\mathrm{r}}(\mathbf{r})\) and
\(\psi(\mathbf{r})\) are the dielectric function and electrostatic
potential at position \(\mathbf{r}\), respectively, and is highly
influenced by the dielectric anisotropy of media \cite{Lu_2016_torque}.
For materials have distinct in- and out-of-plane dielectric functions
(\(\varepsilon^{xx} = \varepsilon^{yy} = \varepsilon^{\parallel}\),
\(\varepsilon^{zz} = \varepsilon^{\perp}\)) , the total interaction
energy is expressed as (details see Supplementary Section
\ref{SI-sec:deriv-mdb}):
\begin{equation}
\label{eq:vdw-vdw-vdw-Phi-aniso}
\Delta \Phi = \frac{k_{\mathrm{B}} T}{8 \pi d^{2}} 
\sideset{}{'} \sum_{n = 0}^{\infty} \frac{1}{g_{\mathrm{m}}^{2}(i \xi_{n})}
\int_{0}^{\infty} x \ln\left[1 - \Delta_{\mathrm{Am}}(i \xi_{n}) \Delta_{\mathrm{Bm}}(i \xi_{n}) e^{-x}\right] \mathrm{d} x 
\end{equation}
where \(\Delta_{\mathrm{jm}} = (\hat{\varepsilon}_{\mathrm{j}} -
\hat{\varepsilon}_{\mathrm{m}}) / (\hat{\varepsilon}_{\mathrm{j}} +
\hat{\varepsilon}_{\mathrm{m}})\) represents the interfacial dielectric
mismatch between materials j (either A or B) and m, and
\(\hat{\varepsilon} = \sqrt{\varepsilon^{\parallel}
\varepsilon^{\perp}}\) is the geometric average of dielectric tensor. A
renormalization factor \(1/g_{\mathrm{m}}^{2} =
\varepsilon^{\perp}_{\mathrm{m}} /
\varepsilon^{\parallel}_{\mathrm{m}}\) emerges as the consequence of
dielectric anisotropy of medium m. It is easy to observe, when
\(\varepsilon^{\perp}_{\mathrm{m}} \ll
\varepsilon^{\parallel}_{\mathrm{m}}\), \(1/g_{\mathrm{m}}^{2} \ll 1\),
which greatly attenuates the total interaction \(\Delta \Phi\) compared
with isotropic medium (\(1/g_{\mathrm{m}}^{2} \equiv 1\)). 



%%% Local Variables:
%%% mode: latex
%%% TeX-master: "../thesis"
%%% End:
