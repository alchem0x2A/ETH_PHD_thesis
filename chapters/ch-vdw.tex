% \chapter{Introduction}
\chapter{Many-Body Transmission of vdW Interactions Through 2D Materials}
\label{ch:vdw}
\renewcommand*\imgdir{img/vdw/}

\dictum[Richard Feynman]{%
  Nature isn't classical, and if you want to make a simulation
  of nature, you'd better make it quantum mechanical, and by golly
  it's a wonderful problem, because it doesn't look so easy.  }%
\worktodo{find the quote or quit}

\vspace{1em}

\chapterabstract{Part of this chapter appears in the following work:
  Tian, T., Naef, F., Lee, Y.-T., Zhang, S.-W., Chiu, Y.-C. \& Shih,
  C.-J. Two-Dimensional Materials as High-Pass Filter of van der Waals
  Interactions, in preparation.
}

\section{Introduction}
\label{sec:vdw-introduction}
% 20191014-1539
The van der Waals (vdW) interactions are inter\-atomic /
inter\-molecular forces that stem from the electromagnetic (EM)
fluctuations~\cite{Israelachvili_2011_book,Woods_2016_rev_vdw,parsegian_van_2010_book}.
%
Unlike trivial electrostatic (Coulombic) forces that depends on the
charge stat, the vdW interaction has a complex electro\-dynamic
origin, and acts ubiquitously between all atoms and molecules,
regardless of their charge
state~\cite{Israelachvili_2011_book,Hermann_2017_vdW_rev}.
%
As briefly discussed in \autoref{sec:inter-forc-at}, although being a
relatively weak force, the vdW interactions can be dominating at
molecule-2D interface due to the absence of dangling bonds.
%
The understanding and engineering of the vdW interactions at the 2D
material interface would greatly benefit molecular
self-assembly~\cite{Kumar_2017_rev_assemb_2D},
electronic~\cite{Wang_2015_phys_chem_tuning_TMDC,Lazar_2013} as well
as optical properties~\cite{Geim_2013_2D_vdw_Het,Novoselov_2016_vdW}
of 2D material-based hetero\-structures.
%
\worktodo{try to say more about the dielectric properties}

As introduced in \autoref{sec:van-der-waals}, one particular question
of interest is the transmission of vdW interactions through the
atomically-thin 2D materials.
%
The effective distance of vdW forces, depending on the medium, can
vary from sub-nm 10$^{1}$ nm~\cite{Israelachvili_2011_book}, and are
relatively longer compared with the sub-nm thickness of 2D materials.
%
Therefore, the vdW forces from an macroscopic substrate may
(partially) penetrate through the 2D layer and be felt by an molecule
on the 2D material interface~\cite{shih_2013_wetting_natmat}.
%
This feature
brings potentially unprecedented surface properties by simply coating
an existing bulk surface with 2D
material(s)~\cite{Prasai_2012_coating,rafiee_2012_transparency,Tsoi_2014_vdW_screening_2D}.
%
Based on classic pair-wise model~\cite{Hamaker_1937_vdW} of vdW
forces, the interaction free energy per area $\Phi_{\mathrm{vdW}}$,
between two semi-infinite bulk bodies (A and B) separated by distance
$d$, is:
\begin{equation}
  \label{eq:vdw-hamaker-res}
  \Phi^{\mathrm{vdW}} = - \frac{A_{\mathrm{AB}}}{12 \pi d^{2}}
\end{equation}
where $A_{\mathrm{AB}}$ is known as the Hamaker constant of vdW
interactions between materials A and B.
%
A quantitative statement can be drawn from such simple model: by inserting a 2D material (\eg graphene
with thickness $\sim{}$3.3 \AA{}~\cite{Shearer_2016}) between two bulk
bodies (3$\sim{}$5 \AA{} \worktodo{cite 1 here}), 20--40\% of the vdW
interactions can still penetrate through the 2D layer.
%
Several unprecedented phenomena concerning the interfacial vdW forces
on 2D materials take credit from the analysis above, including wetting
transparency / translucency
\cite{Shih_2012_prl,rafiee_2012_transparency,Gurarslan_2016_screen_MoS2} and remote
epitaxy \cite{Kim_2017_remote_epi_Gr,Kong_2018_pol}.

Despite the seemingly accordance with experimental observations in
these studies, the prerequisites for the classical model are the
perfect additivity of vdW interactions, and a constant
$A_{\mathrm{AB}}$. 
%
In other words, the classic model assumes that the existence of the
2DEG has no influence on the EM waves that arise from charge
fluctuations.
%
Such picture is however challenged by recent experimental
\cite{Tsoi_2014_vdW_screening_2D} and first-principles
\cite{Ambrosetti_2018_carbon,Liu_2018_gr,Li_2018_screen} studies,
revealing the 2D material-dependent Hamaker coefficient.
%
To correctly capture the many-body picture of vdW interactions at 2D
material interfaces, electro\-dynamic effects beyond the classic model
is needed.
%
In practice, this is usually a challenging task, since state-of-art
density functional theory (DFT) used for first principle simulations
is limited ground state electron density, intrinsically lacking charge
fluctuation~\cite{Woods_2016_rev_vdw}.
%
On the other hand, full-scale interaction energy calculations based on
dynamic non-local dielectric responses are usually time-consuming and
only limited to small
systems.~\cite{Hermann_2017_vdW_rev,Zhou_2017_lifshitz}
%
Therefore, novel theoretical frameworks with both appreciable accuracy
while maintaining simplicity, is of high demand to shed light on the
vdW interactions at 2D material interfaces.

In this chapter, we present
the a generalized model to study the many-body vdW transmission
through 2D materials, based on the modified Lifshitz-vdW theory
\cite{Dzyaloshinskii_1961_lifshitz,parsegian_van_2010_book}.
%
Based on the model, the vdW transmission through a 2D material is
selective, such that a 2D material acts as a high-pass filter of vdW
interaction, filtering the low-frequency interactions.
%
Such phenomenon is related with the anisotropic dielectric nature (\worktodo{ref to chapter
  5}).
%
We further show that such effect is concerted by the bandgap of both the
2D and bulk materials through both large-scale material database
screening.
%
More interestingly, repulsive vdW interactions are possible with careful dielectric engineering of the 2D material interface.
%
As a preliminary experimental demonstration, we show that the such
repulsive vdW forces can be indirectly probed using molecular epitaxy
on gold-supported graphene surfaced, as backed by the model presented
here.

\section{Modified Lifshitz-vdW Model For 2D Material Interfaces}
\label{sec:vdw-model-lifshitz}
% 20191014-1529

\subsection{Generalized Model For Layered Anisotropic Media}
\label{sec:vdw-gener-model-layer}

\worktodo{Add scheme here}
We start the discussion in a system with simplified geometry: consider
two infinitely-large bulk materials (A and B) separated by a medium m
at distance \(d\), the total many-body interaction potential between A
and B over m, \(\Phi\), consists of the contributions from vdW
interactions \(\Phi^{\mathrm{vdW}}\) and mono\-polar surface
interactions \(\Phi^{\pm}\)
\cite{van_Oss_1987_monopolar,Van_Oss_1988}, such that:
\begin{equation}
\label{eq:vdw-phi-oss}
\Phi = \Phi^{\mathrm{vdW}} + \Phi^{\pm}
\end{equation}
where \(\Phi^{\mathrm{vdW}}\) includes the Keesom (dipole -- dipole),
Debye (dipole -- induced dipole) and London dispersion (induced dipole
-- induced dipole) interactions~\cite{Israelachvili_2011_book}, while
\(\Phi^{\pm}\) originates from interfacial monopoles (Coulomb
interactions) \cite{van_Oss_1987_monopolar}.
%
Note that $\Phi^{\mathrm{vdW}}$ and $\Phi^{\pm}$ are all relative
energy scales compared with the infinitely-separated systems.
%
From the Lifshitz-vdW theory \cite{Dzyaloshinskii_1961_lifshitz},
\(\Phi^{\mathrm{vdW}}\) stems from the fluctuations of electromagnetic
(EM) field, and can be regarded as the sum of oscillatory surface EM
mode energies, when the length scale of interest is larger than
atomistic details.
%
When the vdW contribution dominates (\ie chemically inert surface),
and retardation effect (due to limited speed of light, usually seen at
distance $>$20 nm~\cite{parsegian_van_2010_book}) can be ignored,
$\Phi^{\mathrm{vdW}}$ is total energy summed from all allowed EM modes
\cite{Li_2005_diele}:
\begin{equation}
\label{eq:vdw-EM-energy}
\Phi^{\mathrm{vdW}} = k_{\mathrm{B}} T \sum_{j} \ln \left[2 \sinh\left(\frac{\hbar \omega_{j}(\mathbf{k})}{2 k_{\mathrm{B}} T}\right)\right] 
\end{equation}
where \(\omega_{j}\) is EM frequencies of allowed mode $j$ that
corresponds to the (i) geometry and dielectric profile of the system
and (ii) the in-plane wave vector \(\mathbf{k}\).
%
The system-specific information of the EM modes can be generalized by
writing the dispersion relation ($\mathcal{D}$)of system, such that
\(\mathcal{D}(\omega_{j}(\mathbf{k})) \equiv
0\)~\cite{Guttinger_1966_dispersion,Mahanty_1976_dispersion_book}.
%
Analyzing the mode frequency \(\omega_{j}\) is usually a non-trivial
task, while on the other hand, using the Cauchy integral theorem, the
integral in \autoref{eq:vdw-EM-energy} is equivalent to the summation
over the imaginary Matsubara frequencies $\xi$
\cite{Mahanty_1976_dispersion_book} as:
\begin{equation}
\label{eq:vdw-lifshitz-general-2}
\Phi^{\mathrm{vdW}} = \frac{1}{2} k_{\mathrm{B}} T \sum_{n=-\infty}^{\infty} \sum_{\mathbf{k}}\ln \mathcal{D}(i \xi_{n}, \mathbf{k})
= k_{\mathrm{B}} T \sideset{}{'}\sum_{n=0}^{\infty} \sum_{\mathbf{k}} \ln \mathcal{D}(i \xi_{n}, \mathbf{k})
\end{equation}
where \(k_{\mathrm{B}}\) is the Boltzmann constant, \(T\) is
temperature, \(\xi_{n} = 2 \pi n k_{\mathrm{B}} T / \hbar\) is the n-th
Matsubara frequency and the prime in the summation refers to the
prefactor of 1/2 when \(n=0\). Here we assume the dispersion relation
\(\mathcal{D}\) has time-reverse symmetry, i.e.
\(\mathcal{D}(i\xi, \mathbf{k}) = \mathcal{D}(-i\xi, \mathbf{k})\).
%
The summation over \(\mathbf{k}\) can be further carried out in
continuous integral if the system is infinitely large
%
For the 3-layer system considered here (refer to AmB in the remaining
text) that is infinite in \textit{xy}-direction while non-periodic in
\textit{z}-direction, the summation over \(\mathbf{k}\) in
\autoref{eq:vdw-lifshitz-general-2} can be written using in-plane wave
vector \(\mathbf{k} = (k_{x}, k_{y})\):
\begin{equation}
\label{eq:vdw-lifshitz-integral}
\Phi^{\mathrm{vdW}} = \frac{k_{\mathrm{B}} T}{(2 \pi)^{2}} \sideset{}{'} \sum_{n=0}^{\infty} \int_{0}^{\infty} \ln \mathcal{D}(i\xi_{n}, \mathbf{k}) \mathrm{d}^{2} \mathbf{k}
\end{equation}
%
While along the \textit{z}-direction, the electrostatic potential
\(\psi\) follows the Poisson-Laplace equation, such that:
\begin{equation}
\label{eq:vdw-poisson-laplace}
\nabla \cdot [ \varepsilon_{0} \varepsilon_{\mathrm{r}}(\mathbf{r}, z) \cdot \nabla \psi]
= -\rho(\mathbf{r}, z) = 0
\end{equation}
where \(\mathbf{r}=(x, y)\) is the in-plane coordinates,
$\varepsilon_{\mathrm{r}}$ is the dielectric tensor of each material,
and \(\rho\) is the free charge density in the system, and for van der
Waals interactions alone, \(\rho \equiv 0\).
%
The periodicity of the \textit{xy}-plane leads to the form of
plane-wave form of $\psi$~\cite{parsegian_van_2010_book}\worktodo{if
  cite is correct?}:
\begin{equation}
\label{eq:vdw-deriv-pot-pw}
\psi(\mathbf{r}, z) =  
f(z) e^{i \mathbf{k} \cdot \mathbf{r}} 
\end{equation}
where \(f(z)\) is the variation in the \emph{z}-dierction.
%
Here we use a mean-field assumption that $\varepsilon_{\mathrm{r}}$ is
uniform inside each material \worktodo{cite if this is proper}.
%
Due to the dielectric anisotropy discussed in \worktodo{ref to
  previous section}, we consider a general case, that in material $i$,
the dielectric tensor $\varepsilon_{\mathrm{i}}$ consists of different
diagonal components \(\varepsilon^{xx}\), \(\varepsilon^{yy}\) and
\(\varepsilon^{zz}\). \autoref{eq:vdw-poisson-laplace} for individual
layer now writes:
\begin{equation}
\label{eq:vdw-laplace-2}
\varepsilon_{i}^{zz} \frac{\partial^{2} f_{i}(z)}{\partial z^{2}}
- (k_{x}^{2} \varepsilon_{i}^{xx} + k_{y}^{2} \varepsilon_{i}^{yy}) f_{i}(z)
\end{equation}
which is further simplified to:
\begin{equation}
\label{eq:vdw-laplace-3}
\frac{\partial^{2} f_{i}(z)}{\partial z^{2}}
- g_{i}^{2}(\vartheta) \kappa^{2} f_{i}(z) =0
\end{equation}
where $\kappa$ and $\vartheta$ are the polar coordinates of
$\mathbf{k}$ such that
\(\mathbf{k} = (\kappa \cos\mathcal{\vartheta}, \kappa\sin
\mathcal{\vartheta})\), and
\(g_{i}^{2} = {\displaystyle
  [\frac{\varepsilon_{i}^{xx}}{\varepsilon_{i}^{zz}} \cos^{2}
  \vartheta + \frac{\varepsilon_{i}^{yy}}{\varepsilon_{i}^{zz}}
  \sin^{2} \vartheta]}\).
%
\(g_{i}^{2}\) characterizes the dielectric anisotropy of medium $i$
and upscales the wavevector modulus \(\kappa\). Note for an isotropic
medium, $g_{\mathrm{m}}\equiv1$. \worktodo{modify this
  sentence}\worktodo{need to unify the anisotropy with last chapter}

% 
Following the similar procedure of dispersion relations in isotropic
systems \cite{parsegian_van_2010_book}, the dispersion relation
\worktodo{add something in the brackets} of an AmB layered system with
dielectric anisotropy is written as:
\begin{equation}
\label{eq:vdw-disper-D}
\begin{aligned}
\mathcal{D}
&=
1 - 
\underbrace{\left[
\frac{\hat{\varepsilon}_{\mathrm{A}} - \varepsilon_{\mathrm{m}}^{zz} g_{m}(\vartheta) }{\hat{\varepsilon}_{\mathrm{A}} + \varepsilon_{\mathrm{m}}^{zz} g_{\mathrm{m}}(\vartheta)}
\right]}_{\Delta_{\mathrm{Am}}}
\underbrace{\left[
\frac{\hat{\varepsilon}_{\mathrm{B}} - \varepsilon_{\mathrm{m}}^{zz} g_{\mathrm{m}}(\vartheta) }{\hat{\varepsilon}_{\mathrm{B}} + \varepsilon_{\mathrm{m}}^{zz} g_{\mathrm{m}}(\vartheta)}
\right]}_{\Delta_{\mathrm{Bm}}}
e^{-2 g_{\mathrm{m}}(\vartheta) \kappa d} \\
&= 1 - \Delta_{\mathrm{Am}}(\vartheta) \Delta_{\mathrm{Bm}}(\vartheta) e^{-2 g_{\mathrm{m}}(\vartheta) \kappa d}
\end{aligned}
\end{equation}
where \(\hat{\varepsilon}_{i}\) is the geometric averaged dielectric
function of material $i$ ($i$=A, m or B), $\Delta_{\mathrm{Am}}$ and
$\Delta_{\mathrm{Bm}}$ are the dielectric mismatch between Am and Bm
interfaces, respectively.  Combining
with~\autoref{eq:vdw-lifshitz-integral}, we obtain:
\begin{equation}
\label{eq:vdw-lifshitz-aniso-final}
\begin{aligned}
\Phi^{\mathrm{vdW}}
&= \frac{k_{\mathrm{B}} T}{(2 \pi)^{2}} \sideset{}{'}\sum_{n=0}^{\infty}
\int_{0}^{2 \pi} \mathrm{d}\vartheta
\int_{0}^{\infty} \kappa \mathrm{d}\kappa 
\ln[1 - \Delta_{\mathrm{Am}}(\vartheta) 
\Delta_{\mathrm{Bm}}(i\xi_{n},\vartheta) e^{-2 g_{\mathrm{m}}(i\xi_{n},\vartheta) \kappa d}] \\
&= \frac{k_{\mathrm{B}} T}{16 \pi^{2} d^{2}}
\sideset{}{'}\sum_{n=0}^{\infty} \int_{0}^{2 \pi} 
\frac{1}{g^{2}_{\mathrm{m}}(i\xi_{n},\vartheta)}\mathrm{d}\vartheta
\int_{0}^{\infty} x \mathrm{d}x
\ln[1 - \Delta_{\mathrm{Am}}(i\xi_{n},\vartheta) \Delta_{\mathrm{Bm}}(i\xi_{n},\vartheta) e^{-x}] \\
\end{aligned}
\end{equation}
where the last step is obtained using an auxiliary variable \(x = 2
g_{\mathrm{m}}(\mathcal{\theta}) \kappa d\).
%
\worktodo{add figure about convergence}
%
The integral in \autoref{eq:vdw-lifshitz-aniso-final} is dominated by
the contributions from $x < 5$, and its convergence is almost
independent of $\Delta_{\mathrm{Am}}$ or $\Delta_{\mathrm{Bm}}$. In
other words, the contributions from high-\textbf{k} regime to the
total vdW interactions, is negligible.  \worktodo{think about this
  figure}

\begin{figure}[h]
  \centering{}
  \import{\imgdir}{interg_xx.pgf}
  \caption{\label{fig:vdw-integeral-x}%
    Modeled integral \worktodo{more to say}
  }
\end{figure}

An important feature of \autoref{eq:vdw-lifshitz-aniso-final} is that
the total energy $\Phi^{\mathrm{vdW}}$ can be greatly attenuated
through a medium with high dielectric anisotropy (\ie
$g_{\mathrm{m}} \gg 1$).
%
Such high dielectric anisotropy between in- and out-of-plane
permittivities (giant birefringence) is only possible to achieve using
layered \cite{Collin_1958_aniso,Weber_2000_aniso} or low-dimensional
\cite{Niu_2018_aniso,Segura_2018_aniso} materials. Inspired by this,
by inserting a 2D material between bulk materials A and B, the
many-body \worktodo{hyphen or not?} vdW interactions may also be
screened by the high dielectric anisotropy, which contradicts with the classical theory of vdW interactions. \worktodo{another subfigure}
%
In the next section, we extend the modified Lifshitz-vdW model in
order to be applied to 2D material-containing systems.


\subsection{Applying Model to 2D Material Systems}
\label{sec:vdw-model-2D}
% 20191014-2027

The general model presented in \autoref{sec:vdw-model-2D} can be
extended to model the many body vdW transmission through 2D material
systems.
%
The exact solution to the many-body vdW interaction energy of such
mixed-dimensional system requires the knowledge of the full
spatially-varied permittivity profile \cite{Podgornik_2004_continuum},
which can be cumbersome to compute.
%
Instead, we use a mean-field treatment for
the dielectric response of the medium, Inspired by the concept in
\worktodo{ref to previous section}, if medium m consists of a layer of
2D sheet and surrounding vacuum, the average dielectric response
inside the medium m can be calculated using the effective medium
approach, using the 2D polarizability. \worktodo{ref to the equation}
%
Note here two assumptions are made: (i) the dielectric properties of
the 2D material is not influenced by the bulk material (\ie no
substrate doping and interfacial coupling) and (ii) the length scale
of the gap ($d$) is larger than the intrinsic thickness of the 2D
material, such that an uniform description of $\varepsilon_{\mathrm{m}}$ is acceptable.
%
Since both criteria can be fulfilled at larger separation
distance~\cite{Dobson_2012_rev} (\eg $d>$1 nm), when the overlapping
between electron clouds of individual materials are negligible, we
limit our discussions in this regime throughout this chapter.
%
Moreover, As discussed above, the total vdW interaction energy is dominated by
the EM modes at $x \leq 5$, or equivalently
\(\kappa \leq 2.5 (g_{\mathrm{m}} d)^{-1}\).
%
When $d$ is at the order of 2 nm, and the typical anisotropy value for
a 2D material $g_{\mathrm{m}}=2.5$, we see that the majority of interaction comes
from EM modes with \(\kappa<0.0625\) \AA{}\textsuperscript{-1}\worktodo{verify the numbers}, close to the optical limit ($k\to0$).
%
Within this regime, $\alpha_{\mathrm{2D}}$ of the 2D material as well
as $\varepsilon_{\mathrm{A}}$ and $\varepsilon_{\mathrm{B}}$ of bulk
materials can be regarded as
constant~\cite{Li_2005_diele}.
%
In other words, the calculation of $\Phi^{\mathrm{vdW}}$ only requires
dielectric data at the optical limit, which greatly simplifies the
computational expense.\worktodo{refine sentence, fine 1 more citation}

A large variety of 2D materials (\eg materials with P6/mmm,
P$\overline{6}$m2, P$\overline{3}$m1, \worktodo{cite more symmetries}
symmetries) has isotropic in-plane electronic polarizabilities (\ie
$\alpha^{xx}_{\mathrm{2D}} = \alpha_{\mathrm{2D}}^{yy} = \alpha_{\mathrm{2D}}^{\parallel}$) due to
symmetric electronic structure. For an AmB system constructed by these
materials, we have
$\varepsilon_{\mathrm{m}}^{xx} =
\varepsilon_{\mathrm{,}}^{yy}=\varepsilon_{\mathrm{m}}^{\parallel}$,
$\varepsilon_{\mathrm{m}}^{zz} = \varepsilon_{\mathrm{2D}}^{\perp}$ ,
and $g_{\mathrm{m}}$ independent of $\vartheta$. As a result,
\autoref{eq:vdw-lifshitz-aniso-final} can be simplified to:
\begin{equation}
  \begin{aligned}[t]
\label{eq:vdw-Phi-aniso-main}
\Phi^{\mathrm{vdW}} &= \frac{k_{\mathrm{B}} T}{8 \pi d^{2}} 
\sideset{}{'} \sum_{n = 0}^{\infty} \frac{1}{g_{\mathrm{m}}^{2}(i \xi_{n})}
\int_{0}^{\infty} x \ln\left[1 - \hat{\Delta}_{\mathrm{Am}}(i \xi_{n}) \hat{\Delta}_{\mathrm{Bm}}(i \xi_{n}) e^{-x}\right] \mathrm{d} x \\
\hat{\Delta}_{\mathrm{jm}}(i\xi) &= \frac{\hat{\varepsilon}_{\mathrm{j}}(i\xi_{n}) -
\hat{\varepsilon}_{\mathrm{m}}(i\xi_{n})}{\hat{\varepsilon}_{\mathrm{j}}(i\xi_{n}) +
\hat{\varepsilon}_{\mathrm{m}}(i\xi_{n})},\ \mathrm{j=A, B}
\end{aligned}
\end{equation}
where
\(\hat{\Delta}_{\mathrm{jm}}\) represents the
$\vartheta$-independent interfacial dielectric mismatch between
materials $j$ (either A or B) and m.
%
Alternatively, \autoref{eq:vdw-Phi-aniso-main} can also be expressed
using single-frequency interaction energy $G(i\xi_{n})$ as:
\begin{equation}
\begin{aligned}[t]
\label{eq:vdw-Phi-single-point}
\Phi^{\mathrm{vdW}} &= \sideset{}{'} \sum_{n=0}^{\infty} G(i \xi_{n}) \\
G(i \xi_{n}) &= \frac{k_{\mathrm{B}} T}{8 \pi d^{2}} \frac{1}{g_{\mathrm{m}}^{2}(i \xi_{n})}
\int_{0}^{\infty} x \ln\left[1 - \hat{\Delta}_{\mathrm{Am}}(i \xi_{n}) \hat{\Delta}_{\mathrm{Bm}}(i \xi_{n}) e^{-x}\right] \mathrm{d} x
\end{aligned}
\end{equation} 
\autoref{eq:vdw-Phi-single-point} is the main equation for the calculation
of $\Phi^{\mathrm{vdW}}$ in this chapter. As can be seen, it is
correlated with only a few frequency-dependent dielectric properties
$\varepsilon_{\mathrm{A}}$, $\varepsilon_{\mathrm{B}}$ and
$\alpha_{\mathrm{2D}}$ \worktodo{optic or optical?}, as well as distance $d$.


The dielectric function of bulk materials are obtained from both
experimental spectroscopy data~\cite{Palik_1998_handbook} and first
principles simulations.
%
On the other hand, the electronic polarizability of 2D
materials are solely from first principles simulations due to the
problem with experimentally measured optical properties of 2D
materials as mention in \worktodo{ref to previous section}.
%
Note in the framework of Lifshitz-vdW model, the dielectric responses
are sampled on the imaginary frequencies ($i\xi$), while the data
obtained from optical measurements and first principles calculations
are usually on real frequencies ($\omega$). The real and imaginary
frequencies can be unified by the complex frequency
$\omega_{\mathrm{C}} = \omega + i\xi$~\cite{parsegian_van_2010_book}
\worktodo{a better cite than this?}.
%
The conversion between $\varepsilon$ and $\alpha_{\mathrm{2D}}$ from
real frequencies to imaginary frequencies are performed using the
Kramers-Kronig relation (KKR)~\cite{Roessler_1965_KKR}, such that:
\begin{eqnarray}
  \label{eq:vdw-KKR-eps}
  \varepsilon(i\xi) &= 1 + {\displaystyle \frac{2}{\pi}}{\displaystyle \int_{0}^{\infty}} {\displaystyle \frac{\omega \mathrm{Im}(\varepsilon(\omega))}{\omega^{2} + \xi^{2}}} \mathrm{d}\omega \\
  \label{eq:vdw-KKR-alpha}
  \alpha_{\mathrm{2D}}(i\xi) &= {\displaystyle \frac{2}{\pi}} {\displaystyle \int_{0}^{\infty}} {\displaystyle \frac{\omega \mathrm{Im}(\alpha_{\mathrm{2D}}(\omega))}{\omega^{2} + \xi^{2}}} \mathrm{d}\omega
\end{eqnarray}
%
More details about the data acquisition, first principles simulations
and the computation details of the modified Lifshitz model, are
discussed in \autoref{sec:vdw-methods}.  \worktodo{show example of KKR
  conversion} \worktodo{note different between Matsubara and
  continuous}

\begin{figure}[h]
  \centering{}
  \import{\imgdir}{eps_kkr.pgf}
  \caption{\label{fig:vdw-compare-eps-kkr} %
    Comparison between real and imag for hBN \worktodo{more?}
  }
\end{figure}


\subsection{High-Pass vdW Transmission Through 2D Materials}
\label{sec:vdw-high-pass-vdw}

% \subsection{Effect of Dielectric Anisotropy in 2D Material-Containing Medium}
% \label{sec:vdw-effect-diel-anis}

Compared with $\varepsilon(\omega)$ that has a complex value and
multiple transition peaks \worktodo{better choice of words?},
$\varepsilon(i \xi)$ after the KKR are monotonically decaying,
real-value functions, which greatly simplifies the analysis in the Lifshitz-vdW model.
%
As an example, the in- and out-of-plane dielectric functions
$\varepsilon_{\mathrm{m}}^{\parallel}(i \xi)$ and
$\varepsilon_{\mathrm{m}}^{\perp}(i \xi)$ for \worktodo{79} type of 2D
materials in the 2D material-containing medium m at $d=2$ nm, are
shown in \worktodo{figure}, respectively.
%

\begin{figure}[h]
  \centering
  \caption{\label{fig:vdw-eps-iv-all} %
    \worktodo{Add caption}
  }
\end{figure}
For illustration purpose, the data are shown on continuous frequency
domain instead of discrete Matsubara frequencies. Moreover, the curves
of graphene, 2H-MoS\textsubscript{2}, and hBN are
highlighted. \worktodo{better wording}
%
We observe that for a typical 2D medium,
\(\varepsilon_{\mathrm{m}}^{\parallel}\) greatly exceeds
\(\varepsilon_{\mathrm{m}}^{\perp}\) at low frequency range ($\xi <$
10 eV), while the difference diminishes at high frequency regime, when
$\varepsilon_{\mathrm{m}}^{\parallel}$ and
$\varepsilon_{\mathrm{m}}^{\perp}$ approach unity.
%
To characterize the decaying behavior, we introduced the in- and
out-of-plane transition frequencies \(\xi_{\mathrm{tr}}^{\parallel}\)
and \(\xi_{\mathrm{tr}}^{\perp}\), defined as the frequency that
\(\varepsilon_{\mathrm{m}}^{p}(i \xi^{p}_{\mathrm{tr}}) - 1 =
\frac{1}{2}[\varepsilon_{\mathrm{m}}^{p}(i\xi=0) - 1]\), where
$p=\parallel$ or $\perp$.
%

The values of $\xi_{\mathrm{tr}}^{\parallel}$ and
$\xi_{\mathrm{tr}}^{\parallel}$ differ greatly between in- and
out-of-plane directions.
%
Firstly, \(\xi_{\mathrm{tr}}^{\parallel}\) is generally smaller than
\(\xi_{\mathrm{tr}}^{\perp}\), which is related with the quantum
confinement perpendicular to the 2D
plane~\cite{Matthes_2016_effective_PRB}.
%
Moreover, we found that $\xi_{\mathrm{tr}}^{\parallel}$ almost
linearly scales with the bandgap of the 2D material
$E_{\mathrm{g}}^{\mathrm{2D}}$, while $\xi_{\mathrm{tr}}^{\perp}$ is
almost independent of $E_{\mathrm{g}}^{\mathrm{2D}}$. \worktodo{figure insets}
%
Since the frequency-dependent $\varepsilon_{\mathrm{m}}^{\perp}$ varies
much less than $\varepsilon_{\mathrm{m}}^{\parallel}$, the behavior of
\(1/g_{\mathrm{m}}^{2}\) is dominated by
$\varepsilon_{\mathrm{m}}^{\parallel}$, as shown in
\worktodo{figure}.
%
Considering the fact that $1/g_{\mathrm{m}}^{2}$ modulates the
magnitudes of $\Phi^{\mathrm{vdW}}$ at each frequency (see
\autoref{eq:vdw-lifshitz-aniso-final}), its behavior is analogous to a
high-pass band filter~\worktodo{cite 1} \worktodo{think of using G for frequency
  part?}
%
In other words, the vdW interactions are selectively screened at low
frequencies, which arises from the dielectric anisotropy of 2D
materials.
%
More interestingly, the minimum value of $1/g_{\mathrm{m}}^{2}$ always
occurs at $\xi = 0$ due to its monotonic shape.
%
From the analysis in \worktodo{ref to dielectric section},
$1/g_{\mathrm{m}}^{2}(\xi=0)=\varepsilon_{\mathrm{m}}^{\perp} /
\varepsilon_{\mathrm{m}}^{\parallel}$ is dominated by
$E_{\mathrm{g}}^{\mathrm{2D}}$.
% Combine with the fact that
% $\xi_{\mathrm{tr}}^{\parallel}$
\worktodo{mention more about the $\xi_{\mathrm{tr}}$?}
%
As a result, the penetration of vdW interactions is expected to be
more screened by a 2D material with smaller bandgap, which is in good
agreement with recent full-scale \textit{ab initio}
studies~\cite{Liu_2018_gr}.
%
The mechanism behind such high-pass vdW transmission phenomena, is
schematically shown in \worktodo{figure c}.

% 20191015-1415
The model proposed here is capable to explain the
strong vdW force screening by 2D materials observed in AFM experiments
\cite{Tsoi_2014}.
%
To simulate the experimental conditions, we model the
many-body vdW interaction between SiO\textsubscript{2} (A) and Si\textsubscript{3}N\textsubscript{4} (B), by
varying the 2D materials (m).
% according to Equation \ref{eq:Phi-aniso}.
\worktodo{figure here} shows \(\Phi^{\mathrm{vdW}}\) as a function of
\(d\) when varying the 2D material.
%
Clearly, the presence of a 2D material (colored curves) universally
reduces the vdW interaction energy compared with a non-screened system
(\(\Phi^{0}\), gray curve) that the medium m is vacuum.
%
As expected, the reduction of $\Phi^{\mathrm{vdW}}$ is stronger for 2D
materials with smaller \(E_{\mathrm{g}}^{\mathrm{2D}}\).
%
Analogous to the results presented in
\cite{Tsoi_2014_vdW_screening_2D}, the effective Hamaker constants
\(A_{\mathrm{eff}}\) are extracted from fitting of the curve using
\autoref{eq:vdw-hamaker-res}. 
%
Specifically, the effective Hamaker constant in the non-screened system is labeled as $A_{\mathrm{eff}}^{0}$.
%
\worktodo{Figure ??}b shows the normalized Hamaker constant
$A_{\mathrm{eff}} / A_{\mathrm{eff}}^{0}$ for various 2D materials at
$d=$ 5 nm, which almost linearly increases
with \(E_{\mathrm{g}}^{\mathrm{2D}}\).
%
This
observation quantitatively agrees with the experimental ranking of
\(A_{\mathrm{eff}}\) in \cite{Tsoi_2014_vdW_screening_2D}, that graphene (Gr) $<$ 2H-MoS\textsubscript{2} $<$ fluoro\-graphene (F-Gr).
%
More interestingly, the single frequency interaction energies
\(G(i\xi)\), exhibit distinct patterns between 2D materials. As shown
in \worktodo{figure which}, comparing with the non-screened case
($G^{0}$), even at \(d\) = 5 nm, semi\-metallic low-gap 2D materials
can still selectively screens low frequencies interactions with
\(\hbar\xi\) \textless{} 1 eV, corresponding to the Debye and Keesom
interactions \cite{israelachvili_intermolecular_2011}.
%
On the other hand, a wide-gap material like hBN shows much weaker high-pass filtering of vdW interactions.
%

% More insight into the non-linear
% behavior can be drawn from Equation \ref{eq:Phi-aniso}, with two
% factors leading to the material dependency of \(\Delta \Phi\): (i) the
% renormalization factor \(1/g_{\mathrm{m}}^{2}\) and (ii) the
% dielectric mismatch \(\Delta_{\mathrm{Am}}\) and
% \(\Delta_{\mathrm{Bm}}\). By decreasing
% \(E_{\mathrm{g}}^{\mathrm{2D}}\), \(\Delta_{\mathrm{Am}}\) and
% \(\Delta_{\mathrm{Bm}}\) are monotonically suppressed over the whole
% frequency range (Supplementary Figure XX), while the screening of
% \(1/g_{\mathrm{m}}^{2}\) becomes more prominent at lower frequencies
% (Figure \ref{fig1}c).
To unify the understanding of the 2D material-mediated screening of
vdW interactions, we borrow the concept of the field effect
transparency \worktodo{unify the notation} discussed in
\worktodo{chapter 1 section?}, to define the vdW ``transparency''.
%
We introduce two variables to quantify such transparency:
\begin{enumerate}
\item The \textbf{frequency-dependent vdW transparency} index $\tau(i\xi)$,
  defined as the ratio between the single-frequency vdW energy with
  and without the 2D material:
  \begin{equation*}
  \label{eq:vdw-def-tau}
  \tau(i \xi_{n}) = \frac{G(i \xi_{n})}{G^{0}(i\xi_{n})}
\end{equation*}

\item The \textbf{overall vdW transparency} index $\eta$, defined as
  the ratio between total vdW energy with and without the 2D material:
  \begin{equation*}
\label{eq:vdw-def-total-trans}
\eta = \frac{\Phi^{\mathrm{vdW}}}{\Phi^{0}}
\end{equation*}
\end{enumerate}
\worktodo{should we use $G^{vdW, 0}$?}
%
\worktodo{figure which c} shows the values of $\tau(i \xi_{n})$
corresponding to \worktodo{which figure}, which clearly illustrates the
high-pass feature dominated by the dielectric anisotropy of 2D materials.
\worktodo{Add comments about the multilayer?}

\subsection{Bandgap Dependecy of vdW Transparency}
\label{sec:bandg-depend-vdw}

From \autoref{eq:vdw-Phi-aniso-main}, unlike the factor
\(1/g_{\mathrm{m}}^{2}\) that solely depends on the dielectric
anisotropy of 2D material, \(\hat{\Delta}_{\mathrm{Am}}\) and
\(\hat{\Delta}_{\mathrm{Bm}}\) are jointly controlled by the
dielectric properties of both bulk and 2D materials, providing more
dimensions to modulate the many-body vdW transmission.
%
Benefited from
the fast numerical approach in our model, we are able to perform
large-scale evaluation of $\eta^{\mathrm{vdW}}$ on a database with
78 two-dimensional materials and 138 bulk materials, generating a total number of
7.5\texttimes{}10\textsuperscript{5} combinations of AmB systems
(details see \autoref{sec:vdw-methods}).
%
\worktodo{Figure 3?} shows the 3D scatter plot of
\(\eta^{\mathrm{vdW}}\) at \(d\) = 2 nm, with the magnitudes of
$\eta^{\mathrm{vdW}}$ indicated by a color mapping. Inspired by the
fact that dielectric properties of both bulk \cite{Moss_1950} and 2D
materials \worktodo{ref to previous} are highly related to their
electronic band structure, we use the bandgaps of bulk materials
(\(E_{\mathrm{g}}^{\mathrm{A}}\), \(E_{\mathrm{g}}^{\mathrm{B}}\)) and
2D material (\(E_{\mathrm{g}}^{\mathrm{2D}}\)) as the axes in
\worktodo{Figure 3?} \worktodo{better words?}.
%
As expected, increasing
\(E_{\mathrm{g}}^{\mathrm{2D}}\) makes the 2D material more
transparent to vdW interactions (red).
%
Conversely, an increase of \(\eta^{\mathrm{vdW}}\) can also achieved
by lowering both \(E_{\mathrm{g}}^{\mathrm{A}}\) and
\(E_{\mathrm{g}}^{\mathrm{B}}\) simultaneously.
%
Clearly, the results in \worktodo{figure 3?} shows an systematic
approach to tune the vdW transparency by changing the bulk/2D material
combinations.


To get more insights into the complex dependency of
\(\eta^{\mathrm{vdW}}\) on the combination of materials, we first take
horizontal slices of the 3D scatter plot \worktodo{which figure}, to
study the effect of varying the bulk materials while fixing the 2D
material.
%
As an example, the cases corresponding to graphene,
2H-MoS\textsubscript{2} and hBN are shown in \worktodo{another figure,
  a-c}, respectively.
%
Two trends can be observed: (i) hBN shows higher $\eta^{\mathrm{vdW}}$
(more transparent) compared with graphene and 2H-MoS\textsubscript{2}
regardless of the bulk material, which is ascribed to its large
bandgap. (ii) $\eta^{\mathrm{vdW}}$ is generally higher around the
region that \(E_{\mathrm{g}}^{\mathrm{A}} \to 0\) and
\(E_{\mathrm{g}}^{\mathrm{B}} \to 0\) (lower diagonal edge) compared
with \(E_{\mathrm{g}}^{\mathrm{A}} \gg E_{\mathrm{g}}^{\mathrm{B}}\)
or \(E_{\mathrm{g}}^{\mathrm{B}} \gg E_{\mathrm{g}}^{\mathrm{A}}\)
(off-diagonal edges) in the 2D scatter plots.
%
It is intriguing to see such defined patterns even
if the choice of bulk materials spans over wide range of lattice types
and electronic structures.

Here we show it is possible to describe such bandgap dependency of
$\eta^{\mathrm{vdW}}$ using a relatively simple approach, based on the
single Lorentz oscillator model (SLOM).
%
The idea of SLOM is to describe the monotonically decaying dielectric
function on imaginary frequency with few parameters. We describe the
dielectric function $\varepsilon_{\mathrm{Bulk}}$ of a bulk material
based the Drude-Lorentz model of single oscillator\worktodo{cite
  Lorentz} and the Kramers-Krönig relations \worktodo{ref to previous
  chapter}:
\begin{equation}
\label{eq:vdw-lorentz-imag}
\varepsilon_{\mathrm{Bulk}}(i \xi) 
=
1 + \frac{\xi_{\mathrm{p}}^{2}}{\xi_{\mathrm{g}}^{2} + \Gamma \xi + \xi^{2}}
\end{equation}
 where 
\(\xi_{\mathrm{g}}\) is the intrinsic oscillation frequency,
\(\xi_{\mathrm{p}}\) is the plasma frequency related to the valence
electron density, and \(\Gamma\) is the damping parameter.
The parameters \(\xi_{\mathrm{g}}\) and \(\xi_{\mathrm{p}}\) can
be extracted using the following relations:
\begin{enumerate}
\item \(\varepsilon_{\mathrm{Bulk}}(i \xi_{\mathrm{g}}) = \frac{1}{2} (\varepsilon_{\mathrm{Bulk, 0}} - 1)\)
\item \(\xi_{\mathrm{p}}^{2} = (\varepsilon_{\mathrm{Bulk, 0}} - 1) \xi_{\mathrm{g}}^{2}\)
\end{enumerate}
where
$\varepsilon_{\mathrm{Bulk, 0}} = \varepsilon_{\mathrm{Bulk}}(i\xi =
0)$ is the static electronic permittivity.
%
Interestingly, the bulk materials studied show a general linear trend
between \(\hbar \xi_{\mathrm{g}}\) and the bandgap
\(E_{\mathrm{g}}^{\mathrm{Bulk}}\), as shown in \worktodo{which
  figure?}  On the other hand \(\hbar \xi_{\mathrm{p}}\) is almost
independent of \(E_{\mathrm{g}}^{\mathrm{Bulk}}\) and scatters between
10\(\sim\)20 eV.  For simplicity, we propose the following relations
for \(\xi_{\mathrm{p}}\) and \(\xi_{\mathrm{g}}\):
\begin{enumerate}
\item \(\hbar \xi_{\mathrm{g}}\) = 1.04\(E_{\mathrm{g}}^{\mathrm{Bulk}}\) + 2.25 eV
\item \(\hbar \xi_{\mathrm{p}}\) = 12.7 eV (averaged value of \(\hbar
   \xi_{\mathrm{p}}\))
 \end{enumerate}
 In addition, we choose a relative small $\Gamma$ such that
 $\hbar \Gamma$ = 0.05 eV~\worktodo{cite?}.  The simplicity of SLOM
 allows constructing $\varepsilon_{\mathrm{Bulk}}(i \xi)$ profiles
 with arbitrary bandgap, and can be used to explore the vdW
 transparency beyond the existing material combinations.
 

\worktodo{Which figure}d-f show the 2D
contour plots of $\eta^{\mathrm{vdW}}$ for graphene,
2H-MoS\textsubscript{2} and hBN calculated using SLOM corresponding
to \worktodo{which figure}a-c, respectively.
%
To our surprise, the features observed in the scatter plots from
material database can be nicely captured using such simplified
model.
%
Further more, the SLOM resolves even more details: along the diagonal
direction (\ie symmetric system, A=B), a minimum in
$\eta^{\mathrm{vdW}}$ can be seen for graphene and
2H-MoS\textsubscript{2}.
%
The existence of the minima in the vdW transparency of symmetric
systems is an evidence that the vdW transmission through a 2D material
is the competition between the electronic fluctuations in the bulk
material and the dielectric screening through the 2D material.  To see
this effect, we notice \autoref{eq:vdw-Phi-single-point} in this case can be simplified to:
\begin{equation}
  \label{eq:vdw-sym-simply}
  G(i \xi) \propto \int_{0}^{\infty} x \ln \left[ 1 - \hat{\Delta}^{2} e^{-x}\right] \mathrm{d}x
\end{equation}
where
$\hat{\Delta}^{2} =
\hat{\Delta}_{\mathrm{Am}}\hat{\Delta}_{\mathrm{Bm}}$ when A=B.
%
\worktodo{New figure}a indicates the integral in
\autoref{eq:vdw-sym-simply} can be approximated by $\hat{\Delta}^{2}$
when ...\worktodo{make decision after plot}. We further consider the
possible values of \(\Delta^{2}\), given the single frequency
dielectric functions \(\varepsilon_{\mathrm{m}}(i \xi)\) and
\(\varepsilon_{\mathrm{A}}(i \xi)\), as shown in \worktodo{Figure}b.
When the 2D material is fixed, by varying
\(\varepsilon_{\mathrm{m}}\), the value of \(\hat{\Delta}^{2}\)
approaches 0 when
\(\hat{\varepsilon}_{\mathrm{A}} = \hat{\varepsilon}_{\mathrm{A}}
\hat{\varepsilon}_{\mathrm{m}}\). Ideally, if such condition is
fulfilled at any frequency \(i\xi\), the 2D material is completely
opaque to vdW interactions. However this is hard to achieve for real
materials, as the frequency-dependent dielectric functions between
bulk and 2D materials can hardly coincide throughout the frequency
domain. 


We note our theory of bandgap dominance of vdW transparency, does not
violate the recently discovery that 2D materials are more
``transparent'' to remote epitaxy of materials with higher polarity
\cite{Kong_2018_pol}. In such situation where the driving force is
formation of chemical bonding from interfacial danging bonds, the
interaction potential from monopolar surface \(\Phi^{\pm}\) (see \autoref{eq:vdw-phi-oss}), is dominating over the vdW
interactions. Although the Coulomb interactions can also be
effectively screened by 2D material
\cite{Li_2014_screen,Ambrosetti_2019_jpcl}, its strength still
overwhelms the vdW interactions. We also comment that the single
oscillator picture may be oversimplified for many other materials
beyond this study~\worktodo{cite 1}, more precise techniques of the
dielectric properties such as multiple Lorentz oscillator model are
required.

\subsection{Distance Dependency of Many-body vdW Interactions}
\label{sec:vdw-distance}

The last component to control the vdW transmission through 2D
materials, is the distance $d$.  From \worktodo{equations to
  alpha-eps}, \(\hat{\varepsilon}_{\mathrm{m}}\) decreases with the
distance \(d\) roughly by a power law of \(d^{-1}\). When
\(d \to \infty\), we have \(\hat{\varepsilon}_{\mathrm{m}} \to 1\) and
\(1/g_{\mathrm{m}}^{2} \to 1\) on the whole frequency range. In this
case, the many-body vdW interaction recovers its form when the medium
is vacuum ($\Phi^{\mathrm{vdW}, 0}$).
%
In addition to the trivial picture that $\Phi^{\mathrm{vdW}}$
decreases with $d$, the
high-pass transmission feature is also affected.
%
To see this effect, we plot the frequency-dependent interaction energy
\(G(i \xi)\) at different distances for various symmetric bulk-2D
\worktodo{bulk-2D or 2D-bulk?}  materials combinations, as shown in
\worktodo{figure X?}.
%
If m is vacuum, the maximum value of $G(i \xi)$ is always at
$\xi \to 0$, and increasing $\delta$ only trivially down\-scales the
magnitudes of $G(i \xi)$. However if m contains a 2D materials, a
clear change in the shape of \(G(i \xi)\) can be observed upon
increasing $d$,
%
For graphene (\worktodo{figure}a1-a3), a peak exists in the profile of
\(G(i \xi)\) corresponding to the selective screening of vdW
interactions with $\hbar \xi<$5 eV due to the high-pass transmission
discussed in \worktodo{section?}.
%
The peak position drifts to lower energy range with increasing $d$.
%
On the contrary, if the 2D material is 2H-MoS\textsubscript{2} or
h-BN, the high-pass peak only exists with a wide-gap bulk material
like cubic BN (\worktodo{figure}xx-xx). Similar red-shift of the peak
position with increasing \(d\) is also observed in these systems,
indicating the screening effect of the 2D material becomes closer to
vacuum (peak position = 0 eV).


Another exotic behavior concerning the distance-dependency of vdW
interactions through 2D materials, is the power law with $d$
\worktodo{try better wording, and cite Gobre paper}.
%
When the medium m is vacuum, \(|\Phi^{\mathrm{vdW}}|\) always exhibits a
power law of \(d^{-2}\), as can be expected from
\autoref{eq:vdw-hamaker-res}.
%
In other words, the log-log plot for \(\Delta^{\mathrm{vdW}}\) vs
\(d\) in an unscreened system shows a straight line with slope of -2
\worktodo{as shown in figure?}.
%
However, based on our analysis above, the existence of 2D material
causes stronger screening of $\Phi^{\mathrm{vdW}}$ at shorter
distance, and the log-log profile \(|\Phi^{\mathrm{vdW}}|\) vs \(d\)
deviate from that in the non-screened case.
%
Apparently, the vdW interaction in such system cannot be simply
described by the $d^{-2}$ Hamaker-vdW model. Here we define the
derivative $\partial \log(|\Phi^{\mathrm{vdW}}|)/ \partial d$ as the
``local'' power exponent $p_{\mathrm{local}}$, such that
$\Phi^{\mathrm{vdW}}$ approximately decays at a rate of
$d^{-p_{\mathrm{local}}}$ at certain value of $d$.
%
To demonstrate, we choose A=SiO\textsubscript{2} and
B=Si\textsubscript{3}N\textsubscript{4} as demonstrated in
\worktodo{figure xx} to mimic the AFM measurements in
\textcite{Tsoi_2014_vdW_screening_2D}.
%
As shown in \worktodo{another figure}, the vdW interactions through a 2D material have $p_{\mathrm{local}} < 2$ due to the screening.
\(p_{\mathrm{local}}\) asymptotically approaches the vacuum limit of 2 when $d \to \infty$.
%
In the case of graphene and 2H-MoS\textsubscript{2},
\(p_{\mathrm{local}}\) can be as small as $\sim{}$0.5 at \(d\) = 2 nm.
There are several implications from such abnormal power law due to the
existence of 2D material. Firstly, the effective Hamaker constant
approach that frequently used to interpreter experimental
results~\cite{Tsoi_2014_vdW_screening_2D}\worktodo{cite 1+} should be
taken with great caution. Moreover, the small power exponent of
many-body vdW interactions makes them even longer range forces than
non-screened vdW interactions. This may lead to some extraordinary
physical phenomena as we show in the next section.

\section{Repulsive vdW Interactions Through 2D Materials}
\label{sec:vdw-repuls-vdw-inter}

\subsection{Origin of Repulsive vdW Interactions}
\label{sec:vdw-origin-repulsive-vdw}

The model proposed in \autoref{sec:vdw-model-lifshitz} leads to some
even more interesting results.
%
In \worktodo{figure to large scale}, points corresponding to
\(\eta^{\mathrm{vdW}} < 0\) (blue) can be observed at the limit of
small \(E_{\mathrm{g}}^{\mathrm{A}}\) coupled with large
\(E_{\mathrm{g}}^{\mathrm{B}}\) , and \emph{vice versa}. Similar
observations can also be found in \worktodo{figure SLOM} predicted by
the SLOM.
%
This reversal in the sign of $\Phi^{\mathrm{vdW}}$ indicates the
repulsive many-body vdW-Casimir interactions
\cite{Munday_2009_repul,Zhao_2019_casimir_trap} which lead to
fascinating applications including super lubricity
\cite{Feiler_2008_superlubri} and quantum levitation
\cite{MUNDAY_2010_repul}.
%
Historically, the repulsive vdW interactions are only measured in
liquid systems\worktodo{why?}. The predictions from our model
indicates that such repulsive interactions, can also be achieved in
solid state systems, with a proper choice of materials.

The origin of the repulsive vdW forces can be explained by analyzing
\autoref{eq:vdw-Phi-single-point} \worktodo{make sure EQ you want}.
%
As shown in \worktodo{which figure?}, the sign of $G(i \xi)$ is
determined by $\hat{\Delta}_{\mathrm{Am}} \hat{\Delta}_{\mathrm{Bm}}$,
therefore it is possible to have $G(i \xi) > 0$ (repulsive
interaction) \worktodo{check the sign} when
$\hat{\Delta}_{\mathrm{Am}}$ and $\hat{\Delta}_{\mathrm{Bm}}$ have
opposite signs.
%
This can be achieved when a ``dielectric cascade'' exists in the
system, such that
\(\hat{\varepsilon}_{\mathrm{A}} < \hat{\varepsilon}_{\mathrm{m}} <
\hat{\varepsilon}_{\mathrm{B}}\) (or \emph{vice versa}).
%
Inspired by the bandgap-dominance of both
\(\hat{\varepsilon}_{\mathrm{A}}\) and
\(\hat{\varepsilon}_{\mathrm{B}}\), it is intuitive to have a
dielectric cascade system formed by coupling a small-gap and a
large-gap bulk materials.
%
Moreover, the relatively small electronic polarizability of a
large-gap 2D material such as h-BN, makes it difficult to find a bulk
material with
\(\hat{\varepsilon}_{\mathrm{A}} < \hat{\varepsilon}_{\mathrm{m}}\),
which explains the absence of repulsive interactions in \worktodo{this
  figure c - f}.
%
As a demonstration, we consider a model system
that A=gold (Au) and B=lithium fluoride (LiF) separated by graphene at \(d\)
= 2 nm.
%
The cascade of dielectric functions can be observed as low frequency
range (\worktodo{which figure, what range?}), where t
\(G(i \xi)\) becomes repulsive. At higher frequency ranges, the
high-pass feature of 2D material makes \(G(i \xi)\) again attractive,
leading to a total repulsive energy, yet of small magnitude.



\worktodo{mention the temperature}

\subsection{Observation of Repulsive vdW Interactions in Solid-State Systems}
\label{sec:vdw-observ-repuls-vdw}

\worktodo{cite other papers about repulsion}
\worktodo{a short section discussing why repulsive not in solid-state?}

In this section, we show the preliminary experimental results
concerning the evidence of repulsive vdW interactions in solid-state
films.
%
As the analysis in \autoref{sec:vdw-origin-repulsive-vdw} indicates,
the essence of the repulsive many-body vdW interactions is to have the
dielectric cascade system, such that
$\hat{\varepsilon}_{\mathrm{A}} < \hat{\varepsilon}_{\mathrm{m}} <
\hat{\varepsilon}_{\mathrm{B}}$.
%
It is possible to construct such system when A is a organic molecular
crystal, B is a metal with high dielectric function and m is a
semi\-metallic 2D material like graphene.
%
As an example, we choose gold as the metal (B), while the dielectric
function of $A$ is modeled by the SLOM with $\xi_{\mathrm{g}}=$
\worktodo{what} and $\xi_{\mathrm{p}}=$ \worktodo{what} to represent
typical dielectric responses of organic molecules~\worktodo{cite 1
  here}.
%
As shown in \worktodo{what figure}a, our simulation indicates that
repulsive $\Phi^{\mathrm{AmB}}$ indeed exists when $d=$ 1
nm\worktodo{check number}, similar to the case of LiF-graphene-Au
case shown previously.
%

Such system can be experimentally realized using molecular epitaxy on
substrate-supported graphene. As a model study, we deposit the
PTCDI-BPE (\worktodo{full name}) molecules on two different
substrates: SiO\textsubscript{2}-supported graphene
(Gr/SiO\textsubscript{2}) and gold-supported graphene (Gr/Au).
%
The structure of BPE features two phenyl\-ethyl \worktodo{right
  name} groups at the side of the PTCDI basal plane which are freely
rotatable.
%
As discussed in \autoref{sec:inter-forc-at}, epitaxial growth of
organic molecules is the competition between (i) intermolecular vdW
forces, (ii) molecule-2D interactions and (iii) molecule-substrate
interactions.
%
If repulsive interactions exist in (iii), the morphology of the
epitaxial film may be altered.  However, note due to the dominance of (i) and
(ii), we expect such change of morphology, if exists, to be very
small.
%
To accurately determine the crystalline packing in the epitaxial
films, grazing angle X-ray diffraction (GIXD) spectra~\worktodo{cite
  1-2} were used to examine the orientation and inter-plane distance
between the molecules.
%
As shown in \worktodo{Figure ??}a and b, in th 20 nm-thick BPE
films deposited onto both Gr/SiO\textsubscript{2} and Gr/Au surfaces,
the major peak in the GIXD spectra appears around \worktodo{xxx
  nm$^{-1}$} near the $q_{z}$ axis, corresponding to the diffraction from
\worktodo{<hkl>} (basal) plane. \worktodo{cite 1}
%
Such evidence indicates the face-on orientations of the PTCDI-BPE
molecules on both surfaces as mediated by the strong π-π interaction between conjugated organic molecules and graphene. \worktodo{cite 1}
%
A closer inspection of the spectra shows that the diffraction momentum
of \worktodo{hkl} plane $q_{\mathrm{hkl}}$, decreases by \worktodo{0.2
  nm$^{-2}$??} in the BPE/Gr/Au system compared with that in BPE/Gr/SiO2.
%
Equivalently, the distances between the basal plane
$d_{\mathrm{nkl}}$\worktodo{?}, increased from 3.xx\worktodo{\AA{}} in BPE/Gr/SiO2 to 3.yy\worktodo{\AA{}} BPE/Gr/Au. \worktodo{figures}
%
The change of lattice spacing is further verified by comparing the 1D
angle-averaged intensity profile of GIXD. As shown in \worktodo{figure
  yy}, the decreasing of $q$ values (equivalent to longer inter\-plane
spacing) is systematically observed for \worktodo{xxx, yyy and zzz}
lattice planes, which all correspond to the diffraction due to the
PTCDI basal plane.
%

Clearly, such results contradict the trivial theory of vdW
transparency: the vdW interactions between an organic molecule and
gold is in general much stronger than that on a polar surface like
SiO\worktodo{2} \worktodo{cite 1}.
%
To prove this, we further investigate the GIXD patterns of BPE
deposited on bare SiO\textsubscript{2} and Au surfaces, as shown in
\worktodo{figure xxx}.
%
The packing of BPE apparently flips to the edge-on orientation on
SiO\textsubscript{2}, as indicated by the emergence of a single strong
peak at $q_{z}=$ \worktodo{xxx nm$^{-1}$}, corresponding to the <002>
plane.
%
On the other hand, new peaks of \worktodo{q = ??} appears in the case
of BPE/Au corresponding to the diffraction of \worktodo{zzz} plane,
adapting an intermediate orientation.
%
The orientations of BPE/SiO\textsubscript{2} and BPE/Au cases are schematically shown in \worktodo{figure }, respectively.
%
The stronger interaction at BPE/Au interface is revealed by the closer
distance of the PTCDI basal plane to the substrate compared with that
on SiO\textsubscript{2}.
%
To further eliminate the possibility due to substrate doping of
graphene, we monitor the doping density using Raman scattering~\cite{Das_2008_doping} of the
Gr/SiO\textsubscript{2} and Gr/Au systems.
%
The position in the G peak around 1585 cm$^{-1}$ shows no statistic
difference between the two samples, indicating similar doping density.

We note that the results in the GIXD experiments represents the
ensemble packing behavior of the organic molecules within the 20 nm
thickness of the film, in stead of only probing the first few layers.
%
There are two benefits from such experimental setup: (i) the influence
of the interfacial corrugation due to surface roughness and transfer
technique is minimized and (ii) the mean-field treatment of the
modified Lifshitz-vdW model proposed here can be applied to describe
the average many-body interaction potential.
%
Moreover, the change of lattice spacing is constantly observed from
different batches of samples, and the difference in $q$ value is
significantly higher than the spectrum resolution \worktodo{which
  resolution?}
%
\worktodo{figures here} Combining all the evidences
%
above, we propose that the expansion of lattice spacing observed here,
can only be attributed to the existence of repulsive vdW interactions.
%
To our best knowledge, this is the first experimental demonstration of
repulsive vdW interactions in solid-state systems.

Nevertheless, as we see in the examples here, the change of lattice spacing is only $\sim{}$1\%\worktodo{check value}, due to the competition with other attractive forces.
%
In the next sections, we propose several new approach to allow enhancing the repulsive vdW interactions that can be potentially measured experimentally.

\subsection{Designing Novel System for Repulsive vdW Interactions}
\label{sec:proposing-new-system}

As discussed in \autoref{sec:origin-repulsive-vdw}, the repulsive vdW
interactions are elusive due to the attracting energy contributions
at higher frequencies.
%
To enhance the repulsive interactions, one approach is to create systems with dielectric cascade at all frequency ranges.
%
From the material database search, it is particularly difficult to
find a bulk material with $\varepsilon$ constantly smaller than the 2D
material-containing medium throughout the frequency
regime. \worktodo{why?}
% 20191016-0831
Here we propose another approach: potential repulsive vdW interactions
can exist on suspended 2D material layers. If material A is vacuum or
gas, $\varepsilon_{\mathrm{A}}$ approaches unity at all frequency
ranges, $\hat{\Delta}_{\mathrm{Am}}$ will be constantly larger than
0. A cascade system in this case, can be constructed by either (i)
having material B with large dielectric function (such as metal), or
(ii) increase the spacing $d$.
%
\worktodo{A figure here} Due to lack of absolute definition of the
boundary between gas phase and 2D materials, here we treat the
distance $d$ from the bottom of the 2D material's electron cloud to
the closest atomic plane in the bulk material.
%
Such definition, despite being arbitrary, \worktodo{better words?}
allows us to apply the modified Lifshitz-vdW model to the suspended 2D
material systems.
%
Here we propose two possible experimental setups to demonstrate the repulsive many-body vdW interactions.

\subsubsection{Liquid Condensation on Suspended 2D Materials}
\label{sec:liqu-cond-susp}

The first example considered here is the liquid-suspended 2D
interface. \worktodo{figure?}
%
\worktodo{appropriate to use $\Phi^{\mathrm{vdW}}$ still?}
We a condensed water film (B) on top of a suspended 2D material
separated by distance \(d\), while below the 2D layer there is diluted
water vapor (A, \(\hat{\varepsilon}_{\mathrm{A}} \approx 1\)).
%
As a demonstration, we study \(\Phi^{\mathrm{vdW}}\) as function of
\(d\) in such system, when the 2D material is either graphene or hBN ,
as shown in \worktodo{figure}b.
%
In both cases, \(\Phi^{\mathrm{vdW}}\) becomes
repulsive when \(d\) exceeds a certain threshold $d_{0}$ (\worktodo{xx \AA{} for graphene and yy \AA{} for hBN}).
%
This can be understood from the analysis in
\autoref{sec:vdw-distance}, as $\hat{\varepsilon}_{\mathrm{m}}$
shrinks with increasing $d$, and can eventually be smaller than
$\varepsilon_{\mathrm{B}}$ when $d > d_{0}$.
  %
Under such circumstances the relations $\hat{\Delta}_{\mathrm{Am}} >0$
and $\hat{\Delta}_{\mathrm{Bm}}<0$ are fulfilled, leading to the
many-body repulsive interactions in condensed liquids as previously
proposed in \cite{Bostrom_2012_repulsive,Sengupta_2018_rep}.
%
We also notice the repulsive many-body vdW interactions are easier to
be observed on hBN surfaces due to its relatively small electronic
polarizability.

As discussed before, in solid-state systems, the total adhesion forces
are dominated by the vdW interactions between the 2D and bulk
materials. This makes the properties related to repulsive interactions
difficult to be isolated.
%
However, the situation can be different on suspended 2D sheets. From
the discussions in \autoref{sec:vdw-distance}, the many-body vdW
interactions decay slower compared with the non-screened, two-body vdW
interactions. As a consequence of the competition between the forces,
the total interaction energy can be repulsive at a certain distance
range.
%
As shown in \worktodo{which figure?}, the total vdW interactions
$\Phi^{\mathrm{tot}}$ which consists both repulsive 2D interactions
$\Phi^{\mathrm{AmB}}$ and two-body attractive interactions
$\Phi^{\mathrm{2D-B}}$, can have maxima when $d \approx 30 \AA{}$,
corresponding to a energy barrier of XXX mJ$\cdot$m$^{-2}$.
%
Such repulsive total interactions may be potentially observed by
in-situ condensation techniques such as environmental scanning
electron microscope (ESEM), in which the wettability is monitored by
the change of liquid-2D contact angle. \worktodo{choice of words?}
\worktodo{how do we calculate two-body?}


\subsubsection{Epitaxy of Metal on 2D Materials}
\label{sec:epitaxy-metal-2d}

As seen in \autoref{sec:liqu-cond-susp}, although the total adhesion
energy at the liquid-suspended 2D interface can be repulsive, the
energy barrier is relatively small to achieve stable levitation of the
bulk phase~\cite{Zhao_2019_casimir_trap}.
%
To enhance the many-body repulsive interactions, it is desired to have
even larger $\hat{\varepsilon}_{\mathrm{B}}$ compared with water. One
possible approach is epitaxial deposition of metal on suspended 2D
materials.
%
Due to the large dielectric function of metal,
$\hat{\varepsilon}_{\mathrm{B}}$ is constantly larger than
$\hat{\varepsilon}_{\mathrm{m}}$, and the dielectric cascade criteria
$\hat{\Delta}_{\mathrm{Am}} > 0$ and $\hat{\Delta}_{\mathrm{Bm}} < 0$
are always fulfilled, regardless of the choice of 2D material and
distance $d$.
%
This is validated by our simulations in \worktodo{which figure XX?}, when
$\Phi^{\mathrm{AmB}}$ becomes always repulsive for gold deposited on
both suspended graphene and hBN sheets.
%
More interestingly, the large distinction between
$\hat{\varepsilon}_{\mathrm{B}}$ (when B=gold) and
$\hat{\varepsilon}_{\mathrm{m}}$ \worktodo{Should I use gold?}
increases the magnitude of $\Phi^{\mathrm{AmB}}$ by almost 10 times
compared with liquid-2D systems. As a result, an appreciable repulsive
energy barrier of $\sim{}$0.5 mJ$\cdot$m$^{-2}$ is seen in
$\Phi^{\mathrm{tot}}$ at $d_{0} \approx$25 \AA{}.
%
% Such feature makes the repulsive interactions more observable.
%
As a proposal for future experiments, the repulsive interaction is
potentially observable via epitaxial growth of metal on suspended 2D
material sheets, in which the surface wetting behavior, morphology and
nucleation density of the metal film can be used to probe its
existence.



\section{Conclusion}
\label{sec:vdw-conclusion}

In this chapter, we present a general model based on the modified
Lifshitz-vdW theory to describe the many-body transparency of 2D materials
to vdW interactions.
%
The dielectric anisotropy of 2D materials selectively screens the 
low frequency contributions (Keesom and Debye) to the vdW interactions, leading to a high-pass band filter behavior.
%
The formalism presented in this chapter allows fast and large scale
material database screening up to 7.5\texttimes{}10\textsuperscript{5}
material combinations.
%
We find the many-body vdW transparency is a highly non-linear quantity
that dependent majorly on the bandgaps of 2D and bulk materials, which
can be used to quantitatively explain with the force spectroscopy measurements in literature.

More importantly, combining large-scale database screening and
numerical analysis, we further predict the existence of repulsive
interactions when dielectric cascade exists in the system. The
existence of the repulsive vdW interactions is demonstrated by
experimental observations of lattice expansion in molecular epitaxy of
organic molecules on gold-supported graphene interface.
%
Based on our model, we further propose two experimental setups on
suspended 2D material sheets, in which the repulsive vdW interactions
can be potentially observed.

Our theoretical and experimental demonstrations provide a general
picture of 2D material-mediated surface interactions, which can be
used as guidelines for designing vdW heterostructures, 2D
material-based coating as well as wetting phenomena.


\section{Methods}
\label{sec:vdw-methods}
\worktodo{fill}
\subsection{Retrieval of Experimental Dielectric Functions of Bulk Solids}
\label{sec:retr-exper-diel}

\worktodo{fill}


\subsection{First Principles Calculation of Dielectric Properties}
\label{sec:firsc-diel-prop}
\worktodo{fill}


\subsection{Calculation of Two-Body vdW Intercation Energy}
\label{sec:calculation-two-body}
\worktodo{fill}

\subsection{BPE/Graphene Sample Preparation and Measurements}
\label{sec:graph-sample-prep}
\worktodo{fill}

\section{Author Contributions}
\label{sec:vdw-author-contributions}
\worktodo{fill}











%%% Local Variables:
%%% mode: latex
%%% TeX-master: "../thesis"
%%% End:
