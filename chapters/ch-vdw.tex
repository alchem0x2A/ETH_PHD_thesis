% \chapter{Introduction}
\chapter{Many-Body Transmission of vdW Interactions Through 2D Materials}
\label{ch:vdw}
\renewcommand*\imgdir{img/vdw/}

% \dictum[Richard Feynman]{%
  % Nature isn't classical, and if you want to make a simulation
  % of nature, you'd better make it quantum mechanical, and by golly
  % it's a wonderful problem, because it doesn't look so easy.  }%

\vspace{1em}

\chapterabstract{Part of this chapter appears in the following work:
  Tian, T., Naef, F., Lee, Y.-T., Zhang, S.-W., Chiu, Y.-C. \& Shih,
  C.-J. Two-Dimensional Materials as High-Pass Filter of van der Waals
  Interactions, in preparation.
}

\section{Introduction}
\label{sec:vdw-introduction}
% 20191014-1539
The van der Waals (vdW) interactions are inter\-atomic /
inter\-molecular forces that stem from the electromagnetic (EM)
fluctuations~\cite{Israelachvili_2011_book,Woods_2016_rev_vdw,parsegian_van_2010_book}.
%
Unlike trivial electrostatic (Coulombic) forces that depend on the
charge state, the vdW interaction has a complex electro\-dynamic
origin, which acts ubiquitously between all atoms and molecules~\cite{Israelachvili_2011_book,Hermann_2017_vdW_rev}.
%
As briefly discussed in \autoref{sec:inter-forc-at}, despite its weak nature, the vdW interaction can be dominating at
molecule-2D interface due to the absence of dangling bonds.
%
The understanding and engineering of the vdW interactions at the 2D
material interfaces would greatly benefit molecular
self-assembly~\cite{Kumar_2017_rev_assemb_2D},
electronic~\cite{Wang_2015_phys_chem_tuning_TMDC,Lazar_2013} as well
as optical properties~\cite{Geim_2013_2D_vdw_Het,Novoselov_2016_vdW}
of 2D material-based hetero\-structures.
%

As introduced in \autoref{sec:van-der-waals}, one particular question
of interest is the transmission of vdW interactions through the
atomically-thin 2D materials.
%
The effective distance of vdW forces, depending on the medium, can
vary from 10$^{-1}$ to 10$^{1}$ nm~\cite{Israelachvili_2011_book}, and are
relatively longer compared with the sub-nm thickness of 2D materials.
%
Therefore, the vdW forces from an macroscopic substrate may
(partially) penetrate through the 2D layer and be felt by an molecule
on the 2D material interface~\cite{shih_2013_wetting_natmat}.
%
This feature
brings potentially unprecedented properties by simply coating
an existing bulk surface with 2D
material(s)~\cite{Prasai_2012_coating,rafiee_2012_transparency,Tsoi_2014_vdW_screening_2D}.
%


Based on classical pair-wise model~\cite{Hamaker_1937_vdW}
(\autoref{fig:vdw-compare-classic}\lc{a}) of vdW forces, the interaction
free energy per area  between two semi-infinite
bulk bodies (A and B) separated by distance $d$, $\Phi_{\mathrm{AB}}$, is:
\begin{equation}
  \label{eq:vdw-hamaker-res}
  \Phi_{\mathrm{AB}} = - \frac{A_{\mathrm{AB}}}{12 \pi d^{2}}
\end{equation}
where $A_{\mathrm{AB}}$ is known as the Hamaker constant between materials A and B.
%
A quantitative statement can be drawn from such simple model: by
inserting a 2D material (\eg graphene with thickness $\sim{}$3.3
\AA{}~\cite{Shearer_2016}) between two bulk bodies (3$\sim{}$5
\AA{}~\cite{Israelachvili_2011_book}), 20$\sim{}$40\% of the vdW
interactions can still penetrate through the 2D layer.
\begin{figure}[!htbp]
  \centering
  \import{\imgdir}{model_classic_compare.pdf_tex}
  \caption{\label{fig:vdw-compare-classic}%
    Different pictures of vdW interactions. {\bfseries a} The classical
    pair-wise model assuming vdW interaction energy
    $\Phi_{\mathrm{AB}}$ between bulk materials follows
    $\Phi_{\mathrm{AB}} \propto d^{-2}$. {\bfseries b} The Lifshitz-vdW
    model that takes many-body effect into account. In the context of
    2D material medium, in addition to the distance $d$ between the
    two bulk bodies, the many-body vdW interaction energy
    $\Phi_{\mathrm{AmB}}$ is controlled by dielectric mismatch
    $\hat{\Delta}_{\mathrm{Am}}$, $\hat{\Delta}_{\mathrm{Bm}}$ and the
    dielectric anisotropy factor $1/g_{\mathrm{m}}$.}
\end{figure}
%
Several non-trivial phenomena concerning the interfacial vdW forces
take credit from the analysis above, including wetting
transparency / translucency
\cite{Shih_2012_prl,rafiee_2012_transparency,Gurarslan_2016_screen_MoS2} and remote
epitaxy \cite{Kim_2017_remote_epi_Gr,Kong_2018_pol}.

Despite the seemingly accordance with experimental observations in
these studies, the prerequisites for the classical model are (i) the
perfect additivity of vdW interactions, and (ii) a constant
$A_{\mathrm{AB}}$. 
%
In other words, the classical model assumes that the existence of the
2DEG has no influence on the EM waves that arise from charge
fluctuations.
%
Such picture is however challenged by recent experimental
\cite{Tsoi_2014_vdW_screening_2D} and first-principles
\cite{Ambrosetti_2018_carbon,Liu_2018_gr,Li_2018_screen} studies.
%
To correctly capture the many-body picture of vdW interactions at 2D
material interfaces, electro\-dynamic effects beyond the classical model
is needed.
%
In practice, this is usually a challenging task, since state-of-art
density functional theory (DFT) used for first principle simulations
is limited to ground state electron density~\cite{Perdew_1996_GGA},
intrinsically lacking charge fluctuation~\cite{Woods_2016_rev_vdw}.
%
On the other hand, full-scale interaction energy calculations based on
non-local dielectric responses are usually time-consuming and
only limited to small
systems~\cite{Hermann_2017_vdW_rev,Zhou_2017_lifshitz}.
%
Therefore, novel theoretical frameworks with appreciable accuracy
while maintaining simplicity, is of high demand to shed light on the
vdW interactions at 2D material interfaces.

In this chapter, we present
a general model  to study the many-body vdW transmission
through 2D materials, based on the modified Lifshitz-vdW theory
\cite{Dzyaloshinskii_1961_lifshitz,parsegian_van_2010_book}.
%
Our model reveals that the transmission of vdW interactions through a
2D material is frequency-dependent, such that a 2D material acts as a high-pass
filter of vdW interaction, selectively screening the low-frequency interactions.
%
Such phenomenon is related with the anisotropic dielectric nature of 2D materials (~\autoref{sec:diel-unif-geom-repr}).
%
Using both large-scale material database screening and oscillator
mode, we further show that such effect is concerted by the bandgap of
both the 2D and bulk materials
%
More interestingly, the model predicts the possibility to achieve
repulsive vdW interactions  via careful dielectric
engineering of the 2D material interface.
%
As a preliminary experimental demonstration, we show that the such
repulsive vdW forces can be indirectly probed using molecular epitaxy
on gold-supported graphene.

\section{Modified Lifshitz-vdW Model For 2D Material Interfaces}
\label{sec:vdw-model-lifshitz}
% 20191014-1529

\subsection{vdW Interactions  For Layered Anisotropic Media}
\label{sec:vdw-gener-model-layer}

We start the discussion in a system with simplified geometry (as
illustrated in \autoref{fig:vdw-compare-classic}\lc{b}): consider two
infinitely-large bulk materials (A and B) separated by a medium m at
distance \(d\), the total many-body interaction potential between A
and B over m, \(\Phi_{\mathrm{AmB}}\), consists of the contributions
from vdW \(\Phi_{\mathrm{vdW}}\) and mono\-polar \(\Phi_{\pm}\) interactions
\cite{van_Oss_1987_monopolar,Van_Oss_1988}, such that
$\Phi_{\mathrm{AmB}} = \Phi_{\mathrm{vdW}} + \Phi_{\pm}$, where
\(\Phi_{\mathrm{vdW}}\) includes the Keesom (dipole -- dipole), Debye
(dipole -- induced dipole) and London dispersion (induced dipole --
induced dipole) interactions~\cite{Israelachvili_2011_book}, while
\(\Phi_{\pm}\) originates from interfacial monopoles (Coulomb
interactions) \cite{van_Oss_1987_monopolar}.
%
Note that $\Phi_{\mathrm{vdW}}$ and $\Phi_{\pm}$ are both relative
energy scales compared with the infinitely-separated systems.
%
From the Lifshitz-vdW theory \cite{Dzyaloshinskii_1961_lifshitz},
\(\Phi_{\mathrm{vdW}}\) stems from the fluctuations of electromagnetic
(EM) field, and can be regarded as the sum of oscillatory surface 
mode energies, when the length scale of interest is larger than
atomistic details.
%
We further limit our discussion to the regime that the vdW
contribution dominates (\ie chemically inert 2D material interface),
such that $\Phi_{\mathrm{AmB}} \approx \Phi_{\mathrm{vdW}}$.  Further
neglecting the retardation effect~\cite{Dryden_2015_gecko} (due to
limited speed of light, usually seen at distance $>$20
nm~\cite{parsegian_van_2010_book}),
$\Phi_{\mathrm{AmB}}$ is expressed as the total energy summed from all allowed EM modes
\cite{Li_2005_diele}:
\begin{equation}
\label{eq:vdw-EM-energy}
\Phi_{\mathrm{vdW}} = k_{\mathrm{B}} T \sum_{j} \ln \left[2 \sinh\left(\frac{\hbar \omega_{j}(\mathbf{k})}{2 k_{\mathrm{B}} T}\right)\right] 
\end{equation}
where \(\omega_{j}\) is EM frequencies of allowed mode $j$ that
corresponds to the (i) geometry and dielectric profiles of the system
and (ii) the in-plane wave vector \(\mathbf{k}\).
%
The system-specific information of the EM modes can be generalized by
writing the dispersion relation ($\mathcal{D}$) of system, such that
\(\mathcal{D}(\omega_{j}(\mathbf{k})) \equiv
0\)~\cite{Guttinger_1966_dispersion,Mahanty_1976_dispersion_book}.
%
Analyzing the mode frequency \(\omega_{j}\) is usually a non-trivial
task, while on the other hand, using the Cauchy integral theorem, the
integral in \autoref{eq:vdw-EM-energy} is equivalent to the summation
over the imaginary Matsubara frequencies $\xi_{n}$
\cite{Mahanty_1976_dispersion_book} as:
\begin{equation}
\label{eq:vdw-lifshitz-general-2}
\Phi_{\mathrm{vdW}} = \frac{1}{2} k_{\mathrm{B}} T \sum_{n=-\infty}^{\infty} \sum_{\mathbf{k}}\ln \mathcal{D}(i \xi_{n}, \mathbf{k})
= k_{\mathrm{B}} T \sideset{}{'}\sum_{n=0}^{\infty} \sum_{\mathbf{k}} \ln \mathcal{D}(i \xi_{n}, \mathbf{k})
\end{equation}
where \(k_{\mathrm{B}}\) is the Boltzmann constant, \(T\) is
temperature, \(\xi_{n} = 2 \pi n k_{\mathrm{B}} T / \hbar\) is the
$n$-th Matsubara frequency and the prime in the summation
($\sideset{}{'}\sum{}$) refers to the prefactor of 1/2 when
\(n=0\). Here we assume the dispersion relation \(\mathcal{D}\) has
time-reversal symmetry, i.e.
\(\mathcal{D}(i\xi, \mathbf{k}) = \mathcal{D}(-i\xi, \mathbf{k})\).
%
The summation over \(\mathbf{k}\) can be further carried out in
continuous integral if the system is infinitely large.
%
For the 3-layer system considered here (refer to AmB in the remaining
text) that is infinite in \emph{xy}-direction while non-periodic in
\textit{z}-direction, the summation over \(\mathbf{k}\) in
\autoref{eq:vdw-lifshitz-general-2} can be written using in-plane wave
vector \(\mathbf{k} = (k_{x}, k_{y})\):
\begin{equation}
\label{eq:vdw-lifshitz-integral}
\Phi_{\mathrm{AmB}} \approx \Phi_{\mathrm{vdW}} = \frac{k_{\mathrm{B}} T}{(2 \pi)^{2}} \sideset{}{'} \sum_{n=0}^{\infty} \int_{0}^{\infty} \ln \mathcal{D}(i\xi_{n}, \mathbf{k}) \mathrm{d}^{2} \mathbf{k}
\end{equation}
%
While along the \textit{z}-direction, the electric potential
\(\psi\) follows the Poisson-Laplace equation, such that:
\begin{equation}
\label{eq:vdw-poisson-laplace}
\nabla \cdot [ \varepsilon_{0} \varepsilon_{i}(\mathbf{r}, z) \cdot \nabla \psi]
= -\rho(\mathbf{r}, z) = 0
\end{equation}
where \(\mathbf{r}=(x, y)\) is the in-plane coordinates,
$\varepsilon_{i}$ is the dielectric tensor of each material ($i$= A, m
or B), and \(\rho\) is the free charge density in the system. For
van der Waals interactions alone, \(\rho \equiv 0\).
%
The periodicity of the \textit{xy}-plane leads to the form of
plane-wave form of $\psi$~\cite{parsegian_van_2010_book}:
\(\psi(\mathbf{r}, z) =  
f(z) e^{i \mathbf{k} \cdot \mathbf{r}} \),
where \(f(z)\) is the variation in the \emph{z}-dierction.
%
We further use a mean-field assumption that $\varepsilon_{i}$ is
uniform inside each material and is a tensor
consists of different diagonal components \(\varepsilon^{xx}\),
\(\varepsilon^{yy}\) and
\(\varepsilon^{zz}\).
Under such conditions, the Poisson-Laplace equation for individual
layer now writes:
\begin{equation}
  \label{eq:vdw-laplace-2}
  \begin{aligned}[t]
\varepsilon_{i}^{zz} \frac{\partial^{2} f_{i}(z)}{\partial z^{2}}
- (k_{x}^{2} \varepsilon_{i}^{xx} + k_{y}^{2} \varepsilon_{i}^{yy}) f_{i}(z) &= 0    \\
\frac{\partial^{2} f_{i}(z)}{\partial z^{2}}
- g_{i}(\vartheta) \kappa^{2} f_{i}(z) &= 0
  \end{aligned}
\end{equation}
where $\kappa$ and $\vartheta$ are the polar coordinates of
$\mathbf{k}$ such that
\(\mathbf{k} = (\kappa \cos\mathcal{\vartheta}, \kappa\sin
\mathcal{\vartheta})\), and
\(g_{i} = [\frac{\varepsilon_{i}^{xx}}{\varepsilon_{i}^{zz}} \cos^{2}
  \vartheta + \frac{\varepsilon_{i}^{yy}}{\varepsilon_{i}^{zz}}
  \sin^{2} \vartheta]\).
%
The factor \(g_{i}\) characterizes the dielectric anisotropy of
medium $i$ (see discussions in \autoref{sec:diel-unif-geom-repr}) and
upscales the wavevector modulus \(\kappa\). Note for an isotropic
medium, $g_{\mathrm{m}}\equiv1$.

% 
Following the similar procedures in isotropic systems
\cite{parsegian_van_2010_book}, the dispersion relation $\mathcal{D}$
of an AmB layered system and considering the dielectric anisotropy is
written as:
\begin{equation}
\label{eq:vdw-disper-D}
\begin{aligned}
\mathcal{D}
&=
1 - 
\underbrace{\left[
\frac{\hat{\varepsilon}_{\mathrm{A}} - \varepsilon_{\mathrm{m}}^{zz} g_{m}^{1/2}(\vartheta) }{\hat{\varepsilon}_{\mathrm{A}} + \varepsilon_{\mathrm{m}}^{zz} g_{\mathrm{m}}^{1/2}(\vartheta)}
\right]}_{\Delta_{\mathrm{Am}}}
\underbrace{\left[
\frac{\hat{\varepsilon}_{\mathrm{B}} - \varepsilon_{\mathrm{m}}^{zz} g_{\mathrm{m}}^{1/2}(\vartheta) }{\hat{\varepsilon}_{\mathrm{B}} + \varepsilon_{\mathrm{m}}^{zz} g_{\mathrm{m}}^{1/2}(\vartheta)}
\right]}_{\Delta_{\mathrm{Bm}}}
e^{-2 g_{\mathrm{m}}^{1/2}(\vartheta) \kappa d} \\
&= 1 - \Delta_{\mathrm{Am}}(\vartheta) \Delta_{\mathrm{Bm}}(\vartheta) e^{-2 g_{\mathrm{m}}^{1/2}(\vartheta) \kappa d}
\end{aligned}
\end{equation}
where
\(\hat{\varepsilon}_{i} = \sqrt{\varepsilon_{i}^{xx}
  \varepsilon_{i}^{zz}}\) is the geometric averaged dielectric function
  of material $i$ ($i$=A, m or B), $\Delta_{\mathrm{Am}}$ and
  $\Delta_{\mathrm{Bm}}$ quantifies the mismatch of dielectric
  functions between A/m and B/m interfaces, respectively.  Combining
  with~\autoref{eq:vdw-lifshitz-integral}, we obtain:
\begin{equation}
\label{eq:vdw-lifshitz-aniso-final}
\begin{aligned}
  \Phi_{\mathrm{AmB}} &= \dfrac{k_{\mathrm{B}} T}{(2 \pi)^{2}}
  {\displaystyle \sideset{}{'}\sum_{n=0}^{\infty}} {\displaystyle
    \int_{0}^{2 \pi}} \mathrm{d}\vartheta {\displaystyle
    \int_{0}^{\infty}} \kappa \mathrm{d}\kappa \ln[1 -
  \Delta_{\mathrm{Am}}(i\xi_{n}, \vartheta)
  \Delta_{\mathrm{Bm}}(i\xi_{n},\vartheta) e^{-2 g_{\mathrm{m}}^{1/2}(i\xi_{n},\vartheta) \kappa d}] \\
  &= \dfrac{k_{\mathrm{B}} T}{16 \pi^{2} d^{2}}
  \sideset{}{'}\sum_{n=0}^{\infty} \int_{0}^{2 \pi}
  \dfrac{1}{g_{\mathrm{m}}(i\xi_{n},\vartheta)}\mathrm{d}\vartheta
  {\displaystyle \int_{0}^{\infty}} x \mathrm{d}x \ln[1 -
  \Delta_{\mathrm{Am}}(i\xi_{n},\vartheta)
  \Delta_{\mathrm{Bm}}(i\xi_{n},\vartheta) e^{-x}]
\end{aligned}
\end{equation}
where the last step is obtained using an auxiliary variable \(x = 2
g_{\mathrm{m}}^{1/2}(\mathcal{\theta}) \kappa d\).

Although seemingly complicated, the \autoref{eq:vdw-lifshitz-aniso-final} provides insights into the essence of the vdW interaction calculations.
% If we assume $\varepsilon_{i}^{xx} = \varepsilon_{i}^{yy}$, the dependency of $g_{\mathrm{m}}$ on $\vartheta$ is removed.
For each frequency $\xi_{n}$ and $\vartheta$, the interaction energy
will be determined by the integral function $I(\Gamma, x)$ defined as:
\begin{equation}
  \label{eq:vdw-I-gamma}
  I(\Gamma, x) = \int_{0}^{x} x' \ln(1 - \Gamma e^{-x'}) \mathrm{d}x'
\end{equation}
where $\Gamma = \Delta_{\mathrm{Am}} \Delta_{\mathrm{Bm}}$. Using the
above notations, \autoref{eq:vdw-lifshitz-aniso-final} is in fact
calculated when $x \to \infty$.
%
First, we take arbitrary values for $\Gamma$ ranging from -1 to 1 and
calculated the value of $I(\Gamma, \infty)$ as function of $\Gamma$,
as shown in \autoref{fig:vdw-integeral-x}\lc{a}. Interestingly, the
integral is very close to $-\Gamma$, in particular when
$\Gamma \to 0$. This leads to the simplification that
$\Phi_{\mathrm{AmB}} \propto \Delta_{\mathrm{Am}}
\Delta_{\mathrm{Bm}}$ frequently used in
literature~\cite{parsegian_van_2010_book,Rajter_2007_vdW,Dryden_2015_gecko}.
%
The monotonic behavior of $I(\Gamma, \infty)$ is
critical for the analysis of bandgap-dependent properties as will be
shown later.
%
On the other hand, the convergence of the integral is almost
independent of $\Gamma$. As shown in \autoref{fig:vdw-integeral-x}\lc{b},
$I(\Gamma, x)$ reaches over 95\% of its converged value when $x > 5$,
regardless of $\Gamma$.
%
In other words, the contributions from high $x$ (or equivalently,
high-{\bfseries k}) regime to the total vdW interactions are
negligible.

\begin{figure}[!htbp]
  \centering{}
  \import{\imgdir}{interg_xx.pgf}
  \caption{\label{fig:vdw-integeral-x}%
    Behavior of the integral function $I(\Gamma, x)$. {\bfseries a}
    $I(\Gamma, \infty)$ is very close to $-\Gamma$, allowing
    simplification of \autoref{eq:vdw-lifshitz-aniso-final} within low
    $\Gamma$ regime. {\bfseries b} Convergence of $I(\Gamma,
    x)$. $I(\Gamma, x)$ reaches $>95$\% of its converged value when
    $x>5$, regardless of $\Gamma$.}
\end{figure}

An important observation of \autoref{eq:vdw-lifshitz-aniso-final} is that
the total energy $\Phi_{\mathrm{AmB}}$ can be greatly attenuated
through a medium with high dielectric anisotropy (\ie
$g_{\mathrm{m}} \gg 1$).
%
Such high dielectric anisotropy between in- and out-of-plane
permittivities (giant birefringence) is only possible to achieve using
layered \cite{Collin_1958_aniso,Weber_2000_aniso} or low-dimensional
\cite{Niu_2018_aniso,Segura_2018_aniso} materials. Inspired by this,
when inserting a 2D material between bulk materials A and B, the
many-body vdW interactions may also be
attenuated by the high dielectric anisotropy, which contradicts with the classical theory of vdW interactions.
%
In the next section, we extend the modified Lifshitz-vdW model in
order to be applied to 2D material-containing systems.


\subsection{Applying Model to 2D Material Systems}
\label{sec:vdw-model-2D}
% 20191014-2027

The general model presented in
\autoref{sec:vdw-gener-model-layer} can be extended to 2D material systems.
%
The exact solution to the many-body vdW interaction energy of such
mixed-dimensional system requires the knowledge of the full
spatially-varied permittivity profile \cite{Podgornik_2004_continuum},
which can be cumbersome to compute.
%
Instead, we use a mean-field treatment for the dielectric response of
the medium, Inspired by the concept in
\autoref{sec:diel-electr-polar-2d}, if medium m consists of a layer of
2D sheet and surrounding vacuum, the average dielectric response
inside the medium m can be calculated using the effective medium
approach using the 2D polarizability $\alpha_{\mathrm{2D}}$:
\begin{equation}
  \label{eq:vdw-emt-alpha}
  \begin{aligned}[t]
    \varepsilon_{\mathrm{m}}^{\parallel} &\approx 1 + \frac{\alpha_{\mathrm{2D}}^{\parallel}}{\varepsilon_{0} d} \\
        \varepsilon_{\mathrm{m}}^{\perp} &\approx \left(1 - \frac{\alpha_{\mathrm{2D}}^{\perp}}{\varepsilon_{0} d}\right)^{-1} \\
  \end{aligned}
\end{equation}
%
where
$\varepsilon_{\mathrm{m}}^{\parallel} = (\varepsilon_{\mathrm{m}}^{xx}
+ \varepsilon_{\mathrm{m}}^{yy}) / 2$ and
$\varepsilon_{\mathrm{m}}^{\perp} = \varepsilon_{\mathrm{m}}^{zz}$.
Note here two assumptions are made: (i) the dielectric properties of
the 2D material are not influenced by the surrounding bulk material (\ie no
substrate doping and interfacial coupling) and (ii) the length scale
of the gap between A and B ($d$) is larger than the intrinsic
thickness of the 2D material, such that an uniform description of
$\varepsilon_{\mathrm{m}}$ is acceptable.
%
Since both criteria can be fulfilled at larger separation
distance~\cite{Dobson_2012_rev} (\eg $d>$1 nm), when the overlapping
between electron clouds of individual materials is negligible, we
limit our discussions in this regime throughout this chapter.
%
Moreover, As discussed above, the total vdW interaction energy is dominated by
the EM modes at $x \leq 5$, or equivalently
\(\kappa \leq 2.5 (g_{\mathrm{m}} d)^{-1}\).
%
When $d$ is at the order of 2 nm, taking typical anisotropy value $g_{\mathrm{m}}=2.5$, we see that the majority of interaction comes
from EM modes with \(\kappa<0.05\) \AA{}\textsuperscript{-1}, close to the optical limit ($k\to0$).
%
Within this regime, $\alpha_{\mathrm{2D}}$ as well
as $\varepsilon_{\mathrm{A}}$ and $\varepsilon_{\mathrm{B}}$ of bulk
materials can be regarded as
constant~\cite{Li_2005_diele}.
%
In other words, the calculation of $\Phi_{\mathrm{AmB}}$ only requires
dielectric data at the optical limit, which greatly simplifies the
computational expense.

A large variety of 2D materials (\eg materials with P6/mmm,
P$\overline{6}$m2, P$\overline{3}$m1, symmetries) has isotropic
in-plane electronic
polarizabilities~\cite{Kittel_2005_introduction_book} (\ie
$\alpha^{xx}_{\mathrm{2D}} = \alpha_{\mathrm{2D}}^{yy} =
\alpha_{\mathrm{2D}}^{\parallel}$) due to symmetric electronic
structure. For an AmB system constructed by these materials, we have
$\varepsilon_{\mathrm{m}}^{xx} =
\varepsilon_{\mathrm{m}}^{yy}=\varepsilon_{\mathrm{m}}^{\parallel}$,
$\varepsilon_{\mathrm{m}}^{zz} = \varepsilon_{\mathrm{2D}}^{\perp}$ ,
and $g_{\mathrm{m}}$ independent of $\vartheta$. As a result,
\autoref{eq:vdw-lifshitz-aniso-final} can be simplified to:
\begin{equation}
    \label{eq:vdw-Phi-aniso-main}
  \begin{aligned}[t]
    \Phi_{\mathrm{AmB}} &= \sideset{}{'} \sum_{n=0}^{\infty} G(i \xi_{n}) \\
G(i \xi_{n}) &= \frac{k_{\mathrm{B}} T}{8 \pi d^{2}} \frac{1}{g_{\mathrm{m}}(i \xi_{n})}
\int_{0}^{\infty} x \ln\left[1 - \hat{\Delta}_{\mathrm{Am}}(i \xi_{n}) \hat{\Delta}_{\mathrm{Bm}}(i \xi_{n}) e^{-x}\right] \mathrm{d} x \\
\hat{\Delta}_{\mathrm{jm}}(i\xi) &= \frac{\hat{\varepsilon}_{\mathrm{j}}(i\xi_{n}) -
\hat{\varepsilon}_{\mathrm{m}}(i\xi_{n})}{\hat{\varepsilon}_{\mathrm{j}}(i\xi_{n}) +
\hat{\varepsilon}_{\mathrm{m}}(i\xi_{n})},\ \mathrm{j=A, B}
\end{aligned}
\end{equation}
where $G(i\xi_{n})$ is the single-frequency interaction energy,
\(\hat{\Delta}_{\mathrm{jm}}\) represents the
$\vartheta$-independent interfacial dielectric mismatch between
materials $j$ (either A or B) and m.
%
\autoref{eq:vdw-Phi-aniso-main} is the governing equation for the calculation
of $\Phi_{\mathrm{AmB}}$ in this chapter. As can be seen, it is
correlated with only a few frequency-dependent dielectric properties
$\varepsilon_{\mathrm{A}}$, $\varepsilon_{\mathrm{B}}$ and
$\alpha_{\mathrm{2D}}$, as well as the distance $d$.


The dielectric function of bulk materials are obtained from both
experimental spectroscopy data~\cite{Palik_1998_handbook} and first
principles simulations.
%
On the other hand, the electronic polarizability of 2D materials are
solely from first principles simulations due to the problem with
experimentally measured optical properties of 2D materials as mention
in \autoref{ch:diel}.
%
Note in the framework of Lifshitz-vdW model, the dielectric responses
are sampled on the imaginary frequencies ($i\xi$), while the data
obtained from optical measurements and first principles calculations
are usually on real frequencies ($\omega$).
% The real and imaginary
% frequencies can be unified by the complex frequency
% $\omega_{\mathrm{C}} = \omega + i\xi$~\cite{parsegian_van_2010_book}.
%
The conversion between $\varepsilon$ and $\alpha_{\mathrm{2D}}$ from
real frequencies to imaginary frequencies are performed via the
Kramers-Kronig relation (KKR)~\cite{Roessler_1965_KKR}, such that:
\begin{eqnarray}
  \label{eq:vdw-KKR-eps}
  \varepsilon(i\xi) &= 1 + {\displaystyle \frac{2}{\pi}}{\displaystyle \int_{0}^{\infty}} {\displaystyle \frac{\omega \mathrm{Im}(\varepsilon(\omega))}{\omega^{2} + \xi^{2}}} \mathrm{d}\omega \\
  \label{eq:vdw-KKR-alpha}
  \alpha_{\mathrm{2D}}(i\xi) &= {\displaystyle \frac{2}{\pi}} {\displaystyle \int_{0}^{\infty}} {\displaystyle \frac{\omega \mathrm{Im}(\alpha_{\mathrm{2D}}(\omega))}{\omega^{2} + \xi^{2}}} \mathrm{d}\omega
\end{eqnarray}
%
Examples of the KKR can be seen in \autoref{fig:vdw-compare-eps-kkr}, where the complex-value $\varepsilon(\omega)$ is converted to monotonically decaying, real-value $\varepsilon(i \xi)$. 
More details about the data acquisition, first principles simulations
and the computation details of the modified Lifshitz model, are
discussed in \autoref{sec:vdw-methods}.

\begin{figure}[!htbp]
  \centering{}
  \import{\imgdir}{eps_kkr.pgf}
  \caption{\label{fig:vdw-compare-eps-kkr} %
    Example of KKR: conversion from complex-value
    $\mathrm{Im}[\varepsilon_{\mathrm{m}}(\omega)]$ to real-value
    $\varepsilon_{\mathrm{m}}(i \xi)$ when m is 2H-MoS$_{2}$ and $d$=2
    nm. The $\varepsilon(i \xi)$ are monotonically decaying functions,
    and are usually $\varepsilon^{\parallel}(i \xi)$ is much larger
    than $\varepsilon^{\perp}(i \xi)$.}
\end{figure}


\subsection{High-Pass vdW Transmission Through 2D Materials}
\label{sec:vdw-high-pass-vdw}


% Compared with $\varepsilon(\omega)$ that is complex-value and has
% multiple transition peaks,
% $\varepsilon(i \xi)$ after the KKR are monotonically decaying,
% real-value functions, which greatly simplifies the analysis in the Lifshitz-vdW model.
%
Using the KKR, the in- and out-of-plane dielectric functions
$\varepsilon_{\mathrm{m}}^{\parallel}(i \xi)$ and
$\varepsilon_{\mathrm{m}}^{\perp}(i \xi)$ for 78 kinds 2D
materials as the medium m at $d=2$ nm, are
shown in \autoref{fig:vdw-eps-iv-all}\lc{a}.
%

\begin{figure}[!htbp]
  \centering
  \import{\imgdir}{eps_and_gm2.pgf}
  \caption{\label{fig:vdw-eps-iv-all} %
    {\bfseries a} Frequency-dependent
    $\varepsilon_{\mathrm{m}}^{\parallel}$ (top) and
    $\varepsilon_{\mathrm{m}}^{\perp}$ (bottom) of different 2D
    materials studied at $d$ = 2 nm, respectively. The insets show the
    in- and out-of-plane transition energies
    $\hbar \xi_{\mathrm{tr}}^{\parallel}$ and
    $\hbar \xi_{\mathrm{tr}}^{\perp}$ as functions of bandgap
    $E_{\mathrm{g}}^{\mathrm{2D}}$. {\bfseries b} Frequency-dependent
    $1/g_{\mathrm{m}}$ corresponding to {\bfseries a}. The curves
    of Graphene (Gr), 2H-MoS$_{2}$ and hBN are highlighted. %
  }
\end{figure}
For illustration purpose, the data are shown on continuous
$\xi$-domain instead of discrete Matsubara frequencies.
%
The
curves of graphene (Gr), 2H-MoS\textsubscript{2}, and hBN are
highlighted.
%
We observe that for a typical 2D medium,
\(\varepsilon_{\mathrm{m}}^{\parallel}\) greatly exceeds
\(\varepsilon_{\mathrm{m}}^{\perp}\) at low frequency range ($\xi <$
10 eV), while the difference diminishes at high frequency regime, when
both $\varepsilon_{\mathrm{m}}^{\parallel}$ and
$\varepsilon_{\mathrm{m}}^{\perp}$ approach unity.
%
To characterize the decaying of $\varepsilon_{\mathrm{m}}^{\parallel}$ $\varepsilon_{\mathrm{m}}^{\perp}$, we introduced the in- and
out-of-plane transition frequencies \(\xi_{\mathrm{tr}}^{\parallel}\)
and \(\xi_{\mathrm{tr}}^{\perp}\), defined as the frequency that
\(\varepsilon_{\mathrm{m}}^{p}(i \xi^{p}_{\mathrm{tr}}) - 1 =
\frac{1}{2}[\varepsilon_{\mathrm{m}}^{p}(i\xi=0) - 1]\), where
$p=\parallel$ or $\perp$.
%

The values of $\xi_{\mathrm{tr}}^{\parallel}$ and
$\xi_{\mathrm{tr}}^{\perp}$ differ greatly.
%
Firstly, \(\xi_{\mathrm{tr}}^{\parallel}\) is generally smaller than
\(\xi_{\mathrm{tr}}^{\perp}\), which is related with the quantum
confinement perpendicular to the 2D
plane~\cite{Matthes_2016_effective_PRB}.
%
Moreover, we found that $\xi_{\mathrm{tr}}^{\parallel}$ almost
linearly scales with the bandgap of the 2D material
$E_{\mathrm{g}}^{\mathrm{2D}}$, while $\xi_{\mathrm{tr}}^{\perp}$ is
almost independent of $E_{\mathrm{g}}^{\mathrm{2D}}$, as shown in the insets of \autoref{fig:vdw-eps-iv-all}\lc{a}.
%
Since the frequency-dependent $\varepsilon_{\mathrm{m}}^{\perp}$ varies
much less than $\varepsilon_{\mathrm{m}}^{\parallel}$, the behavior of
\(1/g_{\mathrm{m}}\) is dominated by
$\varepsilon_{\mathrm{m}}^{\parallel}$, as shown in
\autoref{fig:vdw-eps-iv-all}\lc{b}.
%
Considering the fact that $1/g_{\mathrm{m}}$ modulates the
magnitudes of $\Phi_{\mathrm{AmB}}$ at each frequency (see
\autoref{eq:vdw-lifshitz-aniso-final}), its behavior is analogous to a
high-pass band filter, such that the vdW interactions are selectively screened at low
frequencies, which arises from the dielectric anisotropy of 2D
materials.
%
More interestingly, the minimum value of $1/g_{\mathrm{m}}$ always
occurs at $\xi = 0$ due to its monotonic shape.
%
Following the analysis above, 
$1/g_{\mathrm{m}}=\varepsilon_{\mathrm{m}}^{\perp} /
\varepsilon_{\mathrm{m}}^{\parallel}$ is also dominated by $\xi_{\mathrm{tr}}^{\parallel}$, or equivalently,
$E_{\mathrm{g}}^{\mathrm{2D}}$.
%
As a result, the vdW interaction is expected to be
more screened by a 2D material with smaller bandgap, which is in good
agreement with recent full-scale {\itshape ab initio}
studies~\cite{Liu_2018_gr}.
%
The mechanism behind such high-pass vdW transmission phenomena, is
schematically summarized in \autoref{fig:vdw-scheme-aniso}.
\begin{figure}[!htbp]
  \centering{}
  \import{\imgdir}{scheme_aniso.pdf_tex}
  \caption{\label{fig:vdw-scheme-aniso} %
    Generalized picture of dielectric functions of 2D
    materials. $\varepsilon_{\mathrm{m}}^{\parallel}$ is in general
    much larger than $\varepsilon_{\mathrm{m}}^{\perp}$, resulting in
    the high-pass behavior of $1/g_{\mathrm{m}}$.  }
\end{figure}



% 20191015-1415
The model proposed here is capable of explaining the
strong vdW force screening by 2D materials observed in AFM experiments
\cite{Tsoi_2014_vdW_screening_2D}.
%
To simulate the experimental conditions, we model the
many-body vdW interaction between SiO\textsubscript{2} (A) and Si\textsubscript{3}N\textsubscript{4} (B), by
varying the 2D materials (m), as schematically shown in \autoref{fig:vdw-afm-model}\lc{a}.
% according to Equation \ref{eq:Phi-aniso}.
\autoref{fig:vdw-afm-model}\lc{b} shows \(\Phi_{\mathrm{AmB}}\) as a function of
\(d\) in the model system, when varying the 2D material.
%
Clearly, the presence of a 2D material (colored curves) universally
reduces the vdW interaction energy compared with a non-screened system
(\(\Phi_{\mathrm{AB}}\), gray curve) that the medium m is vacuum.
%
As expected, the reduction of $\Phi_{\mathrm{AmB}}$ is stronger for 2D
materials with smaller \(E_{\mathrm{g}}^{\mathrm{2D}}\).
%
Analogous to the results presented in
\cite{Tsoi_2014_vdW_screening_2D}, the effective Hamaker constants
\(A_{\mathrm{eff}}\) are extracted from fitting of the curve using
\autoref{eq:vdw-hamaker-res}. 
%
Specifically, the effective Hamaker constant in the non-screened system is labeled as $A_{\mathrm{eff}}^{0}$.
%
As shown in the inset of \autoref{fig:vdw-afm-model}\lc{b}, the normalized
Hamaker constant $A_{\mathrm{eff}} / A_{\mathrm{eff}}^{0}$ for various
2D materials at $d=$5 nm, monotonically increases with
\(E_{\mathrm{g}}^{\mathrm{2D}}\).
%
This
observation quantitatively agrees with the experimental ranking of
\(A_{\mathrm{eff}}\) in \cite{Tsoi_2014_vdW_screening_2D}, that graphene (Gr) $<$ 2H-MoS\textsubscript{2} $<$ fluoro\-graphene (F-Gr).
\begin{figure}[!htbp]
  \centering{}
  \import{\imgdir}{model_afm_transparency.pgf}
  \caption{\label{fig:vdw-afm-model}%
    vdW screening caused by 2D materials. {\bfseries a} Simulate the
    AFM measurements using the AmB model presented in this
    chapter. {\bfseries b} Many-body vdW interaction energy
    $\Phi_{\mathrm{AmB}}$. When the medium m contains a 2D material
    sheet (colored curves), $\Phi_{\mathrm{AmB}}$ is screened compared
    with vacuum (non-screened, $\Phi_{\mathrm{AB}}$, grey curve). The screening
    becomes stronger when the bandgap of the 2D material
    $E_{\mathrm{g}}^{\mathrm{2D}}$ decreases, as shown in the
    inset. {\bfseries c} Single frequency interaction energy,
    $|G(i\xi)|$ of the system in {\bfseries b} at $d = 5$ nm. For
    graphene and 2H-MoS$_{2}$, the low frequency part is more
    screened. {\bfseries d} Frequency- dependent vdW transparency
    $\tau(i \xi)$ corresponding to {\bfseries b}, showing the
    high-pass feature. The curves of graphene, 2H-MoS2 and h-BN are
    highlighted.  }
\end{figure}
%
More interestingly, the single frequency interaction energies
\(G(i\xi)\), exhibit distinct patterns between 2D materials. As shown
in \autoref{fig:vdw-afm-model}, comparing with the non-screened case
($G^{0}$), even at \(d\) = 5 nm, semi\-metallic low-gap 2D materials
can still selectively screens low frequencies interactions with
\(\hbar\xi\) \textless{} 1 eV, corresponding to the Debye and Keesom
interactions \cite{Israelachvili_2011_book}.
%
On the other hand, a wide-gap material like hBN shows much weaker high-pass filtering of vdW interactions.
%

To unify the understanding of the 2D material-mediated screening of
vdW interactions, we borrow the concept of the field effect
transparency discussed in \autoref{ch:qc}, to define the vdW
``transparency''. We introduce two variables to quantify such transparency:
\begin{enumerate}
\item {\bfseries Frequency-dependent vdW transparency} $\tau(i\xi)$,
  defined as the ratio between the single-frequency vdW energy with
  and without the 2D material:
  \begin{equation*}
  \label{eq:vdw-def-tau}
  \tau(i \xi) = \frac{G(i \xi)}{G^{0}(i\xi)}
\end{equation*}

\item {\bfseries Total vdW transparency}  $\eta^{\mathrm{vdW}}$, defined as
  the ratio between total vdW energy with and without the 2D material:
  \begin{equation*}
\label{eq:vdw-def-total-trans}
\eta^{\mathrm{vdW}} = \frac{\Phi_{\mathrm{AmB}}}{\Phi_{\mathrm{AB}}}
\end{equation*}
\end{enumerate}
%
\autoref{fig:vdw-afm-model}\lc{d} shows the values of $\tau(i \xi)$
corresponding to \autoref{fig:vdw-afm-model}\lc{c}, which clearly
illustrates the high-pass feature dominated by the dielectric
anisotropy of 2D materials. It is worth noting that such model is
easily extended to multilayer 2D materials, as long as the
polarizability of individual 2D materials are additive (see
\autoref{sec:diel-apply-electr-polar}).

\subsection{Bandgap Dependecy of vdW Transparency}
\label{sec:bandg-depend-vdw}

From \autoref{eq:vdw-Phi-aniso-main}, unlike the factor
\(1/g_{\mathrm{m}}\) that solely depends on the dielectric
anisotropy of 2D material, \(\hat{\Delta}_{\mathrm{Am}}\) and
\(\hat{\Delta}_{\mathrm{Bm}}\) are jointly controlled by the
dielectric properties of both bulk and 2D materials, providing more
dimensions to modulate the many-body vdW transmission.
%
Benefited from
the fast numerical approach in our model, we are able to perform
large-scale evaluation of $\eta^{\mathrm{vdW}}$ on a database with
78 two-dimensional materials and 138 bulk materials, generating a total number of
7.5\texttimes{}10\textsuperscript{5} combinations of AmB systems
(details see \autoref{sec:vdw-methods}).
%
\begin{figure}[!htbp]
  \centering{}
  \import{\imgdir}{fig3.pdf_tex}
  \caption{\label{fig:vdw-database} %
    2D scatter plot of many-body vdW transparency
    $\eta^{\mathrm{vdW}}$ studied by high-throughput screening of bulk
    and 2D material database at distance of 2 nm}
\end{figure}

\autoref{fig:vdw-database} shows the 3D scatter plot of
\(\eta^{\mathrm{vdW}}\) at \(d=2\) nm. The magnitude of
$\eta^{\mathrm{vdW}}$ is indicated by a color mapping.
%
Inspired by the fact that dielectric properties of both bulk
\cite{Moss_1950_relation} and 2D materials (see
\autoref{sec:diel-univ-scal-laws}) are highly related to their
electronic band structure (\ie fundamental band gap), we use the
bandgaps of bulk (\(E_{\mathrm{g}}^{\mathrm{A}}\),
\(E_{\mathrm{g}}^{\mathrm{B}}\)) and 2D materials
(\(E_{\mathrm{g}}^{\mathrm{2D}}\)) as the axes in
\autoref{fig:vdw-database}.
%
As expected, increasing \(E_{\mathrm{g}}^{\mathrm{2D}}\) makes the 2D
material more transparent (red color) to vdW interactions.
%
Conversely, an increase of \(\eta^{\mathrm{vdW}}\) can also achieved
by lowering both \(E_{\mathrm{g}}^{\mathrm{A}}\) and
\(E_{\mathrm{g}}^{\mathrm{B}}\) simultaneously.
%
Clearly, the results in \autoref{fig:vdw-database} shows an systematic
approach to tune the vdW transparency by changing the bulk/2D material
combinations.


To get more insights into the complex dependency of
\(\eta^{\mathrm{vdW}}\) on the combination of materials, we first take
horizontal slices of the 3D scatter plot, to study the effect of
varying the bulk materials while fixing the 2D material, as shown in
\autoref{fig:vdw-compare-slom}\lc{a}.
%
\begin{figure}[h!]
  \centering{}
  \import{\imgdir}{bulk_compare_slom.pgf}
  \caption{\label{fig:vdw-compare-slom}%
    Many-body vdW transparency influenced by the choice of 2D and bulk
    materials. \textbf{a} 2D scatter plots of the simulated
    $\eta^{\mathrm{vdW}}$ as a function of the bandgaps of bulk
    materials ($E_{\mathrm{g}}^{\mathrm{A}}$ and
    $E_{\mathrm{g}}^{\mathrm{B}}$), when the 2D material is graphene
    (i), 2D-MoS2 (ii), and hBN (iii), respectively.  \textbf{b}
    Simulated results from the single Lorentz oscillator model (SLOM)
    corresponding to \textbf{a}.}
\end{figure}
%

As an example, the cases corresponding to graphene,
2H-MoS\textsubscript{2} and hBN are shown in
\autoref{fig:vdw-compare-slom}\lc{a(i)-a(iii)}, respectively.
%
Two trends can be observed: (i) hBN shows higher $\eta^{\mathrm{vdW}}$
(more transparent) compared with graphene and 2H-MoS\textsubscript{2}
regardless of the bulk material, which is ascribed to its large
bandgap. (ii) $\eta^{\mathrm{vdW}}$ is generally higher around the
region that \(E_{\mathrm{g}}^{\mathrm{A}} \to 0\) and
\(E_{\mathrm{g}}^{\mathrm{B}} \to 0\) (lower diagonal edge) compared
with \(E_{\mathrm{g}}^{\mathrm{A}} \gg E_{\mathrm{g}}^{\mathrm{B}}\)
or \(E_{\mathrm{g}}^{\mathrm{B}} \gg E_{\mathrm{g}}^{\mathrm{A}}\)
(off-diagonal edges) in the 2D scatter plots.
%
It is intriguing to see such defined patterns exist, even if the
choice of bulk materials spans over wide range of lattice types and
electronic structures.

Here we show it is possible to describe such bandgap dependency of
$\eta^{\mathrm{vdW}}$ using a relatively simple approach, based on the
single Lorentz oscillator model (SLOM).
%
The idea of SLOM is to describe the monotonically decaying dielectric
function on imaginary frequency with few parameters. We describe the
dielectric function $\varepsilon_{\mathrm{Bulk}}$ of a bulk material
based on the Drude-Lorentz model of single
oscillator~\cite{jackson_classical_1999} and the Kramers-Krönig
relations (KKR):
\begin{equation}
\label{eq:vdw-lorentz-imag}
\varepsilon_{\mathrm{Bulk}}(i \xi) 
=
1 + \frac{\xi_{\mathrm{p}}^{2}}{\xi_{\mathrm{g}}^{2} + K \xi + \xi^{2}}
\end{equation}
where \(\xi_{\mathrm{g}}\) is the intrinsic oscillation frequency,
\(\xi_{\mathrm{p}}\) is the plasma frequency related to the valence
electron density, and \(K\) is the damping parameter.
%
The parameters
\(\xi_{\mathrm{g}}\) and \(\xi_{\mathrm{p}}\) can be extracted using
the following relations:
\begin{enumerate}
\item \(\varepsilon_{\mathrm{Bulk}}(i \xi_{\mathrm{g}}) = \frac{1}{2} (\varepsilon_{\mathrm{Bulk, 0}} - 1)\)
\item \(\xi_{\mathrm{p}}^{2} = (\varepsilon_{\mathrm{Bulk, 0}} - 1) \xi_{\mathrm{g}}^{2}\)
\end{enumerate}
where
$\varepsilon_{\mathrm{Bulk, 0}} = \varepsilon_{\mathrm{Bulk}}(i\xi =
0)$ is the static electronic permittivity.
%

Interestingly, the bulk materials studied here show a general linear
trend between \(\hbar \xi_{\mathrm{g}}\) and the bandgap
\(E_{\mathrm{g}}^{\mathrm{Bulk}}\), as shown in
\autoref{fig:vdw-slom-param}\lc{a}. On the other hand,
\(\hbar \xi_{\mathrm{p}}\) is almost independent of
\(E_{\mathrm{g}}^{\mathrm{Bulk}}\) and scatters between 10\(\sim\)20
eV (\autoref{fig:vdw-slom-param}\lc{b}).
%
For simplicity, we propose the
following relations for \(\xi_{\mathrm{p}}\) and \(\xi_{\mathrm{g}}\):
\begin{enumerate}
\item \(\hbar \xi_{\mathrm{g}}\) = 1.04\(E_{\mathrm{g}}^{\mathrm{Bulk}}\) + 2.25 eV
\item \(\hbar \xi_{\mathrm{p}}\) = 12.7 eV (averaged value of \(\hbar
   \xi_{\mathrm{p}}\))
 \end{enumerate}
 In addition, we choose a relative small $K$ such that $\hbar K$ =
 0.05 eV~\cite{Dryden_2015_gecko}.  The simplicity of SLOM
 allows constructing $\varepsilon_{\mathrm{Bulk}}(i \xi)$ profiles
 with arbitrary bandgap values, and can be used to explore the vdW
 transparency beyond the existing material combinations.

\begin{figure}[!htbp]
  \centering{}
  \import{\imgdir}{eps_3D_slom.pgf}
  \caption{\label{fig:vdw-slom-param}%
    Dependency of parameters in SLOM with the electronic structure of
    bulk materials. \textbf{a} $\xi_{\mathrm{g}}$ appears almost
    linear with the bandgap
    $E_{\mathrm{g}}^{\mathrm{Bulk}}$. \textbf{b} On the contrary,
    $\xi_{\mathrm{p}}$ does not have a defined pattern with
    $E_{\mathrm{g}}^{\mathrm{Bulk}}$.
  }
\end{figure}

Based on the SLOM, \autoref{fig:vdw-compare-slom}\lc{b(i)-b(iii)} show the 2D
contour plots of $\eta^{\mathrm{vdW}}$ for graphene,
2H-MoS\textsubscript{2} and hBN calculated  corresponding to
\autoref{fig:vdw-compare-slom}\lc{a}(i)-a(iii), respectively.
%
To our surprise, the features observed in the scatter plots from
material database can be nicely captured using such simplified
model.
%
Further more, the SLOM resolves even more details: minima along the
diagonal direction (\ie symmetric system, A=B) of the
$\eta^{\mathrm{vdW}}-E_{\mathrm{g}}$ plots can be seen for graphene
and 2H-MoS\textsubscript{2}.
%
The existence of the minima in the vdW transparency of symmetric
systems is an evidence that the vdW transmission through a 2D material
is the competition between the electronic fluctuations in the bulk
material and the dielectric screening through the 2D material. The
idea of $I(\Gamma, \infty)$ in \autoref{sec:vdw-model-2D} can be used
to explain such scenario. In a symmetric system where A=B,
$\Gamma=\hat{\Delta}_{\mathrm{Am}}\hat{\Delta}_{\mathrm{Bm}} \geq 0$ always holds.
%
Consider the fact that $I(\Gamma, \infty) \sim{} -\Gamma$ when
$\Gamma \to 0$, the frequency-dependent transparency $\tau(i\xi)$ can
be zero when
$\hat{\varepsilon}_{\mathrm{A}}(i\xi) =
\hat{\varepsilon}_{\mathrm{B}}(i\xi) =
\hat{\varepsilon}_{\mathrm{m}}(i\xi)$.
%
Ideally, if such condition is fulfilled at any frequency \(i\xi\), the
2D material is completely \textit{opaque} to many-body vdW
interactions. However this is hard to achieve for real materials, as
the frequency-dependent dielectric functions between bulk and 2D
materials can hardly coincide throughout the frequency domain.


We note our theory of bandgap dominance of vdW transparency, does not
violate the recently discovery that 2D materials are more
``transparent'' to remote epitaxy of materials with higher polarity
\cite{Kong_2018_pol}. In such situation where the driving force is the
formation of chemical bonding from interfacial states, the interaction
potential from monopolar surface
\(\Phi_{\pm}\)~\cite{van_Oss_1987_monopolar}, is dominating over the
vdW interactions. Although the Coulombic interactions can also be
effectively screened by 2D material
\cite{Li_2014_screen,Ambrosetti_2019_jpcl}, its strength still
overwhelms the vdW
interactions~\cite{Israelachvili_2011_book}. We also comment that
the single oscillator picture may be oversimplified for many other
materials beyond this study, where more precise techniques of the
dielectric properties such as multiple Lorentz oscillator model are
required.

\subsection{Distance Dependency of Many-body vdW Interactions}
\label{sec:vdw-distance}

The last  component to control the vdW transmission
through 2D materials, is the distance $d$.  From
\autoref{eq:vdw-emt-alpha}, \(\hat{\varepsilon}_{\mathrm{m}}\)
decreases with the distance \(d\) roughly by 
\(d^{-1}\). When \(d \to \infty\), we have
\(\hat{\varepsilon}_{\mathrm{m}} \to 1\) and
\(1/g_{\mathrm{m}} \to 1\) on the whole frequency range. In this
case, the many-body vdW interaction recovers its form when the medium
is vacuum ($\Phi_{\mathrm{AB}}$).
%
In addition to the trivial picture that $\Phi_{\mathrm{AmB}}$
decreases with $d$, the high-pass transmission feature is also
distance-dependent.
%
To see this effect, we plot the frequency-dependent interaction energy
\(G(i \xi)\) at different distances for various symmetric bulk-2D
combinations, as shown in \autoref{fig:vdw-g-dist}.
%
\begin{figure}[!htbp]
  \centering
  \import{\imgdir}{g_distance_bulk.pgf}
  \caption{\label{fig:vdw-g-dist}%
    Single-frequency interaction energy $G(i\xi)$ at different
    distances $d$ for various symmetric bulk-2D combinations, when the
    medium m is \textbf{a} Vacuum (Vac), \textbf{b} hBN, \textbf{c}
    2H-MoS\textsubscript{2} and \textbf{d} graphene (Gr).  Each column
    represents one bulk material (\textbf{i}: Au, \textbf{ii}: GaAs
    and \textbf{iii}: cubic BN). The color of curves indicate the
    magnitude of $d$.  }
\end{figure}

As expected from the classical picture of vdW interactions, if m is
vacuum (\autoref{fig:vdw-g-dist}\lc{a}), the maximum value of $G(i \xi)$ is
always at $\xi \to 0$, and increasing $d$ only trivially down\-scales
the magnitudes of $G(i \xi)$.
%
However for a 2D material-containing medium, a clear change in the
shape of \(G(i \xi)\) can be observed upon increasing $d$.
%
If the 2D material is hBN (\autoref{fig:vdw-g-dist}\lc{b}) or
2H-MoS\textsubscript{2} (\autoref{fig:vdw-g-dist}\lc{c}), the high-pass
feature exists when a wide-gap bulk material like cubic BN is
involved. The peak position drifts to lower energy range with
increasing $d$, indicating the screening effect of the 2D material
becomes closer to vacuum (peak position = 0 eV).
%
Similar red-shift of the peak position with increasing \(d\) is also
observed in graphene-containing systems (\autoref{fig:vdw-g-dist}\lc{d}),
and is universal in all graphene-bulk combinations due to its higher
screening effect.



Another exotic behavior concerning the distance-dependency of vdW
interactions through 2D materials, is its power law with
$d$, which refers to the power index $p$ such that
$\Phi_{\mathrm{AmB}} \propto d^{-p}$~\cite{Gobre_2013_ncomm}.
%
When m is vacuum, \(|\Phi_{\mathrm{AmB}}|\) always exhibits a
power law of \(d^{-2}\), as can be expected from
\autoref{eq:vdw-hamaker-res}.
%
In other words, the log-log plot for \(\Phi_{\mathrm{AmB}}\) vs
\(d\) in an unscreened system shows a straight line with slope of -2,
as shown in the inset of \autoref{fig:vdw-power-law}.
%
However, based on our analysis above, the existence of 2D material
causes stronger screening of $\Phi_{\mathrm{AmB}}$ at shorter
distance, and the profile of \(|\Phi_{\mathrm{AmB}}|\) vs \(d\)
deviates from that in the non-screened case.
%
Apparently, the vdW interaction in such system cannot be simply
described by the $d^{-2}$ Hamaker-vdW model. Here we define the
derivative $-\partial \log(|\Phi_{\mathrm{AmB}}|)/ \partial d$ as the
``local'' power exponent $p_{\mathrm{loc}}$, such that
$\Phi_{\mathrm{AmB}}$ approximately decays at a rate of
$d^{-p_{\mathrm{loc}}}$ at certain value of $d$.
%

\begin{figure}[!htbp]
  \centering{}
  \import{\imgdir}{power_law.pgf}
  \caption{\label{fig:vdw-power-law}%
    Abnormal power law of vdW interactions in the presence of 2D
    materials. The local power law coefficient $p_{\mathrm{loc}}$ of
    vdW interactions as a function of distance d is shown for graphene
    and hBN, when A=SiO$_{2}$ and
    B=Si$_{3}$N$_{4}$. $p_{\mathrm{loc}}$ becomes as small as 0.5 at
    short distance, deviated from the case of 2 when $d \to
    \infty$. Inset: log-log plot of $\Phi_{\mathrm{AmB}}$ vs $d$ for
    unscreened (broken line) and screened (solid line) systems.  }
\end{figure}
To demonstrate, we choose A=SiO\textsubscript{2} and
B=Si\textsubscript{3}N\textsubscript{4} as used in
\autoref{fig:vdw-afm-model} to simulate the AFM measurements in
\cite{Tsoi_2014_vdW_screening_2D}.
%
As shown in \autoref{fig:vdw-power-law}, the vdW interactions through
a 2D material have $p_{\mathrm{loc}} < 2$ due to the screening, which asymptotically approaches the vacuum limit of 2
when $d \to \infty$.
%
In the case of graphene, \(p_{\mathrm{loc}}\) can be as small as
$\sim{}$0.5 at \(d\) = 2 nm.  There are several implications from such
abnormal power law due to the existence of 2D material. Firstly, the
effective Hamaker constant approach that frequently used to
interpreter experimental
results~\cite{Tsoi_2014_vdW_screening_2D,rafiee_2012_transparency}
should be taken with great caution. Moreover, the small power exponents
of many-body vdW interactions makes them even longer range forces than
non-screened vdW interactions. This may lead to some extraordinary
physical phenomena as we show in the next section.

\section{Repulsive vdW Interactions Through 2D Materials}
\label{sec:vdw-repuls-vdw-inter}

\subsection{Origin of Repulsive vdW Interactions}
\label{sec:vdw-origin-repulsive-vdw}

The model proposed in \autoref{sec:vdw-model-lifshitz} leads to some
even more intriguing results.
%
In \autoref{fig:vdw-database}, points corresponding to
\(\eta^{\mathrm{vdW}} < 0\) (blue) can be observed at the limit of
small \(E_{\mathrm{g}}^{\mathrm{A}}\) coupled with large
\(E_{\mathrm{g}}^{\mathrm{B}}\) , and \emph{vice versa}.
%
Similar observations can also be found in
\autoref{fig:vdw-compare-slom}\lc{b} predicted by the SLOM approach.
%
The sign-reversal of $\Phi_{\mathrm{AmB}}$ indicates the repulsive
many-body vdW-Casimir interactions
\cite{Munday_2009_repul,Zhao_2019_casimir_trap} which lead to
fascinating applications including super lubricity
\cite{Feiler_2008_superlubri} and quantum levitation
\cite{MUNDAY_2010_repul}, may exist in 2D material-based systems.
%
Historically, the repulsive vdW interactions are only measured in
liquid systems (m=liquid) since the control of
$\varepsilon_{\mathrm{m}}$ is difficult in solid-state
systems~\cite{Munday_2009_repul,MUNDAY_2010_repul}. However, the
predictions from our model indicates that such repulsive interactions,
can potentially be achieved in 2D material-based systems,
with a proper choice of their dielectric functions.

The origin of the repulsive vdW forces can be explained by analyzing
\autoref{eq:vdw-Phi-aniso-main}.
%
As indicated in \autoref{fig:vdw-integeral-x}\lc{a}, since the sign of
$G(i \xi)$ is determined by the product of
$\hat{\Delta}_{\mathrm{Am}}$ and $\hat{\Delta}_{\mathrm{Bm}}$, it is
possible to have $G(i \xi) > 0$ (repulsive interaction) when
$\hat{\Delta}_{\mathrm{Am}}$ and $\hat{\Delta}_{\mathrm{Bm}}$ have
opposite signs.
%
This can be achieved when a ``dielectric cascade'' exists in the
system, such that
\(\hat{\varepsilon}_{\mathrm{A}} < \hat{\varepsilon}_{\mathrm{m}} <
\hat{\varepsilon}_{\mathrm{B}}\) (or \emph{vice versa}).
%
Inspired by the bandgap-dominance of both
\(\hat{\varepsilon}_{\mathrm{A}}\) and
\(\hat{\varepsilon}_{\mathrm{B}}\), it is intuitive to have a
dielectric cascade system formed by coupling a small-gap and a
large-gap bulk materials.
%
Moreover, the relatively small electronic polarizability of a
large-gap 2D material such as h-BN, makes it difficult to find a bulk
material with
\(\hat{\varepsilon}_{\mathrm{A}} < \hat{\varepsilon}_{\mathrm{m}}\),
which explains the absence of repulsive interactions for hBN in
\autoref{fig:vdw-compare-slom}.
%
% As a demonstration, we consider a model system
% that A=gold (Au) and B=lithium fluoride (LiF) separated by graphene at \(d\)
% = 2 nm.
% %
% The cascade of dielectric functions can be observed as low frequency
% range (\worktodo{which figure, what range?}), where t
% \(G(i \xi)\) becomes repulsive. At higher frequency ranges, the
% high-pass feature of 2D material makes \(G(i \xi)\) again attractive,
% leading to a total repulsive energy, yet of small magnitude.

\subsection{Observation of Repulsive vdW Interactions in Solid-State Systems}
\label{sec:vdw-observ-repuls-vdw}


In this section, we show the preliminary experimental results
concerning the evidence of repulsive vdW interactions in solid-state
films.
%
As the analysis in \autoref{sec:vdw-origin-repulsive-vdw} indicates,
the essence of the repulsive many-body vdW interactions is to have the
dielectric cascade.
such that $\hat{\varepsilon}_{\mathrm{A}} < \hat{\varepsilon}_{\mathrm{m}} <
\hat{\varepsilon}_{\mathrm{B}}$.
%
It is possible to construct such system when A is an organic molecular
crystal, B is a metal with high dielectric function and m is a
semi\-metallic 2D material like graphene.

\begin{figure}[!htbp]
  \centering{}
  \import{\imgdir}{BPE_exp.pdf_tex}
  \caption{\label{fig:vdw-bpe}%
    Experimental observation of repulsive vdW interactions. \textbf{a}
    Experimental setup and scheme of GIXD characterization. \textbf{b}
    Dielectric function profiles (top) and interaction energy profiles
    (bottom) for BPE/Gr/Au (brown) and BPE/Gr/SiO$_{2}$ (violet)
    systems. The 2D GIXD patterns for BPE/Gr/SiO$_{2}$ (\textbf{c})
    BPE/Gr/Au (\textbf{d}) and 1D diffraction patterns (\textbf{e})
    all indicate the inter\-plane distance between BPE molecules
    increases in BPE/Gr/Au sample. }
\end{figure}

% %
%
% %

As a model study, we experimentally compares the molecular
epitaxy of a conjugated organic molecule (N,N’-bis(2-phenyl\-ethyl)-3,4,9,10- peryl-ene\-tetra\-carboxylic, BPE) on top of two different
substrates: SiO\textsubscript{2}-supported graphene
(Gr/SiO\textsubscript{2}) and gold-supported graphene (Gr/Au).
The structure of BPE features two phenyl\-ethyl groups at the side of
the PTCDI basal plane which are freely rotatable. 
%
As discussed in \autoref{sec:inter-forc-at}, epitaxial growth of
organic molecules is the competition between (i) intermolecular vdW
forces, (ii) molecule-2D interactions and (iii) molecule-substrate
interactions.
%
If repulsive interactions exist in (iii), the morphology of the
epitaxial film may be altered.  However, note due to the dominance of (i) and
(ii), we expect such change of morphology, if exists, to be very
small.
%
To accurately determine the crystalline packing in the epitaxial
films, grazing angle X-ray diffraction (GIXD)
spectra~\cite{Shih_2015_PartiallyScreened} were used to examine the
orientation and inter-plane distance between the molecules, as
schematically shown in \autoref{fig:vdw-bpe}\lc{a}.
%

\autoref{fig:vdw-bpe}\lc{b} shows the dielectric function profiles (top)
and frequency-dependent interaction energy $G(i \xi)$ for two systems
at $d=1$ nm.
%
The dielectric function of A is modeled by the SLOM with
$\hbar \xi_{\mathrm{g}}=2.30$ eV and $\hbar \xi_{\mathrm{p}}=3.16$
(corresponding to optical refractive index of $\sim{}$1.7).
%
As shown in \autoref{fig:vdw-bpe}\lc{b} top, the dielectric cascade exists
BPE/Gr/Au system almost throughout the whole frequency domain, while
the dielectric functions of SiO$_{2}$ and BPE are smaller than graphene-containing medium.
%
As a result, our simulation indicates that repulsive
$\Phi_{\mathrm{AmB}}$ exists in BPE/Gr/Au system (brown circles)
compared with the attractive $\Phi_{\mathrm{AmB}}$ in BPE/Gr/SiO$_{2}$
(violet circles) as shown in \autoref{fig:vdw-bpe}\lc{b} bottom, even if
the interactions  in vacuum
$\Phi_{\mathrm{AB}}$ would be attractive and stronger in
BPE/Gr/Au system.
%
%

%
Next we examine the molecular packing on these surfaces using GIXD.
In th 20 nm-thick BPE films deposited onto both
Gr/SiO\textsubscript{2} (BPE/Gr/Si$_{2}$, \autoref{fig:vdw-bpe}\lc{c}) and
Gr/Au (BPE/Gr/Au, \autoref{fig:vdw-bpe}\lc{d}) surfaces, the major peaks in
the GIXD spectra appears around 19 nm$^{-1}$ near the $q_{z}$ axis,
corresponding to the diffraction from the (1$\overline{2}$4) and
(1$\overline{2}1$) lattice
planes~\cite{Hadicke_1986_BPE,Mizuguchi_1998_BPE}.
%
With a lattice spacing of $\sim{}3.3$ \AA{}, these peaks correspond to
the diffraction between the PTCDI basal planes.
%
Such evidence indicates the face-on orientations of the BPE
molecules on both surfaces as mediated by the strong π-π interaction
between conjugated organic molecules and
graphene~\cite{Chiu_2013_BPE}.
%
A closer inspection of the spectra shows that the diffraction momentum
$|q|$ of both (1$\overline{2}$4) and (1$\overline{2}1$) plane,
decreases by $\sim{}$0.4 nm$^{-2}$ in the BPE/Gr/Au system
compared with that in BPE/Gr/SiO2.
%
Equivalently, the distances between the basal plane increased by
$\sim{}0.09$ \AA{} (2.7\%) in  BPE/Gr/Au sample compared with BPE/Gr/SiO2.
%
The change of lattice spacing is further verified by comparing the 1D
angle-averaged intensity profile of GIXD. As shown in
\autoref{fig:vdw-bpe}\lc{e}, the decreasing of $q$ values (largerinter\-plane spacing) is systematically observed in BPE/Gr/Au
sample at high $q$-regime, compared with that in BPE/Gr/SiO$_{2}$ and
power examples.
%
Clearly, such results contradict the classical theory of vdW
transparency: the vdW interactions between an organic molecule and
gold is in general much stronger than that on a polar surface like
SiO\textsubscript{2}, as shown in \autoref{fig:vdw-bpe}\lc{b}.
%
To prove this, we further investigate the GIXD patterns of BPE
deposited on bare SiO\textsubscript{2} and Au surfaces, as shown in
\autoref{fig:vdw-bare-bpe}\lc{a} and b, respectively.

\begin{figure}[!htbp]
  \centering{}
  \import{\imgdir}{compare_bare_gold.pdf_tex}
  \caption{\label{fig:vdw-bare-bpe} %
    GIXD pattern (top) and proposed packing configuration for
    BPE/SiO$_{2}$ (\textbf{a}) and BPE/Au (\textbf{b}) systems,
    showing stronger molecule-substrate interaction on gold surface.}
\end{figure}
%
The packing of BPE apparently flips to the edge-on orientation on
SiO\textsubscript{2}, as indicated by the emergence of a single strong
peak at $q_{z}=3.8$ nm$^{-1}$, corresponding to the (002)
plane.
%
On the other hand, new peaks of $q_{z}=8.9$ nm$^{-1}$ appears in the
case of BPE/Au corresponding to the diffraction of (013) plane,
adapting an intermediate orientation between face-on and edge on.
%
The comparison of orientations between BPE/SiO\textsubscript{2} and BPE/Au cases are
schematically shown in bottom panels of \autoref{fig:vdw-bare-bpe}\lc{a}
and b, respectively.
%
The stronger interaction at BPE/Au interface is revealed by the closer
distance of the PTCDI basal plane to the substrate compared with that
on SiO\textsubscript{2}.
%
To further eliminate the possibility due to substrate doping of
graphene, we monitor the doping density using Raman scattering~\cite{Das_2008_doping} of the
Gr/SiO\textsubscript{2} and Gr/Au systems.
%
The position in the G peak around 1585 cm$^{-1}$ shows no statistic
difference between the two samples, indicating similar doping density.

We note that the results in the GIXD experiments represents the
ensemble packing behavior of the organic molecules within the 20 nm
thickness of the film, in stead of only probing the first few layers.
%
There are two benefits from such experimental setup: (i) the influence
of the interfacial corrugation due to surface roughness and transfer
technique is minimized and (ii) the mean-field treatment of the
modified Lifshitz-vdW model proposed here can be applied to describe
the average many-body interaction potential.
%
Moreover, the change of lattice spacing is constantly observed from
different batches of samples, and the difference in $q$ value is
significantly higher than the spectrum resolution ($\sim{}$0.02
nm$^{-1}$ per pixel).
%
Combining all the evidences above, we propose that the expansion of
lattice spacing observed here, can only be attributed to the existence
of repulsive vdW interactions.
%
To our best knowledge, this is the first experimental demonstration of
repulsive vdW interactions in solid-state systems.

Nevertheless, as we see in the examples here, the change of lattice
spacing is only $\sim{}$2\% due to the competition with other
attractive forces.
%
In the next sections, we propose several new approaches to allow
enhancing the repulsive vdW interactions that can be potentially
measured experimentally.

\subsection{Designing Novel System for Repulsive vdW Interactions}
\label{sec:proposing-new-system}

As discussed in \autoref{sec:vdw-origin-repulsive-vdw}, the repulsive vdW
interactions are elusive due to the difficulty finding universal
systems with dielectric cascade properties.
%
% To enhance the repulsive interactions, one approach is to create systems with dielectric cascade at all frequency ranges.
%
From the material database search, it is particularly difficult to
find a bulk material with $\varepsilon$ constantly smaller than the 2D
material-containing medium throughout the frequency
regime.
% 20191016-0831
Here we propose another approach: repulsive vdW interactions may exist
on suspended 2D material layers. If material A is vacuum or gas,
$\varepsilon_{\mathrm{A}}$ approaches unity at all frequency ranges,
hence $\hat{\Delta}_{\mathrm{Am}}$ will be constantly smaller than 0. A
cascade system in this case, can be constructed by either (i) having
material B with large dielectric function (such as metal), or (ii)
increase the spacing $d$.
%
Here we treat the
distance $d$ as from the bottom of the 2D material's electron cloud to
the closest atomic plane in the bulk material.
%
% Such definition, despite being arbitrary, \worktodo{better words?}
% allows us to apply the modified Lifshitz-vdW model to the suspended 2D
% material systems.
%
Here we propose two possible experimental setups to demonstrate the
repulsive many-body vdW interactions.

\begin{figure}[!htbp]
  \centering{}
  \import{\imgdir}{suspend_energy.pgf}
  \caption{\label{fig:vdw-repul-suspen} %
    Repulsive vdW interactions on suspended 2D material interfaces by
    \textbf{a} liquid condensation and \textbf{b} metal deposition.
    The total $\Phi_{\mathrm{tot}}$ (circles), many-body
    $\Phi_{\mathrm{AmB}}$ (triangles) and two-body
    $\Phi_{\mathrm{2D-B}}$ (squares) interaction energies are shown
    when the 2D material is graphene (blue) or hBN (orange). Water and
    gold are used as the liquid and metal, respectively. Repulsive and
    attractive regimes are labeled in faint red and green,
    respectively }
\end{figure}

\subsubsection{Liquid Condensation on Suspended 2D Materials}
\label{sec:liqu-cond-susp}

The first example considered here is the liquid-suspended 2D
interface, as shown in \autoref{fig:vdw-repul-suspen}\lc{a}.
%
We condensed a water film (B) on top of a suspended 2D material
separated by distance \(d\), while below the 2D layer there is diluted
water vapor (A, \(\hat{\varepsilon}_{\mathrm{A}} \approx 1\)).
%
As a demonstration, we study \(\Phi_{\mathrm{AmB}}\) as function of
\(d\) in such system, when the 2D material is either graphene or hBN ,
shown as triangles in \autoref{fig:vdw-repul-suspen}\lc{a}.
%
In both cases, \(\Phi_{\mathrm{AmB}}\) becomes positive (repulsive)
when \(d\) exceeds a certain threshold $d_{0}$ ($\sim{}30$ \AA{} for
  graphene and $\sim{}18$ \AA{} for hBN).
%
This can be understood from the analysis in
\autoref{sec:vdw-distance}, as $\hat{\varepsilon}_{\mathrm{m}}$
shrinks with increasing $d$, and can eventually be smaller than
$\varepsilon_{\mathrm{B}}$ when $d > d_{0}$.
  %
Under such circumstances the relations $\hat{\Delta}_{\mathrm{Am}}<0$
and $\hat{\Delta}_{\mathrm{Bm}}>0$ are fulfilled, leading to the
many-body repulsive interactions in condensed liquids as previously
proposed in \cite{Bostrom_2012_repulsive,Sengupta_2018_rep}.
%
We also notice the repulsive many-body vdW interactions are easier to
be observed on hBN surfaces due to its relatively small electronic
polarizability.

As discussed before, in solid-state systems, the total adhesion forces
are dominated by the vdW interactions between the 2D and bulk
materials. This makes the properties related to repulsive interactions
difficult to be isolated.
%
However, the situation can be different on suspended 2D sheets. From
the discussions in \autoref{sec:vdw-distance}, the many-body vdW
interactions decay slower compared with the non-screened, two-body vdW
interactions $\Phi_{\mathrm{2D-B}}$ between the 2D layer and the bulk
phase B.
%
As a consequence of the competition between the forces, the
total interaction energy
$\Phi_{\mathrm{tor}} = \Phi_{\mathrm{AmB}} + \Phi_{\mathrm{2D-B}}$ can
be repulsive at a certain distance range.
%
As shown in  \autoref{fig:vdw-repul-suspen}\lc{a}, the total vdW interactions
$\Phi_{\mathrm{tot}}$ (circles) which consists both repulsive 2D interactions
$\Phi_{\mathrm{AmB}}$ (triangles) and two-body attractive interactions
$\Phi_{\mathrm{2D-B}}$ (squares), can have maxima when $d \approx 30 \AA{}$,
corresponding to a small energy barrier of $\sim{}$0.05 mJ$\cdot$m$^{-2}$.
%
Such repulsive total interactions may be potentially observed by
in-situ condensation techniques such as environmental scanning
electron microscope (ESEM), in which the wettability is monitored by
the change of liquid-2D contact angle.


\subsubsection{Epitaxy of Metal on 2D Materials}
\label{sec:epitaxy-metal-2d}

As seen in \autoref{sec:liqu-cond-susp}, although the total adhesion
energy at the liquid-suspended 2D interface can be repulsive, the
energy barrier is relatively small to achieve stable levitation of the
bulk phase~\cite{Zhao_2019_casimir_trap}.
%
To enhance the many-body repulsive interactions, it is desired to have
even larger $\hat{\varepsilon}_{\mathrm{B}}$ compared with water. One
possible approach is epitaxial deposition of metal on suspended 2D
materials.
%
Due to the large dielectric function of metal,
$\hat{\varepsilon}_{\mathrm{B}}$ is constantly larger than
$\hat{\varepsilon}_{\mathrm{m}}$, and the dielectric cascade criteria
$\hat{\Delta}_{\mathrm{Am}} < 0$ and $\hat{\Delta}_{\mathrm{Bm}} > 0$
are always fulfilled, regardless of the choice of 2D material and
distance $d$.
%
This is validated by our simulations in
\autoref{fig:vdw-repul-suspen}\lc{b}, when $\Phi_{\mathrm{AmB}}$ becomes
always repulsive for gold deposited on both suspended graphene and hBN
sheets.
%
More interestingly, the large distinction between
$\hat{\varepsilon}_{\mathrm{B}}$ (when B=gold) and
$\hat{\varepsilon}_{\mathrm{m}}$
increases the magnitude of $\Phi_{\mathrm{AmB}}$ by almost 10 times
compared with liquid-2D systems. As a result, an appreciable repulsive
energy barrier of $\sim{}$0.5 mJ$\cdot$m$^{-2}$ is seen in
$\Phi_{\mathrm{tot}}$ at $d_{0} \approx$25 \AA{}.
%
% Such feature makes the repulsive interactions more observable.
%
As a proposal for future experiments, the repulsive interaction is
potentially observable via epitaxial growth of metal on suspended 2D
material sheets, in which the surface wetting behavior, morphology and
nucleation density of the metal film can be used to probe its
existence.



\section{Conclusions}
\label{sec:vdw-conclusion}

In this chapter, we present a general theoretical framework based on
the modified Lifshitz-vdW theory to describe the many-body
transparency of 2D materials to vdW interactions.
%
The dielectric anisotropy of 2D materials selectively screens the 
low frequency contributions (Keesom and Debye) to the vdW interactions, leading to a high-pass band filter behavior.
%
The formalism presented in this chapter allows fast and large scale
material database screening up to 7.5\texttimes{}10\textsuperscript{5}
material combinations.
%
We find the many-body vdW transparency is a highly non-linear quantity
that dependent majorly on the bandgaps of 2D and bulk materials, which
can be used to quantitatively explain with the force spectroscopy measurements in literature.

More importantly, combining large-scale database screening and
numerical analysis, we further predict the existence of repulsive
interactions when dielectric cascade exists in the system. The
existence of the repulsive vdW interactions is demonstrated by
experimental observations of lattice expansion in molecular epitaxy of
organic molecules on gold-supported graphene interface.
%
Based on our model, we further propose two experimental setups on
suspended 2D material sheets, in which the repulsive vdW interactions
can be potentially observed.

Our theoretical and experimental demonstrations provide a general
picture of 2D material-mediated surface interactions, which can be
used as guidelines for designing vdW heterostructures, 2D
material-based coating as well as wetting phenomena.


\section{Methods}
\label{sec:vdw-methods}

\subsection*{Building Dielectric Property Database}
\label{sec:retr-exper-diel}

The modified Lifshitz-vdW model presented in this chapter relies on
the bulk material library consisted of 78 types of 2D materials and
138 types of bulk materials. Their dielectric properties are either
retrieved from experimental spectroscopy
data!\cite{Palik_1998_handbook} or using first principles simulations.
%
47 bulk materials have their dielectric properties extracted from
\parencite{Palik_1998_handbook}.
%
All these experimental results have optical refractive index $n$ and
extinction coefficient $k$ measured on at least from infrared (IR)
frequencies to more than 20 eV.
%
The relation $\varepsilon(\omega) = (n + ik)^{2}$ is used for the
conversion between $n$, $k$ and dielectric function $\varepsilon$ on
real frequency domain. KKR is further used to map the dielectric response onto imaginary frequencies.
%
Note some materials in the bulk database has quite large ionic
contribution to the permittivity (such as water), which significantly
increases the value of $\varepsilon(i \xi = 0)$. Nevertheless, due to
the fact low-frequency interactions are highly screened by the 2D
material layer, the final result does not have noticeable difference
with or without the ionic effect.

For bulk materials that do not have corresponding experimental data,
and all 2D materials, first principles simulation are performed to
computer either $\varepsilon(\omega)$ or
$\alpha_{\mathrm{2D}}(\omega)$. All simulations are performed by
density functional theory (DFT) package \texttt{GPAW}~\cite{Mortensen_2005_gpaw}
using the projector augmented wave (PAW)
method~\cite{Blochl_1994_PW}.
%
The geometric relaxation and ground-state electronic structure
calculations are performed using PBE Exchange-Correlation
functional\cite{Perdew_1996_GGA} with a plane-wave energy
cutoff of 800 eV for all elements, and a Monkhorst-Pack k-mesh with
point density close to 10 points/\AA{}$^{-1}$.
%
The dielectric functions for bulk materials are calculated with linear
response scheme presented in \cite{Gajdos_2006_opt_PAW}, with
unoccupied bands up to 3 times of the number of valence bands included
to improve convergence of the dielectric functions.
%
As stated in \autoref{sec:vdw-model-2D}, the dielectric responses are
only calculated at $k = 0$, since vdW interactions in the model
considered here is dominated by the responses at optical limit.

The frequency-dependent 2D polarizability
$\alpha_{\mathrm{2D}}(\omega)$ are calculated instead of
$\varepsilon(\omega)$ for 2D materials, due to the ill-defined
dielectric function for a 2D layer, using the relations presented in
\autoref{eq:diele-alpha-para-def} and
\autoref{eq:diele-alpha-perp-def}.

\subsection*{Calculation of Two-Body vdW Intercation Energy}
\label{sec:calculation-two-body}

In the discussion of total repulsive interactions through 2D
materials, the two-body interaction energy $\Phi_{\mathrm{2D-B}}$
between the 2D layer and bulk phase B is required. In this case the
interaction energy is approximated by the difference between two bulk
systems~\cite{parsegian_van_2010_book}:
\begin{equation}
  \label{eq:vdw-phi-2D-B}
  \Phi_{\mathrm{2D - B}}(d) = \Phi_{\mathrm{A'-B}}(d + l_{\mathrm{2D}}) - \Phi_{\mathrm{A'-B}}(d)
\end{equation}
where system A' represents the layered 2D material, and
$l_{\mathrm{2D}}$ is the inter\-layer distance in the bulk form
of 2D materials. $\Phi_{\mathrm{A'-B}}$ is the trivial vacuum-spaced
interaction energy between bulk bodies A' and B (see
\autoref{eq:vdw-lifshitz-aniso-final} where
$\varepsilon_{\mathrm{m}} \equiv 1$).
%
Although there is uncertainty in the $\delta_{\mathrm{2D}}$ determined
from various methods (see \autoref{sec:diel-comp-with-effect}), due to
the fact $d$ discussed in this chapter is at the order of 10$^{1}$
\AA{}, this does not affect the final results.

\subsection*{BPE/Graphene Sample Preparation}
\label{sec:graph-sample-prep}

Monolayer graphene samples are grown using low-pressure chemical vapor
deposition (CVD)~\cite{Li_2009_science_cvd} and transferred onto
different substrates (SiO$_{2}$ or poly\-crystalline gold) using the
polymer-assisted transfer method.
%
After wet transfer, the samples are annealed under argon at 150
$^{\circ}$C for 10 min to remove interfacial water and improve substrate adhesion.
%
The quality of graphene samples is confirmed by
optical and Raman microscopy.
%
No obviously difference between the substrate doping in the different
samples (Gr/SiO$_{2}$, Gr/Au) are observed.
N,N’-bis(2-phenyl\-ethyl)-3,4,9,10- peryl-ene\-tetra\-carboxylic
diimide (BPE) molecules are deposited using physical vapor deposition
(PVD) at 10$^{-5}$ mbar at a deposition rate lower than 1 \AA·s$^{-1}$
onto heated graphene substrates to allow sufficient annealing.
%
  

\section{Author Contributions}
\label{sec:vdw-author-contributions}
T.T. and C.J.S. designed the theoretical and experimental
studies. T.T. and F.N. derived the formalism and performed
calculations of the modified Lifshitz-vdW model. T.T. performed first
principle calculations of dielectric properties with help from
E.J.G.S. T.T. prepared graphene samples with different substrate
support.  Y.T.L. S.W.Z and Y.C.C. performed BPE deposition and GIXD
measurements.











%%% Local Variables:
%%% mode: latex
%%% TeX-master: "../thesis"
%%% End:
