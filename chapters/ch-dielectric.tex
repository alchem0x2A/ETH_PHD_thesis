% \chapter{Introduction}
\chapter{Unified Understanding of the Dielectric Nature of 2D Materials}
\label{ch:diel}
\renewcommand*\imgdir{img/diel/}

% \dictum[Jorge Luis Borges]{%
  % Reality is not always probable, or likely. But if you're writing a
  % story, you have to make it as plausible as you can, because if not,
  % the reader's imagination will reject it.  }%
% \worktodo{find the quote or quit}

\vspace{1em}

\chapterabstract{Part of this chapter appears in the following work:
  Tian, T., Scullion, D., Hughes, D., Li, L. H., Shih, C.-J., Coleman,
  J., Chhowalla, M. \& Santos, E. Electronic polarizability as the
  fundamental variable in the dielectric properties of two-dimensional
  materials, in revision.  }

\section{Introduction}
\label{sec:diel-introduction}

In this chapter, we aim to answer the open questions raised in
\autoref{sec:diel-prop-2d} concerning the dielectric properties of 2D
materials. In particular, we propose that the macroscopic permittivity
/ dielectric constant may not accurately describe the dielectric
properties of an atomically-thin 2D material. Instead, we propose the
use the electronic polarizability as the true descriptor when dealing
with 2D materials.


The permittivity, or dielectric constant $\varepsilon$ plays a crucial
role in bridging various fundamental material properties, such as
bandgap\autocite{Moss_1950_relation,Moss_1985_n_Eg}, optical
absorption\autocite{Kittel_2005_introduction_book} and
conductivity\autocite{Dressel_2001_electrodynamics} with elemental
interactions.  The central place of $\varepsilon$ in solid-state
physics drives the analysis of several phenomena where is common to
classify a material accordingly to its ability to screen an electric
field $\mathscr{E}$. Such definitions determine a broad range of condensed
matter physics, as well as in related fields in chemistry and
materials science.  The ability to compute and measure $\varepsilon$
in bulk materials is well established via different theoretical
\autocite{Adler_1962_eps,Hybertsen_1987} and experimental techniques
\autocite{Palik_1998_handbook} of distinct flavors.
%
Despite its obvious appeal, however, it is still unknown whether such quantity can determine the 
% electronic and
dielectric properties of two-dimensional (2D) materials \autocite{Novoselov_2016_vdW}.  
%
The confined structure of such atomically-thin 2D crystals associated
with the attenuated and anisotropic character of the dielectric
screening
\autocite{Keldysh_1979_eps_multi,Sharma_1985_semiconductor_slab_eps,Low_2014_screening_BP,Cudazzo_2011_screening_2D,Bechstedt_2012_silicene,Cudazzo_2010_screen2D,Nazarov_2015_2D_3D}
have generated long-standing debates whether the dielectric constant
truly represents dielectric features of such low-dimensional systems.
% or it can be used only as an
% additional parameter from the bulk-counterparts.
The controversy of $\varepsilon$ values reported by both theoretical
and experimental approaches can be widely seen throughput
literature\autocite{Li_2016_screening_rev}, where the variation can be
more than one order of magnitude.
% \worktodo{make a copy of Li's paper}
%
As a
consequence, several key physical parameters that scale with
$\varepsilon$, such as the exciton binding energy and Debye screening
length, cannot be reliably estimated due to the discrepancy of
reported $\varepsilon$ values. In this letter, we give a thorough
study of the underlying issue. Using \textit{ab initio} simulations of
a model system, we demonstrate that the macroscopic dielectric
constant strongly depends on the size of the superlattice, making it
unsuitable to describe the screening of a 2D material. On the
contrary, by analyzing the formalism of dielectric tensors, we propose
that the electronic polarizability is the true descriptor which
uniquely captures the dielectric nature of 2D materials. The 2D
electronic polarizability has unique definition, which overcomes the
disadvantage of ambiguity compared with the conventionally used
effective dielectric medium model for 2D materials. Moreover, we show
the universal scaling relations for in- and out-of-plane 2D electronic
polarizabilities with the electronic and geometric information of the
2D material. The concept of electronic polarizabilities bridges the
gap between the dielectric properties of 2D and 3D systems, and can be
extended to characterize the dielectric anisotropy for any material
dimensions.


%%
%%

\section{Results and Discussions}
\label{sec:diel-results-discussions}

\subsection{Lattice-Dependency of Macroscopic Dielectric Constant}
\label{sec:diel-latt-depend-macr}

We first approach the discrepancy of macroscopic dielectric constant
of 2D materials, by showing that the current definition of
$\varepsilon$ used in layered materials is ill-defined.  This can be
viewed in a model system as illustrated
in \autoref{fig-1}, where an isolated 2D material is placed in
the \textit{xy}-plane of a periodically repeating superlattice (SL)
with a length $L$ along the \textit{z}-direction separating the cell
images.%
\begin{figure}[!htbp]
\centering
\import{\imgdir}{fig-problem.pgf}
\caption{\label{fig-1} %
  Issues with macroscopic dielectric constant
  $\varepsilon_{\mathrm{SL}}$ on 2D materials.  \textbf{a}. 3D
  illustration of the spatial distribution of the charge density
  change $\Delta \rho(z)$ along the z-direction for monolayer
  2H-MoS\textsubscript{2} in a periodic superlattice under external
  electric field of 0.01 V/\AA{}.  The green and red regions represent
  negative and positive induced charges, respectively.
  \textbf{b}.  $\varepsilon^{\parallel}_{\mathrm{SL}}$ (top) and
  $\alpha_{\mathrm{2D}}^{\parallel}$ (bottom) as functions of $L$ for
  the 2H TMDCs. \textbf{c}.  $\varepsilon^{\perp}_{\mathrm{SL}}$ (top)
  and $\alpha_{\mathrm{2D}}^{\perp}$ (bottom) as functions of $L$ for
  the 2H TMDCs. The polarizabilities in \textbf{b} and \textbf{c} are
  constant when $L>$15 \AA{}, compared with the $L$-dependence of
  $\varepsilon_{\mathrm{SL}}$. }
\end{figure}
%
The static macroscopic dielectric tensor from the superlattice
$\varepsilon_{\mathrm{SL}}^{pq}$, is determined through fundamental
electrostatics by the response of the polarization density
$\symbf{P}^{p}$ under small perturbation
$\mathscr{E}^{q}$, where $p$, $q$ are the directions of
polarization and electric field,
respectively~\autocite{Dressel_2001_electrodynamics}:
\begin{subequations}
  \begin{eqnarray}
      \label{eq:diele-def-eps-1}
    &\varepsilon_{\mathrm{SL}}^{pq} &= \kappa^{pq} +
                                 {\displaystyle \frac{\partial \symbf{P}^{p}}
                                 {\varepsilon_{0} \partial \mathscr{E}^{q}}} \\
          \label{eq:diele-def-eps-2}
    &\symbf{P}^{p} &=  {\displaystyle \frac{\symbf{u}^{p}}{\Omega}}
                          = {\displaystyle \frac{{\displaystyle
          \int_{\mathrm{SL}} \rho(\symbf{r}) \symbf{r}^{p} d^{3}\symbf{r}}}
                          {AL}}
  \end{eqnarray}
\end{subequations}
where $\kappa$ is the dielectric tensor of the environment,
$\symbf{u}$ is the total dipole moment within the SL, $\rho$ is the
spatial charge density, $\Omega=AL$ is the volume of the supercell,
$A$ is the \textit{xy}-plane area of the SL and $\varepsilon_{0}$ is
vacuum permittivity. Here we limit our study on the electronic
contribution to the macroscopic dielectric constant where the dipole
$\symbf{P}$ comes from the displacement of electrons under external
field.  On the other hand, the ionic contribution to
$\varepsilon_{\mathrm{SL}}$ that comes from displacement of atomic
nuclei is not covered in this due to its complex dependency on several
factors including phonon dispersion~\autocite{Sohier_2017_phonon} as well
as lattice symmetry~\autocite{Laturia_2018_2D_eps}, which requires
further discussions.  The symmetry of 2D materials lead to the
negligible off-diagonal elements of the dielectric tensor
($p \neq q$), while and diagonal elements
$\varepsilon_{\mathrm{SL}}^{xx}$, $\varepsilon_{\mathrm{SL}}^{yy}$ and
$\varepsilon_{\mathrm{SL}}^{zz}$ can be different
\autocite{Sohier_2016_2D_eps}.  Considering that the 2D material is placed
in vacuum ($\kappa^{pp} = 1$ and $\kappa^{pq} = 0$), we can
distinguish two components of $\varepsilon_{\mathrm{SL}}$, namely the
in-plane ($\varepsilon_{\mathrm{SL}}^{\parallel}$) and out-of-plane
($\varepsilon_{\mathrm{SL}}^{\perp}$) dielectric constants, where
$\varepsilon_{\mathrm{SL}}^{\parallel} =
(\varepsilon_{\mathrm{SL}}^{xx} + \varepsilon_{\mathrm{SL}}^{yy})/2$
and
$\varepsilon_{\mathrm{SL}}^{\perp} = \varepsilon_{\mathrm{SL}}^{zz}$.
The absence of bonding perpendicular to the plane confines the induced
dipoles along the \textit{z}-direction within a range of $\sim{}$5--6
\AA{} into the vacuum (\autoref{fig-1}\lc{a}).
%
%
Under a certain external field, when $L$ increases, the confinement of
the induced dipole moment $\symbf{u}$ causes the integral in the
numerator of ~\autoref{eq:diele-def-eps-2} to be saturated, when the
separation is large enough such that dipole moment from periodic
images do not mutually interfere. On the other hand, the increase of
$L$ in the denominator of ~\autoref{eq:diele-def-eps-2} dilutes the polarization density, and in turn makes both
$\varepsilon^{\parallel}_{\mathrm{SL}}$ and
$\varepsilon^{\perp}_{\mathrm{SL}}$ dropping to
unity when $L$ is infinitely large.
% In the case of bulk materials
% such physical dependence is well described given the periodicity of
% the 3D crystal throughout the space.
%
%since this volume is used to represent the crystal 
%periodically throughout the space.  
%
%
% In the case of 2D materials, however, such repetition is artificial along the direction 
% perpendicular to the plane.
Despite  the simplicity of this argument, any calculation performed
using such definition will intrinsically depend on the magnitude of
$L$, an artificial parameter introduced by simulation system. This
dependence can be demonstrated by plotting
$\varepsilon^{\parallel}_{\mathrm{SL}}$ and
$\varepsilon^{\perp}_{\mathrm{SL}}$ calculated from density functional
theory (DFT) (see \autoref{sec:diel-org8457dbb} for details) as a
function of $L$ for P$\overline{6}$m2 transition metal dichalcogenides
(TMDCs), 2H-MX$_{2}$, where M=Mo, W and X=S, Se, Te (top panels of
\autoref{fig-1}\lc{b} and \autoref{fig-1}\lc{c}, respectively). To get
a better description of the electronic band structure, the
calculations of dielectric properties were performed at the level of
Heyd-Scuseria-Ernzerhof (HSE06) hybrid functional
\autocite{Heyd_2003_HSe,HSE_2006_erratum}.  Both components of the
dielectric constant decreases with $L$ as excepted.  The lattice-size
dependency also exists for dielectric function in the frequency
($\omega$) domain. 
% HEre
\begin{figure}[!htbp]
\centering
\import{\imgdir}{freq-compare.pgf}
\caption{\label{fig:diel-freq}
  %
  $L$-dependency of $\varepsilon_{\mathrm{SL}}$ on frequency
  domain. The real part of in- and out-of-plane
  $\varepsilon_{\mathrm{SL}}(\omega)$ for monolayer hBN at different
  $L$ values are shown in \textbf{a} and \textbf{c}, respectively. The
  magnitudes of $\varepsilon_{\mathrm{SL}}$ drops with increasing $L$
  over the whole frequency range. While the 2D polarizability
  $\alpha_{\mathrm{2D}}$ remains invariant of $L$ (\textbf{b} and
  \textbf{d})}
\end{figure}
%
As seen in \autoref{fig:diel-freq}, the magnitude of dielectric
functions calculated universally decreases with $L$ over the frequency
domain.  These results indicate that quantities that depend on
$\varepsilon^{\parallel}_{\mathrm{SL}}(\omega)$ and
$\varepsilon^{\perp}_{\mathrm{SL}}(\omega)$, including the optical
absorption (abs., equivalent to
$\mathrm{Im}(\varepsilon_{\mathrm{SL}}(\omega))$), refractive index
($n$, $\sim{}\sqrt{\mathrm{Re}(\varepsilon_{\mathrm{SL}}}$) and
electron energy loss spectrum (EELS, equivalent to
$-1 / \mathrm{Im}(\varepsilon_{\mathrm{SL}}(\omega))$), suffers the
same deficiencies as $\varepsilon_{\mathrm{SL}}$ for 2D
materials. This simple model system explains the origin of the
discrepancy of dielectric constant values reported in previous
literature~\autocite{Li_2016_screening_rev}.
%
%%%
%
% We observe that neither
% $\varepsilon^{\parallel}_{\mathrm{SL}}$ nor
% $\varepsilon^{\perp}_{\mathrm{SL}}$ converges with $L$ due to fact
% that the long-range Coulomb interaction could not be completed
% screened within the range of magnitudes computationally accessible
%.
%
%
%As a result, the $L$-dependency of $\varepsilon_{\mathrm{SL}}$ makes
%it impractical to represent the dielectric nature of a 2D material
%both theoretically and experimentally, because the size of vacuum
%layer must always be considered. 
%

\subsection{The Electronic Polarizability of 2D Materials}
\label{sec:diel-electr-polar-2d}

To solve the problem described above, we need to find the
$L$-independent alternative of $\varepsilon_{\mathrm{SL}}$, which is
related to both electrostatic and optical properties of a 2D
material~\autocite{Matthes_2016_effective_PRB}. By multiplying ~\autoref{eq:diele-def-eps-2}
with $L$, we obtain the sheet polarization density that is,
$\symbf{\mu}_{\mathrm{2D}}^{p} =\symbf{u}^{p}/A$, in
direction $p$. Following the discussion in the previous section,
$\symbf{\mu}_{\mathrm{2D}}^{p}$ becomes independent of lattice
size when $L$ is large enough, due to the saturation of
$\symbf{u}^{p}$.
%
Similar to the molecular polarizability \autocite{Israelachvili_2011_book}, we
can define a 2D electronic polarizability $\alpha_{\mathrm{2D}}$ which
characterizes the ability to induce dipole moment of a 2D material. It
is associated with $\symbf{\mu}_{\mathrm{2D}}$ through:
$\symbf{\mu}_{\mathrm{2D}}^{p} = \sum_{q}
\alpha_{\mathrm{2D}}^{pq} \mathscr{E}_{\mathrm{loc}}^{q}$
\autocite{Tobik_2004_perp_polariz}, where $\mathscr{E}_{\mathrm{loc}}$ is the local
electric field causing the polarization. At $L \rightarrow \infty$
limit, $\mathscr{E}_{\mathrm{loc}}$ can be solved using electrostatic boundary
conditions of slab geometry \autocite{Markel_2016_EMT}. The continuity of the
electric field along the in-plane direction gives
$\mathscr{E}^{\parallel}_{\mathrm{loc}}=\mathscr{E}^{\parallel}$,
while for the out-of-plane component, the dipole screening yields
$\mathscr{E}_{\mathrm{loc}}^{\perp}=\mathscr{E}^{\perp}+ \mu_{\mathrm{2D}}^{\perp}/\varepsilon_{0}L$
\autocite{Meyer_2001_dipole_slab,Tobik_2004_perp_polariz}, where
$\mathscr{E}^{\parallel}$ and $\mathscr{E}^{\perp}$ are the
external field in the in-plane and out-of-plane directions,
respectively. Combining with \autoref{eq:diele-def-eps-1} and
\autoref{eq:diele-def-eps-2},
$\alpha_{\mathrm{2D}}^{\parallel}$ and $\alpha_{\mathrm{2D}}^{\perp}$
can be related with $\varepsilon_{\mathrm{SL}}^{\parallel}$ and
$\varepsilon_{\mathrm{SL}}^{\perp}$, respectively:
%
%
\begin{subequations}
\begin{eqnarray}
  \label{eq:diele-alpha-para-def}
  &\varepsilon_{\mathrm{SL}}^{\parallel} &= 1 + \frac{\alpha_{\mathrm{2D}}^{\parallel}}{\varepsilon_{0}L}\\
  \label{eq:diele-alpha-perp-def}
  &\varepsilon_{\mathrm{SL}}^{\perp} &= \left(1 - {\displaystyle \frac{\alpha_{\mathrm{2D}}^{\perp}}{\varepsilon_{\mathrm{0}} L}} \right)^{-1}
\end{eqnarray}
\end{subequations}
% 20191003-16:00
% As pointed before, as we limit the discussion on electronic contributions to the dielectric properties, $\alpha_{\mathrm{2D}}$ characterizes the polarizability of electron cloud of a 2D materials.
% refers to the electronic polarizability (the polarizability of the
% electron cloud \autocite{Israelachvili_2011}).

%
%
Using these relations, we show the calculated
$\alpha_{\mathrm{2D}}^{\parallel}$ and $\alpha_{\mathrm{2D}}^{\perp}$
of the selected TMDCs as a function of $L$ in the bottom panels of
 \autoref{fig-1}b and \autoref{fig-1}c, respectively.  In contrast to
$\varepsilon_{\mathrm{SL}}^{\parallel}$ and
$\varepsilon_{\mathrm{SL}}^{\perp}$, We observe that both
$\alpha_{\mathrm{2D}}^{\parallel}$ and $\alpha_{\mathrm{2D}}^{\perp}$
reach convergence when $L$ exceeds 10 \AA and 15 \AA,
respectively. Such result is in good agreement with the spatially
localized induced dipole moment of a 2D material.
%
%
% Interestingly, we can use \autoref{eq:diele-alpha-para-def} and
% \autoref{eq:diele-alpha-perp-def} to reproduce the trends noticed in the top
% panels in \autoref{fig-1}\lc{b} and \autoref{fig-1}\lc{c} for
% $\varepsilon_{\mathrm{SL}}^{\parallel}$ and
% $\varepsilon_{\mathrm{SL}}^{\perp}$ as a function of $L$ around a
% centered magnitude of $\alpha_{\mathrm{2D}}^{\parallel}$ and
% $\alpha_{\mathrm{2D}}^{\perp}$
% ( \autoref{fig-1}\textbf{d}-\textbf{e}).
% A similar curve profile is obtained in both components of the dielectric constant including the much 
% abrupt variation of $\varepsilon_{\mathrm{SL}}^{\perp}$ with $L$ which is due to 
% interlayer interactions.
%\todo[inline]{The original claim of this sentence, that $\varepsilon_{\mathrm{SL}}^{\perp}$ varies more due to interlayer interactions is not 100\% true. I think it is better not to explicitly mention this.}
These relations can also be used to remove the dependence on $L$ for
frequency-dependent $\varepsilon^{\parallel}_{\mathrm{SL}}(\omega)$
and $\varepsilon^{\perp}_{\mathrm{SL}}(\omega)$, generating
lattice-independent electronic polarizability
$\alpha^{\parallel}_{\mathrm{SL}}(\omega)$ and
$\alpha^{\perp}_{\mathrm{SL}}(\omega)$ in the frequency domain,
respectively (\autoref{fig:diel-freq}).
%
% 
% 
% 
% 
The above findings indicate that the electronic polarizability
$\alpha_{\mathrm{2D}}$ captures the essence of the dielectric
properties of 2D materials. In contrast to the ill-defined macroscopic
$\varepsilon_{\mathrm{SL}}$, $\alpha_{\mathrm{2D}}$ has unique
definition, and does not suffer from the dependency on the lattice
size.

\subsection{Comparison with Effective Dielectric Model (EDM)}
\label{sec:diel-comp-with-effect}

Apart from the 2D electronic polarizability proposed here, it is worth
mentioning that the effective dielectric model (EDM) is widely used in
literature to treat the 2D material as a slab with an effective
dielectric tensor $\varepsilon_{\mathrm{2D}}$ and thickness
$\delta^{*}_{\mathrm{2D}}$. Such method can be found in both
experimental and theoretical studies, such as to interpret
ellipsometry data
\autocite{graphene-epsilon10,Duesberg_2014_opt_MoS2,Chiang13,},
reflectance / transmission spectra
\autocite{Li_2014_opt_ml_mos2,Yoffe-Wilson_1969_TMDC}, optical conductance
\autocite{Matthes_2016_effective_PRB} and many-body interactions
\autocite{Sohier_2016_2D_eps,Markel_2016_EMT} of 2D materials. The EDM
allows applying physical laws of bulk systems directly to 2D materials
using $\varepsilon_{\mathrm{2D}}$. However, there are several
drawbacks of such approach, first, the
wavevector($q$)-dependency of dielectric screening in 2D
sheets~\autocite{Cudazzo_2011_screening_2D,Olsen_2016_hydrogen,Trolle_2017_eps_subst}
is not captured. Moreover, due to the
uncertainty of $\delta^{*}_{\mathrm{2D}}$, the calculated
$\varepsilon_{\mathrm{2D}}$, in particular its out-of-plane component,
is extremely sensitive to the choice of $\delta^{*}_{\mathrm{2D}}$,
making such model questionable.


\begin{figure}[!htbp]
  \centering
  \import{\imgdir}{fig-emt-uncertainty.pgf}
  \caption{\label{fig-emt} %
    Breakdown of the effective dielectric model (EDM)
    approach. \textbf{a}.  Scheme of the EDM picture where an
    effective 2D ``dielectric constant'' $\varepsilon_{\mathrm{2D}}$
    is extracted from $\varepsilon_{\mathrm{SL}}$, which depends on
    the effective thickness $\delta_{\mathrm{2D}}^{*}$ of the 2D
    material that is not well-defined. \textbf{b}.  Estimation error
    between $\delta_{\mathrm{2D}}^{\parallel, \text{fit}}$ and
    $\delta_{\mathrm{2D}}^{\perp, \text{fit}}$ from non-linear fitting
    of \textit{ab initio} $\varepsilon_{\mathrm{SL}} - L$ data for
    selected 2D TMDCs. The uncertainty of $\delta_{\mathrm{2D}}$
    causes considerable variation of estimated
    $\varepsilon_{\mathrm{2D}}^{\parallel}$ (\textbf{c}.) and in
    particular $\varepsilon_{\mathrm{2D}}^{\perp}$ (\textbf{d}.).  }
\end{figure}

The basic assumption of EDM can be seen in \autoref{fig-emt}\lc{a},
where the macroscopic $\varepsilon_{\mathrm{SL}}$ is considered to be
contributed by (i) a 2D slab with effective dielectric constant
$\varepsilon_{\mathrm{2D}}$ and thickness $\delta^{*}_{\mathrm{2D}}$
and (ii) vacuum spacing with distance
$L-\delta^{*}_{\mathrm{2D}}$. Using the effective medium theory
(EMT)~\autocite{Aspnes_1982_EMT,Markel_2016_EMT}, the relation between
$\varepsilon_{\mathrm{SL}}$ and $\varepsilon_{\mathrm{2D}}$ can be
expressed using capacitance-like
equations~\autocite{Matthes_2016_effective_PRB,Laturia_2018_2D_eps} :
\begin{subequations}
  \begin{eqnarray}
    \label{eq:diele-emt-1}
    {\displaystyle \varepsilon_{\mathrm{SL}}^{\parallel}} &= {\displaystyle \frac{\delta^{*}_{\mathrm{2D}}}{L} \varepsilon_{\mathrm{2D}}^{\parallel} + \left(1 - \frac{\delta^{*}_{\mathrm{2D}}}{L} \right)}\\
     \label{eq:diele-emt-2}
    {\displaystyle \frac{1}{\varepsilon_{\mathrm{SL}}^{\perp}}} &= {\displaystyle \frac{\delta^{*}_{\mathrm{2D}}}{L} \frac{1}{\varepsilon_{\mathrm{2D}}^{\perp}} + \left(1 - \frac{\delta^{*}_{\mathrm{2D}}}{L} \right)}
  \end{eqnarray}
\end{subequations}
In principle, both the values of $\varepsilon_{\mathrm{2D}}$ and
$\delta^{*}_{\mathrm{2D}}$ are unknown for a certain 2D material. To
minimize the modeling error, we used non-linear least-square fitting
to extract $\varepsilon_{\mathrm{2D}}$ and $\delta^{*}_{\mathrm{2D}}$
of selected 2H TMDCs simultaneously from \textit{ab initio}
$\varepsilon_{\mathrm{SL}}$ -- $L$ data in ~\autoref{fig-1}\lc{b} and
\autoref{fig-1}\lc{c}. The fitted values of the slab thickness,
$\delta_{\mathrm{2D}}^{\parallel, \mathrm{fit}}$ and
$\delta_{\mathrm{2D}}^{\perp, \mathrm{fit}}$ from in-plane and
out-of-plane data, respectively, are shown in
\autoref{tab:diel-fitting} and \autoref{fig-emt}\lc{b}.

\begin{table}[!htbp]
  \centering
  \caption{Fitted thickness parameter $\delta_{\mathrm{2D}}^{*}$ from
    the EDM approach compared with experimental interlayer distance
    $L_{\mathrm{Bulk}}$}
  \label{tab:diel-fitting}
  \begin{tabular}[htbp]{lccccc}
  \hline{}
  Material & $\delta_{\mathrm{2D}}^{\parallel, \mathrm{fit}}$ (\AA) & $\delta_{\mathrm{2D}}^{\perp, \mathrm{fit}}$ ({\AA})& $L_{\mathrm{Bulk}}$ ({\AA}) \\
  \hline{}
  2H-MoS$_{2}$ & 5.76 & 5.49 & 6.15 \\
  2H-MoSe$_{2}$ & 5.98 & 5.92 & 6.46\\
  2H-MoTe$_{2}$ & 6.43 & 6.85 & 6.98 \\
  2H-WS$_{2}$ & 5.63 & 5.49 & 6.15 \\
  2H-WSe$_{2}$ & 5.84 & 5.92 & 6.49 \\
  2H-WTe$_{2}$ & 6.32 & 6.58 & 7.06 \\
  \hline{}
  \end{tabular}
\end{table}
% 
% As seen in 
% \autoref{fig-emt}\textbf{b}, the estimation error of the thickness
% estimated from different directions is around 0.06$\sim{}$0.27
% \AA{}.
%
Although the uncertainty only corresponds to several percent of the
interlayer spacing in the bulk counterparts of these 2D materials, its
influence on the calculated values
$\varepsilon_{\mathrm{2D}}^{\parallel}$ and
$\varepsilon_{\mathrm{2D}}^{\perp}$ is significant. We estimated the
dispersion of $\varepsilon_{\mathrm{2D}}^{\parallel}$ and
$\varepsilon_{\mathrm{2D}}^{\perp}$ if $\delta^{*}_{\mathrm{2D}}$
deviates from the best fitted value by $\pm{}$7.5\%, as shown in
s \autoref{fig-emt}\lc{c} and \lc{d},
respectively. $\varepsilon_{\mathrm{2D}}^{\parallel}$ decays linearly
with $\delta^{*}_{\mathrm{2D}}$,
while strikingly, the value $\varepsilon_{\mathrm{2D}}^{\perp}$ spans
over more than one order of magnitude. The sensitivity of
$\varepsilon_{\mathrm{2D}}^{\perp}$ to $\delta^{*}_{\mathrm{2D}}$
explains the discrepancy in literature that both
isotropic~\autocite{Sohier_2016_2D_eps} and highly
anisotropic~\autocite{Matthes_2016_effective_PRB,Laturia_2018_2D_eps}
$\varepsilon_{\mathrm{2D}}$ tensors for 2D materials under the EDM are
reported. As a consequence, the estimated values of
$\varepsilon_{\mathrm{2D}}$, in particular its out-of-plane component,
is highly questionable.

On the contrary, the proposed $\alpha_{\mathrm{2D}}$ approach does not
suffer from such deficiencies, despite its relatively simple
formalism. The relative uncertainty of $\alpha_{\mathrm{2D}}$ is
generally at the order of 10\textsuperscript{-4}. In addition, the
calculation of $\alpha_{\mathrm{2D}}$ technically simpler than
$\varepsilon_{\mathrm{2D}}$: (i) $\alpha_{\mathrm{2D}}$ can be
achieved using single-point calculation of macroscopic dielectric
tensor, while $\varepsilon_{\mathrm{2D}}$ requires non-linear fitting
of multiple $\varepsilon_{\mathrm{SL}}$ -- $L$ data points; (ii) the
values of $\alpha_{\mathrm{2D}}$ typically converge well for large $L$
($>$20 \AA{}), while $\varepsilon_{\mathrm{2D}}$ suffers from the
uncertainty as described above.

%\subsection*{Polarizability as the fundamental variable}
%\label{sec:diel-first-principles}

%We consider 

\subsection{Universal Scaling Laws of 2D Electronic Polarizability}
\label{sec:diel-univ-scal-laws}

For bulk materials, pioneering works from the 1950s have demonstrated
empirical equations between $\varepsilon$ and the bandgap
$E_{\mathrm{g}}$, including the
Moss~\autocite{Moss_1950_relation,Moss_1985_n_Eg,Finkenrath_1988_moss1} or
Ravindra~\autocite{Ravindra_1980_model,Ravindra_1979_eps_Eg} relations.
Such universal relations, if exist in the context of
$\alpha_{\mathrm{2D}}$, would be of high importance for studying and
predicting the screening of 2D materials.  Inspired by the random
phase approximation (RPA) theory \autocite{Adler_1962_eps} within the
$\mathbf{k} \cdot \mathbf{p}$
formalism\autocite{Kittel_2005_introduction_book,Jiang_2017_Eg_Eb}, we propose
the universal relations for $\alpha_{\mathrm{2D}}^{\parallel}$ and
$\alpha_{\mathrm{2D}}^{\perp}$, depending on the electronic and
geometric information of a 2D material:
% \begin{subequations}
% \begin{eqnarray}
% \label{eq:diele-2D-Moss-para}
  % &\alpha_{\mathrm{2D}}^{\parallel} &=C^{\parallel} E_{g}^{-1} \\
  % \label{eq:diele-2D-Moss-perp}
  % &\alpha_{\mathrm{2D}}^{\perp} & =C^{\perp} \hat{\delta}_{\mathrm{2D}}
% \end{eqnarray}
% \end{subequations}
\begin{subequations}
\begin{eqnarray}
\label{eq:diele-2D-Moss-para}
  &E_{\mathrm{g}} &= C^{\parallel} / \alpha_{\mathrm{2D}}^{\parallel} \\
  \label{eq:diele-2D-Moss-perp}
  &\hat{\delta}_{\mathrm{2D}} & = C^{\perp} \alpha_{\mathrm{2D}}^{\perp} 
\end{eqnarray}
\end{subequations}
where $E_{\mathrm{g}}$ is the fundamental electronic bandgap and
$\hat{\delta}_{\mathrm{2D}}$ is the intrinsic thickness of the 2D
layer, with coefficients
$C^{\parallel} = \dfrac{Ne^2}{2 \pi}$
\autocite{Jiang_2017_Eg_Eb}, where $N$ is a pre-factor associated with the
band degeneracy, and $C^{\perp} = {\varepsilon_{0}}^{-1}$. It is worth
noting, unlike the parameter $\delta^{*}_{\mathrm{2D}}$ that is
artificially assigned within the EDM picture (see previous section),
$\hat{\delta}_{\mathrm{2D}}$ can be uniquely defined by
$\alpha_{\mathrm{2D}}^{\parallel}$, a quantity that can be
computationally and experimentally determined. Despite the
simplicity of \autoref{eq:diele-2D-Moss-para} and \autoref{eq:diele-2D-Moss-perp},
they generalize direct relationships between the electronic polarizability and
the electronic / structural properties for any 2D material in a new
framework.

Next, we show that these equations are valid for the current
library of known layered materials involving different lattice
symmetry, element composition, optical and electronic properties.
%
\begin{figure}[!htbp]
\centering
\import{\imgdir}{fig-structures.pdf_tex}
\caption{\label{fig:diel-struct}
  Structures of the 2D
  materials investigated in this study.}
\end{figure}
%
A high-throughput screening performed on different families of TMDCs
(MX\(_{\text{2}}\), where M is a metal in groups 4, 6, 10, and X=O, S,
Se, Te) and phases (P\={6}m2, P3m1), metal monochalcogenides
(Ga\textsubscript{2}S\textsubscript{2},
Ga\textsubscript{2}Se\textsubscript{2}), cadmium halides (CdX$_2$,
X=Cl, I), hexagonal boron nitride (BN), graphene derivatives
(fluorographene (C\textsubscript{2}F\textsubscript{2}), graphane
(C\textsubscript{2}H\textsubscript{2})), phosphorene
(P\textsubscript{4}) and thin layer organic-inorganic perovskites
(ABX\textsubscript{3}) (\autoref{fig:diel-struct}) shows that our
method enables full correlation between these disparate variables.

\begin{figure}[!htbp]
\centering
\import{\imgdir}{fig-universal.pgf}
\caption{\label{fig:diel-universal} The universal scaling relations
  concerning dielectric nature of 2D materials.
  \textbf{a}. $\alpha_{\mathrm{2D}}^{\parallel}$,
  $\alpha_{\mathrm{2D}}^{\perp}$ (bar plots) and $E_{\mathrm{g}}$
  (blue dots) for all the 2D materials studied.
  $\alpha_{\mathrm{2D}}^{\parallel}$ is observed to descend with
  increasing $E_{\mathrm{g}}$, while no apparent relation between
  $\alpha_{\mathrm{2D}}^{\perp}$ and $E_{\mathrm{g}}$ is
  observed. \textbf{b}.
  $(4\pi \varepsilon_{0})/\alpha_{\mathrm{2D}}^{\parallel}$ (in
  \AA{}$^{-1}$) as a function of $E_{\mathrm{g}}$, showing a linear
  correlation between each other.  \textbf{c}.
  $\alpha_{\mathrm{2D}}^{\perp}/(\varepsilon_{0})$ (in \AA{}) as a
  function of $\delta_{\mathrm{2D}}^{\mathrm{cov}}$ (definition schematically shown
  in the inset), showing a perfect linear relation with slope very
  close to $1$ . The universal scaling relation is also revealed using
  the data from Ref.~\cite{Haastrup_2018_database} (squares), and
  Ref.~\cite{Mounet_2018_database} (triangles) as superimposed on
  \textbf{b} and \textbf{c}.  }
\end{figure}
\autoref{fig:diel-universal}\lc{a} compares the calculated fundamental bandgap
$E_{\mathrm{g}}$ (blue dots) and 2D electronic polarizabilities (bar
plots) of all the 2D materials investigated, covering a wide spectrum
range from far-infrared to ultraviolet.  Note from dimension analysis,
it is more intuitive to express the polarizability as
$\alpha_{\mathrm{2D}}/(4 \pi \varepsilon_{0})$, which has unit of
\AA{}. We find that $\alpha_{\mathrm{2D}}^{\parallel}$ has a general
descending trend when $E_{\mathrm{g}}$ increases, while no apparent
correlation between $\alpha_{\mathrm{2D}}^{\perp}$ and
$E_{\mathrm{g}}$ is observed.
%
We then examine \autoref{eq:diele-2D-Moss-para} and
\autoref{eq:diele-2D-Moss-perp} using the polarizabilities by
first-principle calculations.  \autoref{fig:diel-universal}\lc{b}
shows $(4 \pi \varepsilon_{0})/\alpha_{\mathrm{2D}}^{\parallel}$ (in
\AA{}$^{-1}$) as a function of $E_{\mathrm{g}}$ (in eV) for the 2D
materials investigated (circle dots) at HSE06 hybrid
functional~\autocite{Heyd_2003_HSe,HSE_2006_erratum} level.  A linear
regression coefficient of $R^{2}=0.84$ indicates a strong correlation
between bandgaps and polarizabilities as predicted in
\autoref{eq:diele-2D-Moss-para}.  We also discover that the linearity
between $(4 \pi \varepsilon_{0})/\alpha_{\mathrm{2D}}^{\parallel}$ and
$E_{\mathrm{g}}$ (measured by the $R^{2}$ value) is higher when the
bandgap is calculated using the HSE06 hybrid functional compared with
that from PBE exchange-correlation functional, due to the better
description of bans structure in the hybrid functional
framework~\autocite{Heyd_2005_HSE_GAP}.



We further examine the validity of ~\autoref{eq:diele-2D-Moss-perp},
i.e., the relation between $\alpha_{\mathrm{2D}}^{\perp}$ and the
thickness of a 2D material. To test if the quantity
$\hat{\delta}_{\mathrm{2D}}$ is physical, we choose the ``covalent''
thickness $\delta_{\mathrm{2D}}^{\mathrm{cov}}$ of a 2D material as a
comparison, which is defined as the
longest distance along the \textit{z}-direction between any two atom
nuclei plus their covalent radii:
%
%
\begin{equation}
  \label{eq:diele-cov-thick}
  \delta_{\mathrm{2D}}^{\mathrm{cov}} = \mathrm{max}(|z^{i} - z^{j}|
  + r^{i}_{\mathrm{cov}} + r^{j}_{\mathrm{cov}})
\end{equation}
where $i$, $j$ are atomic indices in the 2D material and
$r_{\mathrm{cov}}^{i}$ is the covalent radius of atom $i$ (
\autoref{fig:diel-universal}\lc{c} inset). As shown in
~\autoref{fig:diel-universal}\lc{d},
$\alpha_{\mathrm{2D}}^{\perp}/\varepsilon_{0}$ (or equivalently,
$\hat{\delta}_{\mathrm{2D}}$) is very close to
$\delta_{\mathrm{2D}}^{\mathrm{cov}}$ with a good linear correlation
of $R^{2}=0.98$. This result indicates a strong relation between
$\alpha_{\mathrm{2D}}^{\perp}$ and the geometry of the 2D layer, which
can be approximated by $\delta_{\mathrm{2D}}^{\mathrm{cov}}$. Similar
to the molecular polarizability which characterizes the volume of the
electron cloud of an isolated molecule \autocite{Israelachvili_2011_book},
$\alpha_{\mathrm{2D}}^{\perp}/\varepsilon_{0}$ is also naturally
related to the characteristic thickness of the electron cloud of 2D
material. A simple explanation for this behavior can be made from
fundamental electrostatics. Consider the smallest repeating unit of
the 2D material with xy-plane area $A$, under small perturbation field
$E$ along the z-direction.  Note that the surface bound charge
$\sigma_{\mathrm{b}}=n e /A$, where $n$ is the number of unit charges
contributes to the bound charges, comes only from the dipoles of the
outer-most atoms, since the induced charges from inner atoms cancel
out (see \autoref{fig:classic-model}).
\begin{figure}[htbp]
  \centering
  \import{\imgdir}{view-classic.pdf_tex}
  \caption{Fundamental electrostatic model for the
    thickness-dependency of $\alpha_{\mathrm{2D}}^{\perp}$, using 2H-MoS2 as an
    example. Left: induced dipoles from individual atoms along the
    z-direction. The positive and negative induced charges from inner
    atoms cancel out. Right: simplified model for the thickness
    dependency of $\alpha_{\mathrm{2D}}^{\perp}$, where the surface dipole density $\symbf{\mu}_{\mathrm{2D}}$ comes only from the outer-most atoms.}
  \label{fig:classic-model}
\end{figure}
From the definition of
$\alpha_{\mathrm{2D}}^{\perp}$, we have:
\begin{equation}
  \label{eq:alpha-classic}
  \alpha_{\mathrm{2D}}^{\perp} = \frac{\symbf{u}_{z}}{\mathcal{E}_{\mathrm{loc}} A}
  = \frac{(d_{\mathrm{max}} + r_{\mathrm{cov}}^{i} + r_{\mathrm{cov}}^{j}) \sigma_{\mathrm{b}}}{\mathscr{E}}
\end{equation}
where $r_{\mathrm{cov}}^{i}$ and $r_{\mathrm{cov}}^{j}$ are the
covalent radii of the outer-most atoms on top and bottom surfaces of
the 2D material, respectively, and $d_{\mathrm{max}}$ is the
\textit{z}-distance between the nuclei of such atoms.  The field $\mathcal{E}$
counterbalances the field from the surface bound charges and equals
$\mathcal{E} = \sigma_{\mathrm{b}}/\varepsilon_{0}$. Therefore, we have:
\begin{equation}
  \label{eq:alpha-classic-2}
  \alpha_{\mathrm{2D}}^{\perp} = (d_{\mathrm{max}} + r_{\mathrm{cov}}^{i} + r_{\mathrm{cov}}^{j})\varepsilon_{0}
                = \delta_{\mathrm{2D}}^{\mathrm{cov}} \varepsilon_{0}
\end{equation}
which explains the linear relation seen in
\autoref{fig:diel-universal}\lc{c}. We can see that such simple model
nicely captures the thickness feature of
$\alpha_{\mathrm{2D}}^{\perp}$. Nevertheless, a quantum-mechanical
approach is still to verify such relations in a more rigorous manner.


Our proposed definition of
$\hat{\delta_{\mathrm{2D}}}$  based on
\autoref{eq:diele-2D-Moss-perp} may have impact on some long-existing
controversies about the experimental thickness of 2D
materials~\autocite{Shearer_2016}, through the measurable quantity
$\alpha_{\mathrm{2D}}^{\perp}$
\autocite{Antoine_1999_polariz_C60,Cherniavskaya_2003_nanocryst_polariz,Krauss_1999_EFM}.
% 
%
To rule out the possibility that our conclusion are limited by the
number of materials due to the high computation expenses, we further
validate \autoref{eq:diele-2D-Moss-para} and
\autoref{eq:diele-2D-Moss-perp} using two different 2D material
databases \autocite{Haastrup_2018_database,Mounet_2018_database}, from
which we extracted the dielectric properties of over 300 compounds
calculated at the PBE functional~\autocite{Perdew_1996_GGA} level, and
superimpose with our results in \autoref{fig:diel-universal}\lc{b} and
\autoref{fig:diel-universal}\lc{c}. Although the calculated dielectric
properties may be different due to choice of functionals as well as
underestimation of the bandgap compared with the more accurate HSE06
functionals \autocite{Van_Dyck_2017}, similar linear trends can be
observed for both $\alpha^{\parallel}_{\mathrm{2D}}$ and
$\alpha_{\mathrm{2D}}^{\perp}$ with the linear coefficient very close
to the HSE06 results, which further proves the validity of our
proposed relations. We have also searched for additional relations
between the 2D polarizabilities with other physical quantities,
including the effective carrier mass, quantum capacitance (density of
states) and total atomic polarizabilities while no apparent
correlations are found.

\subsection{Application in Multilayer and Bulk Systems}
\label{sec:diel-apply-electr-polar}
The concept of electronic polarizability is not limited to monolayer
materials, and can be applied to multilayer and bulk systems as
well. For a 2D material stack composed of $N$ layers, we can define
the electronic polarizability $\alpha_{\mathrm{NL}}$ similar to
 \eqref{eq:diele-alpha-para-def} and \eqref{eq:diele-alpha-para-def}. Note
 $\alpha_{\mathrm{NL}}(N=1)$ is equivalent to $\alpha_{\mathrm{2D}}$.
 \begin{figure}[!htbp]
\centering
\import{\imgdir}{fig-NL.pgf}
\caption{\label{fig:diel-NL} %
  Application of 2D polarizability to multi-layer systems.  Multilayer
  polarizabilities $\alpha_{\mathrm{NL}}^{\parallel}$ and
  $\alpha_{\mathrm{NL}}^{\perp}$ of selected 2D TMDCs (2H-MX$_{2}$,
  M=Mo, W; X=S, Se, Te) as a function of number of layers $N$
  are shown in \textbf{a} and \textbf{b}, respectively.  Both
  $\alpha_{\mathrm{NL}}^{\parallel}$ and
  $\alpha_{\mathrm{NL}}^{\perp}$ linearly scale with $N$ and the
  electronic polarizability of monolayer.}
\end{figure}
%
 \autoref{fig:diel-NL}\lc{a} and \autoref{fig:diel-NL}\lc{b} show
$\alpha_{\mathrm{NL}}^{\parallel}$ and $\alpha_{\mathrm{NL}}^{\perp}$
as functions of $N$ for the selected 2H TMDCs,
respectively. Interestingly, we find that in all cases,
$\alpha_{\mathrm{NL}}$ exhibits nearly ideal linear relation with
$\alpha_{\mathrm{2D}}$, such that
$\alpha_{\mathrm{NL}}^{\parallel}= N \alpha_{\mathrm{2D}}^{\parallel}$
and $\alpha_{\mathrm{NL}}^{\perp}= N
\alpha_{\mathrm{2D}}^{\perp}$. Due to the relatively small
electric applied (0.01 eV/\AA{}), the interlayer interactions within
the stack are negligible. Under such circumstances, $\alpha_{\mathrm{2D}}$ of individual layers is additive, which leads to the following general relation:
\begin{equation}
  \label{eq:diele-alpha-nl}
  \alpha_{\mathrm{NL}}^{p} = \sum_{i=1}^{N} \alpha_{\mathrm{2D, i}}^{p},\quad p=\parallel\ \mathrm{or}\ \perp
\end{equation}
where $\alpha_{\mathrm{2D, i}}$ is the electronic polarizability of
layer $i$, and $p$ is the direction of polarization. This relation can
be further used to calculate screening inside 2D heterostructures
\autocite{Kumar_2016_jpcc,Andersen_2015_dielec_vdWH}.
%
%
%

We further look into the bulk systems. In a bulk layered material stacked by layers with equilibrium
inter-layer distance $L_{\mathrm{Bulk}}$, we define the polarizability
of individual layer $\alpha_{\mathrm{2D}}(L=L_{\mathrm{Bulk}})$ as
$\alpha_{\mathrm{Bulk}}$. Inspired by \autoref{eq:diele-alpha-para-def} and
\autoref{eq:diele-alpha-perp-def}, the dielectric constants
$\varepsilon^{\parallel}_{\mathrm{Bulk}}$ and
$\varepsilon^{\perp}_{\mathrm{Bulk}}$ of the bulk layered material can
be reconstructed by $\alpha_{\mathrm{Bulk}}^{\parallel}$ and
$\alpha_{\mathrm{Bulk}}^{\perp}$ as:
%
%
\begin{subequations}
\begin{align}
  \label{eq:diele-3D-para}
  \varepsilon^{\parallel}_{\mathrm{Bulk}}
  &= 1 + \frac{\alpha_{\mathrm{Bulk}}^{\parallel}}{\varepsilon_{0} L_{\mathrm{Bulk}}}
  \approx 1 + \frac{\alpha_{\mathrm{2D}}^{\parallel}}{\varepsilon_{0} L_{\mathrm{Bulk}}} \\
  \label{eq:diele-3D-perp}
  \varepsilon^{\perp}_{\mathrm{Bulk}}
  &= \left(1 - \frac{\alpha_{\mathrm{Bulk}}^{\perp}}{\varepsilon_{0} L_{\mathrm{Bulk}}}\right)^{-1}
  \approx \left(1 - \frac{\alpha_{\mathrm{2D}}^{\perp}}{\varepsilon_{0} L_{\mathrm{Bulk}}}\right)^{-1}
\end{align}
\end{subequations}
%
%
Here we neglect the effect of the stacking order of the layers.  The
dielectric constant $\varepsilon$ although not well-defined for a
monolayer 2D material becomes applicable when the 2D layers are put
together.
\begin{figure}[!htbp]
\centering
\import{\imgdir}{fig-2D-3D.pdf_tex}
\caption{\label{fig:diel-2D-3D} %
  DFT calculated $\varepsilon_{\mathrm{Bulk}}^{\parallel}$
  (\textbf{a}) and $\varepsilon_{\mathrm{Bulk}}^{\perp}$ (\textbf{b})
  compared with the model values from the 2D polarizability.  }
\end{figure}
%
We compare the values of
$\varepsilon_{\mathrm{Bulk}}^{\parallel}$ and
$\varepsilon_{\mathrm{Bulk}}^{\perp}$ from DFT calculations (\textit{y}-axis)
with those predicted using \autoref{eq:diele-3D-para} and \autoref{eq:diele-3D-perp}
(\textit{x}-axis) as shown in \autoref{fig:diel-2D-3D}\lc{a} and \autoref{fig:diel-2D-3D}\lc{b}. 
Both HSE06 and PBE datasets give almost identical results.  
%
We observe that
$\varepsilon_{\mathrm{bulk}}^{\parallel}$ values calculated by DFT and
predicted by \autoref{eq:diele-3D-para} are in good agreement with a linear
regression slope of 1.01 and $R^2$ of 0.97. Conversely,
$\varepsilon_{\mathrm{Bulk}}^{\perp}$ values predicted from
 \autoref{eq:diele-3D-perp} agree well with the DFT-calculated values when
$E_{\mathrm{g}}>4$ eV, while the deviation becomes larger when
$E_{\mathrm{g}}$ reduces. The above results indicate that
$\alpha^{\parallel}_{\mathrm{Bulk}}$ can generally be estimated from
its 2D counterpart, while $\alpha^{\perp}_{\mathrm{Bulk}}$ differs due
to the interlayer coupling and overlap between induced
dipoles\autocite{Andersen_2015_dielec_vdWH,Laturia_2018_2D_eps}. 
%With a better model describing
%$\alpha_{\mathrm{bulk}}^{\perp}$ as function of $\alpha_{\mathrm{2D}}^{\perp}$ and
%the degree of interlayer coupling, the dielectric transition from 2D
%to 3D would be smoothly described.

\subsection{Unified Geometric Represenation of $\alpha_{\mathrm{2D}}$}
\label{sec:diel-unif-geom-repr}

%
Lastly, we demonstrate that both $\alpha_{\mathrm{2D}}^{\parallel}$
and $\alpha_{\mathrm{2D}}^{\perp}$ can be unified using a geometric
picture. In merit of the unit analysis,
$\alpha_{\mathrm{2D}}^{\parallel}$ and $\alpha_{\mathrm{2D}}^{\perp}$
both have unit of $4\pi\varepsilon_{0} \cdot$[Length]. In other words,
they represent in- and out-of-plane characteristic lengths,
respectively. It is well-known that the in-plane screened
electrostatic potential (Keldysh potential) $V(r)$ from a point charge
as a function of distance $r$:
$V(r) = {\displaystyle \frac{e}{4 \alpha_{\mathrm{2D}}^{\parallel}}}
\left[H_{0}({\displaystyle \frac{2\varepsilon_{0}
      r}{\alpha_{\mathrm{2D}}^{\parallel}}}) - Y_{0}( {\displaystyle
    \frac{2
      \varepsilon_{0}r}{\alpha_{\mathrm{2D}}^{\parallel}}})\right]$
\autocite{Keldysh_1979_eps_multi,Pulci_2014_exciton}, where $H_{0}$ is the Struve
function and $Y_{0}$ is the Bessel function of second kind, is
associated with the in-plane screening radius
$r_{0}^{\parallel}=\alpha_{\mathrm{2D}}^{\parallel}/(2
\varepsilon_{0})$, such that $V(r,r/r^{\parallel}_{0} \gg 1)$ reduces
to the simple Coulomb potential in vacuum. Combining with the result
that $\alpha_{\mathrm{2D}}^{\perp}/\varepsilon_{0}$ characterizes the
thickness of a 2D material, we can view the dielectric screening of a
point charge sitting in the middle of a 2D material as an ellipsoid
with the radii of principal axes to be
$r_{0}^{\parallel} = \alpha_{\mathrm{2D}}^{\parallel}/(2
\varepsilon_{0})$ and
$r_{0}^{\perp} = \alpha^{\perp}_{\mathrm{2D}}/(2 \varepsilon_{0})$,
respectively, as illustrated in ~\autoref{fig-ellip}\lc{a}.
This is analog to the polarizability ellipsoid picture of molecules
used in spectroscopy \autocite{Banwell_1994_spectro_book}. The polarizability ellipsoid
for a 2D material is in general ultra flat, with
$r_{0}^{\parallel} \gg r_{0}^{\perp}$, as demonstrated by the examples
of group 6 2H TMDCs (~\autoref{fig-ellip}\lc{b} and
~\autoref{fig-ellip}\lc{c}).
%
\begin{figure}[!htbp]
  \centering
  \import{\imgdir}{fig-ellipsoid.pdf_tex}
  \caption{\label{fig-ellip} Geometric representation of the
      2D polarizability. \textbf{a}. Scheme of the polarizability
    ellipsoid of a 2D material, with its in-plane
    ($r_{0}^{\parallel}$) and out-of-plane radii
    ($r_{\mathrm{0}}^{\perp}$) proportional to
    $\alpha_{\mathrm{2D}}^{\parallel}$ and
    $\alpha_{\mathrm{2D}}^{\perp}$, respectively.  The values of
    $r_{0}^{\parallel}$ and $r_{0}^{\perp}$ for selected 2D TMDCs are
    shown in \textbf{b} and \textbf{c}, respectively.  The
    polarizability ellipsoid is highly anisotropic with screening much
    larger at in-plane direction than out-of-plane direction.}
\end{figure}

The polarizability ellipsoid picture provides further insights into
the physical nature of $\alpha_{\mathrm{2D}}$: $r_{0}^{\parallel}$ is
close to the exciton radius that confined within the 2D
plane~\autocite{Pulci_2014_exciton}, which is generally larger for a
smaller bandgap semiconductor, and can be converted through the
exciton binding energy as proposed in
Refs. \cite{Olsen_2016_hydrogen,Jiang_2017_Eg_Eb}. On the other hand,
$r_{0}^{\perp}$ may be indirectly deduced from Stark effect
\autocite{Pedersen_2016_shark_effect_TMDC,Klein_2016_stark_eff,Roch_2018_stark_eff}.
% and the low-frequency Raman
% scattering modes of 2D materials \autocite{Tan_2012,Zhang_2013}.

%\subsection{Bridging the 2D and 3D Dielectric Properties}
%\label{sec:diel-2D-3D}
\begin{figure}[!htbp]
  \centering
  \import{\imgdir}{aniso.pdf_tex}
  \caption{Phase diagram of dielectric anisotropy $g$ as function of
d    bandgap $E_{\mathrm{g}}$. The $g$-$E_{\mathrm{g}}$ values of 2D
    materials (blue triangle) and their bulk counterparts (orange
    square) can be distinguished. The $g-E_{\mathrm{g}}$
    values of semiconducting materials in other dimensions are also
    superimposed for comparison.
    % Isotropic dielectric property is
    % observed for bulk covalent materials (3D, red triangle) and
    % fullerenes (0D, green star), while reduced dimensional materials,
    % including planar organic semiconductor (OSc, 1D-2D, brown
    % triangle), carbon nanotube (CNT, magenta circle) and linear OSc
    % (0D-1D, violet pentagon) are scattered along the boundary
    % line.
    % The dimensionality and structure of typical materials are
    % shown along the axis on the right.
    Compared with other materials,
    2D materials and their bulk counterparts provide more flexibility
    of controlling the dielectric anisotropy.}
  \label{fig:diele-aniso}
\end{figure}

Inspired by the polarizability ellipsoid model, we show that a general picture of the dielectric properties in
any dimension can be drawn by studying the dielectric
anisotropy. We define the dielectric anisotropy index $g$ as:
\begin{equation}
  \label{eq:diele-anisotropy}
  \begin{aligned}[t]
    g =
    \begin{cases}
      {\displaystyle \min_{i \neq j}}
      {\displaystyle
        \left(\frac{\varepsilon^{ii}}{\varepsilon^{jj}}\right)},
      \ \mathrm{Bulk\ Materials}\\
      {\displaystyle \min_{i \neq j}}
      {\displaystyle
        \left(\frac{\alpha_{\mathrm{2D}}^{ii}}{\alpha_{\mathrm{2D}}^{jj}}\right)},
      \ \mathrm{2D\ Materials}\\
    \end{cases}
  \end{aligned}
\end{equation}
$g=1$ indicates the material has isotropic dielectric properties while
$g \to 0$ means the dielectric property is highly anisotropic.
\autoref{fig:diele-aniso} shows the phase diagram of $g$ as function
of $E_{\mathrm{g}}$ for 2D materials and their bulk
counterparts. Interestingly, the 2D materials (blue triangles) can be
clearly distinguished from the bulk layered materials (orange squares)
with the boundary line determined to be
$g =0.048 (E_{\mathrm{g}}/ \mathrm{eV})+0.087$. The much lower $g$
values for 2D materials compared with their bulk counterparts
indicates a high dielectric anisotropy, which is responsible for the
unique 2D optoelectronic properties, such as the electrostatic
transparency phenomena~\autocite{Li_2014_screen} and the large exciton binding energies
\autocite{Pulci_2014_exciton,Tran_2014_gap_ML_BP,Chernikov_2014_EB_MoS2_2D3D,Berkelbach_2013_exciton}. From
\autoref{eq:diele-2D-Moss-para}, \autoref{eq:diele-2D-Moss-perp} and
\autoref{eq:diele-anisotropy} we can see $g$ is roughly proportional
to $\delta \cdot E_{\mathrm{g}}$, which explains the observation that
$g$ for 2D materials increase almost linearly with $E_{\mathrm{g}}$,
since the layer thickness $\delta$ (mostly 3--10 \AA{}) of the 2D
materials investigated varies much less than $E_{\mathrm{g}}$ (0.1--7
eV) ( \autoref{fig:diele-aniso}b and
\autoref{fig:diele-aniso}c). Further analysis shows that the
dielectric anisotropy index of any bulk layered material
$g_{\mathrm{Bulk}}$ obeys
$g_{\mathrm{Bulk}} \geq {\displaystyle \frac{4
    g_{\mathrm{2D}}}{(g_{\mathrm{2D}}+1)^{2}}} \geq g_{\mathrm{2D}}$,
where $g_{\mathrm{2D}}$ is the anisotropy index of corresponding 2D
layer, which is the basis for the separation of bulk and 2D regimes in
the $g-E_{\mathrm{g}}$ phase diagram.  For comparison, we also
superimpose the dielectric anisotropy indices of common semiconducting
materials in other dimensions on the phase diagram in
\autoref{fig:diele-aniso}. Bulk covalent (3D, e.g. Si, GaN) and 0D
(fullerenes) semiconductors show isotropic dielectric properties,
scattered on the line $g=1$. On the other hand, reduced dimensionality
increases the dielectric anisotropy of materials such as planar
organic semiconductor (OSc, 1D-2D, e.g. CuPc), carbon nanotube (CNT,
1D), linear OSc (0D-1D, e.g. polyacene and
polyacetylene). Interestingly, most of these materials also scatter
along the boundary line separating the bulk and 2D regimes, indicating
the criteria distinguishing 2D (more anisotropic) and bulk layered
materials (more isotropic) from the $g-E_{\mathrm{g}}$ diagram, can
also be applied to other dimensions. From the phase diagram, we can
see that 2D and bulk layered materials (also including 2D van der
Waals heterostructures (vdWHs) \autocite{Novoselov_2016_vdW}), provide more
flexibility in controlling the dielectric and electronic properties,
compared with semiconductors in other dimensions.

%
%\subsection{Implications for Experiments}
%\todo[inline]{Wait for comments from JC and MC.}


\section{Conclusions}
%\label{sec:diel-org5fd1f1a}
%\todo[inline]{To be changed later when everything fixed}

Our simulation results and theoretical investigations in this chapter
show that the 2D electronic polarizability $\alpha_{\mathrm{2D}}$ is a
local variable determining the dielectric properties of 2D materials.
There exist well-defined relationships between $\alpha_{\mathrm{2D}}$
and other quantities hidden in the electronic properties.  According
to our analysis, simple scaling equations involving bandgap and layer
thicknesses can be used to describe both dielectric and electronic
features at the same footing. A dielectric anisotropy index is found
to relate any material dimension with its controllability.  Thus, our
results suggest that the challenge of understanding the dielectric
phenomena is in general a geometrical problem mediated by the
bandgap. We believe the principles presented here will benefit both
fundamental understanding of 2D materials and rational device
design and optimization.




\section{Methods}
\label{sec:diel-org8457dbb}

\subsection*{First Principles Calculation of Dielectric Properties}
\label{sec:first-princ-calc}

Simulations were carried out using plane-wave density functional
theory package \texttt{VASP}
\autocite{Kresse_1993_MD_liquid_metal,Kresse_1996_1,Kresse_1996_2} using
the projector augmented wave (PAW) approach with GW pseudopotentials
\autocite{Kresse_1999_pseudopotentials}. Band gaps were calculated using
the Heyd-Scuseria-Ernzerhof hybrid functional (HSE06)
\autocite{Heyd_2003_HSe,HSE_2006_erratum}, with spin orbit coupling (SOC) explicitly
included. The geometries were converged both in cell parameters and
ionic positions, with forces below 0.04 eV/\AA. To ensure the accuracy
of dielectric property of monolayer, a vacuum spacing of $>$ 15 \AA~is
used. A k-point grid of \(7\times7\times1\) was used to relax the
superlattice, with an initial relaxation carried out at the
Perdew-Burke-Ernzerhof
(PBE)\autocite{Perdew_1996_GGA,Ernzerhof_1999,Paier_2005_PBE}
exchange-correlation functional level and a subsequent relaxation
carried out at HSE06 level, allowing both cell parameters and ionic
positions to relax each time. In VASP, the tag PREC=High was used,
giving a plane wave kinetic energy cutoff of 30\% greater than the
highest given in the pseudopotentials used in each material. This
guarantees that absolute energies were converged to a few meV and the
stress tensor to within 0.01 kBar.  Calculation of the macroscopic
ion-clamped dielectric tensor were carried out with an
18$\times$18$\times$1 k-grid and electric field strength of 0.001
eV/\AA.  Local field effect corrections are included at the
exchange-correlation potential $V_{\mathrm{xc}}$ at both PBE and HSE06
levels. The materials from Ref.~\cite{Haastrup_2018_database} for comparison
were chosen with the GW bandgap larger than 0.05 eV. Bulk layered
materials were constructed by relaxing the c-axis length of
corresponding monolayer material with the interlayer van der Waals
(vdW) interactions calculated by non-local vdW correlation
functional\autocite{Lee_2010_vdFD2}.  The dielectric properties of bulk
layered materials using VASP were calculated at HSE06 level with
18$\times$18$\times$6 k-grid with same parameter as for monolayer,
while the dielectric properties of bulk counterparts of
Ref.~\cite{Haastrup_2018_database} are calculated at PBE level with a k-point
density of 10~\AA$^{-1}$. Local field effect corrections are also used
for the dielectric properties of bulk systems.



\section{Author Contributions}
\label{sec:diel-author-contributions}
This study is a joint research with the Santos group in Queen's
University Belfast, UK. E.J.G.S. designed the scope of research. T.T.,
D.S. and D.H. performed first-principles calculations. T.T. developed
the theoretical model for 2D electronic polarizability and performed
the analysis of data.








%%% Local Variables:
%%% mode: latex
%%% TeX-master: "../thesis"
%%% End:
