% \chapter{Introduction}
\chapter{Two-Dimensional Materials and Interfaces}
\label{ch:introduction}
\newcommand*\imgdir{img/ch-intro/}

\dictum[Wolfgang Pauli]{%
  God made the bulk;\\Surfaces were invented by the devil
  }%

\vspace{1em}

\chapterabstract{Part of this chapter appears in the following
  journal article: Tian, T. \& Shih, C.-J. Molecular Epitaxy on
  Two-Dimensional Materials: The Interplay between
  Interactions. Ind. Eng. Chem. Res. 56, 10552--10581 (2017).  }


\section{Overview of Two-Dimensional (2D) Materials}
% \section{Introduction}
\label{sec:ch-intro-2D}
Controlling the dimensionality of materials provides rich
opportunities of tuning the electronic, optical and mechanical
properties which facilitates novel devices and applications.
\worktodo{put more things inside and make claim clear} \worktodo{put
  1-3 refs}. Good example of such material design are the
two-dimensional (2D) materials, which are covalently-bonded
crystalline films with only one- or few-atom thickness. \worktodo{HH??
  Change wording}
%
Starting from the first discovering of graphene, an allotrope of
carbon in 2004\worktodo{cite Geim}, the family of 2D materials has
expanded significantly throughout the last decade. \worktodo{cite
  Novoselov and 2 papers} The electronic properties of 2D materials
cover a wide spectrum ranging from insulators to superconductors,
making them promising candidates to replace conventional bulk
materials in semiconductor industries. \worktodo{cite 1-2}.
%
This section aims to give a brief introduction about the structures,
electronic properties and fabrication techniques, which serves as a
prelude for introducing the 2D materials interfaces.
\worktodo{again, change last part}

% Understanding and engineering the
% interfaces of 2D materials are important to achieve practical uses of
% 2D materials in our 3D world. In this chapter, we briefly discuss the
% fundamental aspects of 2D materials and introduce their interfacial
% phenomena. Several challenges about the 2D materials interfaces are
% also discussed, serving as the prelude to the work of this thesis.


\subsection{Categories of 2D Materials}
\label{sec:categ-2d-mater}

So far, there have been no naturally-existing isolated 2D
materials. The closest form is the layered bulk materials, in which
interlayer interactions are governed by van der Waals (vdW) forces.
%
With proper separation techniques, these layered materials can be
thinned to a few layers or even single layer.
%
The best-known example is the mechanical exfoliation of 2D layer from graphite,
now known as graphene. \worktodo{cite geim}
%
Using such technique, monolayers exfoliated from naturally-existing
layered bulk materials were achievable. Examples include the hexagonal
boron nitride (hBN), transition metal dichalcogenides (TMDCs, with
general chemical composition MX\textsubscript{2}, where M is a
transition metal and X is S, Se or Te), monochalcogenides (GaS, GaSe),
monolayer black phosophorus (BP) and MXene (M = Ti, Nb, V, Ta, etc.,
X = C or N).
%
The family of 2D materials that have been experimentally discovered,
has extended from only 1 in 2004 to over 100 \worktodo{really?} as of
2019, corresponding to an average discovery of about 6 new materials
each year. \worktodo{correct?}
%
Apparently, the known 2D materials are still scarce and limited
compared with the bulk materials,
due to the huge experimental effort
required to synthesize, isolate and characterize 2D materials limit
the speed of novel 2D material discovery.
% 
In view of this, computer-aided material design (CAMD) was
demonstrated in several works to help find new 2D materials either
from high-throughput materials database screening or \textit{ab
  initio} calculations. As a pioneer study, Lebègue et al. performed
data mining on the crystal structures listed in the International
Crystallographic Structural Database (ICSD) to filter out a total
number of 92 layered bulk crystals and to generate corresponding 2D
materials. As an extension, Mounet et al. proposed new algorithm with
more robust criteria for dimensionality and found over 5000 layered
materials, among which over 1800 can be easily or potentially
exfoliated according to \textit{ab initio} calculations.
%
Another example is the combinational design of new 2D materials, which
starts with a known structure (for example MX\textsubscript{2}) and
replace with other elements. This allows the discovery of over 3700
thermo-dynamically stable 2D materials, among which most chemical
compositions are not yet discovered experimentally.

These computational studies broaden our knowledge about 2D
materials. In particular, the large size of database allows
systematical study to search for optimal 2D material according to
their electronic and optical properties, and apply to real-world
applications. As a summary, the most common types of 2D materials from
either experimental or computational discoveries, are listed in
\autoref{tab:category-2D}.

\begin{table}
  \centering
  \caption{Summary of common 2D materials and their structures. The
    formula refer to the chemical composition per unit cell.}
  \label{tab:category-2D}
  \begin{tabularx}{1.05\linewidth}{XXXX}
    \hline
    Prototype  & Example Formula  & Lattice Symmetry & Example Materials \\
    \hline
    Graphene & C\textsubscript{2} &  P6/mmm & Graphene, Silicene, Germanene \\
    Graphane & C\textsubscript{2}H\textsubscript{2} &  P3m1 & Graphane, Fluorographene\\
    hBN      & BN                & P$\overline{3}$m2 & h-BN \\
    2H-MX\textsubscript{2} & MoS\textsubscript{2} & P$\overline{3}$m2 & 2H-MoS\textsubscript{2}, 2H-MoS2\textsubscript{2}, 2H-WS\textsubscript{2} \\
    1T-MX\textsubscript{2} & CdI\textsubscript{2} & P$\overline{3}$m1 & 2H-MoS\textsubscript{2}, CdI\textsubscript{2}\\
    BP & P\textsubscript{4} & Pmna & Phosphorene, Arsenene \\
    Monochalcogenidec & Ga\textsubscript{2}S\textsubscript{2} & P$\overline{3}$m2 & Ga\textsubscript{2}S\textsubscript{2}, Ga\textsubscript{2}Se\textsubscript{2} \\
    Bismuth iodide &  Bi\textsubscript{2}I\textsubscript{6} & P$\overline{3}$m1 & Bi\textsubscript{2}I\textsubscript{6}, Al\textsubscript{2}Cl\textsubscript{6} \\
    Iron oxychloride                &  Fe\textsubscript{2}O\textsubscript{2}Cl\textsubscript{2} & Pmmn & Fe\textsubscript{2}O\textsubscript{2}Cl\textsubscript{2}, Fe\textsubscript{2}O\textsubscript{2}Br\textsubscript{2}  \\
    MXene & Ti\textsubscript{x}C\textsubscript{y} & P$\overline{3}$m2 & Ti\textsubscript{3}C\textsubscript{2}, Ti\textsubscript{4}N\textsubscript{4}, Mo\textsubscript{2}TiC\textsubscript{2} \\
    2D Perovskite & CH\textsubscript{3}NH\textsubscript{3}PbBr\textsubscript{3} & Pm$\overline{3}$m & (MAPbBr\textsubscript{3})\textsubscript{n}\\
   \hline
\end{tabularx}
\end{table}

\subsection{Basic Electronic Properties of 2D Materials}
\label{sec:basic-electr-prop}

The common feature of 2D materials is the absence of dangling bonds at
the surface, as a results, the electrons are highly confined within
the 2D plane.
%
For instance, in graphene and hBN, the sp\textsuperscript{2} orbitals
form covalent (σ) bonds, while the p-orbitals perpendicular to the 2D
plane form delocalized π-electron cloud.
%
Such two-dimensional
electron gas (2DEG) gives rise to dramatic change of its electronic
properties compared with bulk materials.\worktodo{cite Devies}
%
Although the physics about 2DEG has been well developed in the 1980s,
prior to the discovery of 2D materials, the 2DEG can only be achieved
by cumbersome semiconductor quantum well heterostructures. In other
words, the 2D materials are perfect candidates to study the behavior
of 2DEGs. \worktodo{say something more?}

Although the electronic band structures of different 2D materials vary
a lot, one thing in common is that their density of states (DOS). The
DOS is a measure of the number of available states in a certain system
at certain energy level. For an extended system with nearly-continuous
energy distribution, the DOS at energy $E$ is expressed as:
\begin{equation}
  \label{eq:ch-intro-dos}
  \mathrm{DOS}(E) = \frac{\partial N}{\Omega \partial E} =  \frac{1}{\Omega} {\displaystyle \int_{\Omega}} \frac{\mathrm{d}^{n} \mathbf{k}}{(2 \pi)^{n}}
  \delta(E - E(\mathbf{k}))
\end{equation}
where the integral is performed over a system with $d$-dimensionality,
$N$ is the total number of states,
$\Omega$ is the volume of the system, $\mathbf{k}$ is the momentum of
electron, and $\delta$ is the Dirac delta function. Without losing
generality, the DOS can be expressed using the law of chains:
\begin{equation}
  \label{eq:dos-chain}
  \mathrm{DOS} = \frac{\partial N}{\Omega \partial k} \frac{\partial k}{\partial E}
               = \frac{k}{2 \pi} \left(\frac{\partial E(k)}{\partial k}\right)^{-1}
\end{equation}
where $k=|\mathbf{k}|$ is the modulus of $\mathbf{k}$.
%
\autoref{eq:dos-chain} provides a general way linking the DOS to the
energy--momentum ($E-k$) dispersion of a 2D material, which is further
extracted from its band structure. For monolayer 2D materials, two
cases can be distinguished:
\begin{itemize}
\item Parabolic materials: such as TMDCs, hBN, phosphorene

  These are the majority of 2D materials where the $E-k$ dispersion is
  parabolic, such that
  ${\displaystyle E(k) = \frac{\hbar^{2} k^{2}}{2 m^{*}}}$, where
  $\hbar$ is the reduced Planck constant, and $m^{*}$ is the effective
  mass near the band edge. From \autoref{eq:dos-chain}, DOS of a
  parabolic material is \textit{constant}, and proportional
  to $m^{*}$.
  
\item Dirac materials: such as graphene, silicene, germanene

  The Dirac materials are relatively rare among 2D materials,
  \worktodo{cite Wang nsr 2015} where the $E-k$ dispersion is linear:
  $E(k) = \hbar v_{\mathrm{F}}k$, where $v_{\mathrm{F}}$ is the Fermi
  velocity. The Dirac materials has DOS increasing linearly with $k$
  (as well a $E$).
\end{itemize}

The difference between the DOS of ideally parabolic and Dirac
materials can be seen in \worktodo{Figure here}. Unlike parabolic
materials with constant DOS, the Dirac materials have DOS → 0 near the
Dirac cone ($E \to 0$). This feature brings very interesting
properties like electrostatic transparency which we will discuss in
\worktodo{Chapter 2}.

\worktodo{Discuss more about the parabolic and Dirac cone?!}


\subsection{Fabrication of 2D Materials}
\label{sec:fabr-2d-mater}





\section{The 2D Materials Interfaces}
\label{sec:2d-mater-interf}

\subsection{The Variety of Mixed-Dimensional Interfaces}
\label{sec:vari-mixed-dimens}

\subsection{Interactions and Forces at the 2D Materials Interfaces}
\label{sec:inter-forc-at}

\section{Challenging Problems Concerning 2D Interfaces}
\label{sec:chall-probl-conc}

\subsection{Electrostatic Interactions Through 2D Sheet}
\label{sec:electr-inter-thro}

\section{Dielectric Properties of 2D Systems}
\label{sec:diel-prop-2d}

\section{van der Waals (vdW) Interactions and Wetting Phenomena}
\label{sec:van-der-waals}



%%% Local Variables:
%%% mode: latex
%%% TeX-master: "../thesis"
%%% End:
