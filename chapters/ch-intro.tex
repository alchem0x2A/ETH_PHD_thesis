% \chapter{Introduction}
\chapter{Two-Dimensional Materials and Interfaces}
\label{ch:introduction}
\newcommand*\imgdir{img/ch-intro/}

\dictum[Wolfgang Pauli]{%
  God made the bulk;\\Surfaces were invented by the devil
  }%

\vspace{1em}

\chapterabstract{Part of this chapter appears in the following
  journal article: Tian, T. \& Shih, C.-J. Molecular Epitaxy on
  Two-Dimensional Materials: The Interplay between
  Interactions. Ind. Eng. Chem. Res. 56, 10552--10581 (2017).  }


\section{Overview of Two-Dimensional (2D) Materials}
% \section{Introduction}
\label{sec:ch-intro-2D}
Controlling the dimensionality of materials provides rich
opportunities of tuning the electronic, optical and mechanical
properties which facilitates novel devices and applications.
\worktodo{put more things inside and make claim clear} \worktodo{put
  1-3 refs}. Good example of such material design are the
two-dimensional (2D) materials, which are covalently-bonded
crystalline films with only one- or few-atom thickness. \worktodo{HH??
  Change wording}
%
Starting from the first discovering of graphene, an allotrope of
carbon in 2004\worktodo{cite Geim}, the family of 2D materials has
expanded significantly throughout the last decade. \worktodo{cite
  Novoselov and 2 papers} The electronic properties of 2D materials
cover a wide spectrum ranging from insulators to superconductors,
making them promising candidates to replace conventional bulk
materials in semiconductor industries. \worktodo{cite 1-2}.
%
This section aims to give a brief introduction about the structures,
electronic properties and fabrication techniques, which serves as a
prelude for introducing the 2D materials interfaces.
\worktodo{again, change last part}

% Understanding and engineering the
% interfaces of 2D materials are important to achieve practical uses of
% 2D materials in our 3D world. In this chapter, we briefly discuss the
% fundamental aspects of 2D materials and introduce their interfacial
% phenomena. Several challenges about the 2D materials interfaces are
% also discussed, serving as the prelude to the work of this thesis.


\subsection{Categories of 2D Materials}
\label{sec:categ-2d-mater}

So far, there have been no naturally-existing isolated 2D
materials. The closest form is the layered bulk materials, in which
interlayer interactions are governed by van der Waals (vdW) forces.
%
With proper separation techniques, these layered materials can be
thinned to a few layers or even single layer.
%
The best-known example is the mechanical exfoliation of 2D layer from graphite,
now known as graphene. \worktodo{cite geim}
%
Using such technique, monolayers exfoliated from naturally-existing
layered bulk materials were achievable. Examples include the hexagonal
boron nitride (hBN), transition metal dichalcogenides (TMDCs, with
general chemical composition MX\textsubscript{2}, where M is a
transition metal and X is S, Se or Te), monochalcogenides (GaS, GaSe),
monolayer black phosophorus (BP) and MXene (M = Ti, Nb, V, Ta, etc.,
X = C or N).
%
The family of 2D materials that have been experimentally discovered,
has extended from only 1 in 2004 to over 100 \worktodo{really?} as of
2019, corresponding to an average discovery of about 6 new materials
each year. \worktodo{correct?}
%
Apparently, the known 2D materials are still scarce and limited
compared with the bulk materials,
due to the huge experimental effort
required to synthesize, isolate and characterize 2D materials limit
the speed of novel 2D material discovery.
% 
In view of this, computer-aided material design (CAMD) was
demonstrated in several works to help find new 2D materials either
from high-throughput materials database screening or \textit{ab
  initio} calculations. As a pioneer study, Lebègue et al. performed
data mining on the crystal structures listed in the International
Crystallographic Structural Database (ICSD) to filter out a total
number of 92 layered bulk crystals and to generate corresponding 2D
materials. As an extension, Mounet et al. proposed new algorithm with
more robust criteria for dimensionality and found over 5000 layered
materials, among which over 1800 can be easily or potentially
exfoliated according to \textit{ab initio} calculations.
%
Another example is the combinational design of new 2D materials, which
starts with a known structure (for example MX\textsubscript{2}) and
replace with other elements. This allows the discovery of over 3700
thermo-dynamically stable 2D materials, among which most chemical
compositions are not yet discovered experimentally.

These computational studies broaden our knowledge about 2D
materials. In particular, the large size of database allows
systematical study to search for optimal 2D material according to
their electronic and optical properties, and apply to real-world
applications. As a summary, the most common types of 2D materials from
either experimental or computational discoveries, are listed in
\autoref{tab:category-2D}.

\begin{table}
  \centering
  \caption{Summary of common 2D materials and their structures. The
    formula refer to the chemical composition per unit cell.}
  \label{tab:category-2D}
  \begin{tabularx}{1.00\textwidth}{XXXX}
    \hline
    Prototype  & Sample Formula  & Symmetry & Examples \\
    \hline
    Graphene & C\textsubscript{2} &  P6/mmm & Graphene, Silicene, Germanene \\
    Graphane & C\textsubscript{2}H\textsubscript{2} &  P3m1 & Graphane, Fluorographene\\
    hBN      & BN                & P$\overline{3}$m2 & h-BN \\
    2H-MX\textsubscript{2} & MoS\textsubscript{2} & P$\overline{3}$m2 & 2H-MoS\textsubscript{2}, 2H-MoS2\textsubscript{2}, 2H-WS\textsubscript{2} \\
    1T-MX\textsubscript{2} & CdI\textsubscript{2} & P$\overline{3}$m1 & 2H-MoS\textsubscript{2}, CdI\textsubscript{2}\\
    BP & P\textsubscript{4} & Pmna & Phosphorene, Arsenene \\
    Monochalcogenidec & Ga\textsubscript{2}S\textsubscript{2} & P$\overline{3}$m2 & Ga\textsubscript{2}S\textsubscript{2}, Ga\textsubscript{2}Se\textsubscript{2} \\
    Bismuth iodide &  Bi\textsubscript{2}I\textsubscript{6} & P$\overline{3}$m1 & Bi\textsubscript{2}I\textsubscript{6}, Al\textsubscript{2}Cl\textsubscript{6} \\
    Iron oxychloride                &  Fe\textsubscript{2}O\textsubscript{2}Cl\textsubscript{2} & Pmmn & Fe\textsubscript{2}O\textsubscript{2}Cl\textsubscript{2}, Fe\textsubscript{2}O\textsubscript{2}Br\textsubscript{2}  \\
    MXene & Ti\textsubscript{x}C\textsubscript{y} & P$\overline{3}$m2 & Ti\textsubscript{3}C\textsubscript{2}, Ti\textsubscript{4}N\textsubscript{4}, Mo\textsubscript{2}TiC\textsubscript{2} \\
    2D Perovskite & CH\textsubscript{3}NH\textsubscript{3}PbBr\textsubscript{3} & Pm$\overline{3}$m & (MAPbBr\textsubscript{3})\textsubscript{n}\\
   \hline
\end{tabularx}
\end{table}

\subsection{Basic Electronic Properties of 2D Materials}
\label{sec:basic-electr-prop}

The common feature of 2D materials is the absence of dangling bonds at
the surface, as a results, the electrons are highly confined within
the 2D plane.
%
For instance, in graphene and hBN, the sp\textsuperscript{2} orbitals
form covalent (σ) bonds, while the p-orbitals perpendicular to the 2D
plane form delocalized π-electron cloud.
%
Such two-dimensional
electron gas (2DEG) gives rise to dramatic change of its electronic
properties compared with bulk materials.\worktodo{cite Devies}
%
Although the physics about 2DEG has been well developed in the 1980s,
prior to the discovery of 2D materials, the 2DEG can only be achieved
by cumbersome semiconductor quantum well heterostructures. In other
words, the 2D materials are perfect candidates to study the behavior
of 2DEGs. \worktodo{say something more?}

Although the electronic band structures of different 2D materials vary
a lot, one thing in common is that their density of states (DOS). The
DOS is a measure of the number of available states in a certain system
at certain energy level. For an extended system with nearly-continuous
energy distribution, the DOS at energy $E$ is expressed as:
\begin{equation}
  \label{eq:ch-intro-dos}
  \mathrm{DOS}(E) = \frac{\partial N}{\Omega \partial E} =  \frac{1}{\Omega} {\displaystyle \int_{\Omega}} \frac{\mathrm{d}^{n} \mathbf{k}}{(2 \pi)^{n}}
  \delta(E - E(\mathbf{k}))
\end{equation}
where the integral is performed over a system with $d$-dimensionality,
$N$ is the total number of states,
$\Omega$ is the volume of the system, $\mathbf{k}$ is the momentum of
electron, and $\delta$ is the Dirac delta function. Without losing
generality, the DOS can be expressed using the law of chains:
\begin{equation}
  \label{eq:dos-chain}
  \mathrm{DOS}(E) = \frac{\partial N}{\Omega \partial k} \frac{\partial k}{\partial E}
               = \frac{k}{2 \pi} \left(\frac{\partial E(k)}{\partial k}\right)^{-1}
\end{equation}
where $k=|\mathbf{k}|$ is the modulus of $\mathbf{k}$.
%
\autoref{eq:dos-chain} provides a general way linking the DOS to the
energy--momentum ($E-k$) dispersion of a 2D material, which is further
extracted from its band structure. For monolayer 2D materials, two
cases can be distinguished:
\begin{itemize}
\item Parabolic materials: such as TMDCs, hBN, phosphorene

  These are the majority of 2D materials where the $E-k$ dispersion is
  parabolic, such that
  ${\displaystyle E(k) = \frac{\hbar^{2} k^{2}}{2 m^{*}}}$, where
  $\hbar$ is the reduced Planck constant, and $m^{*}$ is the effective
  mass near the band edge. From \autoref{eq:dos-chain}, DOS of a
  parabolic material is \textit{constant}, and proportional
  to $m^{*}$.
  
\item Dirac materials: such as graphene, silicene, germanene

  The Dirac materials are relatively rare among 2D materials,
  \worktodo{cite Wang nsr 2015} where the $E-k$ dispersion is linear:
  $E(k) = \hbar v_{\mathrm{F}}k$, where $v_{\mathrm{F}}$ is the Fermi
  velocity. The Dirac materials has DOS increasing linearly with $k$
  (as well a $E$).
\end{itemize}

The difference between the DOS of ideally parabolic and Dirac
materials can be seen in \worktodo{Figure here}. Unlike parabolic
materials with constant DOS, the Dirac materials have DOS → 0 near the
Dirac cone ($E \to 0$). This feature brings very interesting
properties like electrostatic transparency which we will discuss in
\worktodo{Chapter 2}.

\worktodo{Discuss more about the parabolic and Dirac cone?!}
\worktodo{Some simple discussion based on the band gap etc...}


\subsection{Fabrication of 2D Materials}
\label{sec:fabr-2d-mater}

The development of the 2D material researches cannot be achieved
without appropriate methods to fabricate large-area and high quality
2D materials. The synthesis of 2D materials can be generally
categorized into top-down and bottom-up approaches.

\subsubsection{Top-down Methods}
\label{sec:top-down-methods}

The top-down method is the straightforward approach to exfoliate few-
to single-layer 2D materials from their bulk counterparts. The two
most-used techniques are micro-mechanical exfoliation and liquid-phase
exfoliation. \worktodo{add cites here, from Liu Adv Mater 2018}

The micro-mechanical exfoliation uses mechanical force to overcome the
interlayer vdW interactions in bulk layered materials, and is the
first method used for 2D material exfoliation. Scotch tape is widely
used for such type of exfoliation \worktodo{cite Geim et al}, while
other media as elastic polymers, heat-release tapes are also
employed. If the quality of the bulk layered material is promising
(\ie high chemical purity and low defect density), micro-mechanical
exfoliation usually produces 2D materials with better quality compared
with other methods. However, there are also several key drawbacks of
such methods. First of all, the reliability of micro-mechanical
exfoliation is highly influenced by the interlayer forces, which makes
it difficult to exfoliate materials with high interlayer binding
energy, or in-plane mechanical anisotropy (for instance BP). This
method also leads to broad distribution of flake size and layer
number, making it difficult to be applied to large-scale
applications.

In contrast to micro-mechanical exfoliation where only the top-most
layers are removed, the solution-phase exfoliation method ensures
uniform breaking up of bulk materials, and increases the scalability
of 2D flakes produced. It can be achieved by both physical and
chemical exfoliation approaches. Physical solution-phase exfoliation
makes use of local mechanical stress produced by ultra-sonication or
shearing to overcoming the interlayer vdW interactions. To ensure the
stability of isolated layers in the liquid suspension, high surface
energy liquid, in combination with surfactants are usually
used. \worktodo{cite papers from Coleman et al, Shih et al. Haung et
  al}
%
Centrifugation can be further used to separate flakes to ensure narrow
distribution of layer thickness and size \worktodo{cite coleman}, and
statistic measurement can be performed using the ensemble of such 2D
material suspensions. \worktodo{wording?} However, this method still
suffers from several drawbacks: (i) the flake size is still limited to
μm-scale, (ii) precise control of layer number in suspension is difficult and 
(ii) flake overlay after deposition onto substrate is unavoidable.
%
The interlayer vdW forces can also be overcome by chemically modifying
the 2D layer (for instance, oxidising graphite to produce graphene
oxide (GOx)), or to intercalate ions between the layers in a bulk
material to induce lattice expansion (such as exfoliation of MXenes,
which are otherwise hard to achieve mechanically). Despite the
scalability of chemical solution-phase approaches, they usually
changes the chemical composition and introduce defects in the 2D
materials, which are undesired for high-performance
applications. 

\subsubsection{Bottom-up Methods}
\label{sec:bottom-up-methods}

The bottom-up methods grow 2D materials from precursors, and aims to
produce 2D materials with larger flake size and more controlled layer
numbers. Depending on whether a substrate is involved in the process,
these methods can be categorized into templated or non-templated
growth.

Templated growth uses a bulk surface as a epitaxial template and
support for the 2D material. Due to the absence of dangling bonds, the
2D-substrate interaction is usually much weaker than the in-plane
covalent bonds. Therefore, unlike epitaxy of bulk materials which
require precise control of lattice mismatch between the substrate and
epitaxy layer \worktodo{cite 1-2 paper}, templated growth of 2D
materials can be achieved in various substrates. For example, high
quality single crystal graphene can be epitaxially grown by thermal
annealing of silicon carbide (SiC) surface, or decompositing
hydrocarbons on Ruthenium (Ru) (0001) and Ir (111) substrates, while
epitaxial growth of hBN is achieved by cleavage of borazine
(H\textsubscript{6}B\textsubscript{3}N\textsubscript{3}) on Rh (111)
surface.  Such epitaxial method can also be applied to other 2D
materials like TMDC, monochalcogenides and BP. However, there are
still several limitations. First of all, a clean surface as well as
ultra-high vacuum (UHV) conditions are usually required for the
epitaxial growth methods. Moreover, transferring the 2D material onto
other substrates is generally not easy due to the noble metals
involved. Chemical vapor deposition (CVD) is another widely-used
templated growth method, in which one or more precursors adsorb and
react on a catalytic surface to form covalently-bonded 2D
materials. The essence of CVD process is similar to the epitaxial
growth, while more ambient conditions are used (10$^{1}$ Pa to
atmosphere pressure, ultra clean substrate not necessary). CVD growth
of graphene using hydrocarbon source on copper (Cu) is the most
studied and widely used technique. The self-termination of
second-layer on the Cu surface allows the growth of large area single
layer domains up to centimeter or decimeter scale \worktodo{check if
  explanation is correct}. Another advantage of such method is easy
removal of Cu substrate by standard etching procedure, allowing
transferring graphene onto a large variety of
substrates. \worktodo{cite 1-2} Similar to the case of graphene, large
area single-layer TMDCs and hBN can also be achieved using the CVD technique by proper interfacial engineering. \worktodo{cite TMDC paper; hBN show the Gold sample}
%
% With proper defect control during growth and development of novel
% transferring techniques, the CVD method is promising to 

While the majority of bottom-up growth relies on a substrate to form
the 2D material, there are also cases where colloidal 2D confined
structures can be directed synthesized in liquid phase without
template. Examples of 2D materials grown using the colloidal method
include II-VI semiconductors (CdSe), 2D hybrid perovskites, and
TMDCs. The anisotropic growth is usually modulated by surfactant /
ligand engineering. Recently, colloidal insulating 2D metal oxides are
reported to be synthesized by simultaneous oxidation at the liquid
metal-water interface, further extending the possibility of bottom-up
synthesis of 2D materials. \worktodo{cite Dicky paper}

As a summary, both top-down and bottom-up methods are capable of
fabricating 2D materials with desired purity, flake size and
thickness. A short comparison between different fabrication methods can be seen in \worktodo{table 2?!}



\section{The 2D Materials Interfaces}
\label{sec:2d-mater-interf}
%20191008-0939
the 2D materials do not solely attract the research interests due to
the unique electronic properties, they are also materials with ultra
high surface-area-to-volume ratio intrinsically. As a consequence,
interfaces are almost always required when integrating the 2D
materials into experimental studies and applications in the 3D world,
and the interfacial properties play important roles in determining
their proprieties. For instance, the existence of strict 2D lattices
is long questioned due to the presumed distortion caused by thermal
fluctuation which would break up long-range order. Although recent
studies suggest free-standing 2D materials like graphene can be
stabilized by the phonon coupling that causes 3D ripples in the 2D
layer (\worktodo{find 2 papers to cite}), the majority of studies
still require the 2D materials to be supported by substrate or
encapsulated.  As will be shown later, these 2D interfaces may
significantly alter the intrinsic properties by ways such as
structural corrugation and carrier doping. On the other hand, it
remains unrealistic to find a single 2D material which can satisfy all
the requirements concerning high-performance applications (e.g.,
electronic properties, mechanical strength, chemical stability, and
synthetic difficulty), the flexibility of creating mixed-dimensional
interfaces with existing functional materials may offer opportunities
to fully exploit their potential. Playing with the interfacial
interactions is critical for successful engineering of the interface
dimensionality, morphology, electronic states and transport phenomena.
In this section, we focus the discussion on the interactions involved
at the 2D materials interfaces, and how the interplay between these
interactions influences the mixed-dimensional interfaces with 2D
materials \worktodo{say more?}

\subsection{Interactions and Forces at the 2D Materials Interfaces}
\label{sec:inter-forc-at}

The concepts of interactions on 2D materials interfaces, can be
learned from the field of molecular epitaxy and self-assembly on bulk
interfaces
~\cite{Kowarik_2008_rev_MBE,Barth_2007,Whitesides_2002_assem_rev,Philips_2D_assem_book}.
% 
A molecule in the bulk
form and on a densely-covered surface feels the interactions from the
other epitaxial molecules, known as the intermolecular
interactions. On the other hand, a molecule undergoes various
processes on a 2D surface, including adsorption, diffusion, rotation,
and vibration, which is governed by the molecule-2D material
interactions. Moreover, the effect of the underlying substrate is
usually important where the molecule-substrate interactions come into
play. 

\subsubsection{Intermolecular Interactions}
\label{sec:intro-inter-mole}

The intermolecular interactions govern the packing and orientation
behavior of the molecules several atoms away from the 2D material
surface: the strength and the direction of intermolecular interactions
determine the packing density as well as the orientation of the
molecular epitaxy. The intermolecular interactions can be categorized 
into van der Waals (vdW) interactions, hydrogen bonds (H-bonds), and
covalent bonds depending on their strength. 

\paragraph{van der Waals (vdW) Interaction}

The van der Waals (vdW) interactions are dispersion forces between
charge-neutral molecules, including many organic semiconductors, such
as fullerene (C\(_{\text{60}}\))
~\cite{Corso_2004_C60_hBN,Kim_2015_c60_gr,Chen_2016_c60_mos2},
metal-phthalo\-cyanines (MPcs, where M can be Cu, Fe, Zn, Co, etc.)
~\cite{Xiao_2013_jacs_CuPc_gr,Wang_2010_selec_F16_gr,Zhang_2011_FePc_gr,Hamalainen_2012_CoPc_gr_Ir,Ying_Mao_2011_ge_clAlPc,Ogawa_2013_AlCiPc_gr,Pak_2015_CuPc_MoS2,Avvisati_2017_FePc_intercal,Iannuzzi_2014_MPc_hBN_Rh},
pentacene (PEN)
~\cite{Lee_2011_pentacene,Jariwala_2016_Mos2_pentacene,Shen_2017_DFT_mos2_pent,Kim_2016_trap_Mos2_pent,Nguyen_2015_pent_gr_wett,Betti_2007_orien_pentacene},
perfluoropentacene (PFP)
~\cite{Salzmann_2012_fpen_gr,Breuer_2016_acnene_mos2}, rubrene
~\cite{Lee_2014_rubene_hBN}, perylene-3,4,9,10-tetra\-carboxylic
dianhydride (PTCDA)
~\cite{Wang_2009_STM_PTCDA_Gr,Tian_2010_PTCDA_gr,Huang_2009_PTCDA_gr,Meissner_2012_PTCDA_BLG},
7,7,8,8,-Tetra\-cyanoquino\-dimethane (TCNQ) and its fluorinated
derivative 2,3,5,6-Tetra\-fluoro-7,7,8,8-tetra\-cyanoquino\-dimethane
(F\(_{\text{4}}\)-TCNQ)
~\cite{Chen_2007_TCNQ_gr,Hong_2013_ftcnq_gr,Stradi_2014_TCNQ_gr_Ru,Tsai_2015_TCNQ_gr_hbn}.
(\worktodo{Figure here?})

Due to its non-directional and weak force nature, if the vdW
interactions govern the interfacial molecules (weak interacting 2D
interfaces), the molecules tends to form close-packed structures in 2D
or 3D assemblies. The dimensionality of molecular epitaxy by vdW
interactions is usually dependent on the surface coverage, as the
molecule growth mechanism is similar to that of adsorption
isotherm. Although the vdW interactions usually have an energy less than
4 kJ\(\cdot\)mol\(^{-1}\), the collective interactions between molecules
with large electron cloud can be stronger. For the \(\pi\)-conjugated
aromatic molecules listed above, an effect known as the \(\pi\)-\(\pi\)
interaction, a combined effect of vdW interactions and charge
transfer ~\cite{Hunter_1990_pi}, can lead to preferential stacking and
orientation of the molecules, due to maximal overlapping of
\(\pi\)-electron clouds. 


\paragraph{Hydrogen Bond (H-Bond)}

The hydrogen bond (H-bond) refers to the directional electrostatic
forces between an H atom covalently-bonded to an atom of high
electro\-negativity (such as O, N and F) and another highly
electro\-negative atom in adjacent molecules. Compare the vdW
interactions, hydrogen bonds usually have higher bond energy and
preferred direction, which favors certain assembly structure on 2D
materials. The H-bonds are usually dominating between molecules rich
of N, O and F elements, such as modified PTCDA compounds
~\cite{Mura_2010_DFT_H_bond_PTCDA_gr,Karmel_2014_assembl_hetero_gr},
perylene tetra\-carboxylic diimide (PTCDI) derivatives
~\cite{Pollard_2010_hbond_assembly_gr,Karmel_2014_PTCDI_gr},
carboxylic-substituted aromatic compounds
~\cite{Rochefort_2009_aro_graphene_mech,Addou_2013_TPA_gr}, polycyclic
aromatic compounds
~\cite{Kozlov_2012_polyaro_gr,Roos_2011_BTP_gr,Meier_2010_polycyclic_gr}
and inorganic acids ~\cite{Prado_2011_2D_acid_gr}. The existence of
H-bonds stabilizes the assembled low-dimensional structures on 2D
materials interfaces, such as linear
~\cite{Pollard_2010_hbond_assembly_gr} or two-dimensional
~\cite{Prado_2011_2D_acid_gr} supra\-molecular assemblies. The specific
adsorption sites on 2D materials (such as the moiré patterns) also
play an important role in the assembly of H-bond-governed molecular
epitaxy.


\paragraph{Covalent Bond}
In general, the interactions between the epitaxial molecules and 2D
material (vdW and Coulombic interactions) are much weaker than the
intra\-molecular covalent bond (including metal coordination forces),
resulting in a variety of structures on 2D materials interfaces
~\cite{Bakti_Utama_2013_rev_epitax}. One example is the van der Waals
epitaxy (vdWE) technique which allows 2D or 3D crystalline growth on
2D materials. As discussed before \worktodo{which section?}, the
absence of dangling bonds eliminates the lattice mismatch between
dissimilar materials, leading to a number of 2D vertical
heterostructures including: TMDC/graphene
~\cite{Shi_2012_vdw_epi_MoS2_gr,Liu_2016_epi_MoS2_gr_rotation,Lin_2014_vdW_solid,Lin_2015_Wse2_MoS2_gr,Azizi_2015_Freevdw_Gr_TMDCs,Kim_2016_BiSnTe_gr},
TMDC/hBN
~\cite{Yan_2015_MoS2_on_hBN,Wang_2015_cvd_MoS2_BN,Cattelan_2015_Ws2_hBN},
graphene/hBN
~\cite{Liu_2011_gr_hBN,Zhang_2015_gr_hBN,Driver_2016_MBE_gr_hBN}, and
TMDC/TMDC
~\cite{Zhang_2014_vdw_epi_SnS2_MoS2,Diaz_2015_MoTe2_MoSe2,Gong_2014_WS2_MoS2,Alemayehu_2015_TMDC_vdw}.
%
The vdWE has also been used to grow 3D heterostructures on mono- or
multilayer 2D materials interfaces, including inorganic insulators like Al\(_{\text{2}}\)O\(_{\text{3}}\)
~\cite{Zhang_2014_Al2O3_ALO_Gr,Vaziri_2013_ALD_Al2O3_gr}, and
HfO\(_{\text{2}}\) ~\cite{Alaboson_2011_PTCDA_gr_ALD},
%
and semiconductors including TiO\(_{\text{2}}\)
~\cite{Li_2015_TiO2_GO,Kumar_2011_gr_TiO2_generator,Zhang_2011_TiO2_gr},
ZnO ~\cite{Chung_2010_GaN_ZnO_gr,Oh_2014_ZnO_hBN}, GaN
~\cite{Kobayashi_2012_GaN_hBN,Kim_2014_direct_vdw_GaN_gr,Kim_2017_remote_epi_Gr},
GaAs ~\cite{Alaskar_2015_GaAs_gr_Si_theor,Kim_2017_remote_epi_Gr}
\worktodo{add Kang2018}, and CdS / CdTe
~\cite{Loeher_1994_vdw_epi_CdS_MoTe,Loeher_1996_CdTe_MoWTe}.

Apart from the vdWE approach, covalently bonded structures can also be
formed by on-surface chemical reactions and metal coordination
bonds. Examples of such growth approach include two-dimensional covalent organic frameworks (2D COFs) formed by linking monomers by boron ester
or imine groups
~\cite{Colson_2014_2D_COF_gr,Colson_2011_2DMOF_gr,Sun_2017_cof_gr}, and metal-organic frameworks (MOFs) ~\cite{Urgel_2015_MOF_BN,Kumar_2014_2D_MOF_gr} on weakly interacting or
functionalized 2D materials. The planar sp$^{2}$-type bonds
such as boron ester, imine and square planar metal coordination are
generally required for the formation of stable 2D epitaxial structure.

\subsubsection{Molecule-2D Material Interactions}
\label{sec:intro-mol-2D}

The interactions between the interfacial molecules and 2D material
determine the molecular packing and arrangement of the first few
overlayers. In addition, the interactions also have great impact on
the molecular adsorption process, thereby influencing the
heterogeneous nucleation characteristics. The
ratio between the intermolecular and molecule-2D material
interactions is the key factor in controlling the molecular epitaxial
structure. Here we categorize the molecule-2D material interactions
into weak (dispersion and electrostatic), charge-transfer
interactions, site-specific adsorption, and covalent bond formation.

\paragraph{Weak Interactions}
\label{sec:org68af064}

The weak molecule-2D material interactions involve the short-range
dispersion (vdW) and long-range electrostatic (Coulombic)
interactions. In the case of graphene, the delocalized π-electrons are
the basis for the non-covalent interactions. A large variety of planar
aromatic molecules, including PTCDA, PTCDI, C\(_{\text{60}}\), MPc are
shown to assemble on graphene with their aromatic rings parallel to
the 2D plane, in order to lower the adsorption energy by maximizing
the π-π interaction ~\cite{Grimme_2008_pipi,Zhang_2011_rev_pipi_gr}, a
phenomenon widely known as the graphene template effect
~\cite{Yang_2015_rev_template}. MPc molecules (e.g. M=Cu, Fe, Co and
AlCl) and substituted MPc (e.g. F\(_{\text{16}}\)CuPc) tend
to form a ``face-on'' orientation on graphene interface, relative to
the ``edge-on'' orientation that are usually found on the deposition
of these molecules on amorphous substrates such as SiO\(_{\text{2}}\)
or glass
~\cite{Ying_Mao_2011_ge_clAlPc,Zhang_2011_FePc_gr,Hamalainen_2012_CoPc_gr_Ir,,Xiao_2013_jacs_CuPc_gr}.
Similarly, the graphene template effect is also found  for PEN
~\cite{Zhou_2013_penta_gr_Ru,Lee_2011_pentacene,Lee_2011_pentacene,Zhang_2015_gr_pent_orient},
C\(_{\text{60}}\) ~\cite{Kim_2015_c60_gr,Shih_2015_PartiallyScreened},
p-sexiphenyl (6P) ~\cite{Hlawacek_2011_6P_gr}, and
dibenzotetrathienocoronene (DBTTC) ~\cite{Kim_2016_DBTTC_gr} molecules,
revealing a general mechanism behind their assembly behavior.

Apart from graphene, the weak interactions on hBN and
MoS\(_{\text{2}}\) surfaces are also studied. The π-electron cloud of
hBN resembles that of graphene, causing the 6P molecules to form a
``face on'' configuration ~\cite{Matkovic_2016_6P_hBN} similar to the
case on graphene. However, non-planar molecules such as rubrene
~\cite{Lee_2014_rubene_hBN} adapt the ``edge-on'' configuration over the ``face-on'' configuration, reflecting the fact
that the molecule-hBN interaction is weakly dispersive.
%
On the other hand, the molecular
interactions on MoS$_{2}$ are usually much weaker compared with that on
graphene due to its large dipole moment ~\cite{Rajan_2016_wett_mos2},
and is highly dependent on the lattice symmetry
~\cite{Shen_2017_DFT_mos2_pent} (i.e. 1T- or 2H- phase) and surface
defects ~\cite{Jariwala_2016_Mos2_pentacene,
  Kim_2016_trap_Mos2_pent}.

\paragraph{Charge-Transfer Interaction}
\label{sec:orgebfad7b}

The charge-transfer (CT) interactions, or the donor-acceptor (DA)
interactions, refer to the process that electrons undergo
redistribution between the epitaxial molecules and the underlying 2D
material. Due to the locally enhanced carrier density in the formed CT
complex, the CT interactions tend to be stronger than the dispersion
and electrostatic interactions. The formation of a CT heterostructure
requires alignment of the energy levels between the 2D material and
the overlayer molecules ~\cite{Akiyoshi_2015_DA}, and may also change
the electronic structure of the 2D material through non-covalent
interactions
~\cite{Cai_2015_doping_2D_rev,Wehling_2008_doping,Zhang_2011_rev_pipi_gr}.
TCNQ and its fluorinated derivative
2,3,5,6-Tetra\-fluoro-7,7,8,8-tetra\-cyanoquino\-dimethane (FTCNQ) are
known to form CT complexes with graphene
~\cite{Chen_2007_TCNQ_gr,Voggu_2008_TCNQ,Barja_2010_TCNQ_gr}, with a
degree of charge transfer of $\sim{}$0.3 \textit{e} and $\sim{}$0.4
\textit{e}, respectively. With a stronger CT effect, FTCNQ molecules
on epitaxial graphene tend to be trapped by local
corrugation~\cite{Barja_2010_TCNQ_gr}, compared with closed-packed
TCNQ/graphene assembly.  Since CT may occur when the HOMO and LUMO
energy levels of the epitaxial molecule and 2D material match, it is
also expected to play a role in the molecular epitaxy on 2D
semiconductors, such as TMDCs. Density functional theory (DFT) studies
reveal that PEN adsorbed on 1T-type monolayer MoS\(_{\text{2}}\) has a
large degree of CT ranging from 0.44-0.87 \emph{e}, and can change the
Fermi energy level of MoS\(_{\text{2}}\) by up to 1 eV
~\cite{Shen_2017_DFT_mos2_pent}. Similarly, the interface between
C\(_{\text{60}}\) and MoS\(_{\text{2}}\) is found to be a pn-junction,
with charge depleted at the bottom of the C\(_{\text{60}}\) and
accumulated at the interface ~\cite{Chen_2016_c60_mos2}. On the other
hand, the tendency of forming CT-induced orientation is attenuated on
bulk MoS\(_{\text{2}}\) crystal ~\cite{Sakurai_1991_c60_mos2}, due to
an increase of the DOS compared in bulk crystals. Theoretical studies
also disclose strong CT between phosphorene and electron-donating
tetrathiafulvalene (TTF), as well as electron-accepting TCNQ molecules
~\cite{Zhang_2015_DA_phosphorene}.


\paragraph{Site-Specific Adsorption}
\label{sec:org87b0c12}

The electronic and geometric properties of a 2D material are known to be
influenced by its underlying substrate. When there is a lattice
mismatch between the 2D material and the substrate, a long-range
periodic superposition known as moiré pattern forms, as has been
found graphene/metal ~\cite{Hamalainen_2013_moire_gr} and hBN/metal
~\cite{Schulz_2014_hBN_moire} systems.  The
moiré pattern does not only cause a geometric interference, but
indeed changes the local electronic state and structure of the 2D
material.
%
The height variation within the graphene or hBN layer can be used to
quantify the degree of metal-2D material interaction strength, to
distinguish weakly interacting surfaces include graphene/Ir(111)
~\cite{Pletikosi_2009_gr_Ir,Busse_2011_Gr_Ir,Hamalainen_2013_moire_gr},
graphene/Pt(111) ~\cite{Sutter_2009_Gr_Pt}, hBN/Ir(111)
~\cite{Schulz_2014_hBN_moire}, hBN/Pt(111) ~\cite{Cavar_2008_hBN_Pt},
hBN/Cu(111) ~\cite{Joshi_2012_hBN_Cu} systems, in which the average 2D
material-metal distance is comparable with that in the bulk material
(3.3$\sim{}$3.4 \AA{}) and the corrugation in the 2D layer is
typically small (<0.5 \AA{}). The strongly interacting surfaces
including graphene/\allowbreak{}Ru(0001) ~\cite{Moritz_2010_gr_Ru} \worktodo{sutter
  2}, graphene/Rh(111) ~\cite{Wang_2010_gr_Rh}, hBN/Ru(0001)
~\cite{Wang_2010_gr_Rh}, and hBN/Rh(111) ~\cite{Dil_2008_hBN_Rh}
systems, with structural corrugations as large as 1 \AA{}, and the
electronic fluctuation up to 0.5 eV. In the strongly interacting
systems, the moiré pattern creates a local difference in the
adsorption potential, which in turn results in site-specific
adsorption of small molecules on these surfaces. Such behavior has
been observed in a variety of organic semiconductor molecules
deposited on the graphene/\allowbreak{}Ru(0001) surface, including MPc (M=Fe, Ni,
Zn, Mn) ~\cite{Mao_2009_Pc_gr_kagome,Zhang_2011_FePc_gr}, pentacene
~\cite{Zhou_2013_penta_gr_Ru}, C\(_{\text{60}}\)
~\cite{Li_2012_c60_gr_Ru}, PTCDA ~\cite{Zhou_2011_PTCDA_gr_Ru}, TCNQ
~\cite{Maccariello_2014_TCNQ_gr_Ru}, with similar behavior has also
been found on the surface of hBN/Ru(0001) for MPc (M=H\(_{\text{2}}\),
Cu, Co) ~\cite{Dil_2008_hBN_Rh,Jarvinen_2014_MPc_hBN_Ru}, TCNQ
~\cite{Joshi_2014_TCNQ_hBN}, and C\(_{\text{60}}\)
~\cite{Corso_2004_C60_hBN}. The site-specific adsorption usually lead
to ordered sub-2D assembly, composed of the molecules trapped at the
specific sites, compared with the close-packed assembly on flat and
weakly interacting interfaces.

% Recently, more experimental and theoretical studies have also
% demonstrated the moiré pattern formation on TMDC/metal
% ~\cite{Chen_2013_doping,Sorensen_2014,Le_2012_MoS2_Cu}, TMDC/TMDC
% ~\cite{Kang_2013_TMDC_moire,Zhang_2014_vdw_epi_SnS2_MoS2,Diaz_2015_MoTe2_MoSe2,Fang_2014_intercoupl_vdW,Li_2016_GaSe_MoSe2_vdW},
% and TMDC/hBN ~\cite{Fang_2014_intercoupl_vdW} surfaces. Following the
% discussion of the strongly interacting surface of graphene/\allowbreak{}Ru(0001),
% it is believed that the moiré pattern formed between the strongly
% coupled layers, e.g. TMDC/Ru(0001) ~\cite{Chen_2013_doping} and TMDC/TMDC
% ~\cite{Fang_2014_intercoupl_vdW} heterostructures may also lead to the
% site-specific adsorption phenomenon ~\cite{Diaz_2015_MoTe2_MoSe2}, in
% contrast to the close-packing structure formed on the weakly
% interacting surfaces, as discussed in the previous section.


\paragraph{Covalent Bond}
\label{sec:org6f342a5}

Covalent bonds formed perpendicular to the 2D material plane open an
opportunity for functionalizing 2D materials and provide anchor sites
for modification. However compared with the epitaxy approaches,
chemical modification of 2D material is limited by the choice of
chemical reactions available. Moreover, opening up 2D basal structure
usually destroyed by the geometric change of the molecular orbital
(e.g. planar sp\(^{\text{2}}\) to tetrahedral sp\(^{\text{3}}\) in
graphene). Nevertheless, there are still a few examples showing the
potential of covalent binding and tuning the electronic properties of
the 2D materials
~\cite{Georgakilas_2012_noncoval_gr_rev,Lee_2011_tempo_gr,Zhang_2013_janus_gr,Voiry_2014_cov_TMDC_phase,Vishnoi_2016_ar_mos2_covalent,Liu_2011_rev_chem_dope_gr,Wang_2012_ar_gr_react_rate}.
%
The chemical grafting of graphene mainly involves free-radical
reaction
~\cite{Lee_2011_tempo_gr,Choi_2010_aminotempo_gr,Zhang_2013_janus_gr,Wang_2012_ar_gr_react_rate,Kumar_2014_2D_MOF_gr},
%
with the potential to fabricate asymmetric Janus-type functionalized
graphene by the covalent modification on both sides of a free-standing
graphene sheet ~\cite{Zhang_2013_janus_gr}. The low DOS in a 2D
materials further makes it possible to fine-tune the interfacial
chemical reaction rate by the doping density of 2D materials, for
instance through the substrate doping of graphene
~\cite{Wang_2012_ar_gr_react_rate}. Several approaches have also show
the possibility of functionalizing other 2D materials, including
nucleophilic substitution between anionized TMDCs and organohalides
~\cite{Vishnoi_2016_ar_mos2_covalent} and aryl diazonium
salts. \worktodo{cite these} The chemical modifications are also
frequently used to improve quality of 2D semiconductors.
%
Diazonium modification of BP significantly increases its ambient
stability over several weeks. \worktodo{cite nat chem 2016 8
  597}. Moreover, treating MoS\textsubscript{2} with organic
super\-acids moves its Fermi level towards intrinsic semiconductor,
and significantly improves the photo\-luminescence (PL) quantum yield over two order of magnitudes. \worktodo{Science 2015 350 1065}
%
Future advance of covalently
modified 2D materials with site-specific and programmable chemical
functionalization may combine the 2D with the 3D materials in a
controllable manner.

\subsubsection{Molecule-Substrate Interaction}
\label{sec:intro-mol-subst}

One of the major differences of the interfacial molecules on 2D
materials compared with bulk materials interfaces is significant
influence from the underlying substrate. Note that this phenomenon is
distinguished from the effect of strongly interacting surface or
substrate doping, with the latter two referring to the change of 2D
material's electronic and geometric properties, which then influence
the molecule-2D material interactions. The penetration of the
molecule-substrate interactions through monolayer 2D material is first
observed in the experiments of wettability of substrate-supported
graphene: the water contact angle of water on graphene is found to be
influenced by the vdW force between the water molecules and the
substrate, known as the wetting ``transparency'' or ``translucency''
of graphene
~\cite{rafiee_wetting_2012,shih_breakdown_2012,shih_wetting_2013}. The
transparency can be even pronounced for electrostatic interactions,
which has longer length scale than the vdW force
~\cite{Shih_2015_PartiallyScreened,Tian_2016_multiscale}.  The
influence of the molecule-substrate interactions through a 2D
materials usually can only be examined indirectly.
%
Examples include layer-number-dependent morphology of
6P~\cite{Kratzer_2014_6P_gr_layer} and
PEN~\cite{Chhikara_2014_gr_pent_trans} molecules deposited on
SiO\(_{\text{2}}\)-supported graphene layers. Moreover, the influence
of underlying substrate is also found for PEN when deposited on
graphene supported by substrates with varied surface energy ~\cite{Nguyen_2015_pent_gr_wett} or
electrostatic gating. \worktodo{cite Nguyen paper 2}
%
Recently the concept of vdW transparency has also been
employed in the remote vdWE of III-V semiconductors on graphene supported by highly-crystalline III-V substrate
~\cite{Kim_2017_remote_epi_Gr}. \worktodo{cite Kang 2018 paper 2}
%
The interactions from underlying crystalline III-V semiconductor is
shown to direct the growth of III-V semiconductor on the graphene
interface despite the $\sim{}$ 1 nm gap created by graphene. The
strength of such remote interactions are also shown to be dependent on
the polarity of the underlying material. \worktodo{cite Kang 2018 paper 2}
%
In addition to the vdW and Coulombic interactions,
graphene layer is also found to be transparent to the charge transfer
process ~\cite{Jeong_2015_DA_transparency_gr} when the reduction rate of
AuCl\(_{\text{4}}^{\text{-}}\) on graphene surface are found to be
faster when graphene is coated on a reductive surface, such as Al, Ge
and Cu surfaces. 

% 20191008-1253

\subsubsection{Summary}
\label{sec:org697d552}

To obtain a clear view of the interactions involved in the molecular
epitaxy on 2D materials interfaces, the major forms
of interactions and their energy range are summarized in
\autoref{tbl:intro-interactions}. As can be seen, both strong interactions ($>$ 100
kJ\(\cdot\)mol\(^{\text{-1}}\) for covalent bonds, metal-coordination and some
hydrogen bonds), and weak interactions ($<$ 50 kJ\(\cdot\)mol\(^{\text{-1}}\) vdW
and \(\pi\)-\(\pi\) interactions) exist between the epitaxial molecules and at
the molecule-2D material interface. On the other hand, the
molecule-substrate interactions mainly have a weak nature, and
relatively weaker than the intermolecular and molecule-2D weak
interactions due to the increasing of molecule-substrate distance.

\begin{table}[htbp]
\caption{\label{tbl:intro-interactions}
Types of interfacial interactions involved in the molecular epitaxy on 2D materials interfaces, showing the typical forms of interaction and energy range.}
\small
\centering
  \begin{tabular}[\textwidth]{lll}
\hline
Type of Interaction & Typical Forms & Energy Range  (kJ\(\cdot\)mol\(^{\text{-1}}\))\\
\hline
Intermolecular & van der Waals & \(\le\) 5\\
 & \(\pi\) - \(\pi\) & \(\le\) 50\\
 & H-bonds & 4 - 120 ~\cite{jeffrey_introduction_1997}\\
 & Covalent Bonds & 100 - 400\\
\hline
Molecule-2D & Weak Interactions & 10 - 60 ~\cite{Lazar_2013}\\
 & Charge-Transfer & 50 - 200\\
 & Site-Specific Adsorption & 30 - 100\\
 & Covalent Bonds & 100 - 400\\
\hline
Molecule-Substrate & Weak Interactions & \(\le\) 20\\
\hline
\end{tabular}
\end{table}

\subsection{The Variety of Mixed-Dimensional Interfaces}
\label{sec:vari-mixed-dimens}

%20191008-1405
As shown in the previous section, a variety of interactions exist on 2D
materials, which
essentially dominate the ordering and packing of the molecules in vicinity.
%
The interfacial engineering of the interactions leads to different
dimensionalities ranging from 0D to 3D.
%
In this section, several model systems of mixed-dimensional interfaces
with 2D materials are discussed, with the focus on how the
interactions determine the morphological dimension. The notations for
the interfaces are like ``0D-2D'', with the former indicating the
dimensionality of the heterostructure on 2D materials. In general, two
approaches are usually employed to fabricate such mixed-dimensional
interfaces, namely self-assembly of small molecules, and deposition of
pre-formed nano\-materials. \worktodo{say something more?}

\subsubsection{0D-2D Interface}
\label{sec:intro-0D-2D}

\paragraph{Self-Assembly}
\label{sec:org8117691}

Molecular dynamics (MD) simulations have shown that the weak
intermolecular and molecule-2D interactions alone, do not result in
the formation of sub-monolayer assembly, such as the case of pentacene
and PTCDA on graphene or hBN ~\cite{Zhao_2015_self_assemb_gr_MD}, and
organic semiconductor molecules dominated by vdW force on phosphorene
~\cite{Mukhopadhyay_2017_cryst_BP}.
% 
To form 0D assemblies on 2D materials, specific adsorption sites are
required to exist on the 2D surface.
% 
The moiré pattern formed in the graphene/metal and hBN/metal
interfaces as introduced in \autoref{sec:intro-mol-2D} are shown to
trap metal clusters
~\cite{Goriachko_2007_assembl_hBN_ru,Pan_2009_Pt_cluster_gr,Wang_2011_gr_hBN_metal_cl,Zhang_2014_metal_gr_Ru}
and individual organic semiconductor molecules
~\cite{Joshi_2014_TCNQ_hBN,Dil_2008_hBN_Rh,Lu_2012_c60_gr_moire}.
%
In the case of organic molecular deposition, the site-specific
adsorption energy difference is usually around 10\textsuperscript{2}
meV ~\cite{Lu_2012_c60_gr_moire}, enough to trap the small molecules
within the valley regions of the 2D moiré patterns. More interesting,
it is also found that the site-specific isolation of small molecules
is not limited to strongly interacting surfaces such as graphene and
hBN supported by Ru or Rh, but also weakly interacting surfaces like
hBN/Cu(111) with a small degree of corrugation but strong electronic
patterning ~\cite{Joshi_2012_hBN_Cu,Joshi_2014_TCNQ_hBN}, revealing the
more complex nature of the 0D self-assembly on 2D materials interfaces.
%
The isolated molecules on 2D materials can be used as nucleation sites
for further molecular epitaxy, and facilitate the research of
single-molecular surface reaction. \worktodo{Guess need to cite 1-2
  papers here}


\paragraph{Deposition of nano\-materials}


The 0D-2D interfaces fabricated by deposition usually refer to the 0D
quantum dot (QD)-2D material junction. The quantum dots are
quantum-confined nano\-materials with size of several
nano\-meters. \worktodo{get a cite for this part}
%
The vast majority of the QDs are prepared by colloidal synthesis, and
covered by buffering coating such as ligands. \worktodo{get a cite for this part}
%
As a result, QDs deposited on 2D materials using solution processing
can still sustain the isolate form, making their applications more
versatile than the self-assembled 0D structures.
%
One promising feature of QDs is the optical properties, including high
optical absorption coefficient, PL quantum yield, and size-dependent
modulation of optical bandgap.
%
Combining with the electrostatic tuning of 2DEG, the 0D-2D
heterostructures can be used for multiple light sensitive components
such as photo\-transistors, photo\-diodes, photo\-voltaics (PV) and
light emitting devices (LED). \worktodo{cite}
%
These architectures share a similar feature, that the QDs act as the
main photo-active component, while the 2D materials is design to
modulate the carrier transport / injection at the QD-2D interface.
\worktodo{I guess 2 papers each part may be needed}
%
For instance, in a 0D-2D photo\-transistors based on lead sulfide
(PbS) QDs on graphene \worktodo{nature nanoteh 2012 7 363-368}, the
photo-generated carriers in the QDs are transferred onto the biased
graphene surface and consequently collected at the electrodes. The
benefits from such mixed-dimensional heterostructure are two-folds:
(1) the large optical cross-section of QDs enhanced photo\-detection
limit compared with bare 2D material and (2) the photo\-current is
efficiently modulated by electrostatic gating of graphene.
%
The examples of other 0D-2D photo\-active electronic components can be
found in several recent reviews. \worktodo{cite Jariwala, Dominik
  Kufer ACS Photo}


\subsubsection{1D-2D Interface}
\label{sec:orgeadf57e}

\paragraph{Self-assembly}

With increasing surface coverage on a 2D material interface or
introducing directional intermolecular interactions, self-assembled 1D
and fractal assemblies may be formed on 2D material interfaces, in the
form of nanowires, nanoporous or network structures.
%
As discussed earlier, the strongly interacting surfaces, including
graphene/\allowbreak{}Ru(0001), graphene/Rh(111) and hBN/Ru(0001) result in the
moiré pattern that serves as specific binding sites for trapping small
molecules at low surface coverage.  Starting from the 0D assembly
formed by moiré patterns on graphene/\allowbreak{}Ru(0001), graphene/Rh(111) and
hBN/Ru(0001) as discussed in \autoref{sec:intro-0D-2D}, when further
increasing the surface coverage, the specifically adsorbed molecules
act as nucleation sites for subsequent epitaxial growth.  The intermolecular interactions, in combination with
the geometry of moiré pattern, result in a specific arrangement of the
molecules on the surface, varying from nanowire, nanorope ~\cite{Maccariello_2014_TCNQ_gr_Ru} to Kagome
lattice ~\cite{Atwood_2002_kagome,Mao_2009_Pc_gr_kagome}.
%
The nucleation-induced growth is both substrate- and
molecule-specific. For instance, 1D and fractal molecular assemblies
are rare on hBN/metal
surfaces~\cite{Schulz_2013_copc_hbn_moire,Schulz_2014_hBN_moire,Iannuzzi_2014_MPc_hBN_Rh,Joshi_2014_TCNQ_hBN},
possibly due to the different surface potential distribution compared
with the graphene moiré surface. Furthermore, metal intercalation
between graphene and substrate is shown to affect the assembly pattern
of MPc
~\cite{Bazarnik_2013_tailor_Fe_Co_gr_Ir,Avvisati_2017_FePc_intercal}.
%
In addition to the nucleation-induced growth, a rich set of 1D-2D
heterostructure can be obtained by tailoring the intermolecular and
molecule-substrate interactions.
%
Intermolecular H-bond (such as in PTCDI derivatives) is widely used to
guide the orientation of surface-assisted self-assemblies ranging from
linear to Kagome structures ~\cite{Slater_2014_HBond_assembl_rev,
  Pollard_2010_hbond_assembly_gr, }, due to the relatively high
strength compared with molecule-2D interactions.
%
The assembly structure can be ultra sensitive to the conformation of
molecules, with minor change of functional groups between almost
identical molecules leading to distinct morphologies on graphene/\allowbreak{}Ru
(0001) surface
~\cite{Meier_2010_polycyclic_gr,Roos_2011_BTP_gr,Roos_2011_hiera_org_gr},
revealing the important role of directional interactions in the
formation of low-dimensional assemblies on 2D
interfaces. \worktodo{Add spaces between all Ru instances}
%
In addition to hydrogen bond, several other interactions including
hydrophobic interaction between long alkyl chains
~\cite{De_Feyter_2003_2D_assem_rev, Deshpande_2012_1D_assemb_gr},
Coulombic interactions ~\cite{Prado_2011_2D_acid_gr}, covalent bond
~\cite{Colson_2011_2DMOF_gr,Colson_2014_2D_COF_gr} and metal
coordination bond ~\cite{Urgel_2015_MOF_BN} are also shown to form
1D-2D heterostructures. In these systems, the intermolecular
interactions are generally directional, and much larger than the molecule-2D
interactions, which stabilizes the formed low-dimensional structure.

\paragraph{Deposition of nano\-materials}
% 20191008-1613
Like the 0D quantum dots, 1D quantum-confined nano\-material such as
semiconductor nanowires (NWs), carbon nano\-tubes (CNTs) are the
mostly widely used nano\-structure to build 1D-2D interface.
\worktodo{Cite this part Jairwala 2X + waveguide Si, Appl. Phys. Lett. 100, 223114 (2012), ACS Nano2010 10 5835-5842}
%
The interfaces are usually fabricated using solution processing
(coating of nanowire suspension) or micro\-mechanical manipulation
(for single nanowire devices)
\worktodo{several cite here?}
%
Similar to the idea of 0D-2D heterostructure, the quantum-confinement
of the 1D NW structures is employed to fabricate photo\-diodes and
photo\-transistors with engineering of spectra responsive from
ultraviolet (UV) to infrared (IR) \worktodo{Nie small 2013 9 2872, Gao
  nanoscale 2013 5 5576, miao small 2015 11 936, + jariwala + Spina
  perovskite} by tuning both the bandgap of the NWs or 2D materials.
%
Proper alignment of the 1D-2D heterostructure can also be used as
optical waveguides and cavities to direct in-plane light propagation,
in order to enhance optical responsiveness of pure 2D structures.
\worktodo{cite Nat Photon 2013 Pospischil, Gan and Shiue}

If the 1D structures are aligned vertically to the 2D plane (\ie NW
arrays), such heterostructure further owns high surface-area-to-volume
ratio, and are widely used in applications like electrodes in
batteries and super\-capacitors \worktodo{High-Energy MnO2
  Nanowire/Graphene and Graphene Asymmetric Electrochemical
  Capacitors, Ultralong single crystalline V 2 O 5 nanowire / graphene
  composite fabricated by a facile green approach and its lithium
  storage behavior}.



\subsubsection{2D-2D Interface}
\label{sec:intro-2D-2D}

\paragraph{Monolayer Self-Assembly of Small Molecules}
\label{sec:orgfd77377}

The assembly of small molecules on 2D materials with low geometric and
electronic corrugation have usually been found to form close-packed
structures. The role of intermolecular, molecule-2D material and
molecule-substrate interactions in such assembly process can be
understood by MD simulations. Using PEN and PTCDA as model molecules,
Zhao et al. found that inter\-molecular Coulomb and vdW interactions
are equivalently important to stabilize 2D assemblies, while the
molecule-2D interaction governs the orientation (\ie ``face-on'' vs
``edge-on'' configurations) of molecules on the 2D interface
~\cite{Zhao_2015_self_assemb_gr_MD}. Similar studies are also applied
for molecular assembly on phosphorene
~\cite{Mukhopadhyay_2017_cryst_BP}, which differs from graphene with
its armchair structure and electronic anisotropy. Despite the
non-aromatic nature of phosphorene, the adhesion energy of small
organic molecules such as TCNQ and PEN are comparable with that on
graphene or hBN, which explain similar packing experimentally observed
on phosphorene. Such arguments also correlates with the general trend
observed in experiments: monolayer closed-packed molecular assemblies
are more favored on weakly interacting interfaces such as graphene/Pt
(111) and graphene/Pt (111) compared with strongly interacting
interfaces such as graphene/\allowbreak{}Ru (0001) and graphene/BN, which is applicable for a wide
range of small organic molecules ~\cite{Hamalainen_2012_CoPc_gr_Ir,Xiao_2013_jacs_CuPc_gr,Barja_2010_TCNQ_gr, Jung_2014_C60_gr_Cu,Yang_2012_MPc_gr_metal,Barja_2010_TCNQ_gr,Hamalainen_2012_CoPc_gr_Ir,Tsai_2015_TCNQ_gr_hbn,Stradi_2014_TCNQ_gr_Ru}.
%
Despite the simplicity of such conclusions, in reality, the monolayer
assembly of small molecules is a more complex process which involved
kinetic parameters such as nucleation, local potential barrier and
interfacial diffusion.


\paragraph{2D van der Waals Heterostructures}
\label{sec:org77ea5bc}

The 2D vdW heterostructures (vdWHs) are the heterostructures formed by
sequentially stacking individual 2D layers in controlled manner. The
formation of 2D vdWHs can be seen as the reversal process of 2D
exfoliation, and can be fabricated by either direct mechanical
assembly or epitaxial growth.
%
The mechanical assembly is usually performed by wet transfer and
lift-off techniques,
\worktodo{cite novoselov}
although able to precise determine the stacking
sequence, its process is usually cumbersome and time-consuming, and
the relative orientation between layers is hard to
control. \worktodo{Cite paper from Novoselov} Recent advances like
nanoscale atomic force microscope (AFM) manipulation and folding of 2D
flakes may open avenues to scalable production of 2D vdWHs with
desired stack and orientation. \worktodo{cite 1-2 science papers.}
%
More scalable synthesis of vdWHs include solution-phase assembly of 2D
material flakes by controlling the interlayer charge, which can be
directly used for energy applications.
%
Another important approach to achieve large area vdWHs is through
epitaxy growth techniques like Chemical vapor deposition (CVD) and
van der Waals epitaxy (vdWE)~\cite{Novoselov_2016_vdW}. In the view point of
molecular interactions at the interface, these two methods are
essentially similar. The use of vdWE for both
small molecule ~\cite{Hara_1989_cupc_mos2_vdwe,Sakurai_1991_c60_mos2} and
layered materials
~\cite{Koma_1985_vdWE,Ueno_1990_vdWE,Ohuchi_1990_MoSe2_SnS2,Parkinson_1991_vdWE}
on layered TMDCs has been demonstrated long before the first discovery
of graphene. As discussed before, vdWE is almost not constraint by lattice mismatch compared with conventional heteroepitaxy,
where dangling bonds exist on the substrate surface.
%
The vdWE approach is capable of producing vdWHs pairs including
graphene/hBN ~\cite{Yang_2013_gr_hBN}, TMDC/hBN
~\cite{Yan_2015_MoS2_on_hBN,Wang_2015_cvd_MoS2_BN,
  Cattelan_2015_Ws2_hBN} hBN/graphene ~\cite{Lin_2014_vdW_solid},
TMDC/graphene
~\cite{Shi_2012_vdw_epi_MoS2_gr,McCreary_2014_MoS2_gr,Azizi_2015_Freevdw_Gr_TMDCs,Miwa_2015_MoS2_gr,Ago_2015_MoS2_Gr,Lin_2014_WS2_Gr,Lin_2015_Wse2_MoS2_gr},
, TMDC/TMDC ~\cite{Diaz_2015_MoTe2_MoSe2,Gong_2014_WS2_MoS2},
Monochacolgenide/TMDC
~\cite{Li_2016_GaSe_MoSe2_vdW,Zhang_2014_vdw_epi_SnS2_MoS2} and even
more complex layer combinations
~\cite{Lin_2015_Wse2_MoS2_gr,Alemayehu_2015_TMDC_vdw}.
% 
Despite the versatility of vdWE approaches, one challenge is the
control of layer number  during growth.
%
In many 2D heterostructures systems (mainly TMDC/graphene or TMDC/hBN)
grown by vdWE, the growth of mono- and multi- overlayers both exists
~\cite{Shi_2012_vdw_epi_MoS2_gr,Azizi_2015_Freevdw_Gr_TMDCs,Miwa_2015_MoS2_gr,Yan_2015_MoS2_on_hBN}.
%
The layer number of epitaxy heterostructure is also
also found relevant to the stacking sequence in graphene/hBN systems ~\cite{Wu_2015_Gr_hBN,Yang_2013_gr_hBN,Wu_2015_Gr_hBN}.
%
These results reveal the
importance of chemical kinetics in the vdWE heterostructure growth,
the interplay between the interactions alone cannot explain the
discrepancy.

The beauty of 2D vdWHs is essentially to construct complex materials
with from 2D layer building blocks, while the property of the stack is
strongly influenced by the order and orientation of 2D building
blocks.
%
In other words, the physical property of a vdWH is not trivial
summation / averaging from that of individual layers, with
unprecedented physical phenomena coming from the interlayer coupling.
%
Examples of such interlayer coupling include layer-dependent bandgap
opening, \worktodo{cite} ultra\-fast interlayer charge transfer,
interlayer exciton formation, magnetic coupling and even topological
insulating / superconducting states.
\worktodo{find 1-2 citations each}


\subsubsection{3D-2D Interface}
\label{sec:intro-3D-2D}

Solid-state 3D assembly on 2D materials can be made by layer-by-layer
deposition of small molecules or hetero\-epitaxy of covalently bonded
structure. The intermolecular (or interatomic) interactions become
dominant over the molecule-2D material and molecule-substrate
interactions. However, interfacial interactions still play an
important role in the molecular orientation and morphology. The
packing and morphology of the 3D assembly greatly influence several
key properties including carrier transport, interfacial barrier, which
motivates understanding of the underlying mechanism.  The diversity of
3D epitaxial morphology addresses the question of how the macroscopic
structure is influenced by the interplay between the interactions.

\paragraph{Layer-by-Layer Assembly of Small Molecules}
\label{sec:org2cdd8f0}

Layer-by-layer (LbL) self-assembly of small molecules can be viewed as
the vertical epitaxy of 2D assembled structure. The molecular
orientation and packing of the interfacial layers, i.e., the first few
layers of the molecules, are more influenced by the molecule-2D
material and molecule-substrate interactions, compared to the
molecules far from the interface. The influence of the 2D material and
substrate can be  characterized by the penetration depth of interfacial forces, \ie the
maximum molecule layer number influenced by the 2D material and
substrate.
%
A transition of
molecular orientation is usually observed beyond the penetration depth
as a consequence of vanishing interfacial interactions.
%
The penetration depth is highly system-dependent.
%
For instance, self-assemblies of small organic molecules like
C\(_{\text{60}}\)~\cite{Lu_2012_c60_gr_moire} and TCNQ
~\cite{Maccariello_2014_TCNQ_gr_Ru} on graphene/\allowbreak{}Ru(0001) surface undergo
transition from Kagome lattice to close-packed structures at $\sim$3
monolayer (ML) and $1$ ML, respectively.
%
The orientation of an organic molecular can even fully flip beyond the
penetration depth. PEN molecules deposited graphene/SiC surface
exhibit the face-on orientation with long-range ordering at 1 ML
coverage ~\cite{Jung_2014_pentacene}, while the edge-on orientation
gradually emerges with thicker deposition
~\cite{Chen_2008_transition_pentacene}. After the fifth layer, the PEN
molecules fully adapt the edge-on orientation, consistent with that in
the bulk crystal ~\cite{Ruiz_2004_bulk_pentacene}. Similar structural
transition is also observed in C8-TBTB assembly on
graphene~\cite{He_2014_C8BTBT_gr} and MoS\textsubscript{2}
~\cite{He_2015_C8BTBT_MoS2}, with the penetration depth on graphene slightly longer than on MoS\textsubscript{2} due to stronger π-π interactions.
%
Although in principle, the weak molecule-2D and molecule-substrate
interactions vanishes after a few nano\-meter, the interface-induced orientation may still be preserved over long range.
For aromatic molecules with a large planar structure
such as MPc derivatives, the strong intermolecular interactions lead
to packing along c-axis
~\cite{Ren_2011_DFT_CuPc_epi_gr,Jiang_2014_F16Pc,Yoon_2010_crystal_F16cuPc},
this leads to the propagation of the face-on orientation even
throughout the 3D epitaxy structure, with no apparent penetration
depth. On the contrary, the epitaxy
structure of thin-layer MPc on MoS\(_{\text{2}}\)
~\cite{Zhang_2015_CuPc_MoS2} shows less stability which undergo a
face-on to edge-on transition upon air exposure and forms 1D-ordered
structures. 

The morphological control of 3D molecular epitaxy can also be achieved
by fine-tuning the chemical structure of epitaxial molecules, such as
the cases of
CuPc/F\(_{\text{16}}\)CuPc~\cite{Singha_Roy_2012_CuPc_gr_glass,Xiao_2013_jacs_CuPc_gr,Zhong_2012_gr_F16_pn_junc,Yang_2011_F16CUPc_nanowire}
and
PEN/PFP ~\cite{Salzmann_2012_fpen_gr,Breuer_2011_pent_grap}
on graphene and MoS\textsubscript{2} surfaces.
%
While the nanoscale orientation remains the same, fluorination of the
molecule causes hugh change of micro\-scale file morphology by
decreasing the grain size and increasing of surface roughness.
%
While the detailed mechanism of such phenomenon is still not fully
understood, the change of film interfacial energy caused by the
fluorination may be a plausible explanation.

The atomically flat and chemically inert surfaces of 2D materials
allow the growth of highly crystalline organic thin films. Organic semiconductors such as Rubrene
\worktodo{use capital for R?}
grown by the template effect of
2D materials exhibit high mobility ($>$ 10
cm$^{2}\cdot$V$^{-1}$s$^{-1}$) which is previously only achievable by
single crystal. \worktodo{Cite Lee adv mater 2014 26 2812}
%
The high mobility brought by interfacial molecular packing and
orientation provides rich opportunity for fabricating high performance
electronic devices such as organic field effect transistors (OFET) and
pn-junctions.
%
Current on-off ratio as high as 10$^{8}$ can be achieved in lateral
OFET based on PEN/graphene heterostructure \worktodo{cite lee J am
  chem soc 2011 133 4417}.
%
Vertical OFET made on C\textsubscript{60}/graphene interface also
shows current on-off ratio up to 10$^{5}$. \worktodo{cite Shih}
%
More interestingly, engineering the interfacial molecular orientation
not only enables high carrier mobility, but also changes the work
function of organic layer, providing more flexibility tuning the
electronic properties of the organic/2D material
heterostructure. \worktodo{cite 2 papers from NUS}



\paragraph{Van der Waals Epitaxy of 3D Crystals}
\label{sec:orgeb0161b}
%20191009-1013

The vdWE of 3D crystals on 2D interface s
hares the same mechanism with the 2D
vdWE, while non-planar (such as sp\(^{\text{3}}\)) bonds are involved.
%
A variety of bulk crystalline semiconductors (including
III-V~\cite{Alaskar_2015_GaAs_gr_Si_theor,Kim_2017_remote_epi_Gr,Nepal_2013_GaN_gr,Kim_2014_direct_vdw_GaN_gr,Makimoto_2012_InGaN_hBN},
II-VI~\cite{Loeher_1994_vdw_epi_CdS_MoTe,Loeher_1996_CdTe_MoWTe},
oxides~\cite{Oh_2014_ZnO_hBN,Chung_2010_GaN_ZnO_gr})
can be epitaxially grown on mono- or multilayer graphene, hBN and TMDCs, despite the lattice mismatch between 2D and 3D materials as high as 40\%.
%
In
these systems, the 2D material not only serves as the buffer layer for
vdWE, but also acts as a transferable layer which enables separation
of the 3D crystal from the substrate to allow fabrication of
semiconductor devices
~\cite{Makimoto_2012_InGaN_hBN,Kobayashi_2012_GaN_hBN,Kim_2014_direct_vdw_GaN_gr,Kim_2017_remote_epi_Gr}.
%
Besides the molecule-2D material interactions, the underlying
substrate below 2D material is also shown to have impact on the
epitaxial overlayer, resulting in the epitaxial layer remotely follows
the  lattice of the underlying semiconductor substrate (GaAs, InP and GaP)
~\cite{Kim_2017_remote_epi_Gr}. Moreover such interactions are found to be stronger for materials using higher polarity. \worktodo{cite Kang 2018}
%
The exact mechanism of such phenomena is, however still an open question.


\subsubsection{Summary}
\label{sec:org0b4290f}

As a summary, \autoref{tbl:intro-summary-multidimension} lists the
representative examples of multidimensional molecular epitaxy on 2D
materials interfaces, and outlines the governing interactions and the
comparison between the interactions that discussed in this section. A
clear relation between the interaction and the dimension of assembly
can be observed: molecular assembly of higher dimension is favored
with increasing intermolecular interactions. In the case of 2D-2D and
3D-2D interfaces, the intermolecular interactions (covalent bond) can
be much greater than the molecule-2D interactions (vdW). It can also
be found that the molecule-substrate interactions are constantly
weaker than the other two types of interactions, as a result of
increased interaction distance and partial screening effect of the 2D
material layer. Understanding the role of weak interactions in
determining the molecular epitaxy poses a challenge towards
comprehensive theory framework, and is crucial for the designing of
epitaxial systems on 2D materials interfaces.


\begin{table}[htbp]
\centering
\begin{center}
  \small
\begin{tabularx}{1.0\textwidth}{XXXX}
\hline
Dimensionality & Examples & Governing Interaction(s) & Comparison between Interactions\\
\hline
0D-2D & Strongly interacting surface & Site-specific & Molecule-2D \(\gg\) intermolecular\\
\hline
1D-2D & Strongly interacting surface & H-bond,  site-specific & Molecule-2D \(\approx\) intermolecular\\
 & Weakly interacting surface & Multivalent H-bond & Intermolecular  \textgreater{} molecule-2D\\
\hline
2D-2D & ML on graphene & (H-bond), CT, \(\pi\)-\(\pi\) & Molecule-2D > intermolecular\\
 & ML on TMDC & (H-bond), CT, vdW & Intermolecular  \textgreater{} molecule-2D\\
 & 2D vdWE & Covalent bond,   vdW & Intermolecular  \(\gg\)  molecule-2D\\
\hline
3D-2D & LbL assembly & (H-bond), CT, \(\pi\)-\(\pi\), vdW & Depending on the 2D material\\
 & 3D vdWE & Covalent bond,   vdW & Intermolecular \(\gg\) molecule-2D\\
\hline
\end{tabularx}
\end{center}
\caption{\label{tbl:intro-summary-multidimension}
Summary of representative examples of multidimensional molecular epitaxy on 2D materials interfaces, the governing interactions involved and the comparison between the interactions discussed in this section.}
\end{table}




\section{Open Questions  Concerning 2D Interfaces}
\label{sec:chall-probl-conc}

As discussed in \autoref{sec:2d-mater-interf}, our understandings on
the surfaces of 2D materials have greatly advanced in the past decade
thanks to the development of novel experimental techniques as well as
theoretical frameworks.
%
However, there are still several open questions concerning the
interfacial properties that are not yet well-understood. Many of them
are related with the electromagnetism on these low-dimensional
interfaces, including the electrostatic penetration and screening
through 2D materials, dynamic dielectric properties of 2D material and
heterostructures, many-body exciton effect, and vdW interactions.
%
These phenomena covers broad length scales, ranging from atomistic
interactions to macroscopic wetting behavior.
%
This section aims to give a brief introduction to several selected
topics, which serves as the prelude for the research
work presented in this thesis.
%
It is worth noting that these open questions are not caused by lack of
fundamental theory framework, but rather because current physical
relations in bulk materials can be rather different in these
low-dimensional systems.


\subsection{Electrostatic Interactions Through 2D Sheet}
\label{sec:electr-inter-thro}

% 20191009-1324
The electrostatic (Coulombic) interactions are long-range forces that
can be extended to several nano\-meter or even micro\-meter scale.
\worktodo{Cite griffith book}
%
On the contrary, with only one to few atoms perpendicular to its basal
plane, the thickness of 2D materials can be much smaller than the
length scale of electrostatic interactions.
%
As a consequence, a 2D material cannot fully screen the electric field
(\ie the 2D layer is partially ``transparent'' to electrostatic
interactions).
%
In bulk systems with mobile carries (electrons / holes in solid-state
semiconductor, or ions in electrolyte solution), the characteristic
length of electrostatic interactions is described using the Debye
length $\lambda_{\mathrm{D}}$, which is defined as:
\begin{equation}
  \label{eq:intro-debye}
  \lambda_{\mathrm{D}} = {\displaystyle \sqrt{
      \frac{\varepsilon_{0} \varepsilon_{\mathrm{s}} k_{\mathrm{B}} T}
      {e^{2} n_{\mathrm{s}}}
    }}
\end{equation}
where $\varepsilon_{0}$ is the vacuum permittivity,
$\varepsilon_{\mathrm{s}}$ is the relative permittivity of the system,
$k_{\mathrm{B}}$ is the Boltzmann constant, $T$ is the temperature,
$e$ is the unit charge, and $n_{\mathrm{s}}$ is the overall carrier
density.
%
However, such definition may be not correct for a 2D material due to
(i) the carrier density of a bulk system cannot be readily applied to
a 2D sheet, and (ii) the electronic structures of 2D materials are
completely ignored.
%
Indeed, theoretical investigation shows a more complex picture of the
electrostatic interaction through 2D materials and their stacks.
%
Using a discrete model based on total energy optimization and taking
into account of graphene's Dirac cone structure, Kuroda et
al. demonstrate a non-linear behavior of the effective screening
length $\lambda_{\mathrm{eff}}$ in multi-layer graphene
(MLG). \worktodo{cite Kuroda PRL 2011, Rokni Sci Rep}
%
Unlike $\lambda_{\mathrm{D}}$ in bulk systems which decays with
$n_{\mathrm{s}}^{-1/2}$, $\lambda_{\mathrm{eff}}$ does not show a
single power law with the doping density, and more interestingly,
$\lambda_{\mathrm{eff}}$ saturates when doping density decreases to
zero, in contrast with $\lambda_{\mathrm{D}} \to \infty$ that would be
expected in bulk systems.
%
The complexity of the electrostatic screening through graphene and
other 2D materials has also been shown in both theoretical and
experimental studies.  \worktodo{cite Uesugi, Sci Rep 2012, Goto NL
  2013, Datta 2009 NL, LuHua Li NL 2015}
%

The partial penetration of electric field through 2D materials is
the fundamental mechanism behind several 2D-material-based electronic
devices such as the graphene barristor and vertical field effect
transistor (VFET). \worktodo{cite Yang science \& Shih NL}
%
In these devices, the electric field from a dielectric layer
penetrates graphene (or other 2DEGs) and changes the interfacial
states at the semiconductor/2DEG junction, to enable on/off switch of
current passing through the semiconductor.
%
The operation mechanism of such devices are widely thought to be
modulated by the interfacial transport barrier. \worktodo{cite yang
  science and 2 others} However, this cannot fully explain the other
features such as selective carrier injection at the semiconductor /
2DEG \worktodo{should separate or without blank?} \worktodo{cite Shih NL}
%
Moreover, the influence of the penetrated field on the adjacent
semiconductor is usually ignored.
%
Therefore, a complete theoretical framework taking account of the
previous concerns is of high demand.

An even more challenging issue is the penetration of electric field
through 2D vdWHs. The non-linear electrostatic screening observed for
homogeneous 2D stacks can be more complex, hen combining 2D material
build blocks with distinct electronic properties and stacking order.
%
A few theoretical efforts have been made to elucidate the influence of
external electric field on the 2D vdWHs. \worktodo{cite DTU 2017 NL,
  ACS AMI, Santos paper X2}
%
However, such studies usually employ full-scale first principle
simulations, which is time-consuming and is hard for up-scaling
studies.
%
Simplified models that can capture more insights are therefore of high
interest, in order to help understand the exotic electrostatic behaviors of 2D
vdWHs and to provide guideline for device design.


\subsection{Dielectric Properties of 2D Systems}
\label{sec:diel-prop-2d}

% 20191009-1745
When an insulating material is placed under external electric field
$\mathbf{E}_{\mathrm{ext}}$, the electron cloud is distorted from its
equilibrium state and creates induced dipoles, a process known as
dielectric polarization. The induced dipoles screen the external
electric field and causes the electric field inside the material,
$\mathbf{E}$ to be smaller than $E_{\mathrm{ext}}$.
%
Such response of a material to a dynamic electric field is
characterized by the dielectric function $\varepsilon$ (also known as
relative permittivity).
%
$\varepsilon$ is defined as the ratio between the magnitudes of
$\mathbf{E}$ and $\mathbf{E}_{\mathrm{ext}}$ and has a complex value.
%
It plays a central role in electromagnetism and is connected
to various physical quantities, including optical absorption, electric
conductivity, electric capacitance, impurity energy level, Debye
screening length and vdW interaction coefficient.
\worktodo{think of citing these?}
%
In the context of 2D materials, $\varepsilon$ is widely used to
interpret several energy scales, in particular the exciton binding
energy of direct-gap TMDCs and hBN.
%
However, values of $\varepsilon$ for a certain material in literature
usually has discrepancy of 1$\sim{}$2 orders of magnitude. \worktodo{cite paper 9-13 from Elton ACS Nano, and Li Chem Soc Rev}
%
Such inconsistency causes even worst estimation of corresponding
energy levels (for instance, the exciton binding energy is usually
assumed to be proportional $\varepsilon^{-2}$).
%
Even in the theoretical studies, the dielectric screening of 2D
materials are still under huge debate.  The $\varepsilon$ values
reported for a 2D material range from near unity (\ie no dielectric
screening) to almost equal to $\varepsilon$ in its bulk counterpart
(\ie no difference between 2D and 3D cases). \worktodo{you know what to cite}
%
Moreover, whether the quantum confined structure causes anisotropy in
the $\varepsilon$ of a 2D material, is still questionable.
%
Clearly, precisely determining the dielectric properties of 2D
materials is critical to build correct physical models of these
low-dimensional materials.

%
% 20191010-0947
In addition to the disputed fundamental definition of the dielectric
properties (such as $\varepsilon$) of a 2D material, there are also
complexities \worktodo{I don't this sentence, try to change it?}
%
The electronic structure of a 2D material or vdWH can be greatly
influenced by external electric field, such as modulating the bandgap,
\worktodo{ite paper 11-20 in Kumar phosphorene paper} and even
inducing transition from parabolic dispersion to Dirac cone in the
electronic band structure \worktodo{cite paper 21 and 22}.
%
Similarly, the dielectric properties of 2D materials and vdWHs are
also shown to be electric-field-dependent from first principle
simulations,
%
including multilayer stacks of graphene \worktodo{Elton NL}, TMDCs,
GaS, phosphorene, and heterostructure of hBN/graphene.
%
A commonly
observed trend is the nonlinear increasing of the dielectric response
under stronger fields ($>$ 0.01 ev/\AA{}).
%
Such phenomena have clearly different origin compared with the
field-dependent $\varepsilon$ observed in some bulk para- and
ferro-electric materials, which is usually shows decreasing
$\varepsilon$ upon stronger electric field.  \worktodo{cite Hemberger,
  Maiti and christen}
%
Some studies attribute that the field-dependent dielectric properties
of 2D materials and vdWHs to the change of electronic band structure
and potential insulator-conductor transition, while the exact
description and predictions for other materials combinations, are
still missing.\worktodo{Add some more here}
%

\subsection{Van der Waals Interactions and Wetting Phenomena}
\label{sec:van-der-waals}

% 20191010-1055
As discussed in \autoref{sec:intro-mol-subst}, with the layer
thickness of only few \AA{}, the ultra\-thin 2D material sheets may
not only be partially transparent to long-range electrostatic
interactions, but also short-range vdW forces.
%
The partial penetration of vdW interactions through a 2D material
sheet is in principle possible since the thickness (e.g. $\sim{}$3.3
\AA{} for graphene) is still shorter than the effective length scale
of vdW interactions ($\sim{}$ 1 nm).\worktodo{where to cite?}
%
In literature, such hypothesis is usually examined by two approaches:
the liquid wettability or solid-state molecular epitaxy on supported
2D material surfaces.
%
Although many studies claim the validity of the vdW- or
wetting-transparency through 2D materials, the interpretations of
experimental data usually vary greatly between cases.
%
For instance, Rafiee, Mi et al reported that the liquid contact angle
(CA) on graphene supported by gold or silicon substrates would be
almost identical to that observed on bare substrates, leading to the
claim of complete wetting transparency of graphene \worktodo{cite
  rafiee}.
%
However, this analysis is later shown to be incomplete due to complete
ignoring the liquid-graphene interactions, as discussed in
\autoref{sec:inter-forc-at}.
%
A more complete theoretical model developed by Shih et al
\worktodo{cite} based on the pair-wise vdW interaction and Boltzmann
distribution of liquid molecules predicts that the contact angle on
substrate-supported graphene, would be saturated below 100$^{\circ}$,
which renders the graphene sheet as ``wetting translucent''.
%
The controversy of ``wetting transparency'' or ``wetting
translucency'', is however not completely settled.
%
The cause of the dilemma is two-fold: (i) the wetting properties on
graphene and other 2D materials, depend greatly on the experimental
conditions, and (ii) the physical model describing the interfacial vdW
interactions can be incomplete. We will briefly extend on these topics
in following paragraphs

%
In principle, the intrinsic wettability of a 2D material, should be
measured using suspended sheets to exclude effects from the underlying
substrate.
%
However, in reality this is extremely hard to achieve due to the large
mismatch between the size of a droplet using conventional contact
angle goniometer ($>$0.1 mm) and the largest size of suspended 2D
material so far ($<$10 μm) \worktodo{is it true?}
%
As a result, most reported values of water contact angle of graphene
are still performed on substrate-supported samples, with CA ranging
from below 50$^{\circ}$ to almost 180$^{\circ}$.
%
In addition, the transfer process of 2D materials usually leaves
polymer residues and other defects on the surface, which are also
shown to affect the CA on 2D material surfaces. \worktodo{cite Kobizal
  papers}
%
Moreover, even if polymer- and defect-free 2D material samples can be
obtained, the CA is also shown to be influenced by air-born
hydrocarbon contaminations under ambient, causing a
hydrophilic-to-hydrophobic transition \worktodo{cite Li liu paper}
that ubiquitously found in graphene and TMDC samples.
%
Clearly, any theoretical model dealing with liquid wetting on 2D
material interfaces that makes use of the ``intrinsic'' vdW
interaction coefficients, would be affected by the
large uncertainty of reported wettability.
%
On the contrary, the above issues concerning contamination can be
effectively minimized in molecular epitaxy, which is usually operated
under high vacuum and annealed conditions.
%
\worktodo{cite ME papers}
%
The studies using molecular epitaxy usually compare the morphology and
nucleation density of target molecules deposited onto various 2D
materials supported by different substrates.
%
Although various studies demonstrate the validation of vdW
transparency through molecular epitaxy, the conclusions are in general
indirect and only limited at qualitative level.
%
Moreover, as described in \autoref{sec:intro-3D-2D}, the nanoscale 3D
molecular epitaxy is a collective result of both thermodynamic and
kinetic processes, making the analysis more difficult than the case of
macroscopic wetting phenomena.


Another open question is whether our understandings of interfacial vdW
interactions are complete. For instance, most of the theories about
vdW transparency \worktodo{cite shih and Sci rep paper} assumes the
vdW interactions between the substrate and liquid phase is reduced due
to the thickness of the 2D material layer.
%
The accuracy of such assumption is questionable: using accurate atomic
force spectroscopy, Tsoi et al. showed the molecule-substrate vdW
interaction coefficient (the Hamaker constant \worktodo{cite Hamaker
  paper}), actually changes by varying the 2D material.
%
\worktodo{consider these works \cite{Kong_2012_gr_screen,Cho_2012_c60_gr_decoupl,Tsoi_2014_vdW_screening_2D,Zheng_2016_org_tmdc_screen,Gurarslan_2016_MoS2_vdW_iso}.}
Such many-body effect are not considered in current frameworks of
ground-state density functional theory (DFT) and molecular dynamics
(MD), making the study of such effects particularly difficult.
%
Moreover, the low DOS near the intrinsic Fermi level
~\cite{Das_Sarma_2011_electron_gr,Bhimanapati_2015_2D_rev} of 2D
materials ensures they be easily doped by either substrate-2D material
interactions
~\cite{Varchon_2007_doping,Giovannetti_2008_doping,Chen_2013_doping}
or an electric displacement
field~\cite{Das_2008_doping,Perera_2013_doping}. \worktodo{cite more
  self work}
%
Several works have indicated the potential influence of 2D layer
doping with the change of interfacial wettability and molecular
orientation \worktodo{cite 2 NL paper, Hutmann, Nguyen paper etc}.
%
It is possible that the doping effect can be important for several
experimental systems such as wetting graphene\allowbreak{}/\allowbreak{}metal
surface. \worktodo{cite the 2 MESA+ paper}
%
Therefore In order to precisely determine the vdW transparency , one
needs to decouple it from the substrate-induced doping effect of 2D
materials, as addressed above
\cite{Huttmann_2015_vdw_gr_doping,Muruganathan_2015_tunable_vdw_gr,Hong_2016_mechanism,Ashraf_2016_doping}.









\section{Scope of the thesis}
\label{sec:scope-thesis}



%%% Local Variables:
%%% mode: latex
%%% TeX-master: "../thesis"
%%% End:
