% \chapter{Introduction}
\chapter{Two-Dimensional Materials and Interfaces}
\label{ch:introduction}
\newcommand*\imgdir{img/ch-intro/}

\dictum[Wolfgang Pauli]{%
  God made the bulk;\\Surfaces were invented by the devil
  }%

\vspace{1em}

\chapterabstract{Part of this chapter appears in the following
  journal article: Tian, T. \& Shih, C.-J. Molecular Epitaxy on
  Two-Dimensional Materials: The Interplay between
  Interactions. Ind. Eng. Chem. Res. 56, 10552--10581 (2017).  }


\section{Overview of Two-Dimensional (2D) Materials}
% \section{Introduction}
\label{sec:ch-intro-2D}
Controlling the dimensionality of materials provides rich
opportunities of tuning the electronic, optical and mechanical
properties which facilitates novel devices and applications.
\worktodo{put more things inside and make claim clear} \worktodo{put
  1-3 refs}. Good example of such material design are the
two-dimensional (2D) materials, which are covalently-bonded
crystalline films with only one- or few-atom thickness. \worktodo{HH??
  Change wording}
%
Starting from the first discovering of graphene, an allotrope of
carbon in 2004\worktodo{cite Geim}, the family of 2D materials has
expanded significantly throughout the last decade. \worktodo{cite
  Novoselov and 2 papers} The electronic properties of 2D materials
cover a wide spectrum ranging from insulators to superconductors,
making them promising candidates to replace conventional bulk
materials in semiconductor industries. \worktodo{cite 1-2}.
%
This section aims to give a brief introduction about the structures,
electronic properties and fabrication techniques, which serves as a
prelude for introducing the 2D materials interfaces.
\worktodo{again, change last part}

% Understanding and engineering the
% interfaces of 2D materials are important to achieve practical uses of
% 2D materials in our 3D world. In this chapter, we briefly discuss the
% fundamental aspects of 2D materials and introduce their interfacial
% phenomena. Several challenges about the 2D materials interfaces are
% also discussed, serving as the prelude to the work of this thesis.


\subsection{Categories of 2D Materials}
\label{sec:categ-2d-mater}

So far, there have been no naturally-existing isolated 2D
materials. The closest form is the layered bulk materials, in which
interlayer interactions are governed by van der Waals (vdW) forces.
%
With proper separation techniques, these layered materials can be
thinned to a few layers or even single layer.
%
The best-known example is the mechanical exfoliation of 2D layer from graphite,
now known as graphene. \worktodo{cite geim}
%
Using such technique, monolayers exfoliated from naturally-existing
layered bulk materials were achievable. Examples include the hexagonal
boron nitride (hBN), transition metal dichalcogenides (TMDCs, with
general chemical composition MX\textsubscript{2}, where M is a
transition metal and X is S, Se or Te), monochalcogenides (GaS, GaSe),
monolayer black phosophorus (BP) and MXene (M = Ti, Nb, V, Ta, etc.,
X = C or N).
%
The family of 2D materials that have been experimentally discovered,
has extended from only 1 in 2004 to over 100 \worktodo{really?} as of
2019, corresponding to an average discovery of about 6 new materials
each year. \worktodo{correct?}
%
Apparently, the known 2D materials are still scarce and limited
compared with the bulk materials,
due to the huge experimental effort
required to synthesize, isolate and characterize 2D materials limit
the speed of novel 2D material discovery.
% 
In view of this, computer-aided material design (CAMD) was
demonstrated in several works to help find new 2D materials either
from high-throughput materials database screening or \textit{ab
  initio} calculations. As a pioneer study, Lebègue et al. performed
data mining on the crystal structures listed in the International
Crystallographic Structural Database (ICSD) to filter out a total
number of 92 layered bulk crystals and to generate corresponding 2D
materials. As an extension, Mounet et al. proposed new algorithm with
more robust criteria for dimensionality and found over 5000 layered
materials, among which over 1800 can be easily or potentially
exfoliated according to \textit{ab initio} calculations.
%
Another example is the combinational design of new 2D materials, which
starts with a known structure (for example MX\textsubscript{2}) and
replace with other elements. This allows the discovery of over 3700
thermo-dynamically stable 2D materials, among which most chemical
compositions are not yet discovered experimentally.

These computational studies broaden our knowledge about 2D
materials. In particular, the large size of database allows
systematical study to search for optimal 2D material according to
their electronic and optical properties, and apply to real-world
applications. As a summary, the most common types of 2D materials from
either experimental or computational discoveries, are listed in
\autoref{tab:category-2D}.

\begin{table}
  \centering
  \caption{Summary of common 2D materials and their structures. The
    formula refer to the chemical composition per unit cell.}
  \label{tab:category-2D}
  \begin{tabularx}{1.05\linewidth}{XXXX}
    \hline
    Prototype  & Example Formula  & Lattice Symmetry & Example Materials \\
    \hline
    Graphene & C\textsubscript{2} &  P6/mmm & Graphene, Silicene, Germanene \\
    Graphane & C\textsubscript{2}H\textsubscript{2} &  P3m1 & Graphane, Fluorographene\\
    hBN      & BN                & P$\overline{3}$m2 & h-BN \\
    2H-MX\textsubscript{2} & MoS\textsubscript{2} & P$\overline{3}$m2 & 2H-MoS\textsubscript{2}, 2H-MoS2\textsubscript{2}, 2H-WS\textsubscript{2} \\
    1T-MX\textsubscript{2} & CdI\textsubscript{2} & P$\overline{3}$m1 & 2H-MoS\textsubscript{2}, CdI\textsubscript{2}\\
    BP & P\textsubscript{4} & Pmna & Phosphorene, Arsenene \\
    Monochalcogenidec & Ga\textsubscript{2}S\textsubscript{2} & P$\overline{3}$m2 & Ga\textsubscript{2}S\textsubscript{2}, Ga\textsubscript{2}Se\textsubscript{2} \\
    Bismuth iodide &  Bi\textsubscript{2}I\textsubscript{6} & P$\overline{3}$m1 & Bi\textsubscript{2}I\textsubscript{6}, Al\textsubscript{2}Cl\textsubscript{6} \\
    Iron oxychloride                &  Fe\textsubscript{2}O\textsubscript{2}Cl\textsubscript{2} & Pmmn & Fe\textsubscript{2}O\textsubscript{2}Cl\textsubscript{2}, Fe\textsubscript{2}O\textsubscript{2}Br\textsubscript{2}  \\
    MXene & Ti\textsubscript{x}C\textsubscript{y} & P$\overline{3}$m2 & Ti\textsubscript{3}C\textsubscript{2}, Ti\textsubscript{4}N\textsubscript{4}, Mo\textsubscript{2}TiC\textsubscript{2} \\
    2D Perovskite & CH\textsubscript{3}NH\textsubscript{3}PbBr\textsubscript{3} & Pm$\overline{3}$m & (MAPbBr\textsubscript{3})\textsubscript{n}\\
   \hline
\end{tabularx}
\end{table}

\subsection{Basic Electronic Properties of 2D Materials}
\label{sec:basic-electr-prop}

The common feature of 2D materials is the absence of dangling bonds at
the surface, as a results, the electrons are highly confined within
the 2D plane.
%
For instance, in graphene and hBN, the sp\textsuperscript{2} orbitals
form covalent (σ) bonds, while the p-orbitals perpendicular to the 2D
plane form delocalized π-electron cloud.
%
Such two-dimensional
electron gas (2DEG) gives rise to dramatic change of its electronic
properties compared with bulk materials.\worktodo{cite Devies}
%
Although the physics about 2DEG has been well developed in the 1980s,
prior to the discovery of 2D materials, the 2DEG can only be achieved
by cumbersome semiconductor quantum well heterostructures. In other
words, the 2D materials are perfect candidates to study the behavior
of 2DEGs. \worktodo{say something more?}

Although the electronic band structures of different 2D materials vary
a lot, one thing in common is that their density of states (DOS). The
DOS is a measure of the number of available states in a certain system
at certain energy level. For an extended system with nearly-continuous
energy distribution, the DOS at energy $E$ is expressed as:
\begin{equation}
  \label{eq:ch-intro-dos}
  \mathrm{DOS}(E) = \frac{\partial N}{\Omega \partial E} =  \frac{1}{\Omega} {\displaystyle \int_{\Omega}} \frac{\mathrm{d}^{n} \mathbf{k}}{(2 \pi)^{n}}
  \delta(E - E(\mathbf{k}))
\end{equation}
where the integral is performed over a system with $d$-dimensionality,
$N$ is the total number of states,
$\Omega$ is the volume of the system, $\mathbf{k}$ is the momentum of
electron, and $\delta$ is the Dirac delta function. Without losing
generality, the DOS can be expressed using the law of chains:
\begin{equation}
  \label{eq:dos-chain}
  \mathrm{DOS} = \frac{\partial N}{\Omega \partial k} \frac{\partial k}{\partial E}
               = \frac{k}{2 \pi} \left(\frac{\partial E(k)}{\partial k}\right)^{-1}
\end{equation}
where $k=|\mathbf{k}|$ is the modulus of $\mathbf{k}$.
%
\autoref{eq:dos-chain} provides a general way linking the DOS to the
energy--momentum ($E-k$) dispersion of a 2D material, which is further
extracted from its band structure. For monolayer 2D materials, two
cases can be distinguished:
\begin{itemize}
\item Parabolic materials: such as TMDCs, hBN, phosphorene

  These are the majority of 2D materials where the $E-k$ dispersion is
  parabolic, such that
  ${\displaystyle E(k) = \frac{\hbar^{2} k^{2}}{2 m^{*}}}$, where
  $\hbar$ is the reduced Planck constant, and $m^{*}$ is the effective
  mass near the band edge. From \autoref{eq:dos-chain}, DOS of a
  parabolic material is \textit{constant}, and proportional
  to $m^{*}$.
  
\item Dirac materials: such as graphene, silicene, germanene

  The Dirac materials are relatively rare among 2D materials,
  \worktodo{cite Wang nsr 2015} where the $E-k$ dispersion is linear:
  $E(k) = \hbar v_{\mathrm{F}}k$, where $v_{\mathrm{F}}$ is the Fermi
  velocity. The Dirac materials has DOS increasing linearly with $k$
  (as well a $E$).
\end{itemize}

The difference between the DOS of ideally parabolic and Dirac
materials can be seen in \worktodo{Figure here}. Unlike parabolic
materials with constant DOS, the Dirac materials have DOS → 0 near the
Dirac cone ($E \to 0$). This feature brings very interesting
properties like electrostatic transparency which we will discuss in
\worktodo{Chapter 2}.

\worktodo{Discuss more about the parabolic and Dirac cone?!}
\worktodo{Some simple discussion based on the band gap etc...}


\subsection{Fabrication of 2D Materials}
\label{sec:fabr-2d-mater}

The development of the 2D material researches cannot be achieved
without appropriate methods to fabricate large-area and high quality
2D materials. The synthesis of 2D materials can be generally
categorized into top-down and bottom-up approaches.

\subsubsection{Top-down Methods}
\label{sec:top-down-methods}

The top-down method is the straightforward approach to exfoliate few-
to single-layer 2D materials from their bulk counterparts. The two
most-used techniques are micro-mechanical exfoliation and liquid-phase
exfoliation. \worktodo{add cites here, from Liu Adv Mater 2018}

The micro-mechanical exfoliation uses mechanical force to overcome the
interlayer vdW interactions in bulk layered materials, and is the
first method used for 2D material exfoliation. Scotch tape is widely
used for such type of exfoliation \worktodo{cite Geim et al}, while
other media as elastic polymers, heat-release tapes are also
employed. If the quality of the bulk layered material is promising
(\ie high chemical purity and low defect density), micro-mechanical
exfoliation usually produces 2D materials with better quality compared
with other methods. However, there are also several key drawbacks of
such methods. First of all, the reliability of micro-mechanical
exfoliation is highly influenced by the interlayer forces, which makes
it difficult to exfoliate materials with high interlayer binding
energy, or in-plane mechanical anisotropy (for instance BP). This
method also leads to broad distribution of flake size and layer
number, making it difficult to be applied to large-scale
applications.

In contrast to micro-mechanical exfoliation where only the top-most
layers are removed, the solution-phase exfoliation method ensures
uniform breaking up of bulk materials, and increases the scalability
of 2D flakes produced. It can be achieved by both physical and
chemical exfoliation approaches. Physical solution-phase exfoliation
makes use of local mechanical stress produced by ultra-sonication or
shearing to overcoming the interlayer vdW interactions. To ensure the
stability of isolated layers in the liquid suspension, high surface
energy liquid, in combination with surfactants are usually
used. \worktodo{cite papers from Coleman et al, Shih et al. Haung et
  al}
%
Centrifugation can be further used to separate flakes to ensure narrow
distribution of layer thickness and size \worktodo{cite coleman}, and
statistic measurement can be performed using the ensemble of such 2D
material suspensions. \worktodo{wording?} However, this method still
suffers from several drawbacks: (i) the flake size is still limited to
μm-scale, (ii) precise control of layer number in suspension is difficult and 
(ii) flake overlay after deposition onto substrate is unavoidable.
%
The interlayer vdW forces can also be overcome by chemically modifying
the 2D layer (for instance, oxidising graphite to produce graphene
oxide (GOx)), or to intercalate ions between the layers in a bulk
material to induce lattice expansion (such as exfoliation of MXenes,
which are otherwise hard to achieve mechanically). Despite the
scalability of chemical solution-phase approaches, they usually
changes the chemical composition and introduce defects in the 2D
materials, which are undesired for high-performance
applications. 

\subsubsection{Bottom-up Methods}
\label{sec:bottom-up-methods}

The bottom-up methods grow 2D materials from precursors, and aims to
produce 2D materials with larger flake size and more controlled layer
numbers. Depending on whether a substrate is involved in the process,
these methods can be categorized into templated or non-templated
growth.

Templated growth uses a bulk surface as a epitaxial template and
support for the 2D material. Due to the absence of dangling bonds, the
2D-substrate interaction is usually much weaker than the in-plane
covalent bonds. Therefore, unlike epitaxy of bulk materials which
require precise control of lattice mismatch between the substrate and
epitaxy layer \worktodo{cite 1-2 paper}, templated growth of 2D
materials can be achieved in various substrates. For example, high
quality single crystal graphene can be epitaxially grown by thermal
annealing of silicon carbide (SiC) surface, or decompositing
hydrocarbons on Ruthenium (Ru) (0001) and Ir (111) substrates, while
epitaxial growth of hBN is achieved by cleavage of borazine
(H\textsubscript{6}B\textsubscript{3}N\textsubscript{3}) on Rh (111)
surface.  Such epitaxial method can also be applied to other 2D
materials like TMDC, monochalcogenides and BP. However, there are
still several limitations. First of all, a clean surface as well as
ultra-high vacuum (UHV) conditions are usually required for the
epitaxial growth methods. Moreover, transferring the 2D material onto
other substrates is generally not easy due to the noble metals
involved. Chemical vapor deposition (CVD) is another widely-used
templated growth method, in which one or more precursors adsorb and
react on a catalytic surface to form covalently-bonded 2D
materials. The essence of CVD process is similar to the epitaxial
growth, while more ambient conditions are used (10$^{1}$ Pa to
atmosphere pressure, ultra clean substrate not necessary). CVD growth
of graphene using hydrocarbon source on copper (Cu) is the most
studied and widely used technique. The self-termination of
second-layer on the Cu surface allows the growth of large area single
layer domains up to centimeter or decimeter scale \worktodo{check if
  explanation is correct}. Another advantage of such method is easy
removal of Cu substrate by standard etching procedure, allowing
transferring graphene onto a large variety of
substrates. \worktodo{cite 1-2} Similar to the case of graphene, large
area single-layer TMDCs and hBN can also be achieved using the CVD technique by proper interfacial engineering. \worktodo{cite TMDC paper; hBN show the Gold sample}
%
% With proper defect control during growth and development of novel
% transferring techniques, the CVD method is promising to 

While the majority of bottom-up growth relies on a substrate to form
the 2D material, there are also cases where colloidal 2D confined
structures can be directed synthesized in liquid phase without
template. Examples of 2D materials grown using the colloidal method
include II-VI semiconductors (CdSe), 2D hybrid perovskites, and
TMDCs. The anisotropic growth is usually modulated by surfactant /
ligand engineering. Recently, colloidal insulating 2D metal oxides are
reported to be synthesized by simultaneous oxidation at the liquid
metal-water interface, further extending the possibility of bottom-up
synthesis of 2D materials. \worktodo{cite Dicky paper}

As a summary, both top-down and bottom-up methods are capable of
fabricating 2D materials with desired purity, flake size and
thickness. A short comparison between different fabrication methods can be seen in \worktodo{table 2?!}



\section{The 2D Materials Interfaces}
\label{sec:2d-mater-interf}
%20191008-0939
the 2D materials do not solely attract the research interests due to
the unique electronic properties, they are also materials with ultra
high surface-area-to-volume ratio intrinsically. As a consequence,
interfaces are almost always required when integrating the 2D
materials into experimental studies and applications in the 3D world,
and the interfacial properties play important roles in determining
their proprieties. For instance, the existence of strict 2D lattices
is long questioned due to the presumed distortion caused by thermal
fluctuation which would break up long-range order. Although recent
studies suggest free-standing 2D materials like graphene can be
stabilized by the phonon coupling that causes 3D ripples in the 2D
layer (\worktodo{find 2 papers to cite}), the majority of studies
still require the 2D materials to be supported by substrate or
encapsulated.  As will be shown later, these 2D interfaces may
significantly alter the intrinsic properties by ways such as
structural corrugation and carrier doping. On the other hand, it
remains unrealistic to find a single 2D material which can satisfy all
the requirements concerning high-performance applications (e.g.,
electronic properties, mechanical strength, chemical stability, and
synthetic difficulty), the flexibility of creating mixed-dimensional
interfaces with existing functional materials may offer opportunities
to fully exploit their potential. Playing with the interfacial
interactions is critical for successful engineering of the interface
dimensionality, morphology, electronic states and transport phenomena.
In this section, we focus the discussion on the interactions involved
at the 2D materials interfaces, and how the interplay between these
interactions influences the mixed-dimensional interfaces with 2D
materials \worktodo{say more?}

\subsection{Interactions and Forces at the 2D Materials Interfaces}
\label{sec:inter-forc-at}

The concepts of interactions on 2D materials interfaces, can be
learned from the field of molecular epitaxy and self-assembly on bulk
interfaces
\cite{Kowarik_2008_rev_MBE,Barth_2007,Whitesides_2002_assem_rev,Philips_2D_assem_book}.
% 
A molecule in the bulk
form and on a densely-covered surface feels the interactions from the
other epitaxial molecules, known as the intermolecular
interactions. On the other hand, a molecule undergoes various
processes on a 2D surface, including adsorption, diffusion, rotation,
and vibration, which is governed by the molecule-2D material
interactions. Moreover, the effect of the underlying substrate is
usually important where the molecule-substrate interactions come into
play. 

\subsubsection{Intermolecular Interactions}
\label{sec:orgf0959e1}

The intermolecular interactions govern the packing and orientation
behavior of the molecules several atoms away from the 2D material
surface: the strength and the direction of intermolecular interactions
determine the packing density as well as the orientation of the
molecular epitaxy. The intermolecular interactions can be categorized 
into van der Waals (vdW) interactions, hydrogen bonds (H-bonds), and
covalent bonds depending on their strength. 

\paragraph{van der Waals (vdW) Interaction}

The van der Waals (vdW) interactions are dispersion forces between
charge-neutral molecules, including many organic semiconductors, such
as fullerene (C\(_{\text{60}}\))
\cite{Corso_2004_C60_hBN,Kim_2015_c60_gr,Chen_2016_c60_mos2},
metal-phthalo\-cyanines (MPcs, where M can be Cu, Fe, Zn, Co, etc.)
\cite{Xiao_2013_jacs_CuPc_gr,Wang_2010_selec_F16_gr,Zhang_2011_FePc_gr,Hamalainen_2012_CoPc_gr_Ir,Ying_Mao_2011_ge_clAlPc,Ogawa_2013_AlCiPc_gr,Pak_2015_CuPc_MoS2,Avvisati_2017_FePc_intercal,Iannuzzi_2014_MPc_hBN_Rh},
pentacene (PEN)
\cite{Lee_2011_pentacene,Jariwala_2016_Mos2_pentacene,Shen_2017_DFT_mos2_pent,Kim_2016_trap_Mos2_pent,Nguyen_2015_pent_gr_wett,Betti_2007_orien_pentacene},
perfluoropentacene (PFP)
\cite{Salzmann_2012_fpen_gr,Breuer_2016_acnene_mos2}, rubrene
\cite{Lee_2014_rubene_hBN}, perylene-3,4,9,10-tetra\-carboxylic
dianhydride (PTCDA)
\cite{Wang_2009_STM_PTCDA_Gr,Tian_2010_PTCDA_gr,Huang_2009_PTCDA_gr,Meissner_2012_PTCDA_BLG},
7,7,8,8,-Tetra\-cyanoquino\-dimethane (TCNQ) and its fluorinated
derivative 2,3,5,6-Tetra\-fluoro-7,7,8,8-tetra\-cyanoquino\-dimethane
(F\(_{\text{4}}\)-TCNQ)
\cite{Chen_2007_TCNQ_gr,Hong_2013_ftcnq_gr,Stradi_2014_TCNQ_gr_Ru,Tsai_2015_TCNQ_gr_hbn}.
(\worktodo{Figure here?})

Due to its non-directional and weak force nature, if the vdW
interactions govern the interfacial molecules (weak interacting 2D
interfaces), the molecules tends to form close-packed structures in 2D
or 3D assemblies. The dimensionality of molecular epitaxy by vdW
interactions is usually dependent on the surface coverage, as the
molecule growth mechanism is similar to that of adsorption
isotherm. Although the vdW interactions usually have an energy less than
4 kJ\(\cdot\)mol\(^{-1}\), the collective interactions between molecules
with large electron cloud can be stronger. For the \(\pi\)-conjugated
aromatic molecules listed above, an effect known as the \(\pi\)-\(\pi\)
interaction, a combined effect of vdW interactions and charge
transfer \cite{Hunter_1990_pi}, can lead to preferential stacking and
orientation of the molecules, due to maximal overlapping of
\(\pi\)-electron clouds. 


\paragraph{Hydrogen Bond (H-Bond)}

The hydrogen bond (H-bond) refers to the directional electrostatic
forces between an H atom covalently-bonded to an atom of high
electro\-negativity (such as O, N and F) and another highly
electro\-negative atom in adjacent molecules. Compare the vdW
interactions, hydrogen bonds usually have higher bond energy and
preferred direction, which favors certain assembly structure on 2D
materials. The H-bonds are usually dominating between molecules rich
of N, O and F elements, such as modified PTCDA compounds
\cite{Mura_2010_DFT_H_bond_PTCDA_gr,Karmel_2014_assembl_hetero_gr},
perylene tetra\-carboxylic diimide (PTCDI) derivatives
\cite{Pollard_2010_hbond_assembly_gr,Karmel_2014_PTCDI_gr},
carboxylic-substituted aromatic compounds
\cite{Rochefort_2009_aro_graphene_mech,Addou_2013_TPA_gr}, polycyclic
aromatic compounds
\cite{Kozlov_2012_polyaro_gr,Roos_2011_BTP_gr,Meier_2010_polycyclic_gr}
and inorganic acids \cite{Prado_2011_2D_acid_gr}. The existence of
H-bonds stabilizes the assembled low-dimensional structures on 2D
materials interfaces, such as linear
\cite{Pollard_2010_hbond_assembly_gr} or two-dimensional
\cite{Prado_2011_2D_acid_gr} supra\-molecular assemblies. The specific
adsorption sites on 2D materials (such as the moiré patterns) also
play an important role in the assembly of H-bond-governed molecular
epitaxy.


\paragraph{Covalent Bond}
In general, the interactions between the epitaxial molecules and 2D
material (vdW and Coulombic interactions) are much weaker than the
intra\-molecular covalent bond (including metal coordination forces),
resulting in a variety of structures on 2D materials interfaces
\cite{Bakti_Utama_2013_rev_epitax}. One example is the van der Waals
epitaxy (vdWE) technique which allows 2D or 3D crystalline growth on
2D materials. As discussed before \worktodo{which section?}, the
absence of dangling bonds eliminates the lattice mismatch between
dissimilar materials, leading to a number of 2D vertical
heterostructures including: TMDC/graphene
\cite{Shi_2012_vdw_epi_MoS2_gr,Liu_2016_epi_MoS2_gr_rotation,Lin_2014_vdW_solid,Lin_2015_Wse2_MoS2_gr,Azizi_2015_Freevdw_Gr_TMDCs,Kim_2016_BiSnTe_gr},
TMDC/hBN
\cite{Yan_2015_MoS2_on_hBN,Wang_2015_cvd_MoS2_BN,Cattelan_2015_Ws2_hBN},
graphene/hBN
\cite{Liu_2011_gr_hBN,Zhang_2015_gr_hBN,Driver_2016_MBE_gr_hBN}, and
TMDC/TMDC
\cite{Zhang_2014_vdw_epi_SnS2_MoS2,Diaz_2015_MoTe2_MoSe2,Gong_2014_WS2_MoS2,Alemayehu_2015_TMDC_vdw}.
%
The vdWE has also been used to grow 3D heterostructures on mono- or
multilayer 2D materials interfaces, including inorganic insulators like Al\(_{\text{2}}\)O\(_{\text{3}}\)
\cite{Zhang_2014_Al2O3_ALO_Gr,Vaziri_2013_ALD_Al2O3_gr}, and
HfO\(_{\text{2}}\) \cite{Alaboson_2011_PTCDA_gr_ALD},
%
and semiconductors including TiO\(_{\text{2}}\)
\cite{Li_2015_TiO2_GO,Kumar_2011_gr_TiO2_generator,Zhang_2011_TiO2_gr},
ZnO \cite{Chung_2010_GaN_ZnO_gr,Oh_2014_ZnO_hBN}, GaN
\cite{Kobayashi_2012_GaN_hBN,Kim_2014_direct_vdw_GaN_gr,Kim_2017_remote_epi_Gr},
GaAs \cite{Alaskar_2015_GaAs_gr_Si_theor,Kim_2017_remote_epi_Gr}
\worktodo{add Kang2018}, and CdS / CdTe
\cite{Loeher_1994_vdw_epi_CdS_MoTe,Loeher_1996_CdTe_MoWTe}.

Apart from the vdWE approach, covalently bonded structures can also be
formed by on-surface chemical reactions and metal coordination
bonds. Examples of such growth approach include two-dimensional covalent organic frameworks (2D COFs) formed by linking monomers by boron ester
or imine groups
\cite{Colson_2014_2D_COF_gr,Colson_2011_2DMOF_gr,Sun_2017_cof_gr}, and metal-organic frameworks (MOFs) \cite{Urgel_2015_MOF_BN,Kumar_2014_2D_MOF_gr} on weakly interacting or
functionalized 2D materials. The planar sp$^{2}$-type bonds
such as boron ester, imine and square planar metal coordination are
generally required for the formation of stable 2D epitaxial structure.

\subsubsection{Molecule-2D Material Interactions}
\label{sec:org0047f8d}

The interactions between the interfacial molecules and 2D material
determine the molecular packing and arrangement of the first few
overlayers. In addition, the interactions also have great impact on
the molecular adsorption process, thereby influencing the
heterogeneous nucleation characteristics. The
ratio between the intermolecular and molecule-2D material
interactions is the key factor in controlling the molecular epitaxial
structure. Here we categorize the molecule-2D material interactions
into weak (dispersion and electrostatic), charge-transfer
interactions, site-specific adsorption, and covalent bond formation.

\paragraph{Weak Interactions}
\label{sec:org68af064}

The weak molecule-2D material interactions involve the short-range
dispersion (vdW) and long-range electrostatic (Coulombic)
interactions. In the case of graphene, the delocalized π-electrons are
the basis for the non-covalent interactions. A large variety of planar
aromatic molecules, including PTCDA, PTCDI, C\(_{\text{60}}\), MPc are
shown to assemble on graphene with their aromatic rings parallel to
the 2D plane, in order to lower the adsorption energy by maximizing
the π-π interaction \cite{Grimme_2008_pipi,Zhang_2011_rev_pipi_gr}, a
phenomenon widely known as the graphene template effect
\cite{Yang_2015_rev_template}. MPc molecules (e.g. M=Cu, Fe, Co and
AlCl) and substituted MPc (e.g. F\(_{\text{16}}\)CuPc) tend
to form a ``face-on'' orientation on graphene interface, relative to
the ``edge-on'' orientation that are usually found on the deposition
of these molecules on amorphous substrates such as SiO\(_{\text{2}}\)
or glass
\cite{Ying_Mao_2011_ge_clAlPc,Zhang_2011_FePc_gr,Hamalainen_2012_CoPc_gr_Ir,,Xiao_2013_jacs_CuPc_gr}.
Similarly, the graphene template effect is also found  for PEN
\cite{Zhou_2013_penta_gr_Ru,Lee_2011_pentacene,Lee_2011_pentacene,Zhang_2015_gr_pent_orient},
C\(_{\text{60}}\) \cite{Kim_2015_c60_gr,Shih_2015_PartiallyScreened},
p-sexiphenyl (6P) \cite{Hlawacek_2011_6P_gr}, and
dibenzotetrathienocoronene (DBTTC) \cite{Kim_2016_DBTTC_gr} molecules,
revealing a general mechanism behind their assembly behavior.

Apart from graphene, the weak interactions on hBN and
MoS\(_{\text{2}}\) surfaces are also studied. The π-electron cloud of
hBN resembles that of graphene, causing the 6P molecules to form a
``face on'' configuration \cite{Matkovic_2016_6P_hBN} similar to the
case on graphene. However, non-planar molecules such as rubrene
\cite{Lee_2014_rubene_hBN} adapt the ``edge-on'' configuration over the ``face-on'' configuration, reflecting the fact
that the molecule-hBN interaction is weakly dispersive.
%
On the other hand, the molecular
interactions on MoS$_{2}$ are usually much weaker compared with that on
graphene due to its large dipole moment \cite{Rajan_2016_wett_mos2},
and is highly dependent on the lattice symmetry
\cite{Shen_2017_DFT_mos2_pent} (i.e. 1T- or 2H- phase) and surface
defects \cite{Jariwala_2016_Mos2_pentacene,
  Kim_2016_trap_Mos2_pent}.

\paragraph{Charge-Transfer Interaction}
\label{sec:orgebfad7b}

The charge-transfer (CT) interactions, or the donor-acceptor (DA)
interactions, refer to the process that electrons undergo
redistribution between the epitaxial molecules and the underlying 2D
material. Due to the locally enhanced carrier density in the formed CT
complex, the CT interactions tend to be stronger than the dispersion
and electrostatic interactions. The formation of a CT heterostructure
requires alignment of the energy levels between the 2D material and
the overlayer molecules \cite{Akiyoshi_2015_DA}, and may also change
the electronic structure of the 2D material through non-covalent
interactions
\cite{Cai_2015_doping_2D_rev,Wehling_2008_doping,Zhang_2011_rev_pipi_gr}.
TCNQ and its fluorinated derivative
2,3,5,6-Tetra\-fluoro-7,7,8,8-tetra\-cyanoquino\-dimethane (FTCNQ) are
known to form CT complexes with graphene
\cite{Chen_2007_TCNQ_gr,Voggu_2008_TCNQ,Barja_2010_TCNQ_gr}, with a
degree of charge transfer of $\sim{}$0.3 \textit{e} and $\sim{}$0.4
\textit{e}, respectively. With a stronger CT effect, FTCNQ molecules
on epitaxial graphene tend to be trapped by local
corrugation\cite{Barja_2010_TCNQ_gr}, compared with closed-packed
TCNQ/graphene assembly.  Since CT may occur when the HOMO and LUMO
energy levels of the epitaxial molecule and 2D material match, it is
also expected to play a role in the molecular epitaxy on 2D
semiconductors, such as TMDCs. Density functional theory (DFT) studies
reveal that PEN adsorbed on 1T-type monolayer MoS\(_{\text{2}}\) has a
large degree of CT ranging from 0.44-0.87 \emph{e}, and can change the
Fermi energy level of MoS\(_{\text{2}}\) by up to 1 eV
\cite{Shen_2017_DFT_mos2_pent}. Similarly, the interface between
C\(_{\text{60}}\) and MoS\(_{\text{2}}\) is found to be a pn-junction,
with charge depleted at the bottom of the C\(_{\text{60}}\) and
accumulated at the interface \cite{Chen_2016_c60_mos2}. On the other
hand, the tendency of forming CT-induced orientation is attenuated on
bulk MoS\(_{\text{2}}\) crystal \cite{Sakurai_1991_c60_mos2}, due to
an increase of the DOS compared in bulk crystals. Theoretical studies
also disclose strong CT between phosphorene and electron-donating
tetrathiafulvalene (TTF), as well as electron-accepting TCNQ molecules
\cite{Zhang_2015_DA_phosphorene}.


\paragraph{Site-Specific Adsorption}
\label{sec:org87b0c12}

The electronic and geometric properties of a 2D material are known to be
influenced by its underlying substrate. When there is a lattice
mismatch between the 2D material and the substrate, a long-range
periodic superposition known as moiré pattern forms, as has been
found graphene/metal \cite{Hamalainen_2013_moire_gr} and hBN/metal
\cite{Schulz_2014_hBN_moire} systems.  The
moiré pattern does not only cause a geometric interference, but
indeed changes the local electronic state and structure of the 2D
material.
%
The height variation within the graphene or hBN layer can be used to
quantify the degree of metal-2D material interaction strength, to
distinguish weakly interacting surfaces include graphene/Ir(111)
\cite{Pletikosi_2009_gr_Ir,Busse_2011_Gr_Ir,Hamalainen_2013_moire_gr},
graphene/Pt(111) \cite{Sutter_2009_Gr_Pt}, hBN/Ir(111)
\cite{Schulz_2014_hBN_moire}, hBN/Pt(111) \cite{Cavar_2008_hBN_Pt},
hBN/Cu(111) \cite{Joshi_2012_hBN_Cu} systems, in which the average 2D
material-metal distance is comparable with that in the bulk material
(3.3$\sim{}$3.4 \AA{}) and the corrugation in the 2D layer is
typically small (<0.5 \AA{}). The strongly interacting surfaces
including graphene/Ru(0001) \cite{Moritz_2010_gr_Ru} \worktodo{sutter
  2}, graphene/Rh(111) \cite{Wang_2010_gr_Rh}, hBN/Ru(0001)
\cite{Wang_2010_gr_Rh}, and hBN/Rh(111) \cite{Dil_2008_hBN_Rh}
systems, with structural corrugations as large as 1 \AA{}, and the
electronic fluctuation up to 0.5 eV. In the strongly interacting
systems, the moiré pattern creates a local difference in the
adsorption potential, which in turn results in site-specific
adsorption of small molecules on these surfaces. Such behavior has
been observed in a variety of organic semiconductor molecules
deposited on the graphene/Ru(0001) surface, including MPc (M=Fe, Ni,
Zn, Mn) \cite{Mao_2009_Pc_gr_kagome,Zhang_2011_FePc_gr}, pentacene
\cite{Zhou_2013_penta_gr_Ru}, C\(_{\text{60}}\)
\cite{Li_2012_c60_gr_Ru}, PTCDA \cite{Zhou_2011_PTCDA_gr_Ru}, TCNQ
\cite{Maccariello_2014_TCNQ_gr_Ru}, with similar behavior has also
been found on the surface of hBN/Ru(0001) for MPc (M=H\(_{\text{2}}\),
Cu, Co) \cite{Dil_2008_hBN_Rh,Jarvinen_2014_MPc_hBN_Ru}, TCNQ
\cite{Joshi_2014_TCNQ_hBN}, and C\(_{\text{60}}\)
\cite{Corso_2004_C60_hBN}. The site-specific adsorption usually lead
to ordered sub-2D assembly, composed of the molecules trapped at the
specific sites, compared with the close-packed assembly on flat and
weakly interacting interfaces.

% Recently, more experimental and theoretical studies have also
% demonstrated the moiré pattern formation on TMDC/metal
% \cite{Chen_2013_doping,Sorensen_2014,Le_2012_MoS2_Cu}, TMDC/TMDC
% \cite{Kang_2013_TMDC_moire,Zhang_2014_vdw_epi_SnS2_MoS2,Diaz_2015_MoTe2_MoSe2,Fang_2014_intercoupl_vdW,Li_2016_GaSe_MoSe2_vdW},
% and TMDC/hBN \cite{Fang_2014_intercoupl_vdW} surfaces. Following the
% discussion of the strongly interacting surface of graphene/Ru(0001),
% it is believed that the moiré pattern formed between the strongly
% coupled layers, e.g. TMDC/Ru(0001) \cite{Chen_2013_doping} and TMDC/TMDC
% \cite{Fang_2014_intercoupl_vdW} heterostructures may also lead to the
% site-specific adsorption phenomenon \cite{Diaz_2015_MoTe2_MoSe2}, in
% contrast to the close-packing structure formed on the weakly
% interacting surfaces, as discussed in the previous section.


\paragraph{Covalent Bond}
\label{sec:org6f342a5}

Covalent bonds formed perpendicular to the 2D material plane open an
opportunity for functionalizing 2D materials and provide anchor sites
for modification. However compared with the epitaxy approaches,
chemical modification of 2D material is limited by the choice of
chemical reactions available. Moreover, opening up 2D basal structure
usually destroyed by the geometric change of the molecular orbital
(e.g. planar sp\(^{\text{2}}\) to tetrahedral sp\(^{\text{3}}\) in
graphene). Nevertheless, there are still a few examples showing the
potential of covalent binding and tuning the electronic properties of
the 2D materials
\cite{Georgakilas_2012_noncoval_gr_rev,Lee_2011_tempo_gr,Zhang_2013_janus_gr,Voiry_2014_cov_TMDC_phase,Vishnoi_2016_ar_mos2_covalent,Liu_2011_rev_chem_dope_gr,Wang_2012_ar_gr_react_rate}.
%
The chemical grafting of graphene mainly involves free-radical
reaction
\cite{Lee_2011_tempo_gr,Choi_2010_aminotempo_gr,Zhang_2013_janus_gr,Wang_2012_ar_gr_react_rate,Kumar_2014_2D_MOF_gr},
%
with the potential to fabricate asymmetric Janus-type functionalized
graphene by the covalent modification on both sides of a free-standing
graphene sheet \cite{Zhang_2013_janus_gr}. The low DOS in a 2D
materials further makes it possible to fine-tune the interfacial
chemical reaction rate by the doping density of 2D materials, for
instance through the substrate doping of graphene
\cite{Wang_2012_ar_gr_react_rate}. Several approaches have also show
the possibility of functionalizing other 2D materials, including
nucleophilic substitution between anionized TMDCs and organohalides
\cite{Vishnoi_2016_ar_mos2_covalent} and aryl diazonium
salts. \worktodo{cite these} The chemical modifications are also
frequently used to improve quality of 2D semiconductors.
%
Diazonium modification of BP significantly increases its ambient
stability over several weeks. \worktodo{cite nat chem 2016 8
  597}. Moreover, treating MoS\textsubscript{2} with organic
super\-acids moves its Fermi level towards intrinsic semiconductor,
and significantly improves the photo\-luminescence (PL) quantum yield over two order of magnitudes. \worktodo{Science 2015 350 1065}
%
Future advance of covalently
modified 2D materials with site-specific and programmable chemical
functionalization may combine the 2D with the 3D materials in a
controllable manner.

\subsubsection{Molecule-Substrate Interaction}
\label{sec:org6660f0f}

One of the major differences of the interfacial molecules on 2D
materials compared with bulk materials interfaces is significant
influence from the underlying substrate. Note that this phenomenon is
distinguished from the effect of strongly interacting surface or
substrate doping, with the latter two referring to the change of 2D
material's electronic and geometric properties, which then influence
the molecule-2D material interactions. The penetration of the
molecule-substrate interactions through monolayer 2D material is first
observed in the experiments of wettability of substrate-supported
graphene: the water contact angle of water on graphene is found to be
influenced by the vdW force between the water molecules and the
substrate, known as the wetting ``transparency'' or ``translucency''
of graphene
\cite{rafiee_wetting_2012,shih_breakdown_2012,shih_wetting_2013}. The
transparency can be even pronounced for electrostatic interactions,
which has longer length scale than the vdW force
\cite{Shih_2015_PartiallyScreened,Tian_2016_multiscale}.  The
influence of the molecule-substrate interactions through a 2D
materials usually can only be examined indirectly.
%
Examples include layer-number-dependent morphology of
6P\cite{Kratzer_2014_6P_gr_layer} and
PEN\cite{Chhikara_2014_gr_pent_trans} molecules deposited on
SiO\(_{\text{2}}\)-supported graphene layers. Moreover, the influence
of underlying substrate is also found for PEN when deposited on
graphene supported by substrates with varied surface energy \cite{Nguyen_2015_pent_gr_wett} or
electrostatic gating. \worktodo{cite Nguyen paper 2}
%
Recently the concept of vdW transparency has also been
employed in the remote vdWE of III-V semiconductors on graphene supported by highly-crystalline III-V substrate
\cite{Kim_2017_remote_epi_Gr}. \worktodo{cite Kang 2018 paper 2}
%
The interactions from underlying crystalline III-V semiconductor is
shown to direct the growth of III-V semiconductor on the graphene
interface despite the $\sim{}$ 1 nm gap created by graphene. The
strength of such remote interactions are also shown to be dependent on
the polarity of the underlying material. \worktodo{cite Kang 2018 paper 2}
%
In addition to the vdW and Coulombic interactions,
graphene layer is also found to be transparent to the charge transfer
process \cite{Jeong_2015_DA_transparency_gr} when the reduction rate of
AuCl\(_{\text{4}}^{\text{-}}\) on graphene surface are found to be
faster when graphene is coated on a reductive surface, such as Al, Ge
and Cu surfaces. 

% 20191008-1253

\subsection{The Variety of Mixed-Dimensional Interfaces}
\label{sec:vari-mixed-dimens}




\section{Challenging Problems Concerning 2D Interfaces}
\label{sec:chall-probl-conc}

\subsection{Electrostatic Interactions Through 2D Sheet}
\label{sec:electr-inter-thro}

\section{Dielectric Properties of 2D Systems}
\label{sec:diel-prop-2d}

\section{van der Waals (vdW) Interactions and Wetting Phenomena}
\label{sec:van-der-waals}



%%% Local Variables:
%%% mode: latex
%%% TeX-master: "../thesis"
%%% End:
