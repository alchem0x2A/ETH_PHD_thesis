% \chapter{Introduction}
\chapter{Two-Dimensional Materials and Interfaces}
\label{ch:introduction}
\newcommand*\imgdir{img/ch-intro/}

\dictum[Wolfgang Pauli]{%
  God made the bulk;\\Surfaces were invented by the devil
  }%

\vspace{1em}

\chapterabstract{Part of this chapter appears in the following
  journal article: Tian, T. \& Shih, C.-J. Molecular Epitaxy on
  Two-Dimensional Materials: The Interplay between
  Interactions. Ind. Eng. Chem. Res. 56, 10552--10581 (2017).  }


\section{Overview of Two-Dimensional (2D) Materials}
% \section{Introduction}
\label{sec:ch-intro-2D}
Controlling the dimensionality of materials provides rich
opportunities of tuning the electronic, optical and mechanical
properties which facilitates novel devices and applications.
\worktodo{put more things inside and make claim clear} \worktodo{put
  1-3 refs}. Good example of such material design are the
two-dimensional (2D) materials, which are covalently-bonded
crystalline films with only one- or few-atom thickness. \worktodo{HH??
  Change wording}
%
Starting from the first discovering of graphene, an allotrope of
carbon in 2004\worktodo{cite Geim}, the family of 2D materials has
expanded significantly throughout the last decade. \worktodo{cite
  Novoselov and 2 papers} The electronic properties of 2D materials
cover a wide spectrum ranging from insulators to superconductors,
making them promising candidates to replace conventional bulk
materials in semiconductor industries. \worktodo{cite 1-2}.
%
This section aims to give a brief introduction about the structures,
electronic properties and fabrication techniques, which serves as a
prelude for introducing the 2D materials interfaces.
\worktodo{again, change last part}

% Understanding and engineering the
% interfaces of 2D materials are important to achieve practical uses of
% 2D materials in our 3D world. In this chapter, we briefly discuss the
% fundamental aspects of 2D materials and introduce their interfacial
% phenomena. Several challenges about the 2D materials interfaces are
% also discussed, serving as the prelude to the work of this thesis.


\subsection{Categories of 2D Materials}
\label{sec:categ-2d-mater}

So far, there have been no naturally-existing isolated 2D
materials. The closest form is the layered bulk materials, in which
interlayer interactions are governed by van der Waals (vdW) forces.
%
With proper separation techniques, these layered materials can be
thinned to a few layers or even single layer.
%
The best-known example is the mechanical exfoliation of 2D layer from graphite,
now known as graphene. \worktodo{cite geim}
%
Using such technique, monolayers exfoliated from naturally-existing
layered bulk materials were achievable. Examples include the hexagonal
boron nitride (hBN), transition metal dichalcogenides (TMDCs, with
general chemical composition MX\textsubscript{2}, where M is a
transition metal and X is S, Se or Te), monochalcogenides (GaS, GaSe),
monolayer black phosophorus (BP) and MXene (M = Ti, Nb, V, Ta, etc.,
X = C or N).
%
The family of 2D materials that have been experimentally discovered,
has extended from only 1 in 2004 to over 100 \worktodo{really?} as of
2019, corresponding to an average discovery of about 6 new materials
each year. \worktodo{correct?}
%
Apparently, the known 2D materials are still scarce and limited
compared with the bulk materials,
due to the huge experimental effort
required to synthesize, isolate and characterize 2D materials limit
the speed of novel 2D material discovery.
% 
In view of this, computer-aided material design (CAMD) was
demonstrated in several works to help find new 2D materials either
from high-throughput materials database screening or \textit{ab
  initio} calculations. As a pioneer study, Lebègue et al. performed
data mining on the crystal structures listed in the International
Crystallographic Structural Database (ICSD) to filter out a total
number of 92 layered bulk crystals and to generate corresponding 2D
materials. As an extension, Mounet et al. proposed new algorithm with
more robust criteria for dimensionality and found over 5000 layered
materials, among which over 1800 can be easily or potentially
exfoliated according to \textit{ab initio} calculations.
%
Another example is the combinational design of new 2D materials, which
starts with a known structure (for example MX\textsubscript{2}) and
replace with other elements. This allows the discovery of over 3700
thermo-dynamically stable 2D materials, among which most chemical
compositions are not yet discovered experimentally.

These computational studies broaden our knowledge about 2D
materials. In particular, the large size of database allows
systematical study to search for optimal 2D material according to
their electronic and optical properties, and apply to real-world
applications. As a summary, the most common types of 2D materials from
either experimental or computational discoveries, are listed in
\autoref{tab:category-2D}.

\begin{table}
  \centering
  \caption{Summary of common 2D materials and their structures. The
    formula refer to the chemical composition per unit cell.}
  \label{tab:category-2D}
  \begin{tabularx}{1.05\linewidth}{XXXX}
    \hline
    Prototype  & Example Formula  & Lattice Symmetry & Example Materials \\
    \hline
    Graphene & C\textsubscript{2} &  P6/mmm & Graphene, Silicene, Germanene \\
    Graphane & C\textsubscript{2}H\textsubscript{2} &  P3m1 & Graphane, Fluorographene\\
    hBN      & BN                & P$\overline{3}$m2 & h-BN \\
    2H-MX\textsubscript{2} & MoS\textsubscript{2} & P$\overline{3}$m2 & 2H-MoS\textsubscript{2}, 2H-MoS2\textsubscript{2}, 2H-WS\textsubscript{2} \\
    1T-MX\textsubscript{2} & CdI\textsubscript{2} & P$\overline{3}$m1 & 2H-MoS\textsubscript{2}, CdI\textsubscript{2}\\
    BP & P\textsubscript{4} & Pmna & Phosphorene, Arsenene \\
    Monochalcogenidec & Ga\textsubscript{2}S\textsubscript{2} & P$\overline{3}$m2 & Ga\textsubscript{2}S\textsubscript{2}, Ga\textsubscript{2}Se\textsubscript{2} \\
    Bismuth iodide &  Bi\textsubscript{2}I\textsubscript{6} & P$\overline{3}$m1 & Bi\textsubscript{2}I\textsubscript{6}, Al\textsubscript{2}Cl\textsubscript{6} \\
    Iron oxychloride                &  Fe\textsubscript{2}O\textsubscript{2}Cl\textsubscript{2} & Pmmn & Fe\textsubscript{2}O\textsubscript{2}Cl\textsubscript{2}, Fe\textsubscript{2}O\textsubscript{2}Br\textsubscript{2}  \\
    MXene & Ti\textsubscript{x}C\textsubscript{y} & P$\overline{3}$m2 & Ti\textsubscript{3}C\textsubscript{2}, Ti\textsubscript{4}N\textsubscript{4}, Mo\textsubscript{2}TiC\textsubscript{2} \\
    2D Perovskite & CH\textsubscript{3}NH\textsubscript{3}PbBr\textsubscript{3} & Pm$\overline{3}$m & (MAPbBr\textsubscript{3})\textsubscript{n}\\
   \hline
\end{tabularx}
\end{table}

\subsection{Basic Electronic Properties of 2D Materials}
\label{sec:basic-electr-prop}

The common feature of 2D materials is the absence of dangling bonds at
the surface, as a results, the electrons are highly confined within
the 2D plane.
%
For instance, in graphene and hBN, the sp\textsuperscript{2} orbitals
form covalent (σ) bonds, while the p-orbitals perpendicular to the 2D
plane form delocalized π-electron cloud.
%
Such two-dimensional
electron gas (2DEG) gives rise to dramatic change of its electronic
properties compared with bulk materials.\worktodo{cite Devies}
%
Although the physics about 2DEG has been well developed in the 1980s,
prior to the discovery of 2D materials, the 2DEG can only be achieved
by cumbersome semiconductor quantum well heterostructures. In other
words, the 2D materials are perfect candidates to study the behavior
of 2DEGs. \worktodo{say something more?}

Although the electronic band structures of different 2D materials vary
a lot, one thing in common is that their density of states (DOS). The
DOS is a measure of the number of available states in a certain system
at certain energy level. For an extended system with nearly-continuous
energy distribution, the DOS at energy $E$ is expressed as:
\begin{equation}
  \label{eq:ch-intro-dos}
  \mathrm{DOS}(E) = \frac{\partial N}{\Omega \partial E} =  \frac{1}{\Omega} {\displaystyle \int_{\Omega}} \frac{\mathrm{d}^{n} \mathbf{k}}{(2 \pi)^{n}}
  \delta(E - E(\mathbf{k}))
\end{equation}
where the integral is performed over a system with $d$-dimensionality,
$N$ is the total number of states,
$\Omega$ is the volume of the system, $\mathbf{k}$ is the momentum of
electron, and $\delta$ is the Dirac delta function. Without losing
generality, the DOS can be expressed using the law of chains:
\begin{equation}
  \label{eq:dos-chain}
  \mathrm{DOS} = \frac{\partial N}{\Omega \partial k} \frac{\partial k}{\partial E}
               = \frac{k}{2 \pi} \left(\frac{\partial E(k)}{\partial k}\right)^{-1}
\end{equation}
where $k=|\mathbf{k}|$ is the modulus of $\mathbf{k}$.
%
\autoref{eq:dos-chain} provides a general way linking the DOS to the
energy--momentum ($E-k$) dispersion of a 2D material, which is further
extracted from its band structure. For monolayer 2D materials, two
cases can be distinguished:
\begin{itemize}
\item Parabolic materials: such as TMDCs, hBN, phosphorene

  These are the majority of 2D materials where the $E-k$ dispersion is
  parabolic, such that
  ${\displaystyle E(k) = \frac{\hbar^{2} k^{2}}{2 m^{*}}}$, where
  $\hbar$ is the reduced Planck constant, and $m^{*}$ is the effective
  mass near the band edge. From \autoref{eq:dos-chain}, DOS of a
  parabolic material is \textit{constant}, and proportional
  to $m^{*}$.
  
\item Dirac materials: such as graphene, silicene, germanene

  The Dirac materials are relatively rare among 2D materials,
  \worktodo{cite Wang nsr 2015} where the $E-k$ dispersion is linear:
  $E(k) = \hbar v_{\mathrm{F}}k$, where $v_{\mathrm{F}}$ is the Fermi
  velocity. The Dirac materials has DOS increasing linearly with $k$
  (as well a $E$).
\end{itemize}

The difference between the DOS of ideally parabolic and Dirac
materials can be seen in \worktodo{Figure here}. Unlike parabolic
materials with constant DOS, the Dirac materials have DOS → 0 near the
Dirac cone ($E \to 0$). This feature brings very interesting
properties like electrostatic transparency which we will discuss in
\worktodo{Chapter 2}.

\worktodo{Discuss more about the parabolic and Dirac cone?!}
\worktodo{Some simple discussion based on the band gap etc...}


\subsection{Fabrication of 2D Materials}
\label{sec:fabr-2d-mater}

The development of the 2D material researches cannot be achieved
without appropriate methods to fabricate large-area and high quality
2D materials. The synthesis of 2D materials can be generally
categorized into top-down and bottom-up approaches.

\subsubsection{Top-down Methods}
\label{sec:top-down-methods}

The top-down method is the straightforward approach to exfoliate few-
to single-layer 2D materials from their bulk counterparts. The two
most-used techniques are micro-mechanical exfoliation and liquid-phase
exfoliation. \worktodo{add cites here, from Liu Adv Mater 2018}

The micro-mechanical exfoliation uses mechanical force to overcome the
interlayer vdW interactions in bulk layered materials, and is the
first method used for 2D material exfoliation. Scotch tape is widely
used for such type of exfoliation \worktodo{cite Geim et al}, while
other media as elastic polymers, heat-release tapes are also
employed. If the quality of the bulk layered material is promising
(\ie high chemical purity and low defect density), micro-mechanical
exfoliation usually produces 2D materials with better quality compared
with other methods. However, there are also several key drawbacks of
such methods. First of all, the reliability of micro-mechanical
exfoliation is highly influenced by the interlayer forces, which makes
it difficult to exfoliate materials with high interlayer binding
energy, or in-plane mechanical anisotropy (for instance BP). This
method also leads to broad distribution of flake size and layer
number, making it difficult to be applied to large-scale
applications.

In contrast to micro-mechanical exfoliation where only the top-most
layers are removed, the solution-phase exfoliation method ensures
uniform breaking up of bulk materials, and increases the scalability
of 2D flakes produced. It can be achieved by both physical and
chemical exfoliation approaches. Physical solution-phase exfoliation
makes use of local mechanical stress produced by ultra-sonication or
shearing to overcoming the interlayer vdW interactions. To ensure the
stability of isolated layers in the liquid suspension, high surface
energy liquid, in combination with surfactants are usually
used. \worktodo{cite papers from Coleman et al, Shih et al. Haung et
  al}
%
Centrifugation can be further used to separate flakes to ensure narrow
distribution of layer thickness and size \worktodo{cite coleman}, and
statistic measurement can be performed using the ensemble of such 2D
material suspensions. \worktodo{wording?} However, this method still
suffers from several drawbacks: (i) the flake size is still limited to
μm-scale, (ii) precise control of layer number in suspension is difficult and 
(ii) flake overlay after deposition onto substrate is unavoidable.
%
The interlayer vdW forces can also be overcome by chemically modifying
the 2D layer (for instance, oxidising graphite to produce graphene
oxide (GOx)), or to intercalate ions between the layers in a bulk
material to induce lattice expansion (such as exfoliation of MXenes,
which are otherwise hard to achieve mechanically). Despite the
scalability of chemical solution-phase approaches, they usually
changes the chemical composition and introduce defects in the 2D
materials, which are undesired for high-performance
applications. 

\subsubsection{Bottom-up Methods}
\label{sec:bottom-up-methods}

The bottom-up methods grow 2D materials from precursors, and aims to
produce 2D materials with larger flake size and more controlled layer
numbers. Depending on whether a substrate is involved in the process,
these methods can be categorized into templated or non-templated
growth.

Templated growth uses a bulk surface as a epitaxial template and
support for the 2D material. Due to the absence of dangling bonds, the
2D-substrate interaction is usually much weaker than the in-plane
covalent bonds. Therefore, unlike epitaxy of bulk materials which
require precise control of lattice mismatch between the substrate and
epitaxy layer \worktodo{cite 1-2 paper}, templated growth of 2D
materials can be achieved in various substrates. For example, high
quality single crystal graphene can be epitaxially grown by thermal
annealing of silicon carbide (SiC) surface, or decompositing
hydrocarbons on Ruthenium (Ru) (0001) and Ir (111) substrates, while
epitaxial growth of hBN is achieved by cleavage of borazine
(H\textsubscript{6}B\textsubscript{3}N\textsubscript{3}) on Rh (111)
surface.  Such epitaxial method can also be applied to other 2D
materials like TMDC, monochalcogenides and BP. However, there are
still several limitations. First of all, a clean surface as well as
ultra-high vacuum (UHV) conditions are usually required for the
epitaxial growth methods. Moreover, transferring the 2D material onto
other substrates is generally not easy due to the noble metals
involved. Chemical vapor deposition (CVD) is another widely-used
templated growth method, in which one or more precursors adsorb and
react on a catalytic surface to form covalently-bonded 2D
materials. The essence of CVD process is similar to the epitaxial
growth, while more ambient conditions are used (10$^{1}$ Pa to
atmosphere pressure, ultra clean substrate not necessary). CVD growth
of graphene using hydrocarbon source on copper (Cu) is the most
studied and widely used technique. The self-termination of
second-layer on the Cu surface allows the growth of large area single
layer domains up to centimeter or decimeter scale \worktodo{check if
  explanation is correct}. Another advantage of such method is easy
removal of Cu substrate by standard etching procedure, allowing
transferring graphene onto a large variety of
substrates. \worktodo{cite 1-2} Similar to the case of graphene, large
area single-layer TMDCs and hBN can also be achieved using the CVD technique by proper interfacial engineering. \worktodo{cite TMDC paper; hBN show the Gold sample}
%
% With proper defect control during growth and development of novel
% transferring techniques, the CVD method is promising to 

While the majority of bottom-up growth relies on a substrate to form
the 2D material, there are also cases where colloidal 2D confined
structures can be directed synthesized in liquid phase without
template. Examples of 2D materials grown using the colloidal method
include II-VI semiconductors (CdSe), 2D hybrid perovskite, and
TMDCs. The anisotropic growth is usually modulated by surfactant /
ligand engineering. Recently, colloidal insulating 2D metal oxides are
reported to be synthesized by simultaneous oxidation at the liquid
metal-water interface, further extending the possibility of bottom-up
synthesis of 2D materials. \worktodo{cite Dicky paper}

As a summary, both top-down and bottom-up methods are capable of
fabricating 2D materials with desired purity, flake size and
thickness. A short comparison between different fabrication methods can be seen in \worktodo{table 2?!}




\section{The 2D Materials Interfaces}
\label{sec:2d-mater-interf}

\subsection{The Variety of Mixed-Dimensional Interfaces}
\label{sec:vari-mixed-dimens}

\subsection{Interactions and Forces at the 2D Materials Interfaces}
\label{sec:inter-forc-at}

\section{Challenging Problems Concerning 2D Interfaces}
\label{sec:chall-probl-conc}

\subsection{Electrostatic Interactions Through 2D Sheet}
\label{sec:electr-inter-thro}

\section{Dielectric Properties of 2D Systems}
\label{sec:diel-prop-2d}

\section{van der Waals (vdW) Interactions and Wetting Phenomena}
\label{sec:van-der-waals}



%%% Local Variables:
%%% mode: latex
%%% TeX-master: "../thesis"
%%% End:
