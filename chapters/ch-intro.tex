% \chapter{Introduction}
\chapter{Two-Dimensional Materials and Interfaces}
\label{ch:introduction}
\renewcommand*\imgdir{img/intro/}

\dictum[Wolfgang Pauli]{%
  God made the bulk;\\Surfaces were invented by the devil
  }%

\vspace{1em}

\chapterabstract{Part of this chapter appears in the following
  journal article: Tian, T. \& Shih, C.-J. Molecular Epitaxy on
  Two-Dimensional Materials: The Interplay between
  Interactions. Ind. Eng. Chem. Res. 56, 10552--10581 (2017).  }


\section{Overview of Two-Dimensional (2D) Materials}
% \section{Introduction}
\label{sec:ch-intro-2D}
Controlling the dimensionality of materials provides rich
opportunities of tuning the electronic, optical and mechanical
properties which facilitates novel devices and applications~\cite{Davies_1997_book,Ihn_2009_book}.
%
The two-dimensional (2D)
materials~\nocite{Novoselov_2012_roadmap,Mas_Ballest_2011_review,Bhimanapati_2015_2D_rev,Butler_2013_review,Novoselov_2016_vdW,Mannix_2017}
are good examples of such material design, which are
covalently-bonded crystalline films with only one- or few-atom
thickness.
%
Starting from the first discovery of
graphene~\cite{Novoselov_2004_gr}, an allotrope of carbon, the family
of 2D materials has expanded significantly throughout the last
decade~\cite{Butler_2013_review,Das_2015_beyond_gr,Novoselov_2016_vdW}.
%
The electronic properties of 2D materials cover a wide spectrum
ranging from insulators to superconductors, making them promising
candidates to replace conventional bulk materials in semiconductor
industries~\cite{Xia_2014_2D_nanophoto_rev,Bhimanapati_2015_2D_rev}.
%
This section aims to give a brief introduction about the structures,
electronic properties and fabrication techniques of 2D materials, as a
prelude for introducing the 2D material interfaces in
\autoref{sec:2d-mater-interf}.

% Understanding and engineering the
% interfaces of 2D materials are important to achieve practical uses of
% 2D materials in our 3D world. In this chapter, we briefly discuss the
% fundamental aspects of 2D materials and introduce their interfacial
% phenomena. Several challenges about the 2D materials interfaces are
% also discussed, serving as the prelude to the work of  this thesis.


\subsection{Categories of 2D Materials}
\label{sec:categ-2d-mater}

So far, no naturally-existing isolated 2D
materials have been discovered.
%
The closest form is the layered bulk materials, in which the 2D layer
are coupled by non-covalent van der Waals (vdW)
forces~\cite{Israelachvili_2011_book}.
%
With proper separation techniques, these layered materials can be
thinned to few- or even mono\-layer sheets.
%
The most renowned example is the mechanical exfoliation of 2D layer from graphite,
now known as graphene~\cite{Novoselov_2004_gr}.
%
Using such technique, mono\-layer 2D materials exfoliated from naturally-existing
layered bulk materials are achievable, including the hexagonal boron
nitride\cite{Cavar_2008_hBN_Pt,Gorbachev_2011_BN_monolayer} (hBN), transition metal
dichalcogenides~\cite{Mak_2010_mos2} (TMDCs, with general chemical
composition MX\textsubscript{2}, where M is a transition metal and X
is S, Se or Te), monochalcogenides~\cite{Late_2012_GaS} (GaS, GaSe),
monolayer black phosphorus\cite{Li_2014_BP,Liu_2014_BP} (BP, also
known as phosphorene) and MXene~\cite{Naguib_2011_Mxene} (M = Ti, Nb,
V, Ta, etc., X = C or N).
%
The research interest in graphene and other 2D materials have been
steadily increasing over the past decade, in particular after the
Noble Prize awarding for graphene (~\autoref{fig:intro-papers}).
%
Moreover, the number of 2D materials that have been experimentally
identified has also significantly increased
\cite{Bhimanapati_2015_2D_rev,Mannix_2017}.

\begin{figure}[!htbp]
  \centering
  \import{\imgdir}{scholar.pgf}
  \caption{\label{fig:intro-papers} %
    Research trend of several 2D materials (measured by average number
    of publications per year) including graphene, hBN,
    MoS\textsubscript{2} and phosphorene.  An apparent increase in the
    growth of publication numbers can be ubiquitous seen for all 2D
    materials after the Nobel Prize awarding for graphene in 2010.
    Data are retrieved from Google scholar with interval of 1
    year. }
\end{figure}

%
However, comparing with the bulk materials, the examples of 2D
materials discovered are still scarce, due to the huge experimental
effort required to synthesize, isolate and characterize 2D materials
that limit the speed of novel 2D material discovery.
% 
In view of this, computer-aided material design (CAMD) was
demonstrated in several works to help find new 2D materials from
high-throughput materials database screening and \textit{ab initio}
calculations~\cite{Lebegue_2013_prx,Zhuang_2013_database,Haastrup_2018_database,Mounet_2018_database,Zhou_2019_database}.
%
As a pioneer study, Lebègue et al.~\cite{Lebegue_2013_prx} performed
data mining on the crystal structures listed in the International
Crystallographic Structural Database (ICSD) to filter out a total
number of 92 layered bulk crystals. As an extension, Mounet et al.~\cite{Mounet_2018_database}
proposed new algorithm with more robust criteria for dimensionality
and found over 5000 layered materials, among which over 1800 can be
easily or potentially exfoliated according to \textit{ab initio}
calculations.
%
Another example is the combinatorial design of new 2D materials.
Starting with a known structure (for example MX\textsubscript{2}) and
replace with other elements, Haastrup et
al.~\cite{Haastrup_2018_database} identified over 3700
thermo\-dynamically stable 2D materials, among which most chemical
compositions are not yet discovered experimentally.

These computational studies broaden our knowledge about 2D
materials. In particular, the large size of database allows
perform property-oriented material search, and apply to real-world
applications. As a summary, the most common types of 2D materials from
either experimental or computational discoveries, are listed in
\autoref{tab:category-2D}. For some materials like TMDCs, various
thermo\-dynamically stable phases can exist, including 2H- (hexagonal)
and 1T- (triagonal) symmetries.

\begin{table}[!htbp]
  \centering
  \caption{Summary of common 2D materials and their structures. The
    formula refer to the chemical composition per unit cell. The data
    in this table is adapted from \cite{Haastrup_2018_database}.}
  \label{tab:category-2D}
  \begin{tabularx}{1.00\textwidth}{XXXX}
    \hline
    Prototype  & Sample Formula  & Symmetry & Examples \\
    \hline
    Graphene & C\textsubscript{2} &  P6/mmm & Graphene, Silicene, Germanene \\
    Graphane & C\textsubscript{2}H\textsubscript{2} &  P$\overline{3}$m1 & Graphane, Fluoro\-graphene\\
    hBN      & BN                & P$\overline{6}$m2 & h-BN \\
    2H-MX\textsubscript{2} & MoS\textsubscript{2} & P$\overline{6}$m2 & 2H-MoS\textsubscript{2}, 2H-MoS2\textsubscript{2}, 2H-WS\textsubscript{2} \\
    1T-MX\textsubscript{2} & CdI\textsubscript{2} & P$\overline{3}$m1 & 1T-MoS\textsubscript{2}, CdI\textsubscript{2}\\
    BP & P\textsubscript{4} & Pmna & Phosphorene, Arsenene \\
    Mono\-chalcogenide & Ga\textsubscript{2}S\textsubscript{2} & P$\overline{6}$m2 & Ga\textsubscript{2}S\textsubscript{2}, Ga\textsubscript{2}Se\textsubscript{2} \\
    Bismuth iodide &  Bi\textsubscript{2}I\textsubscript{6} & P$\overline{3}$m1 & Bi\textsubscript{2}I\textsubscript{6}, Al\textsubscript{2}Cl\textsubscript{6} \\
    Iron oxychloride                &  Fe\textsubscript{2}O\textsubscript{2}Cl\textsubscript{2} & Pmmn & Fe\textsubscript{2}O\textsubscript{2}Cl\textsubscript{2}, Fe\textsubscript{2}O\textsubscript{2}Br\textsubscript{2}  \\
    MXene & Ti\textsubscript{x}C\textsubscript{y} & P$\overline{3}$m2 & Ti\textsubscript{3}C\textsubscript{2}, Ti\textsubscript{4}N\textsubscript{4}, Mo\textsubscript{2}TiC\textsubscript{2} \\
    2D Perovskite & CH\textsubscript{3}NH\textsubscript{3}PbBr\textsubscript{3} & Pm$\overline{3}$m & (MAPbBr\textsubscript{3})\textsubscript{n}\\
   \hline
\end{tabularx}
\end{table}

\subsection{Basic Electronic Properties of 2D Materials}
\label{sec:basic-electr-prop}

The large databases generated by both
experimental and theoretical studies cover a broad material spectrum,
ranging from insulator, semiconductor to even
superconductors~\cite{Novoselov_2016_vdW}.
%
The 3D geometries of four commonly-studied 2D materials (graphene,
phosphorene, 2H-MoS\textsubscript{2}, and hBN) and their electronic
band structure plots are shown in \autoref{fig:intro-bs}, distinguishing
zero-gap (graphene), semiconducting (phosphorene and
2H-MoS\textsubscript{2}) and wide-gap (hBN) materials.
%
Moreover, the valence band (VB) - conduction band (CB) transition can
be either direct (monolayer 2H-MoS\textsubscript{2} and hBN) or
indirect (GaS) ~\cite{Xia_2014_2D_nanophoto_rev}, giving rise to a
variety of optical properties.
%
Essentially, the rich selection of electronic and optical properties
makes 2D materials promising candidates for replacing conventional
bulk materials used in semiconductor industry (\eg silicon (Si),
gallium arsenide (GaAs), gallium nitride (GaN)).
%
In addition to the match of bandgap and optical transitions, a 2D
material exhibits various exceptional properties compared with their
bulk counterparts.
%
First, exotic electronic and optical features arise from the quantum
confinement perpendicular to the 2D plane.
%
Examples include the relativistic, mass-less carriers in the Dirac
cone electronic structure of graphene and other group 14 elemental 2D
materials~\cite{Novoselov_2005_massless,Zhang_2005_QHE,Das_Sarma_2011_electron_gr},
strong electron-photon
interaction~\nocite{Nair_2008_transparent,Eda_2013_rev_opt}, and large
exciton binding energy in 2D
semiconductors~\cite{Mak_2010_mos2,Arnaud_2006_exc_hBN}.
%
Secondly, due to the absence of dangling bonds, the 2D materials have
naturally passivated surfaces with less influence from surface states
compared with bulk material surfaces
\cite{Novoselov_2016_vdW,Liu_2016_rev}.


\begin{figure}[!htbp]
  \centering
  \import{\imgdir}{bs.pgf}
  \caption{\label{fig:intro-bs} %
    3D geometry and electronic band structures of \textbf{a} graphene,
    \textbf{b} phosphorene, \textbf{c} MoS\textsubscript{2} and
    \textbf{d} hBN. The band structures are calculated using density
    functional theory (DFT) package GPAW~\cite{Mortensen_2005_gpaw}
    and bandgap corrected according to experimental values.}
\end{figure}

%
In this section, we do not aim to provide a comprehensive comparison
of the fundamental electronic structures between different 2D
material, as has been well-described in several other reviews
\cite{Neto_2009_Electron_gr_rev,Mas_Ballest_2011_review,Das_Sarma_2011_electron_gr,Butler_2013_review,Bhimanapati_2015_2D_rev,Novoselov_2016_vdW}.
%
Instead, one common feature -- the density of states (DOS) -- caused
by the quantum-confined electronic structure of 2D materials, is
discussed.
%
In
one-atom-thick 2D materials like graphene and hBN, the planar
sp\textsuperscript{2} orbitals form covalent (σ) bonds, while the
p-orbitals perpendicular to the 2D plane form delocalized π-electron
cloud\nocite{Ihn_2009_book}.
%
Such two-dimensional
electron gas (2DEG) gives rise to significant change of its electronic
band structures compared with bulk materials \cite{Davies_1997_book}.
%
Although the physics of 2DEGs has already been developed in the
1980s~\cite{Ando_1982_electron_2D}, prior to the discovery of 2D
materials, they can only be achieved by cumbersome semiconductor
quantum well
heterostructures~\cite{Ihn_2009_book,Davies_1997_book}. The quantum
confined structure and chemically inert surface make the 2D materials
perfect candidates to study the behavior of 2DEGs.


Although the electronic band structures strongly depend on the lattice
type, one thing in common is their density of states (DOS). The DOS is
a measure of the number of available states in a certain
materials~\cite{Kittel_2005_introduction_book}. For an extended system
with nearly-continuous energy distribution, the DOS at energy $E$ is
expressed as\cite{Kittel_2005_introduction_book}:
\begin{equation}
  \label{eq:ch-intro-dos}
  \mathrm{DOS}(E) = \frac{\partial N}{\Omega \partial E} =  \frac{1}{\Omega} {\displaystyle \int_{\Omega}} \frac{\mathrm{d}^{n} \mathbf{k}}{(2 \pi)^{n}}
  \delta(E - E(\mathbf{k}))
\end{equation}
where the integral is performed over a system with $d$-dimensionality
($d$=1, 2, 3, etc.), $N$ is the total number of states, $\Omega$ is
the volume of the system, $\mathbf{k}$ is the momentum of electron,
and $\delta$ is the Dirac delta function. Without losing generality,
the DOS can be expressed using the chain rule as:
\begin{equation}
  \label{eq:dos-chain}
  \mathrm{DOS}(E) = \frac{\partial N}{\Omega \partial k} \frac{\partial k}{\partial E}
               = \frac{k}{2 \pi} \left[\frac{\partial E(k)}{\partial k}\right]^{-1}
\end{equation}
where $k=|\mathbf{k}|$.
%
\autoref{eq:dos-chain} provides a general way linking the DOS to the
energy--momentum ($E-k$) dispersion of a 2D material, which is revealed from its band structure.  Two
cases can be distinguished concerning the DOS near the conduction band
(CB) edge of a 2D material, as schematically shown in \autoref{fig:dos}:
\begin{figure}[h]
  \centering
  \import{\imgdir}{dos.pdf_tex}
  \caption{\label{fig:dos}%
    Comparison between the electronic structures of parabolic and
    Dirac 2D materials. \textbf{a} The $E-k$ dispersion diagram of
    parabolic (blue) and Dirac (yellow) materials, the energy $E$
    refers to the offset from the conduction band (CB) edge.
    \textbf{b} As a consequence of different $E-k$ dispersions, DOS of
    a parabolic 2D materials is almost constant near the CB edge,
    while the DOS of a Dirac material linearly increases with $E$.  %
  }
\end{figure}

\begin{itemize}
\item Parabolic materials: TMDCs, hBN, phosphorene, etc.

  These are the majority of 2D materials where the $E-k$ dispersion is
  parabolic (\autoref{fig:dos}\lc{a}), such that
  ${\displaystyle E(k) = \frac{\hbar^{2} k^{2}}{2 m^{*}}}$, where
  $\hbar$ is the reduced Planck constant, and $m^{*}$ is the effective
  mass near the CB edge. From \autoref{eq:dos-chain}, DOS of a
  parabolic material is \textit{constant} at the band edge, and
  proportional to $m^{*}$.
  
\item Dirac materials: graphene, silicene, germanene, etc.

  The Dirac materials are relatively rare among 2D
  materials~\cite{Wang_2015_rare_dirac}.
  %
  The name comes from the exotic electronic band structure known as the
  Dirac cone near the Fermi level (\autoref{fig:dos}\lc{a}), with a linear
  $E-k$ dispersion ($E(k) = \hbar v_{\mathrm{F}}k$), where
  $v_{\mathrm{F}}$ is the Fermi velocity. The Dirac materials has DOS
  increasing linearly with $k$ (as well as $E$).
\end{itemize}

As can be seen in \autoref{fig:dos}\lc{b}, unlike parabolic
materials with constant DOS, the Dirac materials have vanishing DOS near the
Dirac point ($E \to 0$). This feature brings intriguing 
properties like electrostatic transparency which we will discuss in
\autoref{ch:qc}.
%20191013-1701


\subsection{Fabrication of 2D Materials}
\label{sec:fabr-2d-mater}

The development of the 2D material research cannot be achieved
without appropriate methods to fabricate large-area and high quality
2D materials. In this section, we will briefly review the synthesis
techniques of 2D materials, which can be generally categorized into
top-down and bottom-up approaches. 

\subsubsection{Top-down Methods}
\label{sec:top-down-methods}

The top-down method is the straightforward approach to exfoliate few-
to single-layer 2D materials from their bulk counterparts, including micro-mechanical exfoliation and liquid-phase
exfoliation~\cite{Novoselov_2012_roadmap,Liu_2018_rev,Lin_2019_gr_rev_growth}.

\paragraph{Micro\-mechamical exfoliation}

The micro\-mechanical exfoliation uses mechanical force to overcome the
interlayer vdW interactions in bulk layered materials, and is the
first method used for 2D material exfoliation. Scotch tape is widely
used for such type of exfoliation~\cite{Novoselov_2004_gr}, while
other media as elastic polymers, heat-release tapes are also
employed~\cite{Favron_2015_BP_PDMS,Jain_2018_PDMS}. Starting with a high-quality bulk material with high chemical purity
and low defect density, mechanical exfoliation usually produces
2D materials with better quality compared with other
methods~\cite{Lin_2019_gr_rev_growth}.
%
However, there are also several key drawbacks of such methods. First
of all, the reliability of micro-mechanical exfoliation is highly
influenced by the interlayer forces, which makes it difficult for materials with high interlayer binding energy, or in-plane
mechanical anisotropy (for instance
BP~\cite{Favron_2015_BP_PDMS,Liu_2017_aniso_BP}). This method also
leads to broad distribution of lateral flake size and layer number, making it
difficult for to large-scale applications.

\paragraph{Liquid-phase exfoliation}

In contrast to micro\-mechanical exfoliation where only the top-most
layers are removed, the solution-phase exfoliation method ensures
uniform breaking up of bulk materials, and increases the scalability
of 2D flakes produced~\cite{Coleman_2012_rev}.
It can be achieved by
both physical and chemical exfoliation approaches.
%
Physical solution-phase exfoliation makes use of local mechanical
stress produced by ultra-sonication~\cite{Xia_2013_liquidphase}
or shearing~\cite{Paton_2014_shearing} to overcoming the interlayer
vdW interactions. To ensure the stability of isolated layers in 
liquid suspension, high-surface-energy liquid, in combination with
surfactants are usually required
\cite{Coleman_2012_rev,Paton_2014_shearing,Hanlon_2015_shear,Backes_2014_mos2,Shih_2010_exf,Shih_2011_bitri}.
%
Centrifugation can be further used to separate flakes, allowing narrow
distribution of layer thickness and size.
%
Using liquid-phase exfoliation,
statistic measurements
can be performed using the ensemble of such 2D material
suspensions~\cite{Backes_2014_mos2}.
%
However, this method is still in its infancy due the following aspects:
(i) the flake size is still limited to μm-scale, (ii) precise control
of layer number in suspension is difficult to achieve,  and (iii) flake overlay
after deposition onto substrate is unavoidable.
%
The interlayer vdW forces can also be overcome by chemically modifying
the 2D layer (for instance, oxidising graphite to produce graphene
oxide (GOx) \cite{Chen_2013_GO}), or intercalating ions
to induce lattice expansion (such as exfoliation of
MXenes~\cite{Naguib_2011_Mxene}, which are otherwise hard to achieve
mechanically). Despite the scalability of chemical solution-phase
approaches, they usually changes the chemical composition and
introduce defects in the 2D materials, which are undesired for
high-performance applications~\cite{Lin_2019_gr_rev_growth}.

\subsubsection{Bottom-up Methods}
\label{sec:bottom-up-methods}

The bottom-up methods grow 2D materials from precursors, and are able
to produce 2D materials with larger flake size and more control over
layer numbers.
%
Depending on whether a substrate is involved in the
process, these methods can be categorized into templated or
non-templated growth approaches.

\paragraph{Templated growth}
The templated growth uses a bulk surface as a epitaxial template and
support for the 2D material. Due to the absence of dangling bonds, the
2D-substrate interaction is usually much weaker than the in-plane
covalent bonds. Therefore, unlike epitaxy of bulk materials which
require precise control of lattice mismatch between the substrate and
epitaxy
layer~\cite{Koma_1985_vdWE,Ueno_1990_vdWE,Parkinson_1991_vdWE},
templated growth of 2D materials can be achieved on wide selection of
substrates. For example, high quality single crystal graphene can be
epitaxially grown by thermal annealing of silicon carbide (SiC)
surface~\cite{Berger_2004_sic,de_Heer_2007_epi_gr}, or by decomposition of
hydrocarbons on Ruthenium (Ru) (0001) and Pt(111)
substrates~\cite{Sutter_2008_gr_Ru,Sutter_2009_Gr_Pt}, while epitaxial
growth of hBN is achieved by cleavage of borazine
(H\textsubscript{6}B\textsubscript{3}N\textsubscript{3}) on Ru(0001)
and Rh(111)
surfaces~\cite{Goriachko_2007_assembl_hBN_ru,Goriachko_2008_AuNP_moire_hBN}.
%
Such epitaxial method can also be applied to other 2D materials like
elemental 2D
materials~\cite{Aufray_2010_silicene,Liu_2017_assemb_borophene},
TMDCs~\cite{Ugeda_2015_deposition},
monochalcogenides~\cite{Yang_2017_InSe}.
%
However, there are still several limitations: first of all, a clean
surface as well as ultra-high vacuum (UHV) conditions are usually
required for the epitaxial growth
methods~\cite{Liu_2018_rev}.
%
Moreover, transferring the 2D material
onto other substrates is generally not easy due to the noble metals
involved.
%

Chemical vapor deposition (CVD) is another widely-used templated
growth
method~\cite{Li_2016_cvd_rev,Novoselov_2016_vdW,Lin_2019_gr_rev_growth},
in which one or more precursors adsorb and react on a catalytic
surface to form covalently-bonded 2D materials.
%
The essence of CVD process is similar to the epitaxial growth, while
more ambient conditions can be used (10$^{1}$ Pa to atmosphere pressure,
ultra clean substrate not necessary) \cite{Li_2016_cvd_rev}.
%
CVD growth of graphene using hydrocarbon source on copper (Cu) is the
most studied and widely used technique~\cite{Li_2016_cvd_rev}.
%
The self-termination of second-layer on the Cu surface allows the
growth of large area single layer domains up to centimeter or
decimeter
scale~\cite{Li_2011_single_crystal,Bae_2010_gr_roll,Xu_2017_gr_single_large}.
%Another
An advantage of metal-assisted CVD method is easy removal of Cu
substrate by standard etching procedure, allowing transferring
graphene onto a arbitrary substrates~\cite{Kim_2009_gr_transparent}.
%
Similar to the case of graphene, large area single-layer TMDCs and hBN
can also be achieved using the CVD technique by proper interfacial
engineering~\cite{Shi_2012_vdw_epi_MoS2_gr,Lee_2018_gold_BN}.
%
% With proper defect control during growth and development of novel
% transferring techniques, the CVD method is promising to 

\paragraph{Template-free synthesis}
While the majority of bottom-up growth relies on a substrate to form
the 2D material, there are also cases where colloidal 2D confined
structures can be directed synthesized in liquid phase without
template. Examples of 2D materials grown using the colloidal method
include II-VI semiconductors~\cite{Riedinger_2017_2D}, 2D hybrid
perovskites~\cite{Jagielski_2017_SA}, and
several TMDCs~\cite{Altavilla_2011,Plashnitsa_2012_mos2_wet}. The
anisotropic growth of 2D materials is usually modulated by surfactant
/ ligand
engineering\cite{Riedinger_2017_2D,Jagielski_2017_SA}. Recently synthesis of
colloidal insulating 2D metal oxides are also reported by
simultaneous oxidation at the liquid metal-water interface, further
extending the possibility of bottom-up synthesis of 2D
materials~\cite{Zavabeti_2017_GaOx}.

As a summary, both top-down and bottom-up methods are capable of
fabricating 2D materials with desired purity, flake size and
thickness. A short comparison between different fabrication methods is
schematically shown in \autoref{fig:intro-syn}.

\begin{figure}[h]
  \centering
  \import{\imgdir}{synthesis.pdf_tex}
  \caption{\label{fig:intro-syn}%
    Comparison between the different fabrication methods of 2D
    materials: \textbf{a} micro\-mechanical manipulation, \textbf{b}
    solution-phase exfoliation, \textbf{c} chemical vapor deposition
    (CVD) and \textbf{d} template-free synthesis interms of quality, lateral size, layer number control, scalability and cost.
  }
\end{figure}


\section{The 2D Materials Interfaces}
\label{sec:2d-mater-interf}
%20191008-0939
The 2D materials do not solely attract the research interests due to
the unique electronic properties, they are also materials with ultra
high surface-area-to-volume ratio~\cite{Liu_2018_rev,Novoselov_2016_vdW}.
%
As a consequence,
interfaces are almost always required when integrating the 2D
materials into experimental studies and applications in the 3D world,
and the interfacial phenomena play important roles in determining
their proprieties~\cite{Liu_2018_rev}.
% 
The existence of strict 2D lattices is long questioned
due to the presumed distortion caused by thermal fluctuation which
would break up long-range
order~\cite{Peierls_1935_unstable,landau_2009_statistical_Phys_book}.
%
Although recent studies suggest free-standing 2D materials like
graphene can be stabilized by the phonon coupling that causes 3D
ripples in the 2D
plane~\cite{Fasolino_2007_ripple,Brivio_2011_mos2_ripple}, in the
majority of studies, the 2D materials still need to be supported by
substrate or encapsulated.  As will be shown later, these 2D material
interfaces may significantly alter the intrinsic properties by ways
such as structural corrugation and carrier doping.
%
On the other hand, it remains unrealistic to find a single 2D material
which can satisfy all the requirements of high-performance
applications (e.g., electronic properties, mechanical strength,
chemical stability, and synthetic difficulty), the flexibility of
creating mixed-dimensional interfaces with existing functional
materials may offer opportunities to fully exploit their
potential~\cite{Jariwala_2016_mixed_vdw_het}.
%
In this section, we focus the discussion on the
interactions involved at the 2D materials interfaces,
and how playing with the
interfacial interactions are critical for engineering of the
interface dimensionality, morphology, electronic states and transport
phenomena.

\subsection{Interactions and Forces at the 2D Materials Interfaces}
\label{sec:inter-forc-at}

\begin{figure}[!htbp]
  \centering
  \import{\imgdir}{interactions.pdf_tex}
  \caption{\label{fig:intro-interactions} %
    Scheme showing the interplay between various interactions at the
    2D material interface.
  }
\end{figure}

The concepts of 2D materials interfaces, can be
learned from the field of molecular epitaxy and self-assembly on bulk
interfaces
~\cite{Kowarik_2008_rev_MBE,Barth_2007,Whitesides_2002_assem_rev,Philips_2D_assem_book}.
% 
A molecule in the bulk form and on a densely-covered surface feels the
intermolecular interactions~\cite{Israelachvili_2011_book}.
%
On the other hand, a molecule undergoes various
processes on a 2D surface, including adsorption, diffusion, rotation,
and vibration, which is governed by the molecule-2D material
interactions.
%
Moreover, the effect of the underlying substrate is
usually important where the molecule-substrate interactions come into
play.
%
The different types of interactions at 2D material interfaces are
schematically shown in \autoref{fig:intro-interactions}.

\subsubsection{Intermolecular Interactions}
\label{sec:intro-inter-mole}

The intermolecular interactions govern the packing and orientation
behavior of the molecules several atoms away from the 2D material
surface: the strength and the direction of intermolecular interactions
determine the packing density as well as the orientation of the
molecular epitaxy. The intermolecular interactions can be categorized 
into van der Waals (vdW) interactions, hydrogen bonds (H-bonds), and
covalent bonds depending on the strength. 

\paragraph{van der Waals (vdW) Interaction}

The van der Waals (vdW) interactions are dispersion forces between
charge-neutral molecules~\cite{Israelachvili_2011_book}, including
many organic semiconductors, such as fullerene (C\(_{\text{60}}\))
~\cite{Corso_2004_C60_hBN,Kim_2015_c60_gr,Chen_2016_c60_mos2},
metal-phthalo\-cyanines (MPcs, where M can be Cu, Fe, Zn, Co, etc.)
~\cite{Xiao_2013_jacs_CuPc_gr,Wang_2010_selec_F16_gr,Zhang_2011_FePc_gr,Hamalainen_2012_CoPc_gr_Ir,Ying_Mao_2011_ge_clAlPc,Ogawa_2013_AlCiPc_gr,Pak_2015_CuPc_MoS2,Avvisati_2017_FePc_intercal,Iannuzzi_2014_MPc_hBN_Rh},
pentacene (PEN)
~\cite{Lee_2011_pentacene,Jariwala_2016_Mos2_pentacene,Shen_2017_DFT_mos2_pent,Kim_2016_trap_Mos2_pent,Nguyen_2015_pent_gr_wett,Betti_2007_orien_pentacene},
perfluoropentacene (PFP)
~\cite{Salzmann_2012_fpen_gr,Breuer_2016_acnene_mos2}, rubrene
~\cite{Lee_2014_rubene_hBN}, perylene-3,4,9,10-tetra\-carboxylic
dianhydride (PTCDA)
~\cite{Wang_2009_STM_PTCDA_Gr,Tian_2010_PTCDA_gr,Huang_2009_PTCDA_gr,Meissner_2012_PTCDA_BLG},
7,7,8,8,-Tetra\-cyanoquino\-dimethane (TCNQ) and its fluorinated
derivative 2,3,5,6-Tetra\-fluoro-7,7,8,8-tetra\-cyanoquino\-dimethane
(F\(_{\text{4}}\)-TCNQ)
~\cite{Chen_2007_tcnq_gr_transfer,Hong_2013_ftcnq_gr,Stradi_2014_TCNQ_gr_Ru,Tsai_2015_TCNQ_gr_hbn}. These
organic semiconductors are frequently used to form heterostructures
with 2D materials via molecular epitaxy~\cite{Hara_1989_ME}, and their chemical structures are summarized in \autoref{fig:intro-formula}.

\begin{figure}[h]
  \centering
  \import{\imgdir}{scheme-organic.pdf_tex}
  \caption{%
    \label{fig:intro-formula}
    Chemical structures of organic semiconductor molecules frequently
    used in molecular epitaxy on 2D material interfaces }
\end{figure}

Due to its non-directional and weak nature, if the
intermolecular vdW interactions are dominating (\ie weak
interacting 2D interfaces), the molecules tends to form close-packed
2D or 3D assemblies. The dimensionality of molecular
epitaxy by vdW interactions usually depends on the surface
coverage, as the  growth mechanism is similar to that of
adsorption isotherm. Although the vdW interactions usually have an
energy less than 4 kJ\(\cdot\)mol\(^{-1}\)~\cite{Israelachvili_2011_book}, the collective
interactions between molecules with large electron cloud can be
stronger. For the \(\pi\)-conjugated aromatic molecules listed above,
an effect known as the \(\pi\)-\(\pi\) interaction, a combined effect
of vdW interactions and charge transfer ~\cite{Hunter_1990_pi,Ortmann_2005_long_range}, can
lead to preferential stacking and orientation of the molecules, due to
maximal overlapping of \(\pi\)-electron clouds.


\paragraph{Hydrogen Bond (H-Bond)}

The hydrogen bond (H-bond) refers to the directional electrostatic
forces between an H atom covalently-bonded to an atom of high
electro\-negativity (such as O, N and F) and another highly
electro\-negative atom in adjacent molecules. Compare the vdW
interactions, hydrogen bonds usually have higher bond energy and
preferred direction, which favors certain assembly structure on 2D
materials. The H-bonds are usually dominating between molecules rich
of N, O and F elements, such as modified PTCDA compounds
~\cite{Mura_2010_DFT_H_bond_PTCDA_gr,Karmel_2014_assembl_hetero_gr},
perylene tetra\-carboxylic diimide (PTCDI) derivatives
~\cite{Pollard_2010_hbond_assembly_gr,Karmel_2014_PTCDI_gr},
carboxylic-substituted aromatic compounds
~\cite{Rochefort_2009_aro_graphene_mech,Addou_2013_TPA_gr}, polycyclic
aromatic compounds
~\cite{Kozlov_2012_polyaro_gr,Roos_2011_BTP_gr,Meier_2010_polycyclic_gr}
and inorganic acids ~\cite{Prado_2011_2D_acid_gr}. The existence of
H-bonds stabilizes the assembled  linear
~\cite{Pollard_2010_hbond_assembly_gr} or two-dimensional
~\cite{Prado_2011_2D_acid_gr} supra\-molecular assemblies. The specific
adsorption sites on 2D materials (such as the moiré patterns) also
play an important role in the assembly of H-bond-governed molecular
epitaxy.


\paragraph{Covalent Bond}
In general, the interactions between the epitaxial molecules and 2D
material (vdW and Coulombic interactions) are much weaker than
covalent bond (including metal coordination forces), resulting in a
variety of structures on 2D materials interfaces
~\cite{Bakti_Utama_2013_rev_epitax}. One example is the van der Waals
epitaxy (vdWE) technique which allows 2D or 3D crystalline growth on
2D materials. As shown in \autoref{fig:intro-hetero}\lc{a}, in conventional
covalent heteroepitaxy between dissimilar bulk materials, the lattice
mismatch and surface dangling bonds cause interfacial strain that
leads to low-quality crystal growth. Conversely, on
naturally-passivated 2D material surface, the epitaxy of another
crystalline material is possible since the vdW interaction dominates
(\autoref{fig:intro-hetero}\lc{b} and \autoref{fig:intro-hetero}\lc{c}).
\begin{figure}[h]
  \centering
  \import{\imgdir}{hetero-epitaxy.pdf_tex}
  \caption{\label{fig:intro-hetero} %
    Comparison between conventional, covalent hetero\-epitaxy (vdWE)
    (\textbf{a}), and vdW epitaxy (2D-2D epitaxy \textbf{b} and 3D-2D
    epitaxy \textbf{c}). The absence of dangling bonds on 2D material
    interface overcomes the interfacial strain that accompanies
    conventional hetero\-epitaxy, and enables epitaxial growth of both
    2D and 3D heterostructures with lattice mismatch.%
  }
\end{figure}

The versatility of vdWE leads to a number of 2D vertical
heterostructures including: TMDC/graphene
~\cite{Shi_2012_vdw_epi_MoS2_gr,Liu_2016_epi_MoS2_gr_rotation,Lin_2014_vdW_solid,Lin_2015_Wse2_MoS2_gr,Azizi_2015_Freevdw_Gr_TMDCs,Kim_2016_BiSnTe_gr},
TMDC/hBN
~\cite{Yan_2015_MoS2_on_hBN,Wang_2015_cvd_MoS2_BN,Cattelan_2015_Ws2_hBN},
graphene/hBN
~\cite{Liu_2011_gr_hBN,Zhang_2015_gr_hBN,Driver_2016_MBE_gr_hBN}, and
TMDC/TMDC
~\cite{Zhang_2014_vdw_epi_SnS2_MoS2,Diaz_2015_MoTe2_MoSe2,Gong_2014_WS2_MoS2,Alemayehu_2015_TMDC_vdw}.
%
The vdWE technique has also been used to grow 3D heterostructures on
mono- or multilayer 2D materials interfaces, including inorganic
insulators like Al\(_{\text{2}}\)O\(_{\text{3}}\)
~\cite{Zhang_2014_Al2O3_ALO_Gr,Vaziri_2013_ALD_Al2O3_gr}, and
HfO\(_{\text{2}}\) ~\cite{Alaboson_2011_PTCDA_gr_ALD},
%
and semiconductors including TiO\(_{\text{2}}\)
~\cite{Li_2015_TiO2_GO,Kumar_2011_TiO2_piezo_gr,Zhang_2011_TiO2_gr},
ZnO ~\cite{Chung_2010_GaN_ZnO_gr,Oh_2014_ZnO_hBN}, GaN
~\cite{Kobayashi_2012_GaN_hBN,Kim_2014_direct_vdw_GaN_gr,Kim_2017_remote_epi_Gr},
GaAs
~\cite{Alaskar_2015_GaAs_gr_Si_theor,Kim_2017_remote_epi_Gr,Kong_2018_vdw_polar}, and CdS / CdTe
~\cite{Loeher_1994_vdw_epi_CdS_MoTe,Loeher_1996_CdTe_MoWTe}.

Apart from the vdWE approach, covalently bonded structures can also be
formed by on-surface chemical reactions and metal coordination
bonds. Examples of such growth approach include two-dimensional
covalent organic frameworks (2D COFs) formed by linking monomers by
boron ester or imine groups
~\cite{Colson_2014_2D_COF_gr,Colson_2011_2DMOF_gr,Sun_2017_COF_VFET},
and metal-organic frameworks (MOFs)
~\cite{Urgel_2015_MOF_BN,Kumar_2014_2D_MOF_gr} on weakly interacting
or functionalized 2D materials. The planar sp$^{2}$-type bonds such as
boron ester, imine and square planar metal coordination bonds are
generally required for the formation of stable 2D epitaxial structure.

\subsubsection{Molecule-2D Material Interactions}
\label{sec:intro-mol-2D}

The interactions between the interfacial molecules and 2D material
determine the molecular packing and arrangement of the first few
overlayers. In addition, the interactions also have great impact on
the molecular adsorption process, thereby influencing the
heterogeneous nucleation characteristics. The
ratio between the intermolecular and molecule-2D material
interactions is the key factor in controlling the molecular epitaxial
structure. Here we categorize the molecule-2D material interactions
into weak (dispersion and electrostatic), charge-transfer
interactions, site-specific adsorption, and covalent bond formation.

\paragraph{Weak Interactions}
\label{sec:org68af064}

The weak molecule-2D material interactions involve the short-range
dispersion (vdW) and long-range electrostatic (Coulombic)
interactions. In the case of graphene, the delocalized π-electrons are
the basis for the non-covalent interactions. A large variety of planar
aromatic molecules, including PTCDA, PTCDI, C\(_{\text{60}}\), MPc are
shown to assemble on graphene with their aromatic rings parallel to
the 2D plane, in order to lower the adsorption energy by maximizing
the π-π interaction ~\cite{Grimme_2008_pipi,Zhang_2011_rev_pipi_gr}, a
phenomenon widely known as the graphene template effect
~\cite{Yang_2015_rev_template}. MPc molecules (e.g. M=Cu, Fe, Co and
AlCl) and substituted MPc (e.g. F\(_{\text{16}}\)CuPc) tend
to form a ``face-on'' orientation on graphene interface, relative to
the ``edge-on'' orientation that are usually found on the deposition
of these molecules on amorphous substrates such as SiO\(_{\text{2}}\)
or glass
~\cite{Ying_Mao_2011_ge_clAlPc,Zhang_2011_FePc_gr,Hamalainen_2012_CoPc_gr_Ir,,Xiao_2013_jacs_CuPc_gr}.
Similarly, the graphene template effect is also found  for PEN
~\cite{Zhou_2013_penta_gr_Ru,Lee_2011_pentacene,Lee_2011_pentacene,Zhang_2015_gr_pent_orient},
C\(_{\text{60}}\) ~\cite{Kim_2015_c60_gr,Shih_2015_PartiallyScreened},
p-sexiphenyl (6P) ~\cite{Hlawacek_2011_6P_gr}, and
dibenzotetrathienocoronene (DBTTC) ~\cite{Kim_2016_DBTTC_gr} molecules,
revealing a general mechanism behind their assembly behavior.

Apart from graphene, the weak interactions on hBN and
MoS\(_{\text{2}}\) surfaces are also studied. The π-electron cloud of
hBN resembles that of graphene, causing the 6P molecules to form a
``face on'' configuration ~\cite{Matkovic_2016_6P_hBN} similar to the
case on graphene. However, non-planar molecules such as rubrene
~\cite{Lee_2014_rubene_hBN} adopt the ``edge-on'' configuration over the ``face-on'' configuration, reflecting the fact
that the molecule-hBN interaction is weakly dispersive.
%
On the other hand, the molecular
interactions on MoS$_{2}$ are usually much weaker compared with that on
graphene due to its large dipole moment ~\cite{Rajan_2016_wett_mos2},
and is highly dependent on the lattice symmetry
~\cite{Shen_2017_DFT_mos2_pent} (i.e. 1T- or 2H- phase) and surface
defects ~\cite{Jariwala_2016_Mos2_pentacene,
  Kim_2016_trap_Mos2_pent}.

\paragraph{Charge-Transfer Interaction}
\label{sec:orgebfad7b}

The charge-transfer (CT) interactions, or the donor-acceptor (DA)
interactions, refer to the process that electrons undergo
redistribution between the epitaxial molecules and the underlying 2D
material. Due to the locally enhanced carrier density in the formed CT
complex, the CT interactions tend to be stronger than the dispersion
and electrostatic interactions. The formation of a CT heterostructure
requires alignment of the energy levels between the 2D material and
the overlayer molecules ~\cite{Akiyoshi_2015_DA}, and may also change
the electronic structure of the 2D material through non-covalent
interactions
~\cite{Cai_2015_doping_2D_rev,Wehling_2008_doping,Zhang_2011_rev_pipi_gr}.
TCNQ and its fluorinated derivative
2,3,5,6-Tetra\-fluoro-7,7,8,8-tetra\-cyanoquino\-dimethane (FTCNQ) are
known to form CT complexes with graphene
~\cite{Chen_2007_tcnq_gr_transfer,Voggu_2008_TCNQ,Barja_2010_assembl_donor_gr},
with a degree of charge transfer of $\sim{}$0.3 \textit{e} and
$\sim{}$0.4 \textit{e}, respectively. With a stronger CT effect, FTCNQ
molecules on epitaxial graphene tend to be trapped by local
corrugation~\cite{Barja_2010_assembl_donor_gr}, compared with
closed-packed TCNQ/graphene assembly.  Since CT may occur when the
HOMO and LUMO energy levels of the epitaxial molecule and 2D material
match, it is also expected to play a role in the molecular epitaxy on
2D semiconductors, such as TMDCs. Density functional theory (DFT)
studies reveal that PEN adsorbed on 1T-type monolayer
MoS\(_{\text{2}}\) has a large degree of CT ranging from 0.44-0.87
\emph{e}, and can change the Fermi energy level of MoS\(_{\text{2}}\)
by up to 1 eV ~\cite{Shen_2017_DFT_mos2_pent}. Similarly, the
interface between C\(_{\text{60}}\) and MoS\(_{\text{2}}\) is found to
be a pn-junction, with charge depleted at the bottom of the
C\(_{\text{60}}\) and accumulated at the interface
~\cite{Chen_2016_c60_mos2}. On the other hand, the tendency of forming
CT-induced orientation is attenuated on bulk MoS\(_{\text{2}}\)
crystal ~\cite{Sakurai_1991_c60_mos2}, due to an increase of the DOS
compared in bulk crystals. Theoretical studies also disclose strong CT
between phosphorene and electron-donating tetrathiafulvalene (TTF), as
well as electron-accepting TCNQ molecules
~\cite{Zhang_2015_DA_phosphorene}.


\paragraph{Site-Specific Adsorption}
\label{sec:org87b0c12}

The electronic and geometric properties of a 2D material are known to be
influenced by its underlying substrate. When there is a lattice
mismatch between the 2D material and the substrate, a long-range
periodic superposition known as moiré pattern forms, as has been
found graphene/metal ~\cite{Hamalainen_2013_moire_gr} and hBN/metal
~\cite{Schulz_2014_hBN_moire} systems.  The
moiré pattern does not only cause a geometric interference, but
indeed changes the local electronic state and structure of the 2D
material.
%
The height variation within the graphene or hBN layer can be used to
quantify the degree of metal-2D material interaction strength, to
distinguish weakly interacting surfaces include graphene/Ir(111)
~\cite{Pletikosi_2009_gr_Ir,Busse_2011_Gr_Ir,Hamalainen_2013_moire_gr},
graphene/Pt(111) ~\cite{Sutter_2009_Gr_Pt}, hBN/Ir(111)
~\cite{Schulz_2014_hBN_moire}, hBN/Pt(111) ~\cite{Cavar_2008_hBN_Pt},
hBN/Cu(111) ~\cite{Joshi_2012_hBN_Cu} systems, in which the average 2D
material-metal distance is comparable with that in the bulk material
(3.3$\sim{}$3.4 \AA{}) and the corrugation in the 2D layer is
typically small (<0.5 \AA{}). The strongly interacting surfaces
including graphene/\allowbreak{}Ru(0001)
~\cite{Moritz_2010_gr_Ru,Sutter_2008_gr_Ru}, graphene/Rh(111)
~\cite{Wang_2010_gr_Rh}, hBN/Ru(0001) ~\cite{Wang_2010_gr_Rh}, and
hBN/Rh(111) ~\cite{Dil_2008_hBN_Rh} systems, with structural
corrugations as large as 1 \AA{}, and the electronic fluctuation up to
0.5 eV. In the strongly interacting systems, the moiré pattern creates
a local difference in the adsorption potential, which in turn results
in site-specific adsorption of small molecules on these surfaces. Such
behavior has been observed in a variety of organic semiconductor
molecules deposited on the graphene/\allowbreak{}Ru(0001) surface,
including MPc (M=Fe, Ni, Zn, Mn)
~\cite{Mao_2009_Pc_gr_kagome,Zhang_2011_FePc_gr}, pentacene
~\cite{Zhou_2013_penta_gr_Ru}, C\(_{\text{60}}\)
~\cite{Li_2012_c60_gr_Ru}, PTCDA ~\cite{Zhou_2011_PTCDA_gr_Ru}, TCNQ
~\cite{Maccariello_2014_TCNQ_gr_Ru}, with similar behavior has also
been found on the surface of hBN/Ru(0001) for MPc (M=H\(_{\text{2}}\),
Cu, Co) ~\cite{Dil_2008_hBN_Rh,Jarvinen_2014_MPc_hBN_Ru}, TCNQ
~\cite{Joshi_2014_TCNQ_hBN}, and C\(_{\text{60}}\)
~\cite{Corso_2004_C60_hBN}. The site-specific adsorption usually lead
to ordered sub-2D assembly, composed of the molecules trapped at the
specific sites, compared with the close-packed assembly on flat and
weakly interacting interfaces.

% Recently, more experimental and theoretical studies have also
% demonstrated the moiré pattern formation on TMDC/metal
% ~\cite{Chen_2013_doping,Sorensen_2014,Le_2012_MoS2_Cu}, TMDC/TMDC
% ~\cite{Kang_2013_TMDC_moire,Zhang_2014_vdw_epi_SnS2_MoS2,Diaz_2015_MoTe2_MoSe2,Fang_2014_intercoupl_vdW,Li_2016_GaSe_MoSe2_vdW},
% and TMDC/hBN ~\cite{Fang_2014_intercoupl_vdW} surfaces. Following the
% discussion of the strongly interacting surface of graphene/\allowbreak{}Ru(0001),
% it is believed that the moiré pattern formed between the strongly
% coupled layers, e.g. TMDC/Ru(0001) ~\cite{Chen_2013_doping} and TMDC/TMDC
% ~\cite{Fang_2014_intercoupl_vdW} heterostructures may also lead to the
% site-specific adsorption phenomenon ~\cite{Diaz_2015_MoTe2_MoSe2}, in
% contrast to the close-packing structure formed on the weakly
% interacting surfaces, as discussed in the previous section.


\paragraph{Covalent Bond}
\label{sec:org6f342a5}

Covalent bonds formed perpendicular to the 2D material plane open an
opportunity for functionalizing 2D materials and provide anchor sites
for modification. However compared with the epitaxy approaches,
chemical modification of 2D material is limited by the choice of
chemical reactions available. Moreover, opening up 2D basal structure
usually destroyed by the geometric change of the molecular orbital
(e.g. planar sp\(^{\text{2}}\) to tetrahedral sp\(^{\text{3}}\) in
graphene). Nevertheless, there are still a few examples showing the
potential of covalent binding and tuning the electronic properties of
the 2D materials
~\cite{Georgakilas_2012_noncoval_gr_rev,Lee_2011_tempo_gr,Zhang_2013_janus_gr,Voiry_2014_cov_TMDC_phase,Vishnoi_2016_ar_mos2_covalent,Liu_2011_rev_chem_dope_gr,Wang_2012_ar_gr_react_rate}.
%
The chemical grafting of graphene mainly involves free-radical
reaction
~\cite{Lee_2011_tempo_gr,Choi_2010_aminotempo_gr,Zhang_2013_janus_gr,Wang_2012_ar_gr_react_rate,Kumar_2014_2D_MOF_gr},
%
with the potential to fabricate asymmetric Janus-type functionalized
graphene by the covalent modification on both sides of a free-standing
graphene sheet ~\cite{Zhang_2013_janus_gr}. The low DOS in a 2D
materials further makes it possible to fine-tune the interfacial
chemical reaction rate by the doping density of 2D materials, for
instance through the substrate doping of graphene
~\cite{Wang_2012_ar_gr_react_rate}. Several approaches have also show
the possibility of functionalizing other 2D materials, including
nucleophilic substitution between anionized TMDCs and organohalides
~\cite{Vishnoi_2016_ar_mos2_covalent} and aryl diazonium
salts~\cite{Ryder_2016_TMDC_ad,Ryder_2016_phosphorene_ad}. The chemical modifications are also
frequently used to improve quality of 2D semiconductors.
%
Diazonium modification of BP significantly increases its ambient
stability over several
weeks~\cite{Ryder_2016_phosphorene_ad}. Moreover, treating
MoS\textsubscript{2} with organic super\-acids moves its Fermi level
towards mid-gap, and significantly improves the
photo\-luminescence (PL) quantum yield over two order of
magnitudes~\cite{Amani_2015_mos2_QY1}.
%
Future advance of covalently
modified 2D materials with site-specific and programmable chemical
functionalization may combine the 2D with the 3D materials in a
controllable manner.

\subsubsection{Molecule-Substrate Interaction}
\label{sec:intro-mol-subst}

One of the major differences of the interfacial molecules on 2D
materials compared with bulk materials interfaces is significant
influence from the underlying substrate. Note that this phenomenon is
distinguished from the effect of strongly interacting surface or
substrate doping, with the latter two referring to the change of 2D
material's electronic and geometric properties, which in turn influence
the molecule-2D material interactions. The penetration of the
molecule-substrate interactions through monolayer 2D material is first
observed in the experiments of wettability of substrate-supported
graphene: the water contact angle of water on graphene is found to be
influenced by the vdW force between the water molecules and the
substrate, known as the wetting ``transparency'' or ``translucency''
of graphene
~\cite{rafiee_2012_transparency,Shih_2012_prl,shih_2013_wetting_natmat}. The
transparency can be even pronounced for electrostatic interactions,
which has longer length scale than the vdW force
~\cite{Shih_2015_PartiallyScreened,Tian_2016_multiscale}.
\begin{figure}[h]
  \centering
  \import{\imgdir}{mole-sub.pdf_tex}
  \caption{\label{fig:intro-mol-sub}%
    Examples showing the penetration of molecule-substrate
    interactions through the 2D material. \textbf{a}
    Substrate-dependent epitaxial morphology of PEN on various
    graphene-covered surfaces. The morphology changes significantly
    from hydrophobic to hydrophilic substrate. (Reproduced with
    permission from Ref. \cite{Nguyen_2015_pent_gr_wett}. Copyright
    2015 American Chemical Society. \textbf{b} Remote epitaxy of GaAs
    through monolayer graphene. At a distance < 9 Å the electron
    density from substrate GaAs slab can penetrate to influence the
    epitaxial layer (top), resulting in perfect lattice match between
    the epitaxial and substrate slabs (bottom).  (Reproduced with
    permission from Ref. \cite{Kim_2017_remote_epi_Gr}. Copyright 2017
    Springer Nature.)%
  }
\end{figure}
The
influence of the molecule-substrate interactions through a 2D
materials usually can only be examined indirectly.
%
Examples include layer-number-dependent morphology of
6P~\cite{Kratzer_2014_6P_gr_layer} and
PEN~\cite{Chhikara_2014_gr_pent_trans} molecules deposited on
SiO\(_{\text{2}}\)-supported graphene layers. Moreover, the influence
of underlying substrate is also found for PEN when deposited on
graphene supported by substrates with varied surface energy
~\cite{Nguyen_2015_pent_gr_wett} or electrostatic gating
\cite{Nguyen_2019_PEN} (\autoref{fig:intro-mol-sub}\lc{a}).
%
Recently the concept of vdW transparency has also been
employed in the remote vdWE of III-V semiconductors on graphene supported by highly-crystalline III-V substrate
~\cite{Kim_2017_remote_epi_Gr,Kong_2018_vdw_polar}.
%
The interactions from underlying crystalline III-V semiconductor is
shown to direct the growth of III-V semiconductor on the graphene
interface despite the $\sim{}$ 1 nm gap created by graphene (~\autoref{fig:intro-mol-sub}\lc{b}). The
strength of such remote interactions are also shown to be dependent on
the polarity of the underlying material~\cite{Kong_2018_vdw_polar}.
%
In addition to the vdW and Coulombic interactions,
graphene layer is also found to be transparent to the charge transfer
process ~\cite{Jeong_2015_DA_transparency_gr} when the reduction rate of
AuCl\(_{\text{4}}^{\text{-}}\) on graphene surface are found to be
faster when graphene is coated on a reductive surface, such as Al, Ge
and Cu surfaces. 

% 20191008-1253

\subsubsection{Summary}
\label{sec:org697d552}

To obtain a clear view of the interactions involved in the molecular
epitaxy on 2D materials interfaces, the major forms
of interactions and their energy range are summarized in
\autoref{tbl:intro-interactions}. As can be seen, both strong interactions ($>$ 100
kJ\(\cdot\)mol\(^{\text{-1}}\) for covalent bonds, metal-coordination and some
hydrogen bonds), and weak interactions ($<$ 50 kJ\(\cdot\)mol\(^{\text{-1}}\) vdW
and \(\pi\)-\(\pi\) interactions) exist between the epitaxial molecules and at
the molecule-2D material interface. On the other hand, the
molecule-substrate interactions mainly have a weak nature, and
relatively weaker than the intermolecular and molecule-2D weak
interactions due to the increasing of molecule-substrate distance.

\begin{table}[!htbp]
\caption{\label{tbl:intro-interactions}
Types of interfacial interactions involved in the molecular epitaxy on 2D materials interfaces, showing the typical forms of interaction and energy range.}
\small
\centering
  \begin{tabular}[\textwidth]{lll}
\hline
Type of Interaction & Typical Forms & Energy Range  (kJ\(\cdot\)mol\(^{\text{-1}}\))\\
\hline
Intermolecular & van der Waals & \(\le\) 5\\
 & \(\pi\) - \(\pi\) & \(\le\) 50\\
 & H-bonds & 4 - 120 ~\cite{jeffrey_introduction_1997}\\
 & Covalent Bonds & 100 - 400\\
\hline
Molecule-2D & Weak Interactions & 10 - 60 ~\cite{Lazar_2013}\\
 & Charge-Transfer & 50 - 200\\
 & Site-Specific Adsorption & 30 - 100\\
 & Covalent Bonds & 100 - 400\\
\hline
Molecule-Substrate & Weak Interactions & \(\le\) 20\\
\hline
\end{tabular}
\end{table}

\subsection{The Variety of Mixed-Dimensional Interfaces}
\label{sec:vari-mixed-dimens}

%20191008-1405
As shown in the previous section, a variety of interactions exist on 2D
materials, which
essentially dominate the ordering and packing of the molecules in vicinity.
%
The interfacial engineering of the interactions leads to different
dimensionalities ranging from 0D to 3D, as schematically shown in
\autoref{fig:intro-dimensions}.
%
In this section, several model systems of mixed-dimensional interfaces
with 2D materials are discussed, with the focus on how the
interactions determine the morphological dimension. The notations for
the interfaces are like ``0D-2D'', with the former indicating the
dimensionality of the heterostructure on 2D materials. In general, two
approaches are usually employed to fabricate such mixed-dimensional
interfaces, namely self-assembly of small molecules, and deposition of
pre-formed nano\-materials. 

\begin{figure}[!htbp]
  \centering
  \import{\imgdir}{dimensions.pdf_tex}
  \caption{\label{fig:intro-dimensions} %
    Schemes of 2D material interfaces with different dimensionalities:
    \textbf{a} 0D-2D, \textbf{b} 1D-2D, \textbf{c} 2D-2D, and
    \textbf{d} 3D-2D.           %
  }
\end{figure}

\subsubsection{0D-2D Interface}
\label{sec:intro-0D-2D}

\begin{figure}[h]
  \centering
  \import{\imgdir}{0D-2D.pdf_tex}
  \caption{\label{fig:intro-0D-2D}%
    Examples of 0D-2D interface. \textbf{a} Isolated MPc molecules
    trapped in the pore regions of hBN/Ru(0001) surface grown by
    sub-monolayer molecular epitaxy, with both experimental STM
    micro\-graph and first principle simulations (Adapted with
    permission from
    Ref. \cite{Iannuzzi_2014_MPc_hBN_Rh}. Copyright 2014 The PCCP
    Owner Societies.)  \textbf{b} Hybrid graphene–QD photo\-transistor
    made from solution-phase deposition. (Adapted with permission from
    Ref. \cite{Konstantatos_2012_QD_gr_trans}. Copyright 2012
    Macmillan Publishers Limited.)%
  }
\end{figure}

\paragraph{Self-Assembly}
\label{sec:org8117691}

Molecular dynamics (MD) simulations have shown that the weak
intermolecular and molecule-2D interactions alone, do not result in
the formation of sub-monolayer assembly, such as the case of pentacene
and PTCDA on graphene or hBN ~\cite{Zhao_2015_self_assemb_gr_MD}, and
organic semiconductor molecules dominated by vdW force on phosphorene
~\cite{Mukhopadhyay_2017_cryst_BP}.
% 
To form isolated 0D assemblies on 2D materials, specific adsorption sites are
required to exist on the 2D surface.
% 
The moiré pattern formed in the graphene/metal and hBN/metal
interfaces as introduced in \autoref{sec:intro-mol-2D} are shown to
trap metal clusters
~\cite{Goriachko_2007_assembl_hBN_ru,Pan_2009_Pt_cluster_gr,Wang_2011_gr_hBN_metal_cl,Zhang_2014_metal_gr_Ru}
and individual organic semiconductor molecules
~\cite{Joshi_2014_TCNQ_hBN,Dil_2008_hBN_Rh,Lu_2012_c60_gr_moire}.
%
In the case of organic molecular deposition
(~\autoref{fig:intro-0D-2D}\lc{a}), the site-specific adsorption energy
difference is usually around 10\textsuperscript{2} meV
~\cite{Lu_2012_c60_gr_moire,Iannuzzi_2014_MPc_hBN_Rh}, enough to trap
the small molecules within the valley regions of the 2D moiré
patterns. It is also found that the site-specific
isolation of small molecules is not limited to strongly interacting
surfaces such as graphene and hBN supported by Ru or Rh, but also
weakly interacting surfaces like hBN/Cu(111) with a small degree of
corrugation but strong electronic patterning
~\cite{Joshi_2012_hBN_Cu,Joshi_2014_TCNQ_hBN}, revealing the more
complex nature of the 0D self-assembly on 2D materials interfaces.
%
The isolated molecules on 2D materials can be used as nucleation sites
for further molecular epitaxy, and facilitate the research of
single-molecular surface reaction.


\paragraph{Deposition of nano\-materials}


The 0D-2D interfaces fabricated by deposition usually refer to the 0D
quantum dot (QD)-2D material junction. The QDs are
quantum-confined nano\-materials with size of several
nano\-meters~\cite{Sargent_2013_colloidal_book}.
%
The vast majority of QDs are prepared by colloidal synthesis, and
covered by coating layer such as ligands~\cite{Kim_2013_QD_rev}.
%
As a result, QDs deposited on 2D materials using solution processing
can still sustain the isolate form, making their applications more
versatile than the self-assembled 0D
structures~\cite{Jariwala_2016_mixed_vdw_het}.
%
One promising feature of QDs is their optical properties, including
high optical absorption coefficient, PL quantum yield, and
size-dependent modulation of optical
bandgap~\cite{Xia_2014_2D_nanophoto_rev}.
%
Combining with the electrostatic tuning of 2DEG, the 0D-2D
heterostructures can be used for multiple light sensitive components
such as
photo\-transistors~\cite{Kufer_2014_QD-mos2,Konstantatos_2012_QD_gr_trans},
photo\-diodes\cite{Kufer_2014_QD-mos2}, photo\-voltaics
(PV)~\cite{Guo_2010_gr_QD_PV,Wang_2016_QD_PV} and light emitting
devices (LED)~\cite{Son_2012_ZnO-Gr-QD}.
%
These architectures share a similar feature, that the QDs act as the
main photo-active component, while the 2D materials is design to
modulate the carrier transport / injection at the QD-2D interface.
%
For instance, in a 0D-2D photo\-transistors based on lead sulfide
(PbS) QDs on graphene \cite{Konstantatos_2012_QD_gr_trans}, the
photo-generated carriers in the QDs are transferred onto the biased
graphene surface and consequently collected at the electrodes
(~\autoref{fig:intro-0D-2D}\lc{b}). The benefits from such
mixed-dimensional heterostructure are two-folds: (1) the large optical
cross-section of QDs enhanced photo\-detection limit compared with
bare 2D material and (2) the photo\-current is efficiently modulated
by electrostatic gating of graphene.
%
The examples of other 0D-2D photo\-active electronic components can be
found in several recent
reviews~\cite{Jariwala_2016_mixed_vdw_het,Kufer_2016_QD_FET_rev}.


\subsubsection{1D-2D Interface}
\label{sec:orgeadf57e}

\begin{figure}[h]
  \centering
  \import{\imgdir}{1D-2D.pdf_tex}
  \caption{\label{fig:intro-1D-2D}%
    Examples of 1D-2D hetero\-junctions. \textbf{a} Self-assembled
    network structures of TNCQ on graphne/Ru(0001) surface by
    increasing surface coverage 0.3 ML. (Adapted with permission from
    Ref. \cite{Maccariello_2014_TCNQ_gr_Ru}. Copyright 2014 American
    Chemical Society.) \textbf{b} Solution-processed  carbon
    nano\-tube/MoS\textsubscript{2} pn-junction for high performance
    field effect transistor. (Adapted with permission from
    Ref. \cite{Jariwala_2013_CNT-mos2}. Copyright 2013 National
    Academy of Sciences.)}
\end{figure}

\paragraph{Self-assembly}

With increasing surface coverage on a 2D material interface or
introducing directional intermolecular interactions, self-assembled 1D
and fractal assemblies may be formed on 2D material interfaces, in the
form of nanowires, nanoporous or network structures.
%
As discussed earlier, the strongly interacting surfaces, including
graphene/\allowbreak{}Ru(0001), graphene/Rh(111) and hBN/Ru(0001)
result in the moiré pattern that serves as specific binding sites for
trapping small molecules at low surface coverage.  Starting from the
0D assembly formed by moiré patterns on
graphene/\allowbreak{}Ru(0001), graphene/Rh(111) and hBN/Ru(0001) as
discussed in \autoref{sec:intro-0D-2D}, when further increasing the
surface coverage, the specifically adsorbed molecules act as
nucleation sites for subsequent epitaxial growth.  The intermolecular
interactions, in combination with the geometry of moiré pattern,
result in a specific arrangement of the molecules on the surface,
varying from nanowire, nanorope ~\cite{Maccariello_2014_TCNQ_gr_Ru} to
Kagome lattice ~\cite{Atwood_2002_kagome,Mao_2009_Pc_gr_kagome}
(~\autoref{fig:intro-1D-2D}\lc{a}).
%
The nucleation-induced growth is both substrate- and
molecule-specific. For instance, 1D and fractal molecular assemblies
are rare on hBN/metal
surfaces~\cite{Schulz_2013_copc_hbn_moire,Schulz_2014_hBN_moire,Iannuzzi_2014_MPc_hBN_Rh,Joshi_2014_TCNQ_hBN},
possibly due to the different surface potential distribution compared
with the graphene moiré surface. Furthermore, metal intercalation
between graphene and substrate is shown to affect the assembly pattern
of MPc
~\cite{Bazarnik_2013_tailor_Fe_Co_gr_Ir,Avvisati_2017_FePc_intercal}.
%
In addition to the nucleation-induced growth, a rich set of 1D-2D
heterostructure can be obtained by tailoring the intermolecular and
molecule-substrate interactions.
%
Intermolecular H-bond (such as in PTCDI derivatives) is widely used to
guide the orientation of surface-assisted self-assemblies ranging from
linear to Kagome structures ~\cite{Slater_2014_HBond_assembl_rev,
  Pollard_2010_hbond_assembly_gr, }, due to the relatively high
strength compared with molecule-2D interactions.
%
The assembly structure can be ultra sensitive to the conformation of
molecules, with minor change of functional groups between almost
identical molecules leading to distinct morphologies on graphene/\allowbreak{}Ru
(0001) surface
~\cite{Meier_2010_polycyclic_gr,Roos_2011_BTP_gr,Roos_2011_hiera_org_gr},
revealing the important role of directional interactions in the
formation of low-dimensional assemblies on 2D
interfaces.
%
In addition to hydrogen bond, several other interactions including
hydrophobic interaction between long alkyl chains
~\cite{De_Feyter_2003_2D_assem_rev, Deshpande_2012_1D_assemb_gr},
Coulombic interactions ~\cite{Prado_2011_2D_acid_gr}, covalent bond
~\cite{Colson_2011_2DMOF_gr,Colson_2014_2D_COF_gr} and metal
coordination bond ~\cite{Urgel_2015_MOF_BN} are also shown to form
1D-2D heterostructures. In these systems, the intermolecular
interactions are generally directional, and much larger than the molecule-2D
interactions, which stabilizes the formed low-dimensional structure.

\paragraph{Deposition of nano\-materials}
% 20191008-1613
Like the 0D quantum dots, 1D quantum-confined nano\-material such as
semiconductor nanowires (NWs), carbon nano\-tubes (CNTs) are the
mostly widely involved to build 1D-2D interface
~\cite{Jariwala_2013_CNT-mos2,Jariwala_2014_solution_CNT,Fu_2012_gr_ZnONW,Wu_2010_NW_supercap,Gan_2013_Si_WG}.
%
The interfaces can usually be fabricated using solution processing
(\eg coating of nanowire suspension~\cite{Jariwala_2014_solution_CNT},
\autoref{fig:intro-1D-2D}\lc{b})
%
Similar to the idea of 0D-2D heterostructure, the quantum-confinement
of the 1D NW structures is employed to fabricate photo\-diodes and
photo\-transistors with engineering of spectra responsive from
ultraviolet (UV) to infrared (IR) \cite{Nie_2013_NW_array,Gao_2013_NW_gr,Miao_2014_NW_IR,Jariwala_2013_CNT-mos2,Spina_2015_perov_nw}
by tuning both the bandgap of the NWs or 2D materials.
%
Proper alignment of the 1D-2D heterostructure can also be used as
optical waveguides and cavities to direct in-plane light propagation,
in order to enhance optical responsiveness of pure 2D structures~\cite{Gan_2013_Si_WG,Pospischil_2013_CMOS}.
%
If the 1D structures are aligned vertically to the 2D plane (\ie NW
arrays), such heterostructure further owns high surface-area-to-volume
ratio, and are widely used in applications like electrodes in
batteries and super\-capacitors \cite{Wu_2010_NW_supercap,Liu_2011_V2O5NW}


\subsubsection{2D-2D Interface}
\label{sec:intro-2D-2D}

\paragraph{Monolayer Self-Assembly of Small Molecules}
\label{sec:orgfd77377}

The assembly of small molecules on 2D materials with low geometric and
electronic corrugation have usually been found to form close-packed
structures. The role of intermolecular, molecule-2D material and
molecule-substrate interactions in such assembly process can be
understood by MD simulations. Using PEN and PTCDA as model molecules,
Zhao et al. found that inter\-molecular Coulomb and vdW interactions
are equivalently important to stabilize 2D assemblies, while the
molecule-2D interaction governs the orientation (\ie ``face-on'' vs
``edge-on'' configurations) of molecules on the 2D interface
~\cite{Zhao_2015_self_assemb_gr_MD}. Similar studies are also applied
for molecular assembly on phosphorene
~\cite{Mukhopadhyay_2017_cryst_BP}, which differs from graphene with
its armchair structure and electronic anisotropy. Despite the
non-aromatic nature of phosphorene, the adhesion energy of small
organic molecules such as TCNQ and PEN are comparable with that on
graphene or hBN, which explain similar packing experimentally observed
on phosphorene. Such arguments also correlates with the general trend
observed in experiments: monolayer closed-packed molecular assemblies
are more favored on weakly interacting interfaces such as graphene/Pt
interfaces such as graphene/\allowbreak{}Ru(0001) and graphene/BN,
which is applicable for a wide (111) and graphene/Pt (111) compared
with strongly interacting range of small organic molecules
~\cite{Hamalainen_2012_CoPc_gr_Ir,Xiao_2013_jacs_CuPc_gr,Barja_2010_assembl_donor_gr,Jung_2014_C60_gr_Cu,Yang_2012_MPc_gr_metal,Hamalainen_2012_CoPc_gr_Ir,Tsai_2015_TCNQ_gr_hbn,Stradi_2014_TCNQ_gr_Ru}.
%
Despite the simplicity of such conclusions, in reality, the monolayer
assembly of small molecules is a more complex process which involved
kinetic parameters such as nucleation, local potential barrier and
interfacial diffusion.


\paragraph{2D van der Waals Heterostructures}
\label{sec:org77ea5bc}

The 2D vdW heterostructures (vdWHs) are the heterostructures formed by
sequentially stacking individual 2D layers in controlled manner. The
formation of 2D vdWHs can be seen as the reversal process of 2D
exfoliation, and can be fabricated by either direct mechanical
assembly or epitaxial growth.
%
The mechanical assembly is usually performed by wet transfer and
lift-off techniques~\cite{Novoselov_2016_vdW},
although able to precise determine the stacking
sequence, its process is usually cumbersome and time-consuming, and
the relative orientation between layers is hard to
control.
% \worktodo{Cite paper from Novoselov}
Recent advances like atomically-precise scanning tunneling microscope
(STM) manipulation and folding of 2D flakes may open avenues to
scalable production of 2D vdWHs with desired stack and
orientation~\cite{Chen_2019_STM_graphene}.
%
More scalable synthesis of vdWHs include solution-phase assembly of 2D
material flakes by controlling the interlayer charge, which can be
directly used for energy applications.
%
Another important approach to achieve large area vdWHs is through
epitaxy growth techniques like chemical vapor deposition (CVD) and van
der Waals epitaxy (vdWE)~\cite{Novoselov_2016_vdW}. In the view point
of molecular interactions at the interface, these two methods are
essentially similar. The use of vdWE for both small molecule
~\cite{Hara_1989_ME,Sakurai_1991_c60_mos2} and layered materials
~\cite{Koma_1985_vdWE,Ueno_1990_vdWE,Ohuchi_1990_MoSe2_SnS2,Parkinson_1991_vdWE}
has been demonstrated long before the first discovery of graphene. As
discussed before, vdWE is almost not constraint by lattice mismatch
compared with conventional heteroepitaxy, where dangling bonds exist
on the substrate surface.
%
The vdWE approach is capable of producing vdWHs pairs including
graphene/hBN ~\cite{Yang_2013_gr_hBN}, TMDC/hBN
~\cite{Yan_2015_MoS2_on_hBN,Wang_2015_cvd_MoS2_BN,
  Cattelan_2015_Ws2_hBN} hBN/graphene ~\cite{Lin_2014_vdW_solid},
TMDC/graphene
~\cite{Shi_2012_vdw_epi_MoS2_gr,McCreary_2014_MoS2_gr,Azizi_2015_Freevdw_Gr_TMDCs,Miwa_2015_MoS2_gr,Ago_2015_MoS2_Gr,Lin_2014_WS2_Gr,Lin_2015_Wse2_MoS2_gr},
, TMDC/TMDC ~\cite{Diaz_2015_MoTe2_MoSe2,Gong_2014_WS2_MoS2},
Monochacolgenide/TMDC
~\cite{Li_2016_GaSe_MoSe2_vdW,Zhang_2014_vdw_epi_SnS2_MoS2} and even
more complex layer combinations
~\cite{Lin_2015_Wse2_MoS2_gr,Alemayehu_2015_TMDC_vdw}.
% 
Despite the versatility of vdWE approaches, one challenge is the
control of layer number  during growth.
%
In many 2D heterostructures systems (mainly TMDC/graphene or TMDC/hBN)
grown by vdWE, the growth of mono- and multi- overlayers both exists
~\cite{Shi_2012_vdw_epi_MoS2_gr,Azizi_2015_Freevdw_Gr_TMDCs,Miwa_2015_MoS2_gr,Yan_2015_MoS2_on_hBN}.
%
The layer number of epitaxy heterostructure is also
also found relevant to the stacking sequence in graphene/hBN systems ~\cite{Wu_2015_Gr_hBN,Yang_2013_gr_hBN,Wu_2015_Gr_hBN}.
%
These results reveal the
importance of chemical kinetics in the vdWE heterostructure growth,
the interplay between the interactions alone cannot explain the
discrepancy.

The beauty of 2D vdWHs is essentially to construct complex materials
from 2D layer building blocks, while their properties are
strongly influenced by the order and orientation of 2D building
blocks.
%
In other words, the physical property of a vdWH is not trivial
summation / averaging from that of individual layers, with
unprecedented physical phenomena coming from the interlayer coupling.
%
Examples of such interlayer coupling include layer-dependent bandgap
opening~\cite{Mak_2010_mos2,Raja_2017}, ultra\-fast interlayer charge
transfer~\cite{Hong_2014_ultrafast_e_MoS2WS2,Zheng_2017_ultrafast_CT},
interlayer exciton formation~\cite{Latini_2017_interlayer_ex},
magnetic coupling~\cite{Huang_2017_magnet} and even topological
insulating / superconducting
states~\cite{Cao_2018_insulator,Cao_2018_supercond}.


\subsubsection{3D-2D Interface}
\label{sec:intro-3D-2D}

Solid-state 3D assembly on 2D materials can be made by layer-by-layer
deposition of small molecules or hetero\-epitaxy of covalently bonded
structure. The intermolecular (or interatomic) interactions become
dominant over the molecule-2D material and molecule-substrate
interactions. However, interfacial interactions still play an
important role in the molecular orientation and morphology. The
packing and morphology of the 3D assembly greatly influence several
key properties including carrier transport, interfacial barrier, which
motivates understanding of the underlying mechanism.  The diversity of
3D epitaxial morphology addresses the question of how the macroscopic
structure is influenced by the interplay between the interactions.


\paragraph{Layer-by-Layer Assembly of Small Molecules}
\label{sec:org2cdd8f0}

Layer-by-layer (LbL) self-assembly of small molecules can be viewed as
the vertical epitaxy of 2D assembled structure. The molecular
orientation and packing of the interfacial layers, i.e., the first few
layers of the molecules, are more influenced by the molecule-2D
material and molecule-substrate interactions, compared to the
molecules far from the interface. The influence of the 2D material and
substrate can be  characterized by the penetration depth of interfacial forces, \ie the
maximum molecule layer number influenced by the 2D material and
substrate.
%
A transition of
molecular orientation is usually observed beyond the penetration depth
as a consequence of vanishing interfacial interactions.
%
The penetration depth is highly system-dependent.
%
For instance, self-assemblies of small organic molecules like
C\(_{\text{60}}\)~\cite{Lu_2012_c60_gr_moire} and TCNQ
~\cite{Maccariello_2014_TCNQ_gr_Ru} on graphene/\allowbreak{}Ru(0001) surface undergo
transition from Kagome lattice to close-packed structures at $\sim$3
monolayer (ML) and $1$ ML coverage, respectively.
%
The orientation of an organic molecular can even fully flip beyond the
penetration depth. PEN molecules deposited graphene/SiC surface
exhibit the face-on orientation with long-range ordering at 1 ML
coverage ~\cite{Jung_2014_pentacene}, while the edge-on orientation
gradually emerges with thicker deposition
~\cite{Chen_2008_transition_pentacene}. After the fifth layer, the PEN
molecules fully adapt the edge-on orientation, consistent with that in
the bulk crystal ~\cite{Ruiz_2004_bulk_pentacene}. Similar structural
transition is also observed in C8-TBTB assembly on
graphene~\cite{He_2014_C8BTBT_gr} and MoS\textsubscript{2}
~\cite{He_2015_C8BTBT_MoS2}, with the penetration depth on graphene
slightly longer than on MoS\textsubscript{2} due to stronger π-π
interactions (~\autoref{fig:intro-3D-2D}\lc{a} and
\autoref{fig:intro-3D-2D}\lc{b}).
%
Although in principle, the weak molecule-2D and molecule-substrate
interactions vanishes after a few nano\-meter, the interface-induced orientation may still be preserved over long range.
For aromatic molecules with a large planar structure
such as MPc derivatives, the strong intermolecular interactions lead
to packing along its \textit{c}-axis
~\cite{Ren_2011_DFT_CuPc_epi_gr,Jiang_2014_F16Pc,Yoon_2010_crystal_F16cuPc},
propagating of the face-on orientation even
throughout the 3D epitaxy structure, with no apparent penetration
depth. On the contrary, the epitaxy
structure of thin-layer MPc on MoS\(_{\text{2}}\)
~\cite{Zhang_2015_CuPc_MoS2} shows less stability which undergo a
face-on to edge-on transition upon air exposure and forms 1D-ordered
structures.

\begin{figure}[!htbp]
  \centering
  \import{\imgdir}{3D-2D.pdf_tex}
  \caption{\label{fig:intro-3D-2D}%
    Examples of 3D-2D heterostructure by layer-by-layer deposition of
    molecular organic semiconductors. The substrate effect is critical
    for controlling the interfacial orientation by comparing the
    packing of C8-TBTB on \textbf{a} graphene and \textbf{b} MoS2. The
    template effect on MoS2 is less than graphene. (\textbf{a} adapted
    with permission from Ref.~\cite{He_2014_C8BTBT_gr} Copyright 2014
    Springer Nature. \textbf{b} adapted with permission from
    Ref.~\cite{He_2015_C8BTBT_MoS2}. Copyright 2015 American Institute
    of Physics.) \textbf{c} Highly oriented rubrene crystal growth on
    layered hBN for high performance field effect transistor. (Adapted
    with permission from Ref.~\cite{Lee_2014_rubene_hBN} Copyright
    2014 WILEY-VCH Verlag GmbH \& Co)%
  }
\end{figure}

The morphological control of 3D molecular epitaxy can also be achieved
by fine-tuning the chemical structure of epitaxial molecules, such as
the cases of
CuPc/F\(_{\text{16}}\)CuPc~\cite{Singha_Roy_2012_CuPc_gr_glass,Xiao_2013_jacs_CuPc_gr,Zhong_2012_gr_F16_pn_junc,Yang_2011_F16CUPc_nanowire}
and
PEN/PFP ~\cite{Salzmann_2012_fpen_gr,Breuer_2011_pent_graph}
on graphene and MoS\textsubscript{2} surfaces.
%
While the nanoscale orientation remains the same, fluorination of the
molecule causes hugh change of micro\-scale file morphology by
decreasing the grain size and increasing of surface roughness.
%
While the detailed mechanism of such phenomenon is still not fully
understood, the change of film interfacial energy caused by the
fluorination may be a plausible explanation.

The atomically flat and chemically inert surfaces of 2D materials
allow the growth of highly crystalline organic thin films. Organic
semiconductors such as rubrene grown by the template effect of 2D
materials exhibit high mobility ($>$ 10 cm$^{2}\cdot$V$^{-1}$s$^{-1}$)
which is previously only achievable by single
crystal devices~\cite{Lee_2014_rubene_hBN} (~\autoref{fig:intro-3D-2D}\lc{c}).
%
The high mobility brought by interfacial molecular packing and
orientation provides rich opportunities for fabricating high performance
electronic devices such as organic field effect transistors (OFET) and
pn-junctions:
%
current on-off ratio as high as 10$^{8}$ can be achieved in lateral
OFET based on PEN/graphene heterostructure~\cite{Lee_2011_pentacene}.
%
Vertical OFET made on C\textsubscript{60}/graphene interface also
shows current on-off ratio up to
10$^{5}$~\cite{Shih_2015_PartiallyScreened}.
%
In addition, engineering the interfacial molecular orientation
not only enables high carrier mobility, but also changes the work
function of organic layer, providing more flexibility tuning the
electronic properties of the organic/2D material
heterostructure~\cite{Zhong_2014_gr_F16_EF,Wu_2013_CuPc_F16_gr}.



\paragraph{Van der Waals Epitaxy of 3D Crystals}
\label{sec:orgeb0161b}
%20191009-1013

The vdWE of 3D crystals on 2D interface s
hares the same mechanism with the 2D
vdWE, while non-planar (such as sp\(^{\text{3}}\)) bonds are involved.
%
A variety of bulk crystalline semiconductors (including
III-V~\cite{Alaskar_2015_GaAs_gr_Si_theor,Kim_2017_remote_epi_Gr,Nepal_2013_GaN_gr,Kim_2014_direct_vdw_GaN_gr,Makimoto_2012_InGaN_hBN},
II-VI~\cite{Loeher_1994_vdw_epi_CdS_MoTe,Loeher_1996_CdTe_MoWTe},
oxides~\cite{Oh_2014_ZnO_hBN,Chung_2010_GaN_ZnO_gr})
can be epitaxially grown on mono- or multilayer graphene, hBN and TMDCs, despite the lattice mismatch between 2D and 3D materials as high as 40\%.
%
In
these systems, the 2D material not only serves as the buffer layer for
vdWE, but also acts as a transferable layer which enables separation
of the 3D crystal from the substrate to allow fabrication of
semiconductor devices
~\cite{Makimoto_2012_InGaN_hBN,Kobayashi_2012_GaN_hBN,Kim_2014_direct_vdw_GaN_gr,Kim_2017_remote_epi_Gr}.
%
Besides the molecule-2D material interactions, the underlying
substrate below 2D material is also shown to have impact on the
epitaxial overlayer, resulting in the epitaxial layer remotely follows
the  lattice of the underlying semiconductor substrate (GaAs, InP and GaP)
~\cite{Kim_2017_remote_epi_Gr}. Moreover such interactions are found to be stronger for materials using higher polarity~\cite{Kong_2018_vdw_polar}.
%
The exact mechanism of such phenomena is, however still an open question.


\subsubsection{Summary}
\label{sec:org0b4290f}

As a summary, \autoref{tbl:intro-summary-multidimension} lists the
representative examples of multidimensional molecular epitaxy on 2D
materials interfaces, and outlines the governing interactions and the
comparison between the interactions that discussed in this section. A
clear relation between the interaction and the dimension of assembly
can be observed: molecular assembly of higher dimension is favored
with increasing intermolecular interactions. In the case of 2D-2D and
3D-2D interfaces, the intermolecular interactions (covalent bond) can
be much greater than the molecule-2D interactions (vdW).
%
The molecule-substrate interactions are constantly
weaker than the other two types of interactions, as a result of
increased interaction distance and partial screening effect of the 2D
material layer. Understanding the role of weak interactions in
determining the molecular epitaxy poses a challenge towards
comprehensive theory framework, and is crucial for the designing of
epitaxial systems on 2D materials interfaces.


\begin{table}[!htbp]
  \centering
  \caption{\label{tbl:intro-summary-multidimension}%
    Summary of representative examples of multidimensional molecular
    epitaxy on 2D materials interfaces, the governing interactions
    involved and the comparison between the interactions discussed in
    this section.               %
  }
  \small
\begin{tabularx}{1.0\textwidth}{XXXX}
\hline
Dimensionality & Examples & Governing Interaction(s) & Comparison between Interactions\\
\hline
0D-2D & Strongly interacting surface & Site-specific & Molecule-2D \(\gg\) intermolecular\\
\hline
1D-2D & Strongly interacting surface & H-bond,  site-specific & Molecule-2D \(\approx\) intermolecular\\
 & Weakly interacting surface & Multivalent H-bond & Intermolecular  \textgreater{} molecule-2D\\
\hline
2D-2D & ML on graphene & (H-bond), CT, \(\pi\)-\(\pi\) & Molecule-2D > intermolecular\\
 & ML on TMDC & (H-bond), CT, vdW & Intermolecular  \textgreater{} molecule-2D\\
 & 2D vdWE & Covalent bond,   vdW & Intermolecular  \(\gg\)  molecule-2D\\
\hline
3D-2D & LbL assembly & (H-bond), CT, \(\pi\)-\(\pi\), vdW & Depending on the 2D material\\
 & 3D vdWE & Covalent bond,   vdW & Intermolecular \(\gg\) molecule-2D\\
\hline
\end{tabularx}
\end{table}




\section{Open Questions  Concerning 2D Interfaces}
\label{sec:chall-probl-conc}

As discussed in \autoref{sec:2d-mater-interf}, our understandings on
the surfaces of 2D materials have greatly advanced in the past decade
thanks to the development of novel experimental techniques as well as
theoretical frameworks.
%
However, there are still several open questions concerning the
interfacial properties that are not yet well-understood. Many of them 
are related with the electromagnetism on these low-dimensional
interfaces, including the electrostatic penetration and screening
through 2D materials, dynamic dielectric properties of 2D material and
heterostructures, many-body exciton effect, and vdW interactions.
%
These phenomena covers broad length scales, ranging from atomistic
interactions to macroscopic wetting behavior.
%
This section aims to give a brief introduction to several selected
topics, which serves as the prelude for the research
work presented in this thesis.
%
It is worth noting that these open questions are not caused by lack of
fundamental theory framework, but rather because current physical
relations in bulk materials can be different in these
low-dimensional systems.

\begin{figure}[h]
  \centering
  \import{\imgdir}{questions.pdf_tex}
  \caption{\label{fig:intro-questions}%
    Schemes about the open questions for the 2D interfaces, including
    \textbf{a} penetration of field, \textbf{b} dielectric properties
    and \textbf{c} vdW \& wetting phenomena. }
\end{figure}


\subsection{Electrostatic Interactions Through 2D Sheet}
\label{sec:electr-inter-thro}

% 20191009-1324
The electrostatic (Coulombic) interactions are long-range forces that
can be extended to several nano\-meter or even micro\-meter scale~\cite{Lacava_2016-electrodyn}.
%
On the contrary, with only one to few atoms perpendicular to the basal
plane, the thickness of 2D materials can be much smaller than the
length scale of electrostatic interactions.
%
As a consequence, a 2D material cannot fully screen the electric field
(\ie the 2D layer is partially ``transparent'' to electrostatic
interactions).
%
In bulk systems with mobile carries (electrons / holes in solid-state
semiconductor, or ions in electrolyte solution), the characteristic
length of electrostatic interactions is described using the Debye
length $\lambda_{\mathrm{D}}$, which is defined as:
\begin{equation}
  \label{eq:intro-debye}
  \lambda_{\mathrm{D}} = {\displaystyle \sqrt{
      \frac{\varepsilon_{0} \varepsilon_{\mathrm{s}} k_{\mathrm{B}} T}
      {e^{2} n_{\mathrm{s}}}
    }}
\end{equation}
where $\varepsilon_{0}$ is the vacuum permittivity,
$\varepsilon_{\mathrm{s}}$ is the relative permittivity of the system,
$k_{\mathrm{B}}$ is the Boltzmann constant, $T$ is the temperature,
$e$ is the unit charge, and $n_{\mathrm{s}}$ is the overall carrier
density.
%
However, such definition may be incorrect for a 2D material due to
(i) the carrier density of a bulk system cannot be readily applied to
a 2D sheet, and (ii) the electronic structures of 2D materials are
completely ignored.
%
Indeed, theoretical investigation shows a more complex picture of the
electrostatic interaction through 2D materials and their stacks.
%
Using a discrete model based on total energy optimization and taking
into account of graphene's Dirac cone structure, Kuroda et
al. demonstrate a non-linear behavior of the effective screening
length $\lambda_{\mathrm{eff}}$ in multi-layer graphene
(MLG)~\cite{Kuroda_2011_PRL_ML,Kuroda_2011_ML_gr,Rokni_2017_charge_ML_Gr}.
%
Unlike $\lambda_{\mathrm{D}}$ in bulk systems which decays with
$n_{\mathrm{s}}^{-1/2}$, $\lambda_{\mathrm{eff}}$ does not show a
single power law with the doping density, and more interestingly,
$\lambda_{\mathrm{eff}}$ saturates when doping density decreases to
zero, in contrast with $\lambda_{\mathrm{D}} \to \infty$ that would be
expected in bulk systems.
%
The complexity of the electrostatic screening through graphene and
other 2D materials has also been shown in both theoretical and
experimental
studies~\cite{Datta_2009_ML_Screening,Uesugi_2013_EDL_ML,Goto_2013_ML_Gr,Liluhua_2014_hbn}.
%

The partial penetration of electric field is
the fundamental mechanism behind several 2D-material-based electronic
devices such as the graphene barristor and vertical field effect
transistor (VFET)~\cite{Yang_2012_Barristor,Shih_2015_PartiallyScreened}.
%
In these devices, the electric field from a dielectric layer
penetrates graphene (or other 2DEGs) and changes the interfacial
states at the semiconductor/2DEG junction, to enable on/off switch of
current passing through the semiconductor.
%
The operation of such devices is widely thought to be
modulated by the interfacial transport
barrier~\cite{Yang_2012_Barristor,Zhong_2014_SB,Dankert_2017_graphene_spin_SB}.
However, this cannot fully explain other features such as
selective carrier injection at the
semiconductor/\allowbreak{}2DEG~\cite{Shih_2015_PartiallyScreened}
%
Moreover, the influence of the penetrated field on the adjacent
semiconductor is usually ignored.
%
Therefore, a complete theoretical framework taking account of the
previous concerns is of high demand.

An even more challenging issue is the penetration of electric field
through 2D vdWHs. The non-linear electrostatic screening observed for
homogeneous 2D stacks can be more complex, hence combining 2D material
build blocks with distinct electronic properties and stacking order.
%
A few theoretical efforts have been made to elucidate the influence of
external electric field on the 2D
vdWHs~\cite{Santos_2013_tunable_eps_gr,Santos_2013_ACSnano_kaxi,Lu_2017_ami_electro_doping}.
%
However, such studies usually employ full-scale first principle
simulations, which is time-consuming and is hard for up-scaling
studies.
%
Simplified models that can capture more insights are therefore of high
interest, in order to help understand the exotic electrostatic behaviors of 2D
vdWHs and to provide guideline for device design.


\subsection{Dielectric Properties of 2D Systems}
\label{sec:diel-prop-2d}

% 20191009-1745
When an insulating material is placed under external electric field
$\mathscr{E}_{\mathrm{ext}}$, the electron cloud is distorted from its
equilibrium state and creates induced dipoles, a process known as
dielectric polarization. The induced dipoles screen the external
electric field and causes the electric field inside the material,
$\mathscr{E}$ to be smaller than $\mathscr{E}_{\mathrm{ext}}$.
%
Such response of a material to a dynamic electric field is
characterized by the dielectric function $\varepsilon$ (also known as
relative permittivity).
%
$\varepsilon$ is defined as the ratio between the magnitudes of
$\mathscr{E}$ and $\mathscr{E}_{\mathrm{ext}}$ and has a complex value.
%
It plays a central role in electromagnetism and is connected to
various physical quantities, including optical
absorption~\cite{Dressel_2001_electrodynamics}, electric conductivity,
electric capacitance, impurity energy level~\cite{Ihn_2009_book},
Debye screening length~\cite{Israelachvili_2011_book} and vdW
interaction coefficient~\parencite{parsegian_van_2010_book}.
%
In the context of 2D materials, $\varepsilon$ is widely used to
interpret several energy scales, in particular the exciton binding
energy of direct-gap TMDCs and hBN.
%
However, values of $\varepsilon$ for a certain material in literature
usually has discrepancy of 1$\sim{}$2 orders of
magnitude~\cite{Li_2016_screening_rev}.
% \worktodo{cite paper 9-13 from Elton ACS Nano, and Li Chem Soc Rev}
%
Such inconsistency causes even worst estimation of corresponding
energy levels (for instance, the exciton binding energy is usually
assumed to be proportional $\varepsilon^{-2}$).
%
Even in the theoretical studies, the dielectric screening of 2D
materials are still under huge debate.  The $\varepsilon$ values
reported for a 2D material range from near
unity~\cite{Olsen_2016_hydrogen} (\ie no dielectric screening) to
almost equal to $\varepsilon$ in its bulk
counterpart~\cite{Laturia_2018_2D_eps} (\ie no difference between 2D and 3D
cases).
%
Moreover, whether the quantum confined structure causes anisotropy in
the $\varepsilon$ of a 2D material~\cite{Sohier_2016_2D_eps}, is still
questionable.
%
Clearly, precisely determining the dielectric properties of 2D
materials is critical to build correct physical models of these
low-dimensional materials.

%
% 20191010-0947
There are also
complexities even beyond the disputed fundamental definition of the dielectric
properties of a 2D material.
%
The electronic structure of a 2D material or vdWH can be greatly
influenced by external electric field, such as modulating the
bandgap~\cite{Kumar_2016_PRB,Kumar_2016_jpcc,Li_2014_phosphorene_Efield},
and even inducing transition from parabolic dispersion to Dirac cone
in the electronic band structure~\cite{Liu_2015_aniso_dirac}.
%
Similarly, the dielectric properties of 2D materials and vdWHs are
also shown to be electric-field-dependent from first principle
simulations,
%
including multilayer stacks of
graphene~\cite{Santos_2013_tunable_eps_gr},
TMDCs~\cite{Santos_2013_ACSnano_kaxi}, GaS~\cite{Li_2015_GaS}, phosphorene~\cite{Kumar_2016_PRB}, and
heterostructure of hBN/graphene~\cite{Kumar_2016_jpcc}.
%
A commonly
observed trend is the nonlinear increasing of the dielectric response
under stronger fields.
%
Such phenomena have clearly different origin compared with the
field-dependent $\varepsilon$ observed in some bulk para- and
ferro-electric materials, which exhibit decreasing
$\varepsilon$ under stronger electric
field~\cite{Hemberger_1995_bulk_dielectric_efield,Maiti_2006_bulk_eps_batio3}.
%
Some studies attribute that the field-dependent dielectric properties
of 2D materials and vdWHs to the change of electronic band structure
and potential insulator-conductor transition, while the exact
description and predictions for other materials combinations, are
still missing.
%

\subsection{Van der Waals Interactions and Wetting Phenomena}
\label{sec:van-der-waals}

% 20191010-1055
As discussed in \autoref{sec:intro-mol-subst}, with sun-nanometer layer
thickness of only few \AA{}, the ultra\-thin 2D material sheets may
not only be partially transparent to long-range electrostatic
interactions, but also vdW forces.
%
The partial penetration of vdW interactions through a 2D material
sheet is in principle possible since the thickness (e.g. $\sim{}$3.3
\AA{} for graphene) is still shorter than the effective length scale
of vdW interactions~\cite{Israelachvili_2011_book}.
%
In literature, such hypothesis is usually examined by two approaches:
the liquid wettability or solid-state molecular epitaxy on supported
2D material surfaces.
%
Although many studies claim the validity of the vdW- or
wetting-transparency through 2D materials, the interpretations of
experimental data usually vary greatly.
%
For instance, Rafiee et al reported that the liquid contact angle
(CA) on graphene supported by gold or silicon substrates would be
almost identical to that observed on bare substrates, leading to the
claim of complete wetting transparency of
graphene~\cite{rafiee_2012_transparency}.
%
However, this analysis is later shown to be inaccurate due to
ignoring the liquid-graphene interactions, as discussed in
\autoref{sec:inter-forc-at}.
%
A more complete theoretical model developed by Shih et
al.~\cite{Shih_2012_prl}  based on the pair-wise
vdW interaction and Boltzmann distribution of liquid molecules
predicts that the contact angle on substrate-supported graphene, would
be saturated below 100$^{\circ}$, which renders the graphene sheet as
``wetting translucent''.
%
The controversy of ``wetting transparency'' or ``wetting
translucency'', is however not completely settled.
%
The cause of the dilemma is two-fold: (i) the wetting properties on
graphene and other 2D materials, depend greatly on the experimental
conditions, and (ii) the physical model describing the interfacial vdW
interactions can be incomplete. We will briefly extend on these topics
in following paragraphs

%
In principle, the intrinsic wettability of a 2D material, should be
measured using suspended sheets to exclude effects from the underlying
substrate.
%
However, in reality this is extremely hard to achieve due to the large
mismatch between the size of a droplet using conventional contact
angle goniometer ($>$0.1 mm) and the largest size of suspended 2D
material so far ($<$10 μm)~\cite{Zhang_2017_transfer_suspended}.
%
As a result, most reported values of water contact angle of graphene
are still performed on substrate-supported samples, with CA ranging
from below 50$^{\circ}$ to almost
180$^{\circ}$~\cite{Kozbial_study_2014_gr_wetting,Raj_2013_wetting_rev,Wang_2009_wettability}.
%
In addition, the transfer process of 2D materials usually leaves
polymer residues and other defects on the surface, which are also
shown to affect the CA on 2D material surfaces~\cite{Kozbial_2015_wetting_mos2}.
%
Moreover, even if polymer- and defect-free 2D material samples can be
obtained, the CA is also shown to be influenced by air-born
hydrocarbon contaminations under ambient, causing a
hydrophilic-to-hydrophobic transition that ubiquitously found in
graphene~\cite{li_2013_airborne} and TMDC~\cite{Chow_2015_wetting_WS2}
samples.
%
Clearly, any theoretical model dealing with liquid wetting on 2D
material interfaces that makes use of the ``intrinsic'' vdW
interaction coefficients, would be affected by the
large uncertainty of reported wettability.
%
On the contrary, the above issues concerning contamination can be
effectively minimized in molecular epitaxy, which is usually operated
under high vacuum and annealed conditions~\cite{Koma_1985_vdWE}.
%
%
The studies using molecular epitaxy usually compare the morphology and
nucleation density of target molecules deposited onto various 2D
materials supported by different substrates~\cite{Kratzer_2016_6P_gr_trans,Nguyen_2015_pent_gr_wett}.
%
Although various studies demonstrate the validation of vdW
transparency through molecular epitaxy, the conclusions are in general
indirect and only limited at qualitative level.
%
Moreover, as described in \autoref{sec:intro-3D-2D}, the nanoscale 3D
molecular epitaxy is a collective result of both thermodynamic and
kinetic processes, making the analysis more difficult than the case of
macroscopic wetting phenomena.


Another open question is whether our description of interfacial vdW
interactions is satisfactory. For instance, most of the theories
about vdW
transparency~\cite{Shih_2012_prl,Kim_2015_wetting_controversy} assumes
the vdW interactions between the substrate and liquid phase is reduced
due to the thickness of the 2D material layer.
%
The accuracy of such assumption may be questionable: using accurate atomic
force spectroscopy, Tsoi et al. showed the molecule-substrate vdW
interaction coefficient (the Hamaker
constant~\cite{parsegian_van_2010_book}), actually changes by varying the
2D material\cite{Tsoi_2014_vdW_screening_2D}.
%
Such many-body effect is not considered in current frameworks of
ground-state density functional theory (DFT) and molecular dynamics
(MD), making studying of such effects particularly difficult.
%
Moreover, the low DOS near the intrinsic Fermi level
~\cite{Das_Sarma_2011_electron_gr,Bhimanapati_2015_2D_rev} of 2D
materials ensure they be easily doped by either substrate-2D material
interactions
~\cite{Varchon_2007_elec_struc_gr_SiC,Giovannetti_2008_doping,Chen_2013_doping}
or an electric displacement
field~\cite{Das_2008_doping,Perera_2013_doping}.
%
Several works have indicated the potential influence of 2D layer
doping with the change of interfacial wettability and molecular
orientation~\cite{Huttmann_2015_vdw_gr_doping,Nguyen_2019_PEN}.
%
It is possible that the doping effect can be important for several
experimental systems such as wetting
graphene\allowbreak{}/\allowbreak{}metal
surface~\cite{Giovannetti_2008_doping,Pi_2009_metal_doping_gr} .
%
Therefore, in order to precisely determine the vdW transparency , one
needs to decouple it from the substrate-induced doping effect of 2D
materials, as addressed above
\cite{Huttmann_2015_vdw_gr_doping,Muruganathan_2015_tunable_vdw_gr,Hong_2016_mechanism,Ashraf_2016_doping}.


\section{Scope of the thesis}
\label{sec:scope-thesis}

The thesis summaries the author's work of understanding and
engineering the properties of 2D material interfaces through
multiscale modeling and experimental demonstrations. Particular effort
has been made to elucidate several open questions as described in
\autoref{sec:chall-probl-conc}. The rest of the thesis is organized
into three different parts:

\autoref{part:electr-2d-mater} focuses on the fundamental
understandings of 2D materials interfaces under static electric field,
in which governed the quantum capacitance ($C_{\mathrm{Q}}$) of 2D
materials plays an critical role.
%
Two studies are
presented in this part:
%

\autoref{ch:qc} uses multiscale modeling to study the field effect
transparency of a 2DEG. A non-linear dependency of the field effect
transparency index $\eta^{\mathrm{FE}}$ on $C_{\mathrm{Q}}$ and other parameters are
studied in order to guide practical design of 2D material-based
vertical field effect devices.
%

\autoref{ch:asym} Further extends the electrostatic model to
multilayer 2D materials for studying penetration of electric field
through 2D vdWHs. The model captures the experimentally observed
asymmetric electrostatic screening with similar accuracy compared with
full-scale first principles simulations.

\autoref{part:diel-prop-2d} provides fundamental understandings of the
dielectric properties of 2D materials interfaces under both static and
dynamic fields. Two examples are demonstrated in this part:
%

In \autoref{ch:diel}, the 2D electronic polarizability
$\alpha_{\mathrm{2D}}$ is proposed to be the replacement with
$\varepsilon$ to uniquely define the dielectric properties of 2D
materials. Using high-throughput screening based on first principles
simulations, two universal scaling relations for
$\alpha_{\mathrm{2D}}$ are proposed, linking the polarizability of a
2D material with its electronic and structural information.
%
The idea
of $\alpha_{\mathrm{2D}}$ can also be applied for heterostructures
and even bulk systems, allowing quantifying the dielectric anisotropy
for any dimension.
%

\autoref{ch:vdw} takes advantage of the electronic polarizability
in frequency domain, to study the many-body vdW interactions at the 2D
interfaces based on a modified Lifshitz-vdW
formalism. The high dielectric anisotropy of a 2D material selectively
screens the vdW interactions at low frequency regime. More
interestingly, by proper engineering dielectric properties of 2D and
bulk materials, repulsive vdW interactions are predicted by the model,
and validated by experimental investigating using molecular epitaxy.

\autoref{part:mult-probl-conc} Based on these fundamental understandings, several
studies on multiscale phenomena at the 2D materials interfaces are
shown in the last part.
%

\autoref{ch:wet} combines multiscale phenomena including quantum capacitance,
interfacial molecular reorientation, electrical double layer (EDL) to model
the wetting phenomena of a 2D material upon doping are. The
molecular reorientation effect is found to dominate the 2D-liquid
interfacial tension.
%

\autoref{ch:np} aims to the understand ionic transport through
electrostatically gated nano\-porous graphene sheet by
self-consistent transport theory based on Poisson-Nernst-Planck
equation and the quantum capacitance, Gating is found to enable
close-to-unity rejection of ionic species, which is in good agreement
with experimental observations.
%

In \autoref{ch:small} multiscale physical phenomena on 2D material
interfaces is combined to design and fabricate a novel electronic
device named as interfacial field effect transistor (IFET). The IFET
has ultra-sensitive pressure response down below 10 Pa, due to extremely
low elastic modulus of liquid metal droplet. The mechanical response
is harnessed by deformation on superhydrophobic nanowires assembled on
graphene interface.

As a conclusion, \autoref{part:conclusions} summaries the achievements
in this thesis, as well as the remaining open questions concerning 2D
material interfaces. Suggestions for further theoretical and
experimental studies are also proposed.


%%% Local Variables:
%%% mode: latex
%%% TeX-master: "../thesis"
%%% End:
