% \chapter{Introduction}
\chapter{Two-Dimensional Materials and Interfaces}
\label{ch:introduction}
\newcommand*\imgdir{img/ch-intro/}

\dictum[Wolfgang Pauli]{%
  God made the bulk;\\Surfaces were invented by the devil
  }%

\vspace{1em}

\chapterabstract{Part of this chapter appears in the following
  journal article: Tian, T. \& Shih, C.-J. Molecular Epitaxy on
  Two-Dimensional Materials: The Interplay between
  Interactions. Ind. Eng. Chem. Res. 56, 10552--10581 (2017).  }


\section{Overview of Two-Dimensional (2D) Materials}
% \section{Introduction}
\label{sec:ch-intro-2D}
Controlling the dimensionality of materials provides rich
opportunities of tuning the electronic, optical and mechanical
properties which facilitates novel devices and applications.
\worktodo{put more things inside and make claim clear} \worktodo{put
  1-3 refs}. Good example of such material design are the
two-dimensional (2D) materials, which are covalently-bonded
crystalline films with only one- or few-atom thickness. \worktodo{HH??
  Change wording}
%
Starting from the first discovering of graphene, an allotrope of
carbon in 2004\worktodo{cite Geim}, the family of 2D materials has
expanded significantly throughout the last decade. \worktodo{cite
  Novoselov and 2 papers} The electronic properties of 2D materials
cover a wide spectrum ranging from insulators to superconductors,
making them promising candidates to replace conventional bulk
materials in semiconductor industries. \worktodo{cite 1-2}.
%
This section aims to give a brief introduction about the structures,
electronic properties and fabrication techniques, which serves as a
prelude for introducing the 2D materials interfaces.
\worktodo{again, change last part}

% Understanding and engineering the
% interfaces of 2D materials are important to achieve practical uses of
% 2D materials in our 3D world. In this chapter, we briefly discuss the
% fundamental aspects of 2D materials and introduce their interfacial
% phenomena. Several challenges about the 2D materials interfaces are
% also discussed, serving as the prelude to the work of this thesis.


\subsection{Categories of 2D Materials}
\label{sec:categ-2d-mater}

So far, there have been no naturally-existing isolated 2D
materials. The closest form is the layered bulk materials, in which
interlayer interactions are governed by van der Waals (vdW) forces.
%
With proper separation techniques, these layered materials can be
thinned to a few layers or even single layer.
%
The best-known example is the mechanical exfoliation of 2D layer from graphite,
now known as graphene. \worktodo{cite geim}
%
Using such technique, monolayers exfoliated from naturally-existing
layered bulk materials were achievable. Examples include the hexagonal
boron nitride (hBN), transition metal dichalcogenides (TMDCs, with
general chemical composition MX\textsubscript{2}, where M is a
transition metal and X is S, Se or Te), monochalcogenides (GaS, GaSe),
monolayer black phosophorus (BP) and MXene (M = Ti, Nb, V, Ta, etc.,
X = C or N).
%
The family of 2D materials that have been experimentally discovered,
has extended from only 1 in 2004 to over 100 \worktodo{really?} as of
2019, corresponding to an average discovery of about 6 new materials
each year. \worktodo{correct?}
%
Apparently, the known 2D materials are still scarce and limited
compared with the bulk materials,
due to the huge experimental effort
required to synthesize, isolate and characterize 2D materials limit
the speed of novel 2D material discovery.
% 
In view of this, computer-aided material design (CAMD) was
demonstrated in several works to help find new 2D materials either
from high-throughput materials database screening or \textit{ab
  initio} calculations. As a pioneer study, Lebègue et al. performed
data mining on the crystal structures listed in the International
Crystallographic Structural Database (ICSD) to filter out a total
number of 92 layered bulk crystals and to generate corresponding 2D
materials. As an extension, Mounet et al. proposed new algorithm with
more robust criteria for dimensionality and found over 5000 layered
materials, among which over 1800 can be easily or potentially
exfoliated according to \textit{ab initio} calculations.
%
Another example is the combinational design of new 2D materials, which
starts with a known structure (for example MX\textsubscript{2}) and
replace with other elements. This allows the discovery of over 3700
thermo-dynamically stable 2D materials, among which most chemical
compositions are not yet discovered experimentally.

These computational studies broaden our knowledge about 2D
materials. In particular, the large size of database allows
systematical study to search for optimal 2D material according to
their electronic and optical properties, and apply to real-world
applications. As a summary, the most common types of 2D materials from
either experimental or computational discoveries, are listed in
\autoref{tab:category-2D}.

\begin{table}
  \centering
  \caption{Summary of common 2D materials and their structures. The
    formula refer to the chemical composition per unit cell.}
  \label{tab:category-2D}
  \begin{tabularx}{1.00\textwidth}{XXXX}
    \hline
    Prototype  & Example Formula  & Lattice Symmetry & Example Materials \\
    \hline
    Graphene & C\textsubscript{2} &  P6/mmm & Graphene, Silicene, Germanene \\
    Graphane & C\textsubscript{2}H\textsubscript{2} &  P3m1 & Graphane, Fluorographene\\
    hBN      & BN                & P$\overline{3}$m2 & h-BN \\
    2H-MX\textsubscript{2} & MoS\textsubscript{2} & P$\overline{3}$m2 & 2H-MoS\textsubscript{2}, 2H-MoS2\textsubscript{2}, 2H-WS\textsubscript{2} \\
    1T-MX\textsubscript{2} & CdI\textsubscript{2} & P$\overline{3}$m1 & 2H-MoS\textsubscript{2}, CdI\textsubscript{2}\\
    BP & P\textsubscript{4} & Pmna & Phosphorene, Arsenene \\
    Monochalcogenidec & Ga\textsubscript{2}S\textsubscript{2} & P$\overline{3}$m2 & Ga\textsubscript{2}S\textsubscript{2}, Ga\textsubscript{2}Se\textsubscript{2} \\
    Bismuth iodide &  Bi\textsubscript{2}I\textsubscript{6} & P$\overline{3}$m1 & Bi\textsubscript{2}I\textsubscript{6}, Al\textsubscript{2}Cl\textsubscript{6} \\
    Iron oxychloride                &  Fe\textsubscript{2}O\textsubscript{2}Cl\textsubscript{2} & Pmmn & Fe\textsubscript{2}O\textsubscript{2}Cl\textsubscript{2}, Fe\textsubscript{2}O\textsubscript{2}Br\textsubscript{2}  \\
    MXene & Ti\textsubscript{x}C\textsubscript{y} & P$\overline{3}$m2 & Ti\textsubscript{3}C\textsubscript{2}, Ti\textsubscript{4}N\textsubscript{4}, Mo\textsubscript{2}TiC\textsubscript{2} \\
    2D Perovskite & CH\textsubscript{3}NH\textsubscript{3}PbBr\textsubscript{3} & Pm$\overline{3}$m & (MAPbBr\textsubscript{3})\textsubscript{n}\\
   \hline
\end{tabularx}
\end{table}

\subsection{Basic Electronic Properties of 2D Materials}
\label{sec:basic-electr-prop}

The common feature of 2D materials is the absence of dangling bonds at
the surface, as a results, the electrons are highly confined within
the 2D plane.
%
For instance, in graphene and hBN, the sp\textsuperscript{2} orbitals
form covalent (σ) bonds, while the p-orbitals perpendicular to the 2D
plane form delocalized π-electron cloud.
%
Such two-dimensional
electron gas (2DEG) gives rise to dramatic change of its electronic
properties compared with bulk materials.\worktodo{cite Devies}
%
Although the physics about 2DEG has been well developed in the 1980s,
prior to the discovery of 2D materials, the 2DEG can only be achieved
by cumbersome semiconductor quantum well heterostructures. In other
words, the 2D materials are perfect candidates to study the behavior
of 2DEGs. \worktodo{say something more?}

Although the electronic band structures of different 2D materials vary
a lot, one thing in common is that their density of states (DOS). The
DOS is a measure of the number of available states in a certain system
at certain energy level. For an extended system with nearly-continuous
energy distribution, the DOS at energy $E$ is expressed as:
\begin{equation}
  \label{eq:ch-intro-dos}
  \mathrm{DOS}(E) = \frac{\partial N}{\Omega \partial E} =  \frac{1}{\Omega} {\displaystyle \int_{\Omega}} \frac{\mathrm{d}^{n} \mathbf{k}}{(2 \pi)^{n}}
  \delta(E - E(\mathbf{k}))
\end{equation}
where the integral is performed over a system with $d$-dimensionality,
$N$ is the total number of states,
$\Omega$ is the volume of the system, $\mathbf{k}$ is the momentum of
electron, and $\delta$ is the Dirac delta function. Without losing
generality, the DOS can be expressed using the law of chains:
\begin{equation}
  \label{eq:dos-chain}
  \mathrm{DOS} = \frac{\partial N}{\Omega \partial k} \frac{\partial k}{\partial E}
               = \frac{k}{2 \pi} \left(\frac{\partial E(k)}{\partial k}\right)^{-1}
\end{equation}
where $k=|\mathbf{k}|$ is the modulus of $\mathbf{k}$.
%
\autoref{eq:dos-chain} provides a general way linking the DOS to the
energy--momentum ($E-k$) dispersion of a 2D material, which is further
extracted from its band structure. For monolayer 2D materials, two
cases can be distinguished:
\begin{itemize}
\item Parabolic materials: such as TMDCs, hBN, phosphorene

  These are the majority of 2D materials where the $E-k$ dispersion is
  parabolic, such that
  ${\displaystyle E(k) = \frac{\hbar^{2} k^{2}}{2 m^{*}}}$, where
  $\hbar$ is the reduced Planck constant, and $m^{*}$ is the effective
  mass near the band edge. From \autoref{eq:dos-chain}, DOS of a
  parabolic material is \textit{constant}, and proportional
  to $m^{*}$.
  
\item Dirac materials: such as graphene, silicene, germanene

  The Dirac materials are relatively rare among 2D materials,
  \worktodo{cite Wang nsr 2015} where the $E-k$ dispersion is linear:
  $E(k) = \hbar v_{\mathrm{F}}k$, where $v_{\mathrm{F}}$ is the Fermi
  velocity. The Dirac materials has DOS increasing linearly with $k$
  (as well a $E$).
\end{itemize}

The difference between the DOS of ideally parabolic and Dirac
materials can be seen in \worktodo{Figure here}. Unlike parabolic
materials with constant DOS, the Dirac materials have DOS → 0 near the
Dirac cone ($E \to 0$). This feature brings very interesting
properties like electrostatic transparency which we will discuss in
\worktodo{Chapter 2}.

\worktodo{Discuss more about the parabolic and Dirac cone?!}
\worktodo{Some simple discussion based on the band gap etc...}


\subsection{Fabrication of 2D Materials}
\label{sec:fabr-2d-mater}

The development of the 2D material researches cannot be achieved
without appropriate methods to fabricate large-area and high quality
2D materials. The synthesis of 2D materials can be generally
categorized into top-down and bottom-up approaches.

\subsubsection{Top-down Methods}
\label{sec:top-down-methods}

The top-down method is the straightforward approach to exfoliate few-
to single-layer 2D materials from their bulk counterparts. The two
most-used techniques are micro-mechanical exfoliation and liquid-phase
exfoliation. \worktodo{add cites here, from Liu Adv Mater 2018}

The micro-mechanical exfoliation uses mechanical force to overcome the
interlayer vdW interactions in bulk layered materials, and is the
first method used for 2D material exfoliation. Scotch tape is widely
used for such type of exfoliation \worktodo{cite Geim et al}, while
other media as elastic polymers, heat-release tapes are also
employed. If the quality of the bulk layered material is promising
(\ie high chemical purity and low defect density), micro-mechanical
exfoliation usually produces 2D materials with better quality compared
with other methods. However, there are also several key drawbacks of
such methods. First of all, the reliability of micro-mechanical
exfoliation is highly influenced by the interlayer forces, which makes
it difficult to exfoliate materials with high interlayer binding
energy, or in-plane mechanical anisotropy (for instance BP). This
method also leads to broad distribution of flake size and layer
number, making it difficult to be applied to large-scale
applications.

In contrast to micro-mechanical exfoliation where only the top-most
layers are removed, the solution-phase exfoliation method ensures
uniform breaking up of bulk materials, and increases the scalability
of 2D flakes produced. It can be achieved by both physical and
chemical exfoliation approaches. Physical solution-phase exfoliation
makes use of local mechanical stress produced by ultra-sonication or
shearing to overcoming the interlayer vdW interactions. To ensure the
stability of isolated layers in the liquid suspension, high surface
energy liquid, in combination with surfactants are usually
used. \worktodo{cite papers from Coleman et al, Shih et al. Haung et
  al}
%
Centrifugation can be further used to separate flakes to ensure narrow
distribution of layer thickness and size \worktodo{cite coleman}, and
statistic measurement can be performed using the ensemble of such 2D
material suspensions. \worktodo{wording?} However, this method still
suffers from several drawbacks: (i) the flake size is still limited to
μm-scale, (ii) precise control of layer number in suspension is difficult and 
(ii) flake overlay after deposition onto substrate is unavoidable.
%
The interlayer vdW forces can also be overcome by chemically modifying
the 2D layer (for instance, oxidising graphite to produce graphene
oxide (GOx)), or to intercalate ions between the layers in a bulk
material to induce lattice expansion (such as exfoliation of MXenes,
which are otherwise hard to achieve mechanically). Despite the
scalability of chemical solution-phase approaches, they usually
changes the chemical composition and introduce defects in the 2D
materials, which are undesired for high-performance
applications. 

\subsubsection{Bottom-up Methods}
\label{sec:bottom-up-methods}

The bottom-up methods grow 2D materials from precursors, and aims to
produce 2D materials with larger flake size and more controlled layer
numbers. Depending on whether a substrate is involved in the process,
these methods can be categorized into templated or non-templated
growth.

Templated growth uses a bulk surface as a epitaxial template and
support for the 2D material. Due to the absence of dangling bonds, the
2D-substrate interaction is usually much weaker than the in-plane
covalent bonds. Therefore, unlike epitaxy of bulk materials which
require precise control of lattice mismatch between the substrate and
epitaxy layer \worktodo{cite 1-2 paper}, templated growth of 2D
materials can be achieved in various substrates. For example, high
quality single crystal graphene can be epitaxially grown by thermal
annealing of silicon carbide (SiC) surface, or decompositing
hydrocarbons on Ruthenium (Ru) (0001) and Ir (111) substrates, while
epitaxial growth of hBN is achieved by cleavage of borazine
(H\textsubscript{6}B\textsubscript{3}N\textsubscript{3}) on Rh (111)
surface.  Such epitaxial method can also be applied to other 2D
materials like TMDC, monochalcogenides and BP. However, there are
still several limitations. First of all, a clean surface as well as
ultra-high vacuum (UHV) conditions are usually required for the
epitaxial growth methods. Moreover, transferring the 2D material onto
other substrates is generally not easy due to the noble metals
involved. Chemical vapor deposition (CVD) is another widely-used
templated growth method, in which one or more precursors adsorb and
react on a catalytic surface to form covalently-bonded 2D
materials. The essence of CVD process is similar to the epitaxial
growth, while more ambient conditions are used (10$^{1}$ Pa to
atmosphere pressure, ultra clean substrate not necessary). CVD growth
of graphene using hydrocarbon source on copper (Cu) is the most
studied and widely used technique. The self-termination of
second-layer on the Cu surface allows the growth of large area single
layer domains up to centimeter or decimeter scale \worktodo{check if
  explanation is correct}. Another advantage of such method is easy
removal of Cu substrate by standard etching procedure, allowing
transferring graphene onto a large variety of
substrates. \worktodo{cite 1-2} Similar to the case of graphene, large
area single-layer TMDCs and hBN can also be achieved using the CVD technique by proper interfacial engineering. \worktodo{cite TMDC paper; hBN show the Gold sample}
%
% With proper defect control during growth and development of novel
% transferring techniques, the CVD method is promising to 

While the majority of bottom-up growth relies on a substrate to form
the 2D material, there are also cases where colloidal 2D confined
structures can be directed synthesized in liquid phase without
template. Examples of 2D materials grown using the colloidal method
include II-VI semiconductors (CdSe), 2D hybrid perovskites, and
TMDCs. The anisotropic growth is usually modulated by surfactant /
ligand engineering. Recently, colloidal insulating 2D metal oxides are
reported to be synthesized by simultaneous oxidation at the liquid
metal-water interface, further extending the possibility of bottom-up
synthesis of 2D materials. \worktodo{cite Dicky paper}

As a summary, both top-down and bottom-up methods are capable of
fabricating 2D materials with desired purity, flake size and
thickness. A short comparison between different fabrication methods can be seen in \worktodo{table 2?!}



\section{The 2D Materials Interfaces}
\label{sec:2d-mater-interf}
%20191008-0939
the 2D materials do not solely attract the research interests due to
the unique electronic properties, they are also materials with ultra
high surface-area-to-volume ratio intrinsically. As a consequence,
interfaces are almost always required when integrating the 2D
materials into experimental studies and applications in the 3D world,
and the interfacial properties play important roles in determining
their proprieties. For instance, the existence of strict 2D lattices
is long questioned due to the presumed distortion caused by thermal
fluctuation which would break up long-range order. Although recent
studies suggest free-standing 2D materials like graphene can be
stabilized by the phonon coupling that causes 3D ripples in the 2D
layer (\worktodo{find 2 papers to cite}), the majority of studies
still require the 2D materials to be supported by substrate or
encapsulated.  As will be shown later, these 2D interfaces may
significantly alter the intrinsic properties by ways such as
structural corrugation and carrier doping. On the other hand, it
remains unrealistic to find a single 2D material which can satisfy all
the requirements concerning high-performance applications (e.g.,
electronic properties, mechanical strength, chemical stability, and
synthetic difficulty), the flexibility of creating mixed-dimensional
interfaces with existing functional materials may offer opportunities
to fully exploit their potential. Playing with the interfacial
interactions is critical for successful engineering of the interface
dimensionality, morphology, electronic states and transport phenomena.
In this section, we focus the discussion on the interactions involved
at the 2D materials interfaces, and how the interplay between these
interactions influences the mixed-dimensional interfaces with 2D
materials \worktodo{say more?}

\subsection{Interactions and Forces at the 2D Materials Interfaces}
\label{sec:inter-forc-at}

The concepts of interactions on 2D materials interfaces, can be
learned from the field of molecular epitaxy and self-assembly on bulk
interfaces
\cite{Kowarik_2008_rev_MBE,Barth_2007,Whitesides_2002_assem_rev,Philips_2D_assem_book}.
% 
A molecule in the bulk
form and on a densely-covered surface feels the interactions from the
other epitaxial molecules, known as the intermolecular
interactions. On the other hand, a molecule undergoes various
processes on a 2D surface, including adsorption, diffusion, rotation,
and vibration, which is governed by the molecule-2D material
interactions. Moreover, the effect of the underlying substrate is
usually important where the molecule-substrate interactions come into
play. 

\subsubsection{Intermolecular Interactions}
\label{sec:intro-inter-mole}

The intermolecular interactions govern the packing and orientation
behavior of the molecules several atoms away from the 2D material
surface: the strength and the direction of intermolecular interactions
determine the packing density as well as the orientation of the
molecular epitaxy. The intermolecular interactions can be categorized 
into van der Waals (vdW) interactions, hydrogen bonds (H-bonds), and
covalent bonds depending on their strength. 

\paragraph{van der Waals (vdW) Interaction}

The van der Waals (vdW) interactions are dispersion forces between
charge-neutral molecules, including many organic semiconductors, such
as fullerene (C\(_{\text{60}}\))
\cite{Corso_2004_C60_hBN,Kim_2015_c60_gr,Chen_2016_c60_mos2},
metal-phthalo\-cyanines (MPcs, where M can be Cu, Fe, Zn, Co, etc.)
\cite{Xiao_2013_jacs_CuPc_gr,Wang_2010_selec_F16_gr,Zhang_2011_FePc_gr,Hamalainen_2012_CoPc_gr_Ir,Ying_Mao_2011_ge_clAlPc,Ogawa_2013_AlCiPc_gr,Pak_2015_CuPc_MoS2,Avvisati_2017_FePc_intercal,Iannuzzi_2014_MPc_hBN_Rh},
pentacene (PEN)
\cite{Lee_2011_pentacene,Jariwala_2016_Mos2_pentacene,Shen_2017_DFT_mos2_pent,Kim_2016_trap_Mos2_pent,Nguyen_2015_pent_gr_wett,Betti_2007_orien_pentacene},
perfluoropentacene (PFP)
\cite{Salzmann_2012_fpen_gr,Breuer_2016_acnene_mos2}, rubrene
\cite{Lee_2014_rubene_hBN}, perylene-3,4,9,10-tetra\-carboxylic
dianhydride (PTCDA)
\cite{Wang_2009_STM_PTCDA_Gr,Tian_2010_PTCDA_gr,Huang_2009_PTCDA_gr,Meissner_2012_PTCDA_BLG},
7,7,8,8,-Tetra\-cyanoquino\-dimethane (TCNQ) and its fluorinated
derivative 2,3,5,6-Tetra\-fluoro-7,7,8,8-tetra\-cyanoquino\-dimethane
(F\(_{\text{4}}\)-TCNQ)
\cite{Chen_2007_TCNQ_gr,Hong_2013_ftcnq_gr,Stradi_2014_TCNQ_gr_Ru,Tsai_2015_TCNQ_gr_hbn}.
(\worktodo{Figure here?})

Due to its non-directional and weak force nature, if the vdW
interactions govern the interfacial molecules (weak interacting 2D
interfaces), the molecules tends to form close-packed structures in 2D
or 3D assemblies. The dimensionality of molecular epitaxy by vdW
interactions is usually dependent on the surface coverage, as the
molecule growth mechanism is similar to that of adsorption
isotherm. Although the vdW interactions usually have an energy less than
4 kJ\(\cdot\)mol\(^{-1}\), the collective interactions between molecules
with large electron cloud can be stronger. For the \(\pi\)-conjugated
aromatic molecules listed above, an effect known as the \(\pi\)-\(\pi\)
interaction, a combined effect of vdW interactions and charge
transfer \cite{Hunter_1990_pi}, can lead to preferential stacking and
orientation of the molecules, due to maximal overlapping of
\(\pi\)-electron clouds. 


\paragraph{Hydrogen Bond (H-Bond)}

The hydrogen bond (H-bond) refers to the directional electrostatic
forces between an H atom covalently-bonded to an atom of high
electro\-negativity (such as O, N and F) and another highly
electro\-negative atom in adjacent molecules. Compare the vdW
interactions, hydrogen bonds usually have higher bond energy and
preferred direction, which favors certain assembly structure on 2D
materials. The H-bonds are usually dominating between molecules rich
of N, O and F elements, such as modified PTCDA compounds
\cite{Mura_2010_DFT_H_bond_PTCDA_gr,Karmel_2014_assembl_hetero_gr},
perylene tetra\-carboxylic diimide (PTCDI) derivatives
\cite{Pollard_2010_hbond_assembly_gr,Karmel_2014_PTCDI_gr},
carboxylic-substituted aromatic compounds
\cite{Rochefort_2009_aro_graphene_mech,Addou_2013_TPA_gr}, polycyclic
aromatic compounds
\cite{Kozlov_2012_polyaro_gr,Roos_2011_BTP_gr,Meier_2010_polycyclic_gr}
and inorganic acids \cite{Prado_2011_2D_acid_gr}. The existence of
H-bonds stabilizes the assembled low-dimensional structures on 2D
materials interfaces, such as linear
\cite{Pollard_2010_hbond_assembly_gr} or two-dimensional
\cite{Prado_2011_2D_acid_gr} supra\-molecular assemblies. The specific
adsorption sites on 2D materials (such as the moiré patterns) also
play an important role in the assembly of H-bond-governed molecular
epitaxy.


\paragraph{Covalent Bond}
In general, the interactions between the epitaxial molecules and 2D
material (vdW and Coulombic interactions) are much weaker than the
intra\-molecular covalent bond (including metal coordination forces),
resulting in a variety of structures on 2D materials interfaces
\cite{Bakti_Utama_2013_rev_epitax}. One example is the van der Waals
epitaxy (vdWE) technique which allows 2D or 3D crystalline growth on
2D materials. As discussed before \worktodo{which section?}, the
absence of dangling bonds eliminates the lattice mismatch between
dissimilar materials, leading to a number of 2D vertical
heterostructures including: TMDC/graphene
\cite{Shi_2012_vdw_epi_MoS2_gr,Liu_2016_epi_MoS2_gr_rotation,Lin_2014_vdW_solid,Lin_2015_Wse2_MoS2_gr,Azizi_2015_Freevdw_Gr_TMDCs,Kim_2016_BiSnTe_gr},
TMDC/hBN
\cite{Yan_2015_MoS2_on_hBN,Wang_2015_cvd_MoS2_BN,Cattelan_2015_Ws2_hBN},
graphene/hBN
\cite{Liu_2011_gr_hBN,Zhang_2015_gr_hBN,Driver_2016_MBE_gr_hBN}, and
TMDC/TMDC
\cite{Zhang_2014_vdw_epi_SnS2_MoS2,Diaz_2015_MoTe2_MoSe2,Gong_2014_WS2_MoS2,Alemayehu_2015_TMDC_vdw}.
%
The vdWE has also been used to grow 3D heterostructures on mono- or
multilayer 2D materials interfaces, including inorganic insulators like Al\(_{\text{2}}\)O\(_{\text{3}}\)
\cite{Zhang_2014_Al2O3_ALO_Gr,Vaziri_2013_ALD_Al2O3_gr}, and
HfO\(_{\text{2}}\) \cite{Alaboson_2011_PTCDA_gr_ALD},
%
and semiconductors including TiO\(_{\text{2}}\)
\cite{Li_2015_TiO2_GO,Kumar_2011_gr_TiO2_generator,Zhang_2011_TiO2_gr},
ZnO \cite{Chung_2010_GaN_ZnO_gr,Oh_2014_ZnO_hBN}, GaN
\cite{Kobayashi_2012_GaN_hBN,Kim_2014_direct_vdw_GaN_gr,Kim_2017_remote_epi_Gr},
GaAs \cite{Alaskar_2015_GaAs_gr_Si_theor,Kim_2017_remote_epi_Gr}
\worktodo{add Kang2018}, and CdS / CdTe
\cite{Loeher_1994_vdw_epi_CdS_MoTe,Loeher_1996_CdTe_MoWTe}.

Apart from the vdWE approach, covalently bonded structures can also be
formed by on-surface chemical reactions and metal coordination
bonds. Examples of such growth approach include two-dimensional covalent organic frameworks (2D COFs) formed by linking monomers by boron ester
or imine groups
\cite{Colson_2014_2D_COF_gr,Colson_2011_2DMOF_gr,Sun_2017_cof_gr}, and metal-organic frameworks (MOFs) \cite{Urgel_2015_MOF_BN,Kumar_2014_2D_MOF_gr} on weakly interacting or
functionalized 2D materials. The planar sp$^{2}$-type bonds
such as boron ester, imine and square planar metal coordination are
generally required for the formation of stable 2D epitaxial structure.

\subsubsection{Molecule-2D Material Interactions}
\label{sec:intro-mol-2D}

The interactions between the interfacial molecules and 2D material
determine the molecular packing and arrangement of the first few
overlayers. In addition, the interactions also have great impact on
the molecular adsorption process, thereby influencing the
heterogeneous nucleation characteristics. The
ratio between the intermolecular and molecule-2D material
interactions is the key factor in controlling the molecular epitaxial
structure. Here we categorize the molecule-2D material interactions
into weak (dispersion and electrostatic), charge-transfer
interactions, site-specific adsorption, and covalent bond formation.

\paragraph{Weak Interactions}
\label{sec:org68af064}

The weak molecule-2D material interactions involve the short-range
dispersion (vdW) and long-range electrostatic (Coulombic)
interactions. In the case of graphene, the delocalized π-electrons are
the basis for the non-covalent interactions. A large variety of planar
aromatic molecules, including PTCDA, PTCDI, C\(_{\text{60}}\), MPc are
shown to assemble on graphene with their aromatic rings parallel to
the 2D plane, in order to lower the adsorption energy by maximizing
the π-π interaction \cite{Grimme_2008_pipi,Zhang_2011_rev_pipi_gr}, a
phenomenon widely known as the graphene template effect
\cite{Yang_2015_rev_template}. MPc molecules (e.g. M=Cu, Fe, Co and
AlCl) and substituted MPc (e.g. F\(_{\text{16}}\)CuPc) tend
to form a ``face-on'' orientation on graphene interface, relative to
the ``edge-on'' orientation that are usually found on the deposition
of these molecules on amorphous substrates such as SiO\(_{\text{2}}\)
or glass
\cite{Ying_Mao_2011_ge_clAlPc,Zhang_2011_FePc_gr,Hamalainen_2012_CoPc_gr_Ir,,Xiao_2013_jacs_CuPc_gr}.
Similarly, the graphene template effect is also found  for PEN
\cite{Zhou_2013_penta_gr_Ru,Lee_2011_pentacene,Lee_2011_pentacene,Zhang_2015_gr_pent_orient},
C\(_{\text{60}}\) \cite{Kim_2015_c60_gr,Shih_2015_PartiallyScreened},
p-sexiphenyl (6P) \cite{Hlawacek_2011_6P_gr}, and
dibenzotetrathienocoronene (DBTTC) \cite{Kim_2016_DBTTC_gr} molecules,
revealing a general mechanism behind their assembly behavior.

Apart from graphene, the weak interactions on hBN and
MoS\(_{\text{2}}\) surfaces are also studied. The π-electron cloud of
hBN resembles that of graphene, causing the 6P molecules to form a
``face on'' configuration \cite{Matkovic_2016_6P_hBN} similar to the
case on graphene. However, non-planar molecules such as rubrene
\cite{Lee_2014_rubene_hBN} adapt the ``edge-on'' configuration over the ``face-on'' configuration, reflecting the fact
that the molecule-hBN interaction is weakly dispersive.
%
On the other hand, the molecular
interactions on MoS$_{2}$ are usually much weaker compared with that on
graphene due to its large dipole moment \cite{Rajan_2016_wett_mos2},
and is highly dependent on the lattice symmetry
\cite{Shen_2017_DFT_mos2_pent} (i.e. 1T- or 2H- phase) and surface
defects \cite{Jariwala_2016_Mos2_pentacene,
  Kim_2016_trap_Mos2_pent}.

\paragraph{Charge-Transfer Interaction}
\label{sec:orgebfad7b}

The charge-transfer (CT) interactions, or the donor-acceptor (DA)
interactions, refer to the process that electrons undergo
redistribution between the epitaxial molecules and the underlying 2D
material. Due to the locally enhanced carrier density in the formed CT
complex, the CT interactions tend to be stronger than the dispersion
and electrostatic interactions. The formation of a CT heterostructure
requires alignment of the energy levels between the 2D material and
the overlayer molecules \cite{Akiyoshi_2015_DA}, and may also change
the electronic structure of the 2D material through non-covalent
interactions
\cite{Cai_2015_doping_2D_rev,Wehling_2008_doping,Zhang_2011_rev_pipi_gr}.
TCNQ and its fluorinated derivative
2,3,5,6-Tetra\-fluoro-7,7,8,8-tetra\-cyanoquino\-dimethane (FTCNQ) are
known to form CT complexes with graphene
\cite{Chen_2007_TCNQ_gr,Voggu_2008_TCNQ,Barja_2010_TCNQ_gr}, with a
degree of charge transfer of $\sim{}$0.3 \textit{e} and $\sim{}$0.4
\textit{e}, respectively. With a stronger CT effect, FTCNQ molecules
on epitaxial graphene tend to be trapped by local
corrugation\cite{Barja_2010_TCNQ_gr}, compared with closed-packed
TCNQ/graphene assembly.  Since CT may occur when the HOMO and LUMO
energy levels of the epitaxial molecule and 2D material match, it is
also expected to play a role in the molecular epitaxy on 2D
semiconductors, such as TMDCs. Density functional theory (DFT) studies
reveal that PEN adsorbed on 1T-type monolayer MoS\(_{\text{2}}\) has a
large degree of CT ranging from 0.44-0.87 \emph{e}, and can change the
Fermi energy level of MoS\(_{\text{2}}\) by up to 1 eV
\cite{Shen_2017_DFT_mos2_pent}. Similarly, the interface between
C\(_{\text{60}}\) and MoS\(_{\text{2}}\) is found to be a pn-junction,
with charge depleted at the bottom of the C\(_{\text{60}}\) and
accumulated at the interface \cite{Chen_2016_c60_mos2}. On the other
hand, the tendency of forming CT-induced orientation is attenuated on
bulk MoS\(_{\text{2}}\) crystal \cite{Sakurai_1991_c60_mos2}, due to
an increase of the DOS compared in bulk crystals. Theoretical studies
also disclose strong CT between phosphorene and electron-donating
tetrathiafulvalene (TTF), as well as electron-accepting TCNQ molecules
\cite{Zhang_2015_DA_phosphorene}.


\paragraph{Site-Specific Adsorption}
\label{sec:org87b0c12}

The electronic and geometric properties of a 2D material are known to be
influenced by its underlying substrate. When there is a lattice
mismatch between the 2D material and the substrate, a long-range
periodic superposition known as moiré pattern forms, as has been
found graphene/metal \cite{Hamalainen_2013_moire_gr} and hBN/metal
\cite{Schulz_2014_hBN_moire} systems.  The
moiré pattern does not only cause a geometric interference, but
indeed changes the local electronic state and structure of the 2D
material.
%
The height variation within the graphene or hBN layer can be used to
quantify the degree of metal-2D material interaction strength, to
distinguish weakly interacting surfaces include graphene/Ir(111)
\cite{Pletikosi_2009_gr_Ir,Busse_2011_Gr_Ir,Hamalainen_2013_moire_gr},
graphene/Pt(111) \cite{Sutter_2009_Gr_Pt}, hBN/Ir(111)
\cite{Schulz_2014_hBN_moire}, hBN/Pt(111) \cite{Cavar_2008_hBN_Pt},
hBN/Cu(111) \cite{Joshi_2012_hBN_Cu} systems, in which the average 2D
material-metal distance is comparable with that in the bulk material
(3.3$\sim{}$3.4 \AA{}) and the corrugation in the 2D layer is
typically small (<0.5 \AA{}). The strongly interacting surfaces
including graphene/Ru(0001) \cite{Moritz_2010_gr_Ru} \worktodo{sutter
  2}, graphene/Rh(111) \cite{Wang_2010_gr_Rh}, hBN/Ru(0001)
\cite{Wang_2010_gr_Rh}, and hBN/Rh(111) \cite{Dil_2008_hBN_Rh}
systems, with structural corrugations as large as 1 \AA{}, and the
electronic fluctuation up to 0.5 eV. In the strongly interacting
systems, the moiré pattern creates a local difference in the
adsorption potential, which in turn results in site-specific
adsorption of small molecules on these surfaces. Such behavior has
been observed in a variety of organic semiconductor molecules
deposited on the graphene/Ru(0001) surface, including MPc (M=Fe, Ni,
Zn, Mn) \cite{Mao_2009_Pc_gr_kagome,Zhang_2011_FePc_gr}, pentacene
\cite{Zhou_2013_penta_gr_Ru}, C\(_{\text{60}}\)
\cite{Li_2012_c60_gr_Ru}, PTCDA \cite{Zhou_2011_PTCDA_gr_Ru}, TCNQ
\cite{Maccariello_2014_TCNQ_gr_Ru}, with similar behavior has also
been found on the surface of hBN/Ru(0001) for MPc (M=H\(_{\text{2}}\),
Cu, Co) \cite{Dil_2008_hBN_Rh,Jarvinen_2014_MPc_hBN_Ru}, TCNQ
\cite{Joshi_2014_TCNQ_hBN}, and C\(_{\text{60}}\)
\cite{Corso_2004_C60_hBN}. The site-specific adsorption usually lead
to ordered sub-2D assembly, composed of the molecules trapped at the
specific sites, compared with the close-packed assembly on flat and
weakly interacting interfaces.

% Recently, more experimental and theoretical studies have also
% demonstrated the moiré pattern formation on TMDC/metal
% \cite{Chen_2013_doping,Sorensen_2014,Le_2012_MoS2_Cu}, TMDC/TMDC
% \cite{Kang_2013_TMDC_moire,Zhang_2014_vdw_epi_SnS2_MoS2,Diaz_2015_MoTe2_MoSe2,Fang_2014_intercoupl_vdW,Li_2016_GaSe_MoSe2_vdW},
% and TMDC/hBN \cite{Fang_2014_intercoupl_vdW} surfaces. Following the
% discussion of the strongly interacting surface of graphene/Ru(0001),
% it is believed that the moiré pattern formed between the strongly
% coupled layers, e.g. TMDC/Ru(0001) \cite{Chen_2013_doping} and TMDC/TMDC
% \cite{Fang_2014_intercoupl_vdW} heterostructures may also lead to the
% site-specific adsorption phenomenon \cite{Diaz_2015_MoTe2_MoSe2}, in
% contrast to the close-packing structure formed on the weakly
% interacting surfaces, as discussed in the previous section.


\paragraph{Covalent Bond}
\label{sec:org6f342a5}

Covalent bonds formed perpendicular to the 2D material plane open an
opportunity for functionalizing 2D materials and provide anchor sites
for modification. However compared with the epitaxy approaches,
chemical modification of 2D material is limited by the choice of
chemical reactions available. Moreover, opening up 2D basal structure
usually destroyed by the geometric change of the molecular orbital
(e.g. planar sp\(^{\text{2}}\) to tetrahedral sp\(^{\text{3}}\) in
graphene). Nevertheless, there are still a few examples showing the
potential of covalent binding and tuning the electronic properties of
the 2D materials
\cite{Georgakilas_2012_noncoval_gr_rev,Lee_2011_tempo_gr,Zhang_2013_janus_gr,Voiry_2014_cov_TMDC_phase,Vishnoi_2016_ar_mos2_covalent,Liu_2011_rev_chem_dope_gr,Wang_2012_ar_gr_react_rate}.
%
The chemical grafting of graphene mainly involves free-radical
reaction
\cite{Lee_2011_tempo_gr,Choi_2010_aminotempo_gr,Zhang_2013_janus_gr,Wang_2012_ar_gr_react_rate,Kumar_2014_2D_MOF_gr},
%
with the potential to fabricate asymmetric Janus-type functionalized
graphene by the covalent modification on both sides of a free-standing
graphene sheet \cite{Zhang_2013_janus_gr}. The low DOS in a 2D
materials further makes it possible to fine-tune the interfacial
chemical reaction rate by the doping density of 2D materials, for
instance through the substrate doping of graphene
\cite{Wang_2012_ar_gr_react_rate}. Several approaches have also show
the possibility of functionalizing other 2D materials, including
nucleophilic substitution between anionized TMDCs and organohalides
\cite{Vishnoi_2016_ar_mos2_covalent} and aryl diazonium
salts. \worktodo{cite these} The chemical modifications are also
frequently used to improve quality of 2D semiconductors.
%
Diazonium modification of BP significantly increases its ambient
stability over several weeks. \worktodo{cite nat chem 2016 8
  597}. Moreover, treating MoS\textsubscript{2} with organic
super\-acids moves its Fermi level towards intrinsic semiconductor,
and significantly improves the photo\-luminescence (PL) quantum yield over two order of magnitudes. \worktodo{Science 2015 350 1065}
%
Future advance of covalently
modified 2D materials with site-specific and programmable chemical
functionalization may combine the 2D with the 3D materials in a
controllable manner.

\subsubsection{Molecule-Substrate Interaction}
\label{sec:intro-mol-subst}

One of the major differences of the interfacial molecules on 2D
materials compared with bulk materials interfaces is significant
influence from the underlying substrate. Note that this phenomenon is
distinguished from the effect of strongly interacting surface or
substrate doping, with the latter two referring to the change of 2D
material's electronic and geometric properties, which then influence
the molecule-2D material interactions. The penetration of the
molecule-substrate interactions through monolayer 2D material is first
observed in the experiments of wettability of substrate-supported
graphene: the water contact angle of water on graphene is found to be
influenced by the vdW force between the water molecules and the
substrate, known as the wetting ``transparency'' or ``translucency''
of graphene
\cite{rafiee_wetting_2012,shih_breakdown_2012,shih_wetting_2013}. The
transparency can be even pronounced for electrostatic interactions,
which has longer length scale than the vdW force
\cite{Shih_2015_PartiallyScreened,Tian_2016_multiscale}.  The
influence of the molecule-substrate interactions through a 2D
materials usually can only be examined indirectly.
%
Examples include layer-number-dependent morphology of
6P\cite{Kratzer_2014_6P_gr_layer} and
PEN\cite{Chhikara_2014_gr_pent_trans} molecules deposited on
SiO\(_{\text{2}}\)-supported graphene layers. Moreover, the influence
of underlying substrate is also found for PEN when deposited on
graphene supported by substrates with varied surface energy \cite{Nguyen_2015_pent_gr_wett} or
electrostatic gating. \worktodo{cite Nguyen paper 2}
%
Recently the concept of vdW transparency has also been
employed in the remote vdWE of III-V semiconductors on graphene supported by highly-crystalline III-V substrate
\cite{Kim_2017_remote_epi_Gr}. \worktodo{cite Kang 2018 paper 2}
%
The interactions from underlying crystalline III-V semiconductor is
shown to direct the growth of III-V semiconductor on the graphene
interface despite the $\sim{}$ 1 nm gap created by graphene. The
strength of such remote interactions are also shown to be dependent on
the polarity of the underlying material. \worktodo{cite Kang 2018 paper 2}
%
In addition to the vdW and Coulombic interactions,
graphene layer is also found to be transparent to the charge transfer
process \cite{Jeong_2015_DA_transparency_gr} when the reduction rate of
AuCl\(_{\text{4}}^{\text{-}}\) on graphene surface are found to be
faster when graphene is coated on a reductive surface, such as Al, Ge
and Cu surfaces. 

% 20191008-1253

\subsubsection{Summary}
\label{sec:org697d552}

To obtain a clear view of the interactions involved in the molecular
epitaxy on 2D materials interfaces, the major forms
of interactions and their energy range are summarized in
\autoref{tbl:intro-interactions}. As can be seen, both strong interactions ($>$ 100
kJ\(\cdot\)mol\(^{\text{-1}}\) for covalent bonds, metal-coordination and some
hydrogen bonds), and weak interactions ($<$ 50 kJ\(\cdot\)mol\(^{\text{-1}}\) vdW
and \(\pi\)-\(\pi\) interactions) exist between the epitaxial molecules and at
the molecule-2D material interface. On the other hand, the
molecule-substrate interactions mainly have a weak nature, and
relatively weaker than the intermolecular and molecule-2D weak
interactions due to the increasing of molecule-substrate distance.

\begin{table}[htbp]
  \centering
\caption{\label{tbl:intro-interactions}
Types of interfacial interactions involved in the molecular epitaxy on 2D materials interfaces, showing the typical forms of interaction and energy range.}
\centering
\begin{tabularx}{1.0\textwidth}{XXX}
\hline
Type of Interaction & Typical Forms & Energy Range  (kJ\(\cdot\)mol\(^{\text{-1}}\))\\
\hline
Intermolecular & van der Waals & \(\le\) 5\\
 & \(\pi\) - \(\pi\) & \(\le\) 50\\
 & H-bonds & 4 - 120 \cite{jeffrey_introduction_1997}\\
 & Covalent Bonds & 100 - 400\\
\hline
Molecule-2D & Weak Interactions & 10 - 60 \cite{Lazar_2013}\\
 & Charge-Transfer & 50 - 200\\
 & Site-Specific Adsorption & 30 - 100\\
 & Covalent Bonds & 100 - 400\\
\hline
Molecule-Substrate & Weak Interactions & \(\le\) 20\\
\hline
\end{tabularx}
\end{table}

\subsection{The Variety of Mixed-Dimensional Interfaces}
\label{sec:vari-mixed-dimens}

%20191008-1405
As shown in the previous section, a variety of interactions exist on 2D
materials, which
essentially dominate the ordering and packing of the molecules in vicinity.
%
The interfacial engineering of the interactions leads to different
dimensionalities ranging from 0D to 3D.
%
In this section, several model systems of mixed-dimensional interfaces
with 2D materials are discussed, with the focus on how the
interactions determine the morphological dimension. The notations for
the interfaces are like ``0D-2D'', with the former indicating the
dimensionality of the heterostructure on 2D materials. In general, two
approaches are usually employed to fabricate such mixed-dimensional
interfaces, namely self-assembly of small molecules, and deposition of
pre-formed nano\-materials. \worktodo{say something more?}

\subsubsection{0D-2D Interface}
\label{sec:intro-0D-2D}

\paragraph{Self-Assembly}
\label{sec:org8117691}

Molecular dynamics (MD) simulations have shown that the weak
intermolecular and molecule-2D interactions alone, do not result in
the formation of sub-monolayer assembly, such as the case of pentacene
and PTCDA on graphene or hBN \cite{Zhao_2015_self_assemb_gr_MD}, and
organic semiconductor molecules dominated by vdW force on phosphorene
\cite{Mukhopadhyay_2017_cryst_BP}.
% 
To form 0D assemblies on 2D materials, specific adsorption sites are
required to exist on the 2D surface.
% 
The moiré pattern formed in the graphene/metal and hBN/metal
interfaces as introduced in \autoref{sec:intro-mol-2D} are shown to
trap metal clusters
\cite{Goriachko_2007_assembl_hBN_ru,Pan_2009_Pt_cluster_gr,Wang_2011_gr_hBN_metal_cl,Zhang_2014_metal_gr_Ru}
and individual organic semiconductor molecules
\cite{Joshi_2014_TCNQ_hBN,Dil_2008_hBN_Rh,Lu_2012_c60_gr_moire}.
%
In the case of organic molecular deposition, the site-specific
adsorption energy difference is usually around 10\textsuperscript{2}
meV \cite{Lu_2012_c60_gr_moire}, enough to trap the small molecules
within the valley regions of the 2D moiré patterns. More interesting,
it is also found that the site-specific isolation of small molecules
is not limited to strongly interacting surfaces such as graphene and
hBN supported by Ru or Rh, but also weakly interacting surfaces like
hBN/Cu(111) with a small degree of corrugation but strong electronic
patterning \cite{Joshi_2012_hBN_Cu,Joshi_2014_TCNQ_hBN}, revealing the
more complex nature of the 0D self-assembly on 2D materials interfaces.
%
The isolated molecules on 2D materials can be used as nucleation sites
for further molecular epitaxy, and facilitate the research of
single-molecular surface reaction. \worktodo{Guess need to cite 1-2
  papers here}


\paragraph{Deposition of nano\-materials}


The 0D-2D interfaces fabricated by deposition usually refer to the 0D
quantum dot (QD)-2D material junction. The quantum dots are
quantum-confined nano\-materials with size of several
nano\-meters. \worktodo{get a cite for this part}
%
The vast majority of the QDs are prepared by colloidal synthesis, and
covered by buffering coating such as ligands. \worktodo{get a cite for this part}
%
As a result, QDs deposited on 2D materials using solution processing
can still sustain the isolate form, making their applications more
versatile than the self-assembled 0D structures.
%
One promising feature of QDs is the optical properties, including high
optical absorption coefficient, PL quantum yield, and size-dependent
modulation of optical bandgap.
%
Combining with the electrostatic tuning of 2DEG, the 0D-2D
heterostructures can be used for multiple light sensitive components
such as photo\-transistors, photo\-diodes, photo\-voltaics (PV) and
light emitting devices (LED). \worktodo{cite}
%
These architectures share a similar feature, that the QDs act as the
main photo-active component, while the 2D materials is design to
modulate the carrier transport / injection at the QD-2D interface.
\worktodo{I guess 2 papers each part may be needed}
%
For instance, in a 0D-2D photo\-transistors based on lead sulfide
(PbS) QDs on graphene \worktodo{nature nanoteh 2012 7 363-368}, the
photo-generated carriers in the QDs are transferred onto the biased
graphene surface and consequently collected at the electrodes. The
benefits from such mixed-dimensional heterostructure are two-folds:
(1) the large optical cross-section of QDs enhanced photo\-detection
limit compared with bare 2D material and (2) the photo\-current is
efficiently modulated by electrostatic gating of graphene.
%
The examples of other 0D-2D photo\-active electronic components can be
found in several recent reviews. \worktodo{cite Jariwala, Dominik
  Kufer ACS Photo}


\subsubsection{1D-2D Interface}
\label{sec:orgeadf57e}

\paragraph{Self-assembly}

With increasing surface coverage on a 2D material interface or
introducing directional intermolecular interactions, self-assembled 1D
and fractal assemblies may be formed on 2D material interfaces, in the
form of nanowires, nanoporous or network structures.
%
As discussed earlier, the strongly interacting surfaces, including
graphene/Ru(0001), graphene/Rh(111) and hBN/Ru(0001) result in the
moiré pattern that serves as specific binding sites for trapping small
molecules at low surface coverage.  Starting from the 0D assembly
formed by moiré patterns on graphene/Ru(0001), graphene/Rh(111) and
hBN/Ru(0001) as discussed in \autoref{sec:intro-0D-2D}, when further
increasing the surface coverage, the specifically adsorbed molecules
act as nucleation sites for subsequent epitaxial growth.  The intermolecular interactions, in combination with
the geometry of moiré pattern, result in a specific arrangement of the
molecules on the surface, varying from nanowire, nanorope \cite{Maccariello_2014_TCNQ_gr_Ru} to Kagome
lattice \cite{Atwood_2002_kagome,Mao_2009_Pc_gr_kagome}.
%
The nucleation-induced growth is both substrate- and
molecule-specific. For instance, 1D and fractal molecular assemblies
are rare on hBN/metal
surfaces\cite{Schulz_2013_copc_hbn_moire,Schulz_2014_hBN_moire,Iannuzzi_2014_MPc_hBN_Rh,Joshi_2014_TCNQ_hBN},
possibly due to the different surface potential distribution compared
with the graphene moiré surface. Furthermore, metal intercalation
between graphene and substrate is shown to affect the assembly pattern
of MPc
\cite{Bazarnik_2013_tailor_Fe_Co_gr_Ir,Avvisati_2017_FePc_intercal}.
%
In addition to the nucleation-induced growth, a rich set of 1D-2D
heterostructure can be obtained by tailoring the intermolecular and
molecule-substrate interactions.
%
Intermolecular H-bond (such as in PTCDI derivatives) is widely used to
guide the orientation of surface-assisted self-assemblies ranging from
linear to Kagome structures \cite{Slater_2014_HBond_assembl_rev,
  Pollard_2010_hbond_assembly_gr, }, due to the relatively high
strength compared with molecule-2D interactions.
%
The assembly structure can be ultra sensitive to the conformation of
molecules, with minor change of functional groups between almost
identical molecules leading to distinct morphology on graphene/Ru
(0001) surface
\cite{Meier_2010_polycyclic_gr,Roos_2011_BTP_gr,Roos_2011_hiera_org_gr},
revealing the important role of directional interactions in the
formation of low-dimensional assemblies on 2D
interfaces. \worktodo{Add spaces between all Ru instances}
%
In addition to hydrogen bond, several other interactions including
hydrophobic interaction between long alkyl chains
\cite{De_Feyter_2003_2D_assem_rev, Deshpande_2012_1D_assemb_gr},
Coulombic interactions \cite{Prado_2011_2D_acid_gr}, covalent bond
\cite{Colson_2011_2DMOF_gr,Colson_2014_2D_COF_gr} and metal
coordination bond \cite{Urgel_2015_MOF_BN} are also shown to form
1D-2D heterostructures. In these systems, the intermolecular
interactions are generally directional, and much larger than the molecule-2D
interactions, which stabilizes the formed low-dimensional structure.

\paragraph{Deposition of nanomaterials}
% 20191008-1613




\subsubsection{2D-2D Interface}
\label{sec:org1d44ccd}

In principle, under sub-ML coverage, when the diffusion of molecules on  2D
material is not limited by interfacial traps, close-packed 2D assembly
can be formed in molecular epitaxy. The molecular epitaxy with
2D assembled structures can be categorized into two classes, namely the
self-assembled small organic molecules on 2D material, and the
2D heterostructures grown by vdWE.

\paragraph{Monolayer Self-Assembly of Small Molecules}
\label{sec:orgfd77377}

The assembly of small molecules on 2D materials with low geometric and
electronic corrugation have usually been found to form close-packed
structures. Molecular dynamics (MD) simulations is a good tool to
rationalize the role of intermolecular, molecule-2D material and
molecule-substrate interactions. Zhao et al. studied the self-assembly
of non-polar pentacene and polar PTCDA molecules on planar graphene
and hBN \cite{Zhao_2015_self_assemb_gr_MD}. Starting from a disordered
state with sub-ML adsorption, both pentacene and PTCDA molecules ended
up in forming an ordered 2D assembled structure, within the timescale
of 100 ns. Subsequent adsorption of molecules was found to fill the
gaps in the assembled structures within 1 ns, regardless of the
initial orientation. For non-polar pentacene, the intermolecular vdW
interactions which decreased dominate structure, while the Coulombic
interactions, on the other hand, have much less effect on the
packing. However both the Coulombic and vdW interactions were found to
play important roles in stabilizing the epitaxy structure of
PTCDA. The molecule-2D material interactions determine the orientation
of the interfacial molecules. Decreasing the pentacene-graphene
potential to half of its optimized force field value changed the
packing of pentacene molecules from the face-on to edge-on
configuration (see Figure \ref{fig-2D-MD}). Using a similar approach,
Mukhopadhyay et al. studied the interactions involved in the
self-assembly of various small organic molecules (including benzene
derivatives, TCNQ, pentacene, C\(_{\text{60}}\)) on top of phosphorene
\cite{Mukhopadhyay_2017_cryst_BP}. For non-polar molecules such as
pentacene and mesitylene, the vdW potential decreases upon assembly
while the Coulombic potential slightly increases. For polar molecules
(1,3,5-trifluorobenzene, 1,3,5-trihydroxybenzene and TCNQ), the
long-range intermolecular Coulombic interactions are more
dominant. The calculated molecule-phosphorene interaction free
energies for TCNQ (-45.4 kJ/mol) and pentacene (-28.6 kJ/mol) were
comparable to those on graphene and hBN, which explain the similar
packing behavior observed on phosphorene as compared to graphene and
hBN systems. The results indicate that the non-aromatic nature of
phosphorene does not affect the stability of epitaxial molecules in
assemblies, and therefore the experimentally observed packing
structures on phosphorene may be analog to those on graphene and
hBN. To our knowledge, the effect of underlying substrate on 2D
epitaxial assembly has not been investigated under the scope of MD
simulations. However with the recently-developed knowledge of
substrate-influenced effects in wettability of 2D materials by MD
simulations \cite{shih_breakdown_2012,Hung_2015_MD_water_sub}, it would
be straightforward to consider the substrate effect in further work.

Although the above findings that are based on the MD-calculated
interaction energies have not taken into account the kinetic phenomena
involved in epitaxy experiments, the formation of ordered 2D
crystalline organic films on 2D materials generally correlates well to
the theoretical framework. A general feature in the 2D epitaxy of
small molecules is that the sub-2D packing formed on the strongly
interacting 2D surfaces, as discussed in the previous sections,
changes to close-packed arrangements on a weakly interacting 2D
material surface for molecules including MPc
\cite{Jarvinen_2013_assembl_SiO2_hBN,Hamalainen_2012_CoPc_gr_Ir,Singha_Roy_2012_CuPc_gr_glass,Xiao_2013_jacs_CuPc_gr,Wu_2013_CuPc_F16_gr,Wang_2010_selec_F16_gr},
TCNQ \cite{Barja_2010_TCNQ_gr}, C\(_{\text{60}}\)
\cite{Jung_2014_C60_gr_Cu,Kim_2015_c60_gr,Chen_2016_c60_mos2}. The
comparisons between the epitaxial behavior on the strongly and weakly
interacting 2D material substrate have been addressed, such as FePc on
graphene/Ru(0001) and graphene/Pt(111) \cite{Yang_2012_MPc_gr_metal},
TCNQ on graphene/Ru(0001) \cite{Maccariello_2014_TCNQ_gr_Ru} and
graphene/Ir(111) \cite{Barja_2010_TCNQ_gr}, CoPc on graphene/Ru(0001)
\cite{Cai_2015_CoPc_gr_Ru} and graphene/Ir(111)
\cite{Hamalainen_2012_CoPc_gr_Ir}, F\(_{\text{4}}\)-TCNQ on graphene/Ru(0001)
\cite{Stradi_2014_TCNQ_gr_Ru} and graphene/hBN
\cite{Tsai_2015_TCNQ_gr_hbn}, as shown in Figure
\ref{fig-2D-strong-weak}.

Due to a low surface corrugation for 2D materials on the weakly
interacting surfaces, more freedom for surface adsorption and
diffusion of the small molecule is observed, resulting in the
close-packed structures. Note that a strongly interacting 2D material
does not always lead to sub-2D packing, as the molecular geometry and
intermolecular interactions also come into play. Such phenomena has
been observed in the systems of PTCDA molecules epitaxially grown on
graphene
\cite{Wang_2009_STM_PTCDA_Gr,Lauffer_2008_PTCDA_gr_sic,Emery_2011_PTCDA_gr,Tian_2010_PTCDA_gr},
and supported by \emph{ab initio} simulations
\cite{Mura_2010_DFT_H_bond_PTCDA_gr}. Due to the existence of
intermolecular C-H \dots{} O hydrogen bonds, the moiré pattern of
graphene/Ru(0001) surface is not sufficiently strong to trap
individual PTCDA molecules \cite{Wang_2009_STM_PTCDA_Gr}, and therefore
a close-packed 2D assembly forms, similar to that found on weakly
interacting graphene/SiC surface
\cite{Lauffer_2008_PTCDA_gr_sic,Emery_2011_PTCDA_gr}. The hydrogen bond
energy per unit PTCDA herringbone lattice (400 \textasciitilde{} 600 meV)
\cite{Tian_2010_PTCDA_gr} was found to be comparable with the adsorption
energy on the top sites of the moiré pattern (\textasciitilde{}700 meV)
\cite{Roos_2011_BTP_gr}, revealing a competition between the
intermolecular and molecule-2D interactions. Therefore, there is a
small degree of morphological change by changing the weak interacting
substrate to the strongly-interacting substrate, as vacancies are be
found in the 2D PTCDA deposited on graphene/Ru(0001) located on the
top sites. The close-packed assemblies on the strongly interacting 2D
surface are also be found in other molecules including pentacene
\cite{Zhou_2013_pent_gr_Ru} and C\(_{\text{60}}\)
\cite{Lu_2012_c60_gr_moire,Li_2012_c60_gr_Ru}, in which the potential
barrier on the surface was overcome by kinetic energy. On the other
hand, the influence of molecule-substrate interaction has been
demonstrated in the systems of CoPc deposited on graphene/SiO\(_{\text{2}}\) and
graphene/hBN \cite{Jarvinen_2013_assembl_SiO2_hBN}. Specifically,
although CoPc forms cubic close-packed structures on both surfaces,
the domain size on graphene/SiO\(_{\text{2}}\) was found to be significantly
smaller than that on graphene/hBN. The LUMO energy level fluctuation
of CoPc on graphene/hBN was found to be less than that on
graphene/SiO\(_{\text{2}}\), revealing the influence of the underlying substrate.

Following the discussions of 2D assembly, molecular packing of small
organic molecules on 2D materials other than graphene is essentially
determined by the molecule-2D material interactions, as well as the
geometry and electronic structure of the 2D material. Shen et
al. predicted the phase-dependent charge transfer between pentacene
and MoS\(_{\text{2}}\) monolayer \cite{Shen_2017_DFT_mos2_pent}, as briefly
mentioned in Section 2.2. A considerable degree of charge
transfer between pentacene and 1T MoS\(_{\text{2}}\) (i.e. enhanced molecule-2D
material interactions) may lead to unprecedented 2D self-assembly. The
dipole MoS\(_{\text{2}}\) surface has shown to orient the butyl-substituted PTCDI
derivative (PTCDI-C\(_{\text{4}}\)) differently from the graphene surface
\cite{Arramel_2017_Ptcdi_mos2}. X-ray photoelectron spectroscopy (XPS),
NEXAFS and resonant photoemission spectroscopy (RPES) where used to
reveal the edge-on orientation of PTCDI-C\(_{\text{4}}\), compared with the
normally found face-on configuration of PTCDI compounds on graphene
\cite{Karmel_2014_PTCDI_gr}. The molecule-MoS\(_{\text{2}}\) interactions have been
found to be weaker than that between the alkyl chains, leading to a
tilted packing configuration. Another example is the
dioctylbenzothienobenzothiophene (C8-BTBT) epitaxy on MoS\(_{\text{2}}\)
\cite{He_2015_C8BTBT_MoS2}, compared with that on graphene
\cite{He_2014_C8BTBT_gr}. The ML thickness of C8-BTBT on MoS\(_{\text{2}}\) was
found to be \textasciitilde{}1.2 nm, significantly larger than that on graphene (\textasciitilde{}0.7
nm), because the interfacial C8-BTBT molecules are with the ``edge-on''
orientation on MoS\(_{\text{2}}\). Clearly, the reduced vdW molecular-2D material
interaction is responsible for the substrate-dependent packing
configuration. Due to the fact that monolayer molecular assembly on 2D
materials other than graphene and hBN has not been well studied by
STM, more detailed studies will be required to uncover the molecule-2D
interaction on these 2D materials.

\paragraph{2D van der Waals Heterostructures}
\label{sec:org77ea5bc}

We refer the 2D vdW heterostructures to the epitaxial assembly of
covalently bonded 2D materials. Chemical vapor deposition (CVD) and
van der Waals epitaxy (vdWE) are the two major methods to grow the 2D
vdW heterostructures \cite{Novoselov_2016_vdW}. In the view point of
molecular interactions at the interface, these two methods are
essentially similar. Therefore here we do not specify the preparation
method for the 2D vdW heterostructure in this review. Note that the
definition of 2D vdWE refers to the growth of single- or few-layer 2D
materials on top of another 2D material. The use of vdWE for both
small molecule \cite{Hara_1989_cupc_mos2_vdwe,Sakurai_1991_c60_mos2} and
layered materials
\cite{Koma_1985_vdWE,Ueno_1990_vdWE,Ohuchi_1990_MoSe2_SnS2,Parkinson_1991_vdWE}
on layered TMDCs has been demonstrated long before the first discovery
of graphene. The benefit of vdWE over conventional heteroepitaxy is
less constraint in lattice mismatch. In conventional heteroepitaxy,
dangling bonds exist on the substrate surface, thereby limiting the
growth of lattice-mismatch overlayer. On the other hand for 2D
materials, the interactions perpendicular to the 2D plane are mainly
vdW or Coulombic interaction, so that the overlaying 2D material can
be grown with less constraint in lattice mismatch. Figure
\ref{fig-2D-vdW}(a) schematically shows the principle of van der Waals
epitaxy process.  As introduced in the section of intermolecular
covalent bond, a large variety of 2D vdW heterostructures have been
synthesized by vdWE approach. In this section we discuss the interplay
between the covalent bond and inter-layer interactions in the vdWE
growth of 2D vdW heterostructures.

Graphene, hBN and TMDC (in particular MoS\(_{\text{2}}\)) are the most-studied 2D
materials for vdWE growth of 2D heterostructures, due to their large
area accessibility. Epitaxial graphene \cite{Yang_2013_gr_hBN}, MoS\(_{\text{2}}\)
\cite{Yan_2015_MoS2_on_hBN,Wang_2015_cvd_MoS2_BN} and WS\(_{\text{2}}\)
\cite{Cattelan_2015_Ws2_hBN} have been grown by vdWE on multilayer
hBN. The epitaxial graphene on hBN was grown by plasma-enhanced CVD at
\textasciitilde{}500 \(^{\circ} \mathrm{C}\), which is much lower than that required for
metal-catalyzed CVD growth of graphene (\textasciitilde{}1000 \(^{\circ} \mathrm{C}\))
\cite{Yang_2013_gr_hBN}.  The orientation of epitaxial graphene was
found to be uniform on hBN, as revealed by the moiré pattern formation
in STM (see Figure \ref{fig-2D-vdW}(b)). Similar to the previous
discussion about the 2D assembly of organic molecules, the interlayer
interaction is also shown to govern the stacking of 2D
heterostructure. On epitaxial or CVD-grown graphene, the vdWE
technique has been used to grow hBN \cite{Lin_2014_vdW_solid}, MoS\(_{\text{2}}\)
\cite{Shi_2012_vdw_epi_MoS2_gr,Lin_2014_vdW_solid,McCreary_2014_MoS2_gr,Azizi_2015_Freevdw_Gr_TMDCs,Miwa_2015_MoS2_gr,Ago_2015_MoS2_Gr},
WS\(_{\text{2}}\) \cite{Azizi_2015_Freevdw_Gr_TMDCs}, WSe\(_{\text{2}}\)
\cite{Lin_2014_WS2_Gr,Lin_2015_Wse2_MoS2_gr}, and non-layered
Pb\(_{\text{1-x}}\)Sn\(_{\text{x}}\)Se \cite{Wang_2015_vdw_non_layer}. Monolayer epitaxial
MoS\(_{\text{2}}\) was successfully grown following on graphene, whereas the
growth on bulk SiC surface remains negligible, following the principle
of vdWE \cite{Lin_2014_vdW_solid}. Monolayer TMDCs were also used for
vdWE of TMDC/TMDC hybrid heterostructure, including MoTe\(_{\text{2}}\)/MoS\(_{\text{2}}\)
\cite{Diaz_2015_MoTe2_MoSe2}, WS\(_{\text{2}}\)/MoS\(_{\text{2}}\) \cite{Gong_2014_WS2_MoS2},
GaSe/MoSe\(_{\text{2}}\) \cite{Li_2016_GaSe_MoSe2_vdW} (see Figure
\ref{fig-2D-vdW}(c)), MoS\(_{\text{2}}\)/SnS\(_{\text{2}}\)
\cite{Zhang_2014_vdw_epi_SnS2_MoS2}. More complex vdW heterostructures
can also be synthesized using the vdWE approach. Lin et al. used
repeated vdWE on epitaxial graphene at different temperatures for the
synthesis of MoS\(_{\text{2}}\)/WSe\(_{\text{2}}\)/graphene and WS\(_{\text{2}}\)/MoSe\(_{\text{2}}\)/graphene
heterostructures \cite{Lin_2015_Wse2_MoS2_gr}. Alemayehu et
al. synthesized ordered stacks of GeSe\(_{\text{2}}\)/VSe\(_{\text{2}}\) heterostructures
\cite{Alemayehu_2015_TMDC_vdw} with controlled GeSe\(_{\text{2}}\)
layer number modulation by tuning the nucleation process.

One challenge in the vdWE growth of 2D vdW heterostructures is the
controlled growth of monolayer 2D overlayer. We note that in many 2D
heterostructures systems (mainly TMDC/graphene or TMDC/hBN) grown by
vdWE, the growth of mono- and multi- overlayers have both been
reported, including MoS\(_{\text{2}}\)/graphene
\cite{Shi_2012_vdw_epi_MoS2_gr,Lin_2014_vdW_solid,Azizi_2015_Freevdw_Gr_TMDCs,Miwa_2015_MoS2_gr,Liu_2016_epi_MoS2_gr_rotation,McCreary_2014_MoS2_gr},
WSe\(_{\text{2}}\)/graphene \cite{Lin_2014_vdW_solid,Azizi_2015_Freevdw_Gr_TMDCs}
and MoS\(_{\text{2}}\)/hBN \cite{Yan_2015_MoS2_on_hBN}. The stacking sequence seems
also relevant to the layer number of the epitaxial layers. For example
almost all hBN/graphene heterostructures reported showed multilayer
hBN growth \cite{Wu_2015_Gr_hBN,Lin_2014_vdW_solid}, while graphene
grown on hBN tend to be monolayer
\cite{Yang_2013_gr_hBN,Wu_2015_Gr_hBN}. These results reveal the
importance of chemical kinetics in the vdWE heterostructure growth, since
the interplay between the interactions alone cannot explain the
discrepancy. The proposed mechanisms include
preferred nucleation sites \cite{Yan_2015_MoS2_on_hBN}, non-epitaxial
growth \cite{Azizi_2015_Freevdw_Gr_TMDCs} and synthesis method
\cite{Azizi_2015_Freevdw_Gr_TMDCs}. Further studies are required to
elucidate the mechanism underlying the layer-controlled growth.



\subsubsection{3D-2D Interface}
\label{sec:org321dcef}

3D assembly on 2D material can be made by layer-by-layer deposition of
small molecules or epitaxy of covalently bonded structure. When the
thickness of 3D assembly increases, the intermolecular (or
interatomic) interactions become dominant over the molecule-2D
material and molecule-substrate interactions. However, this does not
mean the interfacial interactions are negligible. In fact, as will be
discussed later, the interfacial layer plays an important role in
determining the molecular orientation and morphology in the 3D
assembled structures. 3D epitaxy on 2D material interfaces may result
in various forms of nanostructures other than simple stacked
layers. The morphology of the 3D assembly greatly influence several
key properties in organic semiconductors, including carrier transport,
interfacial barrier, which motivates understanding of the underlying
mechanism.  The diversity of 3D epitaxial morphology addresses the
question of how the macroscopic structure is influenced by the
interplay between the interactions. In this section, we review the
interactions involved in a variety of 3D epitaxial structures, with
more focus on the theoretical work in this field.

\paragraph{Layer-by-Layer Assembly of Small Molecules}
\label{sec:org2cdd8f0}

Layer-by-layer (LbL) self-assembly of small molecules can be viewed as
the vertical epitaxy of 2D assembled structure. The molecular
orientation and packing of the interfacial layers, i.e., the first few
layers of the molecules, are strongly influenced by the molecule-2D
material and molecule-substrate interactions, compared to the
molecules far from the interface.  The maximum molecule layer number
influenced by the 2D material and substrate highly depends on the
choice of molecule. The transition of orientation is a consequence of
the competition between the interfacial and the bulk packing
orientations. In the case of C\(_{\text{60}}\) on corrugated graphene/Ru(0001)
surface \cite{Lu_2012_c60_gr_moire}, the first layer of C\(_{\text{60}}\) forms
isolated single molecules trapped in the HCP valley of the moiré
pattern, which also serves as the nucleation sites for the second
layer. The second layer C\(_{\text{60}}\) molecules pack in a hexagonal pattern
around the trapped C60 molecules, namely the corrugation-guided
packing. From the third layer on, subsequent packing of C\(_{\text{60}}\)
molecules experience from dendritic to compact growth; in other words,
the influence of graphene corrugation does not go beyond the third
layer (Figure \ref{fig-3D-layer-dep}(a)). Similarly, TCNQ assembly on
graphene/Ru(0001) also show transition from sub-2D Kagome packing to
bulk phase close-packed order when coverage is slight higher than 1ML
\cite{Maccariello_2014_TCNQ_gr_Ru}.  On the other hand, the second layer
CoPc molecules adsorbed on hBN/Ir(111) still prefer to occupy the pore
regions of hBN moiré pattern which have already been covered by the
first layer CoPc molecules \cite{Schulz_2013_copc_hbn_moire}. Multilayer
assembly of pentacene molecules on graphene is another interesting
example of layer-dependent molecule-2D material interactions. On
epitaxial graphene/SiC surface, the first layer of pentacene shows the
face-on orientation, with long-range ordering
\cite{Jung_2014_pentacene}. The second pentacene layer shows a tiled
angle with respect to the graphene plane, corresponding to partial
turn-over of the pentacene molecules
\cite{Chen_2008_transition_pentacene}. After the fifth layer, the
pentacene molecules pack in an edge-on manner, consistent with that in
the bulk pentacene film \cite{Ruiz_2004_bulk_pentacene} (Figure
\ref{fig-3D-layer-dep}(b)). By increasing pentacene-graphene
interactions, a strained polymorph of pentacene was observed
\cite{Kim_2015_pentacene_gr_strain}, stabilizing the epitaxial structure
with consistent face-on orientation throughout the entire
film. C8-TBTB is another example that exhibits a structural transition
during epitaxial assembly. The first layer of C8-TBTB molecules packs
with the polycyclic TBTB group in contact with graphene, forming
long-range ordered linear structure \cite{He_2014_C8BTBT_gr}, as a
result of the \(\pi\)-\(\pi\) interactions between TBTB core and graphene,
together with the alignment of the alkyl chains. In the bulk phase,
the C8-TBTB molecules assemble with the edge-on orientation and form
cubic lattice. The transition of orientation is found exclusively at
the second layer, as revealed by the average layer thickness change
(Figure \ref{fig-3D-layer-dep}(c)). The observation slightly changes for
C8-TBTB deposited on single layer MoS\(_{\text{2}}\) \cite{He_2015_C8BTBT_MoS2}
(Figure \ref{fig-3D-layer-dep}(d)). Due to a weaker molecule-2D material
interaction, the first layer of C8-TBTB already shows an intermediate
degree of tilted orientation between the face-on and edge-on
configurations. From second layer on, the C8-TBTB molecules completely
become the bulk phase packing. A general trend of the mono- to
multi-layer transition can be observed: the molecule-2D material
interactions have more influence on the first ML and decays rapidly
with the increase of layer numbers. When the direction of molecule-2D
material interaction differs from that of intermolecular interaction
in the bulk phase, a transition of the molecular orientation can be
observed from monolayer to multi-layer. The effective depth of the
molecule-2D material interactions is usually less than 2\textasciitilde{}3 ML, and
dependent on the type of epitaxial molecules.

For molecules with a large planar structure such as CuPc and
F\(_{\text{16}}\)CuPc, the strong intermolecular interactions tend to pack the
molecules along their c-axis
\cite{Ren_2011_DFT_CuPc_epi_gr,Jiang_2014_F16Pc,Yoon_2010_crystal_F16cuPc}. Due
to the fact that MPc molecules also take face-on orientation on
graphene surface
\cite{Xiao_2013_jacs_CuPc_gr,Mativetsky_2014_CuPc_gr,Zhong_2012_gr_F16_pn_junc},
the face-on orientation can be maintained after the second layer and
continue in its 3D epitaxy structure. On the other hand, the MPc
molecules prefer to form the edge-on orientation on amorphous polar
surface such as SiO\(_{\text{2}}\) and glass
\cite{Singha_Roy_2012_CuPc_gr_glass,Xiao_2013_jacs_CuPc_gr,de_Oteyza_2006_F16CuPc_sio2}. As
a result, molecular orientation of epitaxial CuPc and F\(_{\text{16}}\)CuPc is be
templated by the graphene layer, screening the interactions from the
SiO\(_{\text{2}}\) or glass substrates. The idea of graphene-templated growth is
also applied to the epitaxy of pentacene \cite{Lee_2011_pentacene},
perfluoropentacene \cite{Salzmann_2012_fpen_gr}, C\(_{\text{60}}\)
\cite{Kim_2015_c60_gr} and DBTTC \cite{Kim_2016_DBTTC_gr}. On the
contrary, the epitaxy structure of thin-layer CuPc on MoS\(_{\text{2}}\)
\cite{Zhang_2015_CuPc_MoS2} shows less stability which undergo a face-on
to edge-on transition upon air exposure and forming 1D-ordered
structures. The face-on orientation, however, can be maintained at
higher deposition temperature to create a
compact structure and high turn-over barrier.

The morphology of the grown 3D epitaxial film is a consequence of both
thermodynamic and kinetic effects \cite{Kowarik_2008_rev_MBE}. Classical
theory of thin film organic molecular epitaxy identifies three typical
growth mechanisms, depending on their morphology, as follows: (i)
Volmer-Weber (VW) type, where the assemblies are majorly island-like,
(ii) Frank-van der Merwe (FM) type, where flat epitaxial layers are
formed, and (iii) Stranski-Krastanov (SK) type, where island formation
follows flat interfacial layer. The macroscopic phenomenon of
morphology can be explained by the interplay of surface energies of
different interfaces, including the 2D material surface
\(\gamma_{\mathrm{2D}}\), surface of epitaxial molecules
\(\gamma_{\mathrm{M}}\), and molecule-2D interface
\(\gamma_{\mathrm{M-2D}}\), under the scope of 2D-interfacial
epitaxy. In VW type growth, it's usually observed that
\(\gamma_{\mathrm{M}} + \gamma_{\mathrm{M-2D}} > \gamma_{\mathrm{2D}}\),
indicating a non-wetting of molecules at the interface, while in the
FM type growth, \(\gamma_{\mathrm{M}} + \gamma_{\mathrm{M-2D}} <
\gamma_{\mathrm{2D}}\), corresponding to wetting of the molecules. The
different models of thin film growth is shown in Figure
\ref{fig-morphology}(a). Since the surface energy is a macroscopic
property related to the intermolecular and interfacial interactions,
the growth models can also be rationalized by the interplay between
the interactions. In VW-type growth, the intermolecular interaction is
usually stronger than the molecule-2D and molecule-substrate
interactions, as the molecules tend to grow on nuclei and form
islands, while in FM- and SK- type growth, the strong interfacial
interactions lead to the formation of interfacial wetting layer. After
the first few layers, the kinetic parameters, in particular the
surface diffusivity of epitaxial molecules, play a crucial role
\cite{Chatraphorn_2001_limit_diff}.

The morphological control of 3D molecular epitaxy is also studied by
fine-tuning the chemical structure of epitaxial molecules, such as the
cases of CuPc/F\(_{\text{16}}\)CuPc, and pentacene/perfluoropentacene. Several
reports have indicated that the fluorination has great impact on the
assembly morphology. On graphene surface, both CuPc
\cite{Singha_Roy_2012_CuPc_gr_glass,Xiao_2013_jacs_CuPc_gr} and
F\(_{\text{16}}\)CuPc \cite{Zhong_2012_gr_F16_pn_junc} molecules form the face-on
orientation, as indicated by X-ray diffraction (XRD) and grazing
incidence X-ray diffraction (GIXD) spectroscopy. For the following
layers of both molecules, a uniform diffraction peak corresponds to
the face-on orientation is observed. However the macroscopic
morphology shows different features: CuPc tends to form a dense film
on graphene, with the grain size larger than that on SiO\(_{\text{2}}\)
\cite{Singha_Roy_2012_CuPc_gr_glass,Xiao_2013_jacs_CuPc_gr}, F\(_{\text{16}}\)CuPc
forms vertical nanowires on PTCDA-template surface (similar to
graphene with rich \(\pi\)-electrons) \cite{Yang_2011_F16CUPc_nanowire}
(Figure \ref{fig-morphology}(b)).  Pentacene and perfluoropentacene on
MoS\(_{\text{2}}\) are also shown to have different morphologies
\cite{Breuer_2016_acnene_mos2}. Individual close-packed islands were
formed in 30 nm-thick the pentacene/MoS\(_{\text{2}}\) film, while the deposited
film of perfluoropentacene on MoS\(_{\text{2}}\) showed much a smoother surface,
with the island size larger than 10 \(\mathrm{\mu m}\), as has been
observed in the case of pentacene and perfluoropentacene deposited on
graphene or graphite \cite{Salzmann_2012_fpen_gr,Breuer_2011_pent_graph}
(Figure \ref{fig-morphology}(c)). However the interfacial layer for both
molecules were found to be the face-on orientation. Indeed, the
morphology change caused by fluorination demonstrate how minor
chemical composition change of the epitaxial molecule may affect the
macroscopic behavior in 3D molecular epitaxy. While the detailed
mechanism is still not fully understood, the change of film surface
energy \(\gamma_{\mathrm{M}}\) caused by the fluorine atoms may be a
possible explanation, as indicated by the different
epitaxial growth models.

Another variable to control the macroscopic morphology is the
interfacial energy \(\gamma_{\mathrm{M-2D}}\), following the previous
discussion of molecular-substrate interactions. The wetting
transparency of graphene layer enables the modulation of interfacial
energy between pentacene and graphene by changing the wettability of
the underlying substrate \cite{Nguyen_2015_pent_gr_wett}. Although the
growth mechanism remains to be VW-type, epitaxy on a graphene-coated
hydrophobic surface shows a larger domain size, by reducing the
heterogeneous nucleation rate, since \(\gamma_{\mathrm{M-2D}}\) is
small. A smaller domain size, on the other hand, has been observed on
the graphene-coated hydrophobic surface.


\paragraph{Van der Waals Epitaxy of 3D Crystals}
\label{sec:orgeb0161b}

3D vdWE on 2D material interface shares the same mechanism with the 2D
vdWE. Here we limit the discussion of 3D vdWE to the growth of
crystalline material with non-planar (such as sp\(^{\text{3}}\)) bonding. The absence of dangling bonds at the
interface enables the growth of 3D bulk crystal with lattice
mismatch. Similar to the vdWE of 2D heterostructures, the growth of 3D
vdWE structure on layered 2D material is not a new idea. Layered TMDCs
(MoTe\(_{\text{2}}\), WSe\(_{\text{2}}\)) have been used for the growth of 3D crystalline
semiconductors like CdS and CdTe
\cite{Loeher_1994_vdw_epi_CdS_MoTe,Loeher_1996_CdTe_MoWTe}. In the case
of epitaxial CdTe/MoTe\(_{\text{2}}\) and CdTe/WSe\(_{\text{2}}\), the lattice mismatch is
as high as 30\% and 40\%, respectively \cite{Loeher_1996_CdTe_MoWTe},
showing the power of vdWE in fabricating 3D crystalline
structures. Recent examples have shown using 2D material as template
for 3D vdWE of crystalline semiconductors, include GaAs/graphene
\cite{Alaskar_2015_GaAs_gr_Si_theor,Kim_2017_remote_epi_Gr},
GaN/graphene
\cite{Chung_2010_GaN_ZnO_gr,Chung_2012_GaN_gr,Chung_2012_GaN_gr,Yoo_2013_GaN_gr_defect,Nepal_2013_GaN_gr,Kim_2014_direct_vdw_GaN_gr},
GaN/hBN \cite{Kobayashi_2012_GaN_hBN,Makimoto_2012_InGaN_hBN}, InGaN/hBN
\cite{Makimoto_2012_InGaN_hBN}, ZnO/graphene \cite{Chung_2010_GaN_ZnO_gr}
and ZnO/hBN \cite{Oh_2014_ZnO_hBN}. In these systems, the 2D material
not only serves as the buffer layer for vdWE, but also acts as a
transferable layer which enables separation of the 3D crystal from the
substrate to allow fabrication of semiconductor devices
\cite{Makimoto_2012_InGaN_hBN,Kobayashi_2012_GaN_hBN,Kim_2014_direct_vdw_GaN_gr,Kim_2017_remote_epi_Gr}.
However a general problem encountered in the 3D vdWE process is the
low density of nucleation sites and formation of smaller clusters
instead of continuous film
\cite{Loeher_1994_vdw_epi_CdS_MoTe,Loeher_1996_CdTe_MoWTe,Chung_2012_GaN_gr,Kobayashi_2012_GaN_hBN}.
In 2D vdWE process, the covalent bonds are confined in-plane which
enable the growth of large area 2D layer. On the contrary, in 3D vdWE
process, the 3D covalent bonds lead to comparable growth rate in all
directions, so that the crystal size highly depends on the interfacial
nucleation density. On the other hand, the growth of metal oxides
(e.g. Al\(_{\text{2}}\)O\(_{\text{3}}\), ZnO, TiO\(_{\text{2}}\)) on graphene or hBN lead to more
uniform films
\cite{Dlubak_2012_gr_sub_assist_Al2O3,Vaziri_2013_ALD_Al2O3_gr,Zhang_2014_Al2O3_ALO_Gr,Chung_2010_GaN_ZnO_gr,Oh_2014_ZnO_hBN,Zhang_2011_TiO2_gr,Kumar_2011_gr_TiO2_generator,Li_2015_TiO2_GO},
due to an increased interfacial adsorption of precursors. 2D materials
with a ZnO or AlN buffer layers have thus been used for vdWE of large
area crystalline semiconductors
\cite{Chung_2010_GaN_ZnO_gr,Yoo_2013_GaN_gr_defect,Nepal_2013_GaN_gr,Kobayashi_2012_GaN_hBN}. Kim
et al. showed the nucleation density of GaN on graphene can be
enhanced by the periodic step edges of epitaxial graphene/SiC surface
\cite{Kim_2014_direct_vdw_GaN_gr}. Large-area single crystalline film of
epitaxial GaN with high lattice ordering has been achieved, despite
the 23\% lattice mismatch between graphene and GaN. For epitaxy of
metal-oxide films, the monolayer assembled organic molecules can also
be used for seeded growth of high-quality metal oxide films on 2D
materials. For instance, Alaboson et al. used single layer
close-packed PTCDA molecules adsorbed on the epitaxial graphene/SiC
for the seeding growth of Al\(_{\text{2}}\)O\(_{\text{3}}\) and HfO\(_{\text{2}}\)
\cite{Alaboson_2011_PTCDA_gr_ALD}. The ALD-grown high-k metal oxide
films showed smoother morphology and less current leakage compared to
those grown in identical ALD conditions without the PTCDA seeding
layer. Besides the molecule-2D material interactions, recently Kim et
al. also showed that the 3D vdWE growth can be guided by the
interaction between the epitaxial overlayer and the underlying
substrate, in the case of the homoepitaxy of III-V semiconductors
(GaAs, InP and GaP) \cite{Kim_2017_remote_epi_Gr}. In other words,
graphene serves as a buffer layer for 3D vdWE. Since the thickness of
graphene (\textasciitilde{}3 \AA{}) is smaller than the vdW interaction distance of the
bulk material (\textasciitilde{}9 \AA{} for GaAs), the lattice of the underneath
substrate can guide remote homoepitaxy of the overlayer. In principle,
a series of single layer 2D materials can serve as the vdW buffer
layer for such homoepitaxy, and benefit the design and fabrication of
tandem 2D-3D heterostructures.


\subsubsection{Summary}
\label{sec:org0b4290f}

Table \ref{tbl-summary-multidimension} summarizes the representative
examples of multidimensional molecular epitaxy on 2D materials
interfaces, and outlines the governing interactions and the comparison
between the interactions that discussed in this section. A clear
relation between the interaction and the dimension of assembly can be
observed: molecular epitaxy of higher dimension is favored with
increasing intermolecular interactions. In the case of 2D and 3D vdWE,
the intermolecular interactions (covalent bond) can be much greater
than the molecule-2D interactions (vdW). It can also be found that the
molecule-substrate interactions are constantly weaker than the other
two types of interactions, as a result of increased interaction
distance and partial screening effect of the 2D material
layer. However as seen in the examples in this section, the absolute
order of interaction strength is not the only factor that determines
the packing and orientation of the molecular epitaxy. Understanding
the role of weak interactions in determining the molecular epitaxy
poses a challenge towards comprehensive theory framework, and is
crucial for the designing of epitaxial systems on 2D materials
interfaces.

\begin{table}[htbp]
\caption{\label{tbl-summary-multidimension}
Summary of representative examples of multidimensional molecular epitaxy on 2D materials interfaces, the governing interactions involved and the comparison between the interactions discussed in this section.}
\centering
\begin{tabularx}{0.95\textwidth}{lp{3cm}lp{4cm}lp{4cm}lp{5cm}}
\hline
Dimension & Examples & Governing interaction(s) & Comparison between interactions\\
\hline
0D & Strongly interacting surface & Site-specific & Molecule-2D \(\gg\) intermolecular\\
\hline
1D & Strongly interacting surface & H-bond,  site-specific & Molecule-2D \(\approx\) intermolecular\\
\hline
Fractal & Strongly interacting surface & H-bond, site-specific & Molecule-2D \(\approx\) intermolecular\\
 & Weakly interacting surface & Multivalent H-bond & Intermolecular  \textgreater{} molecule-2D\\
\hline
2D & ML on graphene & (H-bond), CT, \(\pi\)-\(\pi\) & Molecule-2D > intermolecular\\
 & ML on TMDC & (H-bond), CT, vdW & Intermolecular  \textgreater{} molecule-2D\\
 & 2D vdWE & Covalent bond,   vdW & Intermolecular  \(\gg\)  molecule-2D\\
\hline
3D & LbL assembly & (H-bond), CT, \(\pi\)-\(\pi\), vdW & Depending on the 2D material\\
 & 3D vdWE & Covalent bond,   vdW & Intermolecular \(\gg\) molecule-2D\\
\hline
\end{tabularx}
\end{table}




\section{Open Questions  Concerning 2D Interfaces}
\label{sec:chall-probl-conc}

\subsubsection{Electrostatic Interactions Through 2D Sheet}
\label{sec:electr-inter-thro}

\subsubsection{Dielectric Properties of 2D Systems}
\label{sec:diel-prop-2d}

\subsubsection{van der Waals (vdW) Interactions and Wetting Phenomena}
\label{sec:van-der-waals}



%%% Local Variables:
%%% mode: latex
%%% TeX-master: "../thesis"
%%% End:
