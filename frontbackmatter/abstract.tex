%*******************************************************
% Abstract
%*******************************************************
%\renewcommand{\abstractname}{Abstract}
\pdfbookmark[1]{Abstract}{Abstract}
\begingroup
\let\clearpage\relax
\let\cleardoublepage\relax
\let\cleardoublepage\relax

\chapter*{Abstract}

% English abstract
Two-dimensional (2D) materials, the crystalline films with one- to
few-atom thickness, have attracted the spotlight of material research
in recent years.
% 
Due to the quantum-confined electronic structures, the 2D materials
serve as ideal platforms for studying low-dimensional physics and
chemistry.
% 
Despite extensive research focusing on the intrinsic characteristics
of 2D materials, little has been studied regarding their interfacial
properties, which are essential for the integration into modern 
technology platforms.
%
The aim of this thesis to provide fundamental insights into the
atomically thin interfaces by developing theoretical frameworks that
bridge multiscale phenomena and providing guidelines for experimental
demonstrations.  The thesis is organized
in four main parts as follows:

The first part gives a brief introduction of the 2D materials and
their interfaces.
%
The focus lies on the various types of interactions
involved at the interfaces, as well as their connection with the
fundamental electronic structures of 2D materials.
%
In particular, several open questions in the context of
2D material interfaces are also reviewed.

The second part focuses on the properties of 2D materials interfaces
under static electric field, which is governed by the quantum
capacitance ($C_{\mathrm{Q}}$) of the 2D materials.
%
Two studies are presented in this part to demonstrate the practical
impact of quantum capacitance:
%
(i) Using a self-consistent Poisson-Boltzmann model, the penetration of field effect through 2D material is studied. The field-effect transparency
$\eta^{\mathrm{FE}}$ of a 2D material is quantified. The
non-linear dependency of $\eta^{\mathrm{FE}}$ on $C_{\mathrm{Q}}$ and other
parameters are studied in order to guide practical design of 2D
material-based vertical field effect devices.
%
(ii) A multilayer quantum capacitor model is proposed to study the
penetration of field effect in 2D heterostructures. Using relatively
simple parameters including density of states (DOS), band alignment
and interlayer dielectric constant, the model captures the
experimentally observed asymmetric electrostatic screening with
similar accuracy compared with full-scale \textit{ab initio}
simulations.

The third part focuses on another fundamental aspect of 2D materials
interfaces: the dielectric properties.
%
%
Two examples are
demonstrated in this part:
%
(i) Using first principles calculations, we show that instead of the
macroscopic dielectric constant, the 2D electronic polarizability
$\alpha_{\mathrm{2D}}$, is the true descriptor of the dielectric
properties for a 2D material.  Using high-throughput material database
screening, two universal scaling relations for $\alpha_{\mathrm{2D}}$
are proposed, linking the polarizability of a 2D material with its
electronic and structural information.
%
The idea
of $\alpha_{\mathrm{2D}}$ is further to be valid for heterostructures
and even bulk systems, allowing quantifying the dielectric anisotropy
for any dimension.
%
(ii) Using a modified Lifshitz model and knowledge of
frequency-dependent dielectric properties, the van der Waals (vdW)
interactions at the 2D material interfaces are studied.
The 
dielectric anisotropy of a 2D material selectively screens the vdW
interactions at low frequency regime. More interestingly, by proper
engineering dielectric properties of 2D and bulk materials, repulsive
vdW interactions are predicted by the model, and validated by
experimental investigating using molecular epitaxy.

Based on these fundamental understandings, several
studies on multiscale phenomena at the 2D material interfaces are
shown in the last part.
%
(i) By combining multiscale phenomena including quantum capacitance,
interfacial molecular reorientation, electrical double layer (EDL),
the wetting phenomena of a 2D material upon doping are studied. The
molecular reorientation effect is found to dominate the 2D-liquid
interfacial tension.
%
(ii) Using self-consistent transport theory based on
Poisson-Nernst-Planck equation and Quantum capacitance, the ionic
transport through nanopores in a gated graphene sheet is
studied. Gating is found to enable close-to-unity rejection of ionic
species, which is in good agreement with experimental observations.
%
(iii) Taking advantage of multiscale phenomena on 2D material
interfaces, a novel electronic device named as interfacial field
effect transistor (IFET), is proposed and fabricated. The IFET has
ultra-sensitive pressure response down below 10 Pa, due to extremely low
elastic modulus of liquid metal droplet. Mechanical response is
harnessed by deformation on superhydrophobic nanowires assembled on
graphene interface.

The studies presented in thesis aim to provide insights into the
atomically thin interfaces, as well as providing guidelines and design
rules for novel electronic devices and applications.


\endgroup

\cleardoublepage%

\begingroup
\let\clearpage\relax
\let\cleardoublepage\relax
\let\cleardoublepage\relax

\begin{otherlanguage}{ngerman}
\pdfbookmark[1]{Zusammenfassung}{Zusammenfassung}
\chapter*{Zusammenfassung}

Zweidimensionale (2D) Materialien, die kristalline Filme mit mehere Atome in der Dicke haben das Rampenlicht der Materialforschung angezogen
in den letzten Jahren.
%
Aufgrund der quantenbeschränkten elektronischen Strukturen sind die 2D-Materialien
als ideale Plattform für das Studium der niederdimensionalen Physik und
Chemie.
%
Trotz umfangreicher Forschung konzentriert sich auf die intrinsischen Eigenschaften
von 2D-Materialien, nur wenig Forschung auf ihre Grenzfläche wurde untersucht.
Die Grenzfläche der 2D Materialien sind unerlässlich für die Integration in die moderne
Technologieplattformen.
%
Ziel dieser Arbeit ist, grundlegende Einblicke in die
atomar dünne Grenzflächen zu vermitteln.
Es werden theoretische Rahmenbedingungen entwickelt, die multiskalige Phänomene verbinden.
Sie bieten auch Richtlinien für experimentelle
Demonstrationen.
Die Arbeit ist organisiert
in vier Hauptteilen wie folgt:

Der erste Teil gibt eine kurze Einführung in die 2D-Materialien und
ihre Grenzfläche.
%
Der Fokus liegt auf den verschiedenen Arten von Interaktionen
an der Grenzfläche beteiligt, sowie deren Verbindung mit der
grundlegende elektronische Strukturen von 2D-Materialien.
%
Insbesondere einige offene Fragen im Zusammenhang mit
2D-Materialsgrenzfläche werden ebenfalls diskutiert.

Der zweite Teil befasst sich mit den Eigenschaften von
2D-Materialgrenzfläche unter statischem elektrischem Feld, das vom
Quantenkapazität ($C_{\mathrm{Q}}$) der 2D-Materialien bestimmt
werde.
%
In diesem Teil werden zwei Studien vorgestellt, um die praktische Wirkung der Quantenkapazität zu demonstrieren:
%
(i) Unter Verwendung eines selbstkonsistenten
Poisson-Boltzmann-Modells wird die Penetration des Feldeffekts durch
2D-Material untersucht. Die Feldeffekttransparenz $\eta^{\mathrm{FE}}$
eines 2D-Materials wird quantifiziert. Die nichtlineare Abhängigkeit
von $\eta^{\mathrm{FE}}$ von $C_{\mathrm{Q}}$ und anderen Parameter
werden untersucht, um die praktische Gestaltung für vertikale
Feldeffektgeräte zu leiten.
%
(ii) Ein mehrschichtiges Quantenkondensatormodell um die
Penetration des Feldeffekts in 2D-Heterostrukturen wird vorgeschlagen.
%
Mit meheren relativ einfachen Parametern, einschließlich Zustandsdichte (DOS), Bandausrichtung
und zwischenschicht Dielektrizitätskonstante erfasst das Modell die
experimentell beobachtete asymmetrische elektrostatische.
Die Methode zeigt auch eine ähnliche Genauigkeit im Vergleich zu \textit{ab initio}
Simulationen.

Der dritte Teil konzentriert sich auf einen weiteren grundlegenden Aspekt von 2D-Materialien
Grenzflächen: die dielektrischen Eigenschaften.
%
%
Zwei Beispiele sind
in diesem Teil demonstriert:
%
(i) Anhand von Berechnungen der ersten Prinzipien zeigen wir, dass anstelle der
makroskopische Dielektrizitätskonstante, die 2D elektronische Polarisierbarkeit
$\alpha_{\mathrm{2D}}$ ist der wahre Deskriptor des Dielektrikums
Eigenschaften für ein 2D-Material.
%
Mit Hochdurchsatz-Screening auf Materialdatenbank, zwei universelle Skalierungsrelationen für $\alpha_{\mathrm{2D}}$
werden vorgeschlagen, die Polarisierbarkeit eines 2D-Materials mit seiner
elektronische und strukturelle Informationen zu verbinden.
%
Die Idee
von $\alpha_{\mathrm{2D}}$ gilt weiterhin für Heterostrukturen
und sogar Bulk-Systeme, mit denen die dielektrische Anisotropie für jede dimension quantifiziert werden kann.
%
(ii) Verwendung eines modifizierten Lifshitz-Modells und Kenntnis von
frequenzabhängige dielektrische Eigenschaften, van der Waals (vdW)
Wechselwirkungen an den 2D-Materialgrenzflächen werden untersucht.
%
Die dielektrische Anisotropie eines 2D-Materials reduziert selektiv den vdW
Wechselwirkungen im Niederfrequenzbereich.
%
Durch die Entwicklung der dielektrischen Eigenschaften von 2D- und Bulk-Materialen werden
 abstoßend
 vdW-Wechselwirkungen vom Modell vorhergesagt.
 Die abstoßende vdW-Wechselwirkungen werden ebenfalls von 
experimentelle Untersuchung mittels molekularer Epitaxie validiert.


Basierend auf diesen grundlegenden Erkenntnissen werden im letzten Teil mehrere Studien zu Multiskalenphänomenen an den 2D-Materialgrenzflächen gezeigt.
%
(i) Durch Kombinieren von Multiskalenphänomenen, einschließlich der Quantenkapazität,
molekulare Orientierung an der Grenzfläche, elektrische Doppelschicht (EDL),
Die Benetzungsphänomene eines 2D-Materials beim Dotieren werden untersucht.
Der molekulare Umorientierungseffekt dominiert die 2D-Flüssigkeit
Grenzflächenspannung.
%
(ii) Mit der Poisson-Nernst-Planck-Gleichung und der Quantenkapazität
wird der Ionentransport durch Nanoporen in einer geschlossenen
Graphenschicht untersucht.
Es wurde festgestellt, dass das Gating die
Ionenzurückweisung ermöglicht, was mit experimentellen Beobachtungen übereinstimmt

%
(iii) Basierend auf der Theorie der 2D-Materialengrenzflä wird  ein neuartiges elektronisches Gerät, als Interfacial Field Effect Transistor (IFET) vorgeschlagen und hergestellt.
%
Das IFET hat
ultraempfindliche Druckreaktion unter 10 Pa aufgrund extrem niedriger Elastizitätsmodul von Flüssigmetalltröpfchen. Die mechanische Reaktion ist
nutzbar gemacht durch Verformung des Flüssigmetalltröpfchen an superhydrophoben Nanodrähten auf die Graphen-grenzfläche.

Die vorgestellten Studien sollen Einblicke in die
atomar dünne Grenzfläche sowie Richtlinien und Design
Regeln für neuartige elektronische Geräte und Anwendungen.

\end{otherlanguage}

\endgroup

\vfill