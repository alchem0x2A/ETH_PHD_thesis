%*******************************************************
% Abstract
%*******************************************************
%\renewcommand{\abstractname}{Abstract}
\pdfbookmark[1]{Abstract}{Abstract}
\begingroup
\let\clearpage\relax
\let\cleardoublepage\relax
\let\cleardoublepage\relax

\chapter*{Abstract}

% English abstract
Two-dimensional (2D) materials, the crystalline film with one- to
few-atom thickness, have attracted the spotlight of material research
in recent years.
% 
Due to the quantum-confined electronic structures, the 2D materials
serve as ideal platforms for studying low-dimensional physics and
chemistry.
% 
Despite extensive research focusing on the intrinsic characteristics
of 2D materials, little has been studied of their interfacial
properties, which are essential for the integration into modern 3D
technology platforms.
%
The aim of this thesis to provide fundamental insights into the
atomically thin interfaces by developing theoretical frameworks that
bridge multiscale phenomena and providing guidelines for experimental
demonstrations.  The thesis is organized
in four main parts as follows:

The first part gives a brief introduction of the 2D materials and
their interfaces.
%
The focus lies on the various types of interactions
involved at the interfaces, as well as the connection with the
fundamental electronic structures of 2D materials.
%
In particular, several open questions in the context of
2D material interfaces are also reviewed.

The second part focuses on the properties of 2D materials interfaces
under static electric field, which is governed by the quantum
capacitance ($C_{\mathrm{Q}}$) of the 2D materials.
%


Two studies are presented in this part to demonstrate the practical
impact of quantum capacitance:
%
(i) Using a self-consistent Poisson-Boltzmann model, the penetration of field effect through 2D material is studied. The field-effect transparency
$\eta^{\mathrm{FE}}$ of a 2D material is quantified. The
non-linear dependency of $\eta^{\mathrm{FE}}$ on $C_{\mathrm{Q}}$ and other
parameters are studied in order to guide practical design of 2D
material-based vertical field effect devices.
%
(ii) A multilayer quantum capacitor model is proposed to study the
penetration of field effect in 2D heterostructures. Using relatively
simple parameters including density of states (DOS), band alignment
and interlayer dielectric constant, the model captures the
experimentally observed asymmetric electrostatic screening with
similar accuracy compared with full-scale \textit{ab initio}
simulations.

The third part focuses on another fundamental aspect of 2D materials
interfaces: the dielectric properties.
%

%
Two examples are
demonstrated in this part:
%
(i) Using first principles calculations, we show that instead of the
macroscopic dielectric constant, the 2D electronic polarizability
$\alpha_{\mathrm{2D}}$, is the true descriptor of the dielectric
properties for a 2D material.  Using high-throughput material database
screening, two universal scaling relations for $\alpha_{\mathrm{2D}}$
are proposed, linking the polarizability of a 2D material with its
electronic and structural information.
%
The idea
of $\alpha_{\mathrm{2D}}$ is further to be valid for heterostructures
and even bulk systems, allowing quantifying the dielectric anisotropy
for any dimension.
%
(ii) Using a modified Lifshitz model and knowledge of
frequency-dependent dielectric properties, the van der Waals (vdW)
interactions at the 2D material interfaces are studied.
The high
dielectric anisotropy of a 2D material selectively screens the vdW
interactions at low frequency regime. More interestingly, by proper
engineering dielectric properties of 2D and bulk materials, repulsive
vdW interactions are predicted by the model, and validated by
experimental investigating using molecular epitaxy.

Based on these fundamental understandings, several
studies on multiscale phenomena at the 2D material interfaces are
shown in the last part.
%
(i) By combining multiscale phenomena including quantum capacitance,
interfacial molecular reorientation, electrical double layer (EDL),
the wetting phenomena of a 2D material upon doping are studied. The
molecular reorientation effect is found to dominate the 2D-liquid
interfacial tension.
%
(ii) Using self-consistent transport theory based on
Poisson-Nernst-Planck equation and Quantum capacitance, the ionic
transport through nanopores in a gated graphene sheet is
studied. Gating is found to enable close-to-unity rejection of ionic
species, which is in good agreement with experimental observations.
%
(iii) Taking advantage of multiscale phenomena on 2D material
interfaces, a novel electronic device named as interfacial field
effect transistor (IFET), is proposed and fabricated. The IFET has
ultra-sensitive pressure response down to 1 Pa, due to extremely low
elastic modulus of liquid metal droplet. Mechanical response is
harnessed by deformation on superhydrophobic nanowires assembled on
graphene interface.

The studies presented in thesis aim to provide insights into the
atomically thin interfaces, as well as providing guidelines and design
rules for novel electronic devices and applications.


\endgroup

\cleardoublepage%

\begingroup
\let\clearpage\relax
\let\cleardoublepage\relax
\let\cleardoublepage\relax

\begin{otherlanguage}{ngerman}
\pdfbookmark[1]{Zusammenfassung}{Zusammenfassung}
\chapter*{Zusammenfassung}

Zweidimensionale (2D) Materialien, der kristalline Film mit Eins-zu-Eins
wenige Atome Dicke haben das Rampenlicht der Materialforschung angezogen
in den vergangenen Jahren.
%
Die quantenbeschränkte elektronische Struktur von 2D-Materialien macht sie zu idealen Plattformen für das Studium der niedrigdimensionalen Physik und Chemie.
%
Trotz umfangreicher Forschung konzentriert sich auf die intrinsischen Eigenschaften
Von 2D-Materialien wurde bisher nur wenig über ihre Grenzflächen untersucht
Eigenschaften, die für die Integration in modernes 3D unerlässlich sind
Technologieplattformen.
%
Ziel dieser Arbeit ist es, grundlegende Einsichten in die
atomar dünne Grenzflächen durch die Entwicklung theoretischer Rahmenbedingungen
Überbrückung von Multiskalenphänomenen und Bereitstellung von Richtlinien für experimentelle
Demonstration. \ worktodo {letzten Teil polieren}. Die Arbeit ist organisiert
in vier Hauptteilen wie folgt:

Der erste Teil gibt eine kurze Einführung in die 2D-Materialien und
ihre Schnittstellen.
%
Der Fokus liegt auf den verschiedenen Arten von Interaktionen
an der Schnittstelle beteiligt, sowie die Verbindung mit der
grundlegende elektronische Strukturen von 2D-Materialien.
%
Insbesondere bei der Erforschung des
2D-Schnittstellen werden somit überprüft. \ worktodo {Poliere diese Sätze} \ worktodo {Erwähne
elektrostatische Wechselwirkungen} \ worktodo {Erwähne das Multiskalen
Eigenschaft beeinflusst durch fundamentale Eigenschaften}


Im Gegensatz zu ihrer Masse
Gegenstücke, 2D-Materialien können das teilweise Eindringen von außen ermöglichen
Feld aufgrund der relativ kleinen ($ C _ {\ mathrm {Q}} $).
%
Zwei Studien sind
in diesem Teil vorgestellt, um die praktischen Auswirkungen solcher zu demonstrieren
Phänomene:
%
(i) Verwenden eines selbstkonsistenten Modells basierend auf
Poisson-Boltzmann-Gleichungen, der Feldeffekttransparenzindex
$ \ eta $ eines 2D-Materials zu einem externen Feld wird quantifiziert. Das
nichtlineare Abhängigkeit von $ \ eta $ von $ C _ {\ mathrm {Q}} $ und anderen
Parameter werden untersucht, um die praktische Gestaltung von 2D zu leiten
materialbasierte vertikale Feldeffektgeräte.
%
(ii) Die Selbstkonsistenz
Das elektrostatische Modell wird für das Studium weiter ausgebaut
Feldeffekt in 2D-Heterostrukturen. Mit relativ einfach
Parameter einschließlich Zustandsdichte (DOS), Bandausrichtung und
Zwischenschicht-Dielektrizitätskonstante erfasst das Modell die experimentell beobachtete Asymmetrie
elektrostatische Abschirmung mit ähnlicher Genauigkeit wie im Originalmaßstab
\ textit {ab initio} Simulationen.


Der dritte Teil über einen weiteren grundlegenden Aspekt von 2D-Materialien
Grenzflächen: die dielektrischen Eigenschaften.
%
Im Gegensatz zu Schüttgütern ist das
Die makroskopische Dielektrizitätskonstante $ \ varepsilon $ für ein 2D-Material ist

%
Als Ersatz,
Die elektronische Polarisierbarkeit wird als der wahre Deskriptor von vorgeschlagen
die dielektrischen Eigenschaften für ein 2D-Material.
%
%
(i) Durch Hochdurchsatz-Screening basierend auf ersten Prinzipien
Simulationen, zwei universelle Skalierungsrelationen für
2D-Material mit seinen elektronischen und strukturellen Informationen.
%
Die Idee
von $ \ alpha _ {\ mathrm {2D}} $ gilt weiterhin für Heterostrukturen
und sogar Massensysteme, die die dielektrische Anisotropie quantifizieren
für jede dimension.
%
(ii) Ausnutzen der elektronischen Polarisierbarkeit und weiter
Die van der Waals - Wechselwirkungen (vdW) erstrecken sich auf den Frequenzbereich
Die 2D-Materialschnittstellen werden untersucht. Der Grad des VD-Screenings
auf die Existenz eines 2D-Materials wird anhand eines modifizierten quantifiziert
Lifshitz vdW Formalismus. Die hohe dielektrische Anisotropie einer 2D
materialscreenings screenen die vdW-wechselwirkungen mit niedriger frequenz
Regime. Interessanter, durch richtige Technik
Eigenschaften von 2D- und Schüttgütern sind abstoßende vdW-Wechselwirkungen
vom Modell vorhergesagt und durch experimentelle Untersuchungen validiert
mit molekularer Epitaxie.

Basierend auf diesen grundlegenden Erkenntnissen, mehrere
Untersuchungen zu Multiskalenphänomenen an den 2D-Materialgrenzflächen liegen vor
im letzten Teil gezeigt.
%
(i) Durch Kombinieren von Multiskalenphänomenen, einschließlich der Quantenkapazität,
molekulare Neuorientierung an der Grenzfläche, elektrische Doppelschicht (EDL),
Die Benetzungsphänomene eines 2D-Materials beim Dotieren werden untersucht. Das
Es wurde festgestellt, dass der molekulare Umorientierungseffekt die 2D-Flüssigkeit dominiert
Grenzflächenspannung.
%
(ii) Verwenden einer auf
Poisson-Nernst-Planck-Gleichung und Quantenkapazität, die ionische
Transport durch Nanoporen in einer mit einem Gatter versehenen Graphenschicht ist
von studiert. Es wurde festgestellt, dass das Gating die Zurückweisung von Ionen nahe an der Einheit ermöglicht
Spezies, was gut mit experimentellen Beobachtungen übereinstimmt.
%
(iii) Ausnutzen von Multiskalenphänomenen auf 2D-Material
interfaces, ein neuartiges elektronisches Gerät, das als Grenzflächenfeld bezeichnet wird
Effekttransistor (IFET) vorgeschlagen und hergestellt.

\end{otherlanguage}

\endgroup

\vfill